
% Abstract

\cleardoublepage

\addchap*{Abstract}

% An abstract of not more than 400 words should be bound into each copy of the examinable outcome. The abstract should indicate the problem investigated, the procedures followed, the general result obtained and the major conclusions reached. It should not contain any illustrative material or tables.

A method of simulating the full quantum field dynamics of multimode multi-component Bose-Einstein condensates is developed.
The truncated Wigner representation is used to obtain a probabilistic theory.
The representation is extended to to use a functional calculus and can be applied to quantum fields in the arbitrary number of dimensions.
Detailed proofs of the corresponding theorems are given.

This method produces c-number stochastic equations which may be solved using conventional stochastic integration methods.
The technique is valid for large mode occupation numbers.
It describes spatial evolution of spinor components and properly accounts for nonlinear interactions and losses.
The method is further used to model several experiments where quantum effects are significant, and cannot be calculated with simpler approximations.
It is shown to describe leading quantum corrections accurately for effects such as phase noise, quantum squeezing, entanglement, and interactions with engineered nonlinear reservoirs.

In conclusion, the functional Wigner representation framework developed in this thesis provides a relatively fast and accurate method of simulating dynamics of large nonlinear quantum systems, such as trapped Bose-Einstein condensates.
The methods of solving the resulting stochastic equations are easily and scalably computed in parallel.
Methods for efficient computation on modern graphical processing unit hardware are presented.

% Acknowledgements

\cleardoublepage

\addchap*{Acknowledgements}

I would like to thank both of my supervisors, Prof.~Peter~D. Drummond and Prof.~Andrei~I. Sidorov, for many helpful discussions.

My primary supervisor, Peter Drummond, provided a fundamental outlook on the methods that I developed and used, employing both mathematical arguments and physical considerations.
He could always give a piece of practical advice on the application of numerical methods to various problems.
Besides that, I appreciate him encouraging me to venture into the field of parallel computing, which turned out to be perfectly suited to solve the problems we were investigating.

My secondary supervisor, Andrei Sidorov, supplied expertise in the experimental side of my research, explaining where and how the theoretical models I used clash with reality.

I thank my fellow theorists: Rodney Polkinghorne, Dr~Qiong-Yi He, Dr~Laura Rosales-Z\'arate, who gave me advice on tricky problems I was facing, and with whom I enjoyed co-authoring papers.
I also thank Dr~Michael Egorov and Dr~Valentine Ivannikov from the experimental part of our research group, who had a hands-on expertise on the atom interferometry experiments in Swinburne, and who supplied me with data necessary to test my models.
