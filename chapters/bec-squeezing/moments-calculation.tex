% =============================================================================
\section{Moments of field operator in Wigner representation}
% =============================================================================


This chapter shows how to calculate different kinds of observables from wavefunctions in Wigner representation.
From the definition of Wigner function~\cite{Gardiner2004}:
\[
	\langle \symprod{ \hat{a}^r ( \hat{a}^\dagger)^s } \rangle
	= \int \alpha^r (\alpha^*)^s W (\alpha, \alpha^*) d^2\alpha ,
\]
where $\{\}_{\mathrm{sym}}$ stands for symmetrically ordered operator product.
It can be shown that similar relation applies for the multimode field operator:
\[
	\langle \symprod{ \Psiop^r ( \Psiop^\dagger)^s } \rangle
	= \int \Psi^r (\Psi^*)^s W (\Psi, \Psi^*) \delta^2\Psi.
\]
This equation can be further generalised for multi-component field.
In simulations, Wigner function $W$ can be treated as the probability distribution, allowing to replace the integral by average over simulation paths:
\[
	\int \Psi^r (\Psi^*)^s W (\Psi, \Psi^*) \delta^2\Psi
	= \pathavg{ \Psi^r (\Psi^*)^s }
	= \frac{1}{N_{\mathrm{paths}}} \sum\limits_{j=1}^{N_{\mathrm{paths}}}
		\Psi^{(j)r} (\Psi^{(j)*})^s,
\]
where superscript $(j)$ denotes the value taken from $j$-th simulation path.


% =============================================================================
\subsection{Number of atoms}
% =============================================================================

First example is the calculation of atom density:
\begin{equation*}
\begin{split}
		\langle \hat{n} (\xvec) \rangle
		& = \langle \Psiop^\dagger (\xvec) \Psiop (\xvec) \rangle \\
		& = \langle
				\symprod{ \Psiop^\dagger \Psiop }
			\rangle - \frac{1}{2} \delta_P (\xvec, \xvec) \\
		& = \pathavg{ \Psi^* (\xvec) \Psi (\xvec) }
			- \frac{1}{2} \delta_P (\xvec, \xvec) \\
		& = \pathavg{ n (\xvec) }
			- \frac{1}{2} \delta_P (\xvec, \xvec).
\end{split}
\end{equation*}
Defining population operator $\hat{N}$ as
\[
	\hat{N} = \int \hat{n} (\xvec) d\xvec,
\]
we can get the average of total population:
\[
		\langle \hat{N} \rangle
		= \int \langle \hat{n}(\xvec) \rangle d\xvec
		= \int \pathavg{ n(\xvec) } d\xvec - \frac{M}{2}
		= \pathavg{ \int n(\xvec) d\xvec } d\xvec - \frac{M}{2}
		= \pathavg{N} - \frac{M}{2},
\]
where $\delta_P (\xvec, \xvec)$ is a restricted delta function from \defref{func-calculus:restricted-delta}.
Its integral over space equals to the number of modes $M$ in the restricted basis.

Variance of total number $N$ is expressed in a slightly more complicated way.
\[
	(\Delta N)^2
		= \langle \hat{N}^2 \rangle - \langle \hat{N} \rangle^2
\]
Average of $\hat{N}^2$ requires some work.
Denoting $\Psiop(\xvec) \equiv \Psiop$ and $\Psiop(\xvec^\prime) \equiv \Psiop^\prime$ for simplicity:
\[
	\hat{N}^2
		= \int \Psiop^\dagger \Psiop d\xvec
			\int \Psiop^\dagger \Psiop d\xvec
		= \int
			\Psiop^\dagger \Psiop
			\Psiop^{\prime\dagger} \Psiop^\prime
			d\xvec d\xvec^\prime
\]
\begin{equation*}
\begin{split}
	\langle
		\Psiop^\dagger \Psiop \Psiop^{\prime\dagger} \Psiop^\prime
	\rangle
	& = \langle
		\symprod{ \Psiop^{\prime\dagger} \Psiop^\prime \Psiop^\dagger \Psiop}
		- \frac{\delta_P(\xvec^\prime,\xvec^\prime)}{2} \symprod{\Psiop^\dagger \Psiop}
		- \frac{\delta_P(\xvec,\xvec)}{2} \symprod{\Psiop^{\prime\dagger} \Psiop^\prime} \\
	& - \frac{\delta_P(\xvec,\xvec^\prime)}{2} \symprod{\Psiop^{\prime\dagger} \Psiop}
		+ \frac{\delta_P(\xvec^\prime,\xvec)}{2} \symprod{\Psiop^\dagger \Psiop^\prime}
		+ \frac{\delta_P(\xvec,\xvec) \delta_P(\xvec^\prime,\xvec^\prime)}{2}
	\rangle.
\end{split}
\end{equation*}
Therefore the average of $\hat{N}^2$ is:
\begin{eqn*}
	\langle \hat{N}^2 \rangle & = \int
		\langle
			\Psiop^\dagger \Psiop \Psiop^{\prime\dagger} \Psiop^\prime
		\rangle
	d\xvec d\xvec^\prime \\
	& = \int \pathavgleft
		\Psi^* \Psi \Psi^{\prime *} \Psi^\prime
		- \frac{\delta_P(\xvec^\prime,\xvec^\prime)}{2} \Psi^* \Psi
		- \frac{\delta_P(\xvec,\xvec) }{2} \Psi^{\prime *} \Psi^\prime \right. \\
	&	\left. - \frac{\delta_P(\xvec,\xvec^\prime)}{2} \Psi^{\prime *} \Psi
		+ \frac{\delta_P(\xvec^\prime,\xvec)}{2} \Psi^* \Psi^\prime
		+ \frac{\delta_P(\xvec,\xvec) \delta_P(\xvec^\prime,\xvec^\prime)}{2}
	\pathavgright d\xvec d\xvec^\prime \\
	& = \pathavgleft
		\int \Psi^* \Psi d\xvec \int \Psi^* \Psi d\xvec
		- \frac{M}{2} \int \Psi^* \Psi d\xvec
		- \frac{M}{2} \int \Psi^* \Psi d\xvec \right. \\
	&	\left. - \int \frac{\delta_P(\xvec,\xvec^\prime)}{2} \Psi^{\prime *} \Psi d\xvec d\xvec^\prime
		+ \int \frac{\delta_P(\xvec^\prime,\xvec)}{2} \Psi^* \Psi^\prime d\xvec d\xvec^\prime
		+ \frac{M^2}{2}
	\pathavgright \\
	& = \pathavg{N^2 - M N + \frac{M^2}{2}}.
\end{eqn*}
Here we used the correspondence between the average of the symmetric product of field operators and the average of wavefunctions over simulation paths.
Then we can split variables in double integrals, allowing us to group terms and simplify the whole equation.
Substituting this into equation for $(\Delta N)^2$:
\begin{equation}
\label{eqn:moments-calculation:delta-N}
	(\Delta N)^2
		= \langle \hat{N}^2 \rangle - \langle \hat{N} \rangle^2
		= \pathavg{N^2 - M N + \frac{M^2}{2}} - (\pathavg{N} - \frac{M}{2})^2
		= \pathavg{N^2} - \pathavg{N}^2 + \frac{M^2}{4}
\end{equation}


% =============================================================================
\subsection{Spin vector}
% =============================================================================

Another example is the spin vector, whose averages and variances are required for squeezing calculation~\cite{Li2009}.
Spin operators are defined as following:
\begin{equation}
\label{eqn:moments-calculation:spin-operators}
\begin{split}
	\hat{S}_x & = \frac{1}{2} \int \left(
		\Psiop^\dagger_2 \Psiop_1 + \Psiop^\dagger_1 \Psiop_2
	\right) d\xvec, \\
	\hat{S}_y & = \frac{i}{2} \int \left(
		\Psiop^\dagger_2 \Psiop_1 - \Psiop^\dagger_1 \Psiop_2
	\right) d\xvec, \\
	\hat{S}_z & = \frac{1}{2} \int \left(
		\Psiop^\dagger_1 \Psiop_1 - \Psiop^\dagger_2 \Psiop_2
	\right) d\xvec.
\end{split}
\end{equation}
Averages of spin operators can be calculated straightforwardly (using the fact that interspecies commutators $[\Psiop_1, \Psiop_2] = [\Psiop^\dagger_1, \Psiop_2] = 0$):
\begin{equation*}
\begin{split}
	\langle \hat{S}_x \rangle
		& = \pathavg{\Real \int \Psi^*_1 \Psi_2 d\xvec }
		= \pathavg{\Real I}
		= \pathavg{S_x}, \\
	\langle \hat{S}_y \rangle
		& = \pathavg{\Imag \int \Psi^*_1 \Psi_2 d\xvec }
		= \pathavg{\Imag I}
		= \pathavg{S_y}, \\
	\langle \hat{S}_z \rangle
		& = \frac{1}{2} \pathavg{\int \Psi^*_1 \Psi_1 d\xvec - \int \Psi^*_2 \Psi_2 d\xvec}
		= \frac{1}{2} (\pathavg{N_1 - N_2})
		= \pathavg{S_z},
\end{split}
\end{equation*}
where we introduced auxiliary per-path interaction values $I^{j}$ and per-path spin component values, whose definitions are an intuitive consequence of equations~\eqnref{moments-calculation:spin-operators}:
\begin{equation*}
\begin{split}
	S^{(j)}_x & = \frac{1}{2} \int \left(
		\Psi^{(j)*} \Psi^{(j)}_1 + \Psi^{(j)*}_1 \Psi^{(j)}_2
	\right) d\xvec, \\
	S^{(j)}_y & = \frac{i}{2} \int \left(
		\Psi^{(j)*}_2 \Psi^{(j)}_1 - \Psi^{(j)*}_1 \Psi^{(j)}_2
	\right) d\xvec, \\
	S^{(j)}_z & = \frac{1}{2} \int \left(
		\Psi^{(j)*}_1 \Psi^{(j)}_1 - \Psi^{(j)*}_2 \Psi^{(j)}_2
	\right) d\xvec,
\end{split}
\end{equation*}
where $j$ stands for the number of the simulation path.

Second-order moments of spin operators can be obtained similarly to second-order moment of population operator, by transforming normally ordered field operator products to symmetrically ordered ones, substituting them for path averages of wavefunction moments and grouping terms.
\begin{equation*}
\begin{split}
	\langle \hat{S}^2_x \rangle
	& = \frac{1}{4} \langle \int \left(
		\Psiop^\dagger_2 \Psiop_1 + \Psiop^\dagger_1 \Psiop_2
	\right)
	\left(
		\Psiop^{\prime\dagger}_2 \Psiop^\prime_1 + \Psiop^{\prime\dagger}_1 \Psiop^\prime_2
	\right) d\xvec d\xvec^\prime \rangle \\
	& = \frac{1}{4} \langle \int \left(
		\symprod{ \Psiop^\dagger_2 \Psiop_1 \Psiop^{\prime\dagger}_2 \Psiop^\prime_1 }
		+ \symprod{ \Psiop^\dagger_1 \Psiop_2 \Psiop^{\prime\dagger}_2 \Psiop^\prime_1 }
		+ \symprod{ \Psiop^\dagger_1 \Psiop_2 \Psiop^{\prime\dagger}_1 \Psiop^\prime_2 }
		+ \symprod{ \Psiop^\dagger_2 \Psiop_1 \Psiop^{\prime\dagger}_1 \Psiop^\prime_2 }
	\right. \\
	& \left.
		+ \frac{\delta_P(\xvec,\xvec^\prime)}{2} \left(
			- \symprod{ \Psiop_2 \Psiop^{\prime\dagger}_2 }
			- \symprod{ \Psiop_1 \Psiop^{\prime\dagger}_1 }
			+ \symprod{ \Psiop^\dagger_2 \Psiop^\prime_1 }
			+ \symprod{ \Psiop^\dagger_1 \Psiop^\prime_2 }
		\right)
	\right) d\xvec d\xvec^\prime \rangle \\
	& = \frac{1}{4} \pathavgleft
		\int \Psi^*_2 \Psi_1 d\xvec \int \Psi^*_2 \Psi_1 d\xvec
		+ \int \Psi^*_1 \Psi_2 d\xvec \int \Psi^*_2 \Psi_1 d\xvec \right. \\
	&	\left. + \int \Psi^*_1 \Psi_2 d\xvec \int \Psi^*_1 \Psi_2 d\xvec
		+ \int \Psi^*_2 \Psi_1 d\xvec \int \Psi^*_1 \Psi_2 d\xvec \pathavgright \\
	& = \frac{1}{4} \pathavg{ (I^*)^2 + I I^* + I^2 + I^* I } \\
	& = \pathavg{ (\Real I)^2 } = \pathavg{ S^2_x }
\end{split}
\end{equation*}

\begin{equation*}
\begin{split}
	\langle \hat{S}^2_y \rangle
	& = - \frac{1}{4} \langle \int \left(
		\Psiop^\dagger_2 \Psiop_1 - \Psiop^\dagger_1 \Psiop_2
	\right)
	\left(
		\Psiop^{\prime\dagger}_2 \Psiop^\prime_1 - \Psiop^{\prime\dagger}_1 \Psiop^\prime_2
	\right) d\xvec d\xvec^\prime \rangle \\
	& = - \frac{1}{4} \langle \int \left(
		\symprod{ \Psiop^\dagger_2 \Psiop_1 \Psiop^{\prime\dagger}_2 \Psiop^\prime_1 }
		- \symprod{ \Psiop^\dagger_1 \Psiop_2 \Psiop^{\prime\dagger}_2 \Psiop^\prime_1 }
		+ \symprod{ \Psiop^\dagger_1 \Psiop_2 \Psiop^{\prime\dagger}_1 \Psiop^\prime_2 }
		- \symprod{ \Psiop^\dagger_2 \Psiop_1 \Psiop^{\prime\dagger}_1 \Psiop^\prime_2 }
	\right. \\
	& \left.
		+ \frac{\delta_P(\xvec,\xvec^\prime)}{2} \left(
			\symprod{ \Psiop_2 \Psiop^{\prime\dagger}_2 }
			+ \symprod{ \Psiop_1 \Psiop^{\prime\dagger}_1 }
			- \symprod{ \Psiop^\dagger_2 \Psiop^\prime_2 }
			- \symprod{ \Psiop^\dagger_1 \Psiop^\prime_1 }
		\right)
	\right) d\xvec d\xvec^\prime \rangle \\
	& = - \frac{1}{4} \pathavgleft
		\int \Psi^*_2 \Psi_1 d\xvec \int \Psi^*_2 \Psi_1 d\xvec
		- \int \Psi^*_1 \Psi_2 d\xvec \int \Psi^*_2 \Psi_1 d\xvec \right. \\
	&	\left. + \int \Psi^*_1 \Psi_2 d\xvec \int \Psi^*_1 \Psi_2 d\xvec
		- \int \Psi^*_2 \Psi_1 d\xvec \int \Psi^*_1 \Psi_2 d\xvec \pathavgright \\
	& = - \frac{1}{4} \pathavg{ (I^*)^2 - I I^* + I^2 - I^* I } \\
	& = \pathavg{ ( \Imag I )^2 } = \pathavg{ S^2_y }
\end{split}
\end{equation*}

\begin{equation*}
\begin{split}
	\langle \hat{S}^2_z \rangle
	& = \frac{1}{4} \langle \int \left(
		\Psiop^\dagger_1 \Psiop_1 - \Psiop^\dagger_2 \Psiop_2
	\right)
	\left(
		\Psiop^{\prime\dagger}_1 \Psiop^\prime_1 - \Psiop^{\prime\dagger}_2 \Psiop^\prime_2
	\right) d\xvec d\xvec^\prime \rangle \\
	& = \frac{1}{4} \langle \int \left(
		\symprod{ \Psiop^\dagger_1 \Psiop_1 \Psiop^{\prime\dagger}_1 \Psiop^\prime_1 }
		- \symprod{ \Psiop^\dagger_1 \Psiop_1 \Psiop^{\prime\dagger}_2 \Psiop^\prime_2 }
		- \symprod{ \Psiop^\dagger_2 \Psiop_2 \Psiop^{\prime\dagger}_1 \Psiop^\prime_1 }
		+ \symprod{ \Psiop^\dagger_2 \Psiop_2 \Psiop^{\prime\dagger}_2 \Psiop^\prime_2 }
	\right. \\
	& \left.
		+ \frac{\delta_P(\xvec,\xvec^\prime)}{2} \left(
			\symprod{ \Psiop^\dagger_1 \Psiop^\prime_1 }
			- \symprod{ \Psiop_2 \Psiop^{\prime\dagger}_2 }
			+ \symprod{ \Psiop^\dagger_2 \Psiop^\prime_2 }
			- \symprod{ \Psiop_1 \Psiop^{\prime\dagger}_1 }
		\right)
	\right) d\xvec d\xvec^\prime \rangle \\
	& = \frac{1}{4} \pathavgleft
		\int \Psi^*_1 \Psi_1 d\xvec \int \Psi^*_1 \Psi_1 d\xvec
		- \int \Psi^*_1 \Psi_1 d\xvec \int \Psi^*_2 \Psi_2 d\xvec \right. \\
	&	\left. - \int \Psi^*_2 \Psi_2 d\xvec \int \Psi^*_1 \Psi_1 d\xvec
		+ \int \Psi^*_2 \Psi_2 d\xvec \int \Psi^*_2 \Psi_2 d\xvec \pathavgright \\
	& = \frac{1}{4} \pathavg{ N^2_1 - N_1 N_2 - N_2 N_1 + N^2_2 } \\
	& = \frac{1}{4} \pathavg{ (N_1 - N_2)^2 } = \pathavg{ S^2_z }
\end{split}
\end{equation*}

\begin{equation*}
\begin{split}
	\langle \hat{S}_x \hat{S}_y + \hat{S}_y \hat{S}_x \rangle
	& = \frac{i}{4} \langle \int \left(
		\left(
			\Psiop^\dagger_2 \Psiop_1 + \Psiop^\dagger_1 \Psiop_2
		\right)
		\left(
			\Psiop^{\prime\dagger}_2 \Psiop^\prime_1 - \Psiop^{\prime\dagger}_1 \Psiop^\prime_2
		\right)
		+ \left(
			\Psiop^\dagger_2 \Psiop_1 - \Psiop^\dagger_1 \Psiop_2
		\right)
		\left(
			\Psiop^{\prime\dagger}_2 \Psiop^\prime_1 + \Psiop^{\prime\dagger}_1 \Psiop^\prime_2
		\right)
	\right) d\xvec d\xvec^\prime \rangle \\
	& = \frac{i}{2} \langle \int \left(
		\symprod{ \Psiop^\dagger_2 \Psiop_1 \Psiop^{\prime\dagger}_2 \Psiop^\prime_1 }
		- \symprod{ \Psiop^\dagger_1 \Psiop_2 \Psiop^{\prime\dagger}_1 \Psiop^\prime_2 }
	\right) d\xvec d\xvec^\prime \rangle \\
	& = \frac{i}{2} \pathavg{
		\int \Psi^*_2 \Psi_1 d\xvec \int \Psi^*_2 \Psi_1 d\xvec
		- \int \Psi^*_1 \Psi_2 d\xvec \int \Psi^*_1 \Psi_2 d\xvec } \\
	& = \frac{i}{2} \pathavg{ (I^*)^2 - I^2 } \\
	& = 2 \pathavg{ \Real I \, \Imag I } = 2 \pathavg{ S_x S_y }
\end{split}
\end{equation*}

\begin{equation*}
\begin{split}
	\langle \hat{S}_x \hat{S}_z + \hat{S}_z \hat{S}_x \rangle
	& = \frac{1}{4} \langle \int \left(
		\left(
			\Psiop^\dagger_2 \Psiop_1 + \Psiop^\dagger_1 \Psiop_2
		\right)
		\left(
			\Psiop^{\prime\dagger}_1 \Psiop^\prime_1 - \Psiop^{\prime\dagger}_2 \Psiop^\prime_2
		\right)
		+ \left(
			\Psiop^\dagger_1 \Psiop_1 - \Psiop^\dagger_2 \Psiop_2
		\right)
		\left(
			\Psiop^{\prime\dagger}_2 \Psiop^\prime_1 + \Psiop^{\prime\dagger}_1 \Psiop^\prime_2
		\right)
	\right) d\xvec d\xvec^\prime \rangle \\
	& = \frac{1}{4} \langle \int \left(
		\symprod{ \Psiop^\dagger_1 \Psiop_1 \Psiop^{\prime\dagger}_2 \Psiop^\prime_1 }
		- \symprod{ \Psiop^\dagger_2 \Psiop_2 \Psiop^{\prime\dagger}_1 \Psiop^\prime_2 }
		+ \symprod{ \Psiop^\dagger_2 \Psiop_1 \Psiop^{\prime\dagger}_1 \Psiop^\prime_1 }
		+ \symprod{ \Psiop^\dagger_1 \Psiop_2 \Psiop^{\prime\dagger}_1 \Psiop^\prime_1 }
	\right. \\
	& \left.
		- \symprod{ \Psiop^\dagger_2 \Psiop_1 \Psiop^{\prime\dagger}_2 \Psiop^\prime_2 }
		+ \symprod{ \Psiop^\dagger_1 \Psiop_1 \Psiop^{\prime\dagger}_1 \Psiop^\prime_2 }
		- \symprod{ \Psiop^\dagger_2 \Psiop_2 \Psiop^{\prime\dagger}_2 \Psiop^\prime_1 }
		- \symprod{ \Psiop^\dagger_1 \Psiop_2 \Psiop^{\prime\dagger}_2 \Psiop^\prime_2 }
	\right) d\xvec d\xvec^\prime \rangle \\
	& = \frac{1}{4} \pathavgleft
		\int \Psi^*_1 \Psi_1 d\xvec \int \Psi^*_2 \Psi_1 d\xvec
		- \int \Psi^*_2 \Psi_2 d\xvec \int \Psi^*_1 \Psi_2 d\xvec
		+ \int \Psi^*_2 \Psi_1 d\xvec \int \Psi^*_1 \Psi_1 d\xvec
		+ \int \Psi^*_1 \Psi_2 d\xvec \int \Psi^*_1 \Psi_1 d\xvec \right. \\
	&	\left. - \int \Psi^*_2 \Psi_1 d\xvec \int \Psi^*_2 \Psi_2 d\xvec
		+ \int \Psi^*_1 \Psi_1 d\xvec \int \Psi^*_1 \Psi_2 d\xvec
		- \int \Psi^*_2 \Psi_2 d\xvec \int \Psi^*_2 \Psi_1 d\xvec
		- \int \Psi^*_1 \Psi_2 d\xvec \int \Psi^*_2 \Psi_2 d\xvec
	\pathavgright \\
	& = \frac{1}{4} \pathavg{
		N_1 I^*
		- N_2 I
		+ I^* N_1
		+ I N_1
		- I^* N_2
		+ N_1 I
		- N_2 I^*
		- I N_2
	} \\
	& = \pathavg{ (N_1 - N_2) \Real I } = 2 \pathavg{ S_x S_z }
\end{split}
\end{equation*}

\begin{equation*}
\begin{split}
	\langle \hat{S}_y \hat{S}_z + \hat{S}_z \hat{S}_y \rangle
	& = \frac{i}{4} \langle \int \left(
		\left(
			\Psiop^\dagger_2 \Psiop_1 - \Psiop^\dagger_1 \Psiop_2
		\right)
		\left(
			\Psiop^{\prime\dagger}_1 \Psiop^\prime_1 - \Psiop^{\prime\dagger}_2 \Psiop^\prime_2
		\right)
		+ \left(
			\Psiop^\dagger_1 \Psiop_1 - \Psiop^\dagger_2 \Psiop_2
		\right)
		\left(
			\Psiop^{\prime\dagger}_2 \Psiop^\prime_1 - \Psiop^{\prime\dagger}_1 \Psiop^\prime_2
		\right)
	\right) d\xvec d\xvec^\prime \rangle \\
	& = \frac{i}{4} \langle \int \left(
		\symprod{ \Psiop^\dagger_1 \Psiop_1 \Psiop^{\prime\dagger}_2 \Psiop^\prime_1 }
		+ \symprod{ \Psiop^\dagger_2 \Psiop_2 \Psiop^{\prime\dagger}_1 \Psiop^\prime_2 }
		+ \symprod{ \Psiop^\dagger_2 \Psiop_1 \Psiop^{\prime\dagger}_1 \Psiop^\prime_1 }
		- \symprod{ \Psiop^\dagger_1 \Psiop_2 \Psiop^{\prime\dagger}_1 \Psiop^\prime_1 }
	\right. \\
	& \left.
		- \symprod{ \Psiop^\dagger_2 \Psiop_1 \Psiop^{\prime\dagger}_2 \Psiop^\prime_2 }
		- \symprod{ \Psiop^\dagger_1 \Psiop_1 \Psiop^{\prime\dagger}_1 \Psiop^\prime_2 }
		- \symprod{ \Psiop^\dagger_2 \Psiop_2 \Psiop^{\prime\dagger}_2 \Psiop^\prime_1 }
		+ \symprod{ \Psiop^\dagger_1 \Psiop_2 \Psiop^{\prime\dagger}_2 \Psiop^\prime_2 }
	\right) d\xvec d\xvec^\prime \rangle \\
	& = \frac{i}{4} \pathavgleft
		\int \Psi^*_1 \Psi_1 d\xvec \int \Psi^*_2 \Psi_1 d\xvec
		+ \int \Psi^*_2 \Psi_2 d\xvec \int \Psi^*_1 \Psi_2 d\xvec
		+ \int \Psi^*_2 \Psi_1 d\xvec \int \Psi^*_1 \Psi_1 d\xvec
		- \int \Psi^*_1 \Psi_2 d\xvec \int \Psi^*_1 \Psi_1 d\xvec \right. \\
	&	\left. - \int \Psi^*_2 \Psi_1 d\xvec \int \Psi^*_2 \Psi_2 d\xvec
		- \int \Psi^*_1 \Psi_1 d\xvec \int \Psi^*_1 \Psi_2 d\xvec
		- \int \Psi^*_2 \Psi_2 d\xvec \int \Psi^*_2 \Psi_1 d\xvec
		+ \int \Psi^*_1 \Psi_2 d\xvec \int \Psi^*_2 \Psi_2 d\xvec
	\pathavgright \\
	& = \frac{i}{4} \pathavg{
		N_1 I^*
		+ N_2 I
		+ I^* N_1
		- I N_1
		- I^* N_2
		- N_1 I
		- N_2 I^*
		+ I N_2
	} \\
	& = \pathavg{ (N_1 - N_2) \Imag I } = 2 \pathavg{ S_y S_z }
\end{split}
\end{equation*}

As it turns out, unlike the equation~\eqnref{moments-calculation:delta-N}, formulas for second-order moments for spin operators do not contain any additional terms depending on $M$.
Now we can calculate all spin correlations from~\cite{Li2009}:
\begin{equation*}
\begin{split}
	\Delta S^2_i
		& = \langle \hat{S}^2_i \rangle - \langle \hat{S}_i \rangle^2
		= \pathavg{ S^2_i } - \pathavg{ S_i }^2, \\
	\Delta_{ij}
		& = \langle \hat{S}_i \hat{S}_j + \hat{S}_j \hat{S}_i \rangle
		- 2 \langle \hat{S}_i \rangle \langle \hat{S}_j \rangle
		= 2 ( \pathavg{ S_i S_j } - \pathavg{ S_i } \pathavg { S_j } )
\end{split}
\end{equation*}
In other words, we proved that in the simulator application we can first calculate spin components $S^{(j)}_i$ for each simulation path, and then use common average and variance functions on resulting arrays to obtain required correlations.
