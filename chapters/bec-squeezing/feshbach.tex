% =============================================================================
\section{Squeezing near a Feshbach resonance}
% =============================================================================

In the previous section we have discussed an experiment with component-dependent potentials, which helped to reduce the inter-component interaction for the majority of time and thus avoid ``oversqueezing''.
A similar effect can be achieved differently~--- by manipulating the external magnetic field near a Feshbach resonance for a chosen pair of hyperfine states~\cite{Gross2010}.
This allows one to vary the interaction in a wide range with relative ease.
The downside of this approach is a significant increase of the inter-component nonlinear loss rate accompanying the change in the interaction strength.
This results in the destruction of the squeezing as the two components interfere with each other and lose coherence.

The truncated Wigner approach allows us to investigate the combined effect of a reduced interaction strength and increased losses.
In this section we will consider a hypothetical interferometry experiment with two hyperfine states of \Rb{} and demonstrate how with the help of the Wigner method we can pick the optimal value of the magnetic field that leads to the maximum squeezing.

The experiment follows the general scheme used in this and the previous chapters.
We start from a \abbrev{bec} of $N = 55000$ \Rb{} atoms in the hyperfine state ${\ket{F=1,\, m_F=+1}}$ in a cigar-shaped trap with the frequencies $f_x = f_y = 97.0\un{Hz}$ and $f_z = 11.69\un{Hz}$.
The first $\pi/2$-pulse creates an equal superposition of two states, ${\ket{F=1,\, m_F=+1}}$ and ${\ket{F=2,\, m_F=-1}}$.
The external magnetic field of strength $B \approx B_0$ is applied, where $B_0 = 9.1047\un{G}$ corresponds to the Feshbach resonance for the two hyperfine states used~\cite{Kaufman2009}.
We then investigate the time dependence of the maximum degree of spin squeezing.
Intra-component interaction strengths do not depend on the external magnetic field, and are taken to be $a_{11} = 100.4\,r_B$ and $a_{22} = 95.44\,r_B$.
The numerical simulations used a $8\times8\times64$ spatial grid, $20,000$ time steps and $2560$ stochastic trajectories.

The dependence of the inter-component interaction and loss rate can be described with a single equation for the complex scattering length~\cite{Kaufman2009}
\begin{eqn}
    a(B)
    = a_{\mathrm{bg}} \left(
        1 - \frac{\Delta B}{(B - B_0) - i \gamma_B / 2}
        \right),
\end{eqn}
where $a_{\mathrm{bg}}$ is the background scattering length, $\Delta B$ is the resonance width, and $\gamma_B$ is the decay width.
For a given $B$, the real part of this value acts as the $s$-wave scattering length $a_{12}$ for the interaction coefficient in~\eqnref{bec-noise:system:g}:
\begin{eqn}
\label{eqn:bec-squeezing:feshbach:g}
    g_{12}(B)
    = \frac{4 \pi \hbar^2 \Real a(B)}{m}
    = \frac{4 \pi \hbar^2 a_{\mathrm{bg}}}{m} \left(
        1 - \frac{\Delta B (B - B_0)}{(B - B_0)^2 + \gamma_B^2 / 4}
    \right),
\end{eqn}
and the imaginary part can be connected to the loss rate by substituting it into~\eqnref{bec-noise:system:g} as well and comparing the resulting expression with the corresponding loss term in~\eqnref{bec-noise:mean-field:cgpes-simplified}:
\begin{eqn}
\label{eqn:bec-squeezing:feshbach:gamma}
    \gamma_{12}(B)
    = -\frac{8 \pi \hbar \Imag a(B)}{m}
    = \frac{4 \pi \hbar a_{\mathrm{bg}} \Delta B \gamma_B}{m((B - B_0)^2 + \gamma_B^2 / 4)}.
\end{eqn}

\begin{figure}
    \centerline{\includegraphics{figures_generated/bec_squeezing/feshbach_scattering.pdf}}

    \caption[Inter-component scattering lenth and loss rate near a Feshbach resonance]{
    Real (blue solid line) and imaginary (red dashed line, negated value is plotted for compactness) parts of the complex scattering length near the Feshbach resonance at $B_0 = 9.1047\un{G}$.
    Four pairs of points show the distances from the resonance chosen for the simulation.
    }%endcaption

    \label{fig:bec-squeezing:feshbach:scattering}
\end{figure}

For the Feshbach resonance we use the reported parameters are $\Delta B = 2\times10^{-3}\un{G}$, $\gamma_B = 4.7\times10^{-3}\un{G}$, and $a_{\mathrm{bg}} = 97.7\,r_B$~\cite{Kaufman2009}.
The behavior of the real and imaginary parts of $a(B)$ near the resonance is shown in~\figref{bec-squeezing:feshbach:scattering}.
From the equations above, as well as from the figure, it is obvious that the minimum inter-species scattering length is achieved when $B - B_0 = 0.5 \gamma_B$.
Unfortunately, this value also corresponds to a relatively large value of the imaginary part, and, correspondingly, an unacceptable loss rate.
Therefore, we pick values of $B$ further from the resonance, as displayed in the figure, where the inter-component interaction is somewhat stronger, but the loss rate is much lower.
Equations~\eqnref{bec-squeezing:feshbach:g} and~\eqnref{bec-squeezing:feshbach:gamma} give us the following values for the simulation:
\begin{eqn}
    B - B_0 & = 2.24 \gamma_B, \quad
        a_{12} = 80.0\,r_B, \quad \gamma_{12} = 3.85\times10^{-12}\un{cm^3/s},\\
    B - B_0 & = 3.20 \gamma_B, \quad
        a_{12} = 85.0\,r_B, \quad \gamma_{12} = 1.93\times10^{-12}\un{cm^3/s},\\
    B - B_0 & = 5.35 \gamma_B, \quad
        a_{12} = 90.0\,r_B, \quad \gamma_{12} = 7.00\times10^{-13}\un{cm^3/s},\\
    B - B_0 & = 15.4 \gamma_B, \quad
        a_{12} = 95.0\,r_B, \quad \gamma_{12} = 8.53\times10^{-14}\un{cm^3/s}.
\end{eqn}

\begin{figure}
    \centerline{%
    \includegraphics{figures_generated/bec_squeezing/feshbach_squeezing_no_losses.pdf}%
    \includegraphics{figures_generated/bec_squeezing/feshbach_squeezing.pdf}}

    \caption[Spin squeezing near a Feshbach resonance]{
    Truncated Wigner simulations of the squeezing in the vicinity of the $B_0 = 9.1047\un{G}$ Feshbach resonance in \Rb{} with different values of the external magnetic field, with \textbf{(a)}~losses turned off, and \textbf{(b)}~inter-component $1,2$-losses turned on.
    Results corresponding to the scattering lengths $a_{12}=80.0\,r_B$ (blue solid line), $a_{12}=85.0\,r_B$ (red dashed line), $a_{12}=90.0\,r_B$ (green dotted line), and $a_{12}=95.0\,r_B$ (yellow dash-dotted line) are plotted.
    The same-colored bands show the estimated sampling error.}%endcaption

    \label{fig:bec-squeezing:feshbach:squeezing}
\end{figure}

First, we run the simulations with $\gamma_{12}$ set to zero in order to test the squeezing in ideal conditions.
As expected, the lower cross-term $a_{12}$ is, the stronger is the squeezing, as shown in~\figref{bec-squeezing:feshbach:squeezing},~(a).

With the inclusion of inter-component losses, the picture is different.
Feshbach tuning to $a_{12} = 85.0\,r_B$ ensures the best squeezing of the four variants ($-12\un{dB}$ at $60\un{ms}$), whereas long lasting squeezing is predicted for the variant with $a_{12} = 95.0\,r_B$ (\figref{bec-squeezing:feshbach:squeezing},~(b)).
In practice, it is possible to run the simulation for other values of $B$, thus finding the ideal balance between the interaction strength and the loss rate which produces the maximum squeezing.
