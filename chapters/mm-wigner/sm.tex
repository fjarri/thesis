% =============================================================================
\section{Single-mode Wigner representation}
% =============================================================================

We will need the displacement operator which was first introduced by Weyl~\cite{Weyl1950}.

\begin{definition}
	\begin{eqn*}
	\label{eqn:wigner:sm:displacement-op}
		\hat{D}(\lambda, \lambda^*) = \exp(\lambda \hat{a}^\dagger - \lambda^* \hat{a}),
	\end{eqn*}
	where $\hat{a}^\dagger$ and $\hat{a}$ are bosonic creation and annihilation operators, and $\lambda$ is a complex variable.
\end{definition}
Using Baker-Hausdorff theorem to split non-commuting operators in the exponent,
one can find that
\begin{eqn}
\label{eqn:wigner:sm:displacement-derivatives}
	\frac{\partial}{\partial \lambda} \hat{D}(\lambda, \lambda^*)
	& = \hat{D}(\lambda, \lambda^*) (\hat{a}^\dagger + \frac{1}{2} \lambda^*)
	= (\hat{a}^\dagger - \frac{1}{2} \lambda^*) \hat{D}(\lambda, \lambda^*), \\
	-\frac{\partial}{\partial \lambda^*} \hat{D}(\lambda, \lambda^*)
	& = \hat{D}(\lambda, \lambda^*) (\hat{a} + \frac{1}{2} \lambda)
	= (\hat{a} - \frac{1}{2} \lambda) \hat{D}(\lambda, \lambda^*).
\end{eqn}

Using the displacement operator we can define Wigner transformation.

\begin{definition}
\label{def:wigner:sm:w-transformation}
	Wigner transformation converts an operator $\hat{A}$ on a Hilbert space to a complex-valued function $\mathcal{W}[\hat{A}](\alpha, \alpha^*)$ on phase space.
	\begin{eqn*}
		\mathcal{W}\left[\hat{A}\right]
		= \frac{1}{\pi^2} \int d^2 \lambda \exp(-\lambda \alpha^* + \lambda^* \alpha)
			\Trace{ \hat{A} \hat{D}(\lambda, \lambda^*) }.
	\end{eqn*}
	The backward transformation (called the Weyl transformation) gives back the operator:
	\begin{eqn*}
		\mathcal{W}^{-1}[f]
		= \frac{1}{\pi} \int d^2 \xi \hat{D}^{\dagger}(\xi, \xi^*)
			\int d^2 \eta \exp(-\eta \xi^* + \eta^* \xi) f(\eta, \eta^*).
	\end{eqn*}
\end{definition}

It is easy to demonstrate that Wigner and Weyl transformations define a bijection $\mathbb{H} \leftrightarrow (\mathbb{C} \rightarrow \mathbb{C})$ \todo{at least, for Hilbert-Schmidt operators and square-integrable functions~\cite{Cahill1969}}.
Let us assume $\hat{A} \equiv \mathcal{W}^{-1}[f]$.
Then, using the fact that $\Trace{\hat{D}^{\dagger}(\xi, \xi^*) \hat{D}(\lambda, \lambda^*)} = \pi \delta(\Real \xi - \Real \lambda) \delta(\Imag \xi - \Imag \lambda)$~\cite{Cahill1969} and applying \lmmref{c-numbers:fourier-of-moments}:
\begin{eqn}
	\mathcal{W}[\hat{A}]
	={} & \frac{1}{\pi^3} \int d^2 \lambda \exp(-\lambda \alpha^* + \lambda^* \alpha) \\
	&	\times \Trace{
			\int d^2 \xi \hat{D}^{\dagger}(\xi, \xi^*)
				\int d^2 \eta \exp(-\eta \xi^* + \eta^* \xi) f(\eta, \eta^*)
			\hat{D}(\lambda, \lambda^*)
		} \\
	={} & \frac{1}{\pi^2} \int d^2 \lambda \int d^2 \eta
	 	\exp(-\lambda \alpha^* + \lambda^* \alpha)
		\exp(-\eta \lambda^* + \eta^* \lambda) f(\eta, \eta^*) \\
	={} & \frac{1}{\pi^2} \int d^2 \eta \int d^2 \lambda
	 	\exp(-\lambda (\alpha^* - \eta^*) + \lambda^* (\alpha - \eta)) f(\eta, \eta^*) \\
	={} & \int d^2 \eta \delta(\Real \alpha - \Real \eta) \delta(\Imag \alpha - \Imag \eta) f(\eta, \eta^*) \\
	={} & f(\alpha, \alpha^*).
\end{eqn}
The proof for the other direction looks the same.

\begin{theorem}
\label{thm:wigner:sm:w-real}
	If an operator $\hat{A}$ is Hermitian, its Wigner transformation $W[\hat{A}]$ is a real function.
	\todo{It is a sufficient condition, but is it necessary?}
\end{theorem}
\begin{proof}
Let us calculate the conjugation of $W[\hat{A}]$:
\begin{eqn}
	(W[\hat{A}])^*
	& = \frac{1}{\pi^2} \int d^2 \lambda \exp(-\lambda^* \alpha + \lambda \alpha^*)
		(\Trace{ \hat{A} \exp(\lambda \hat{a}^\dagger - \lambda^* \hat{a}) })^*.
\end{eqn}
Changing variables as $\lambda \rightarrow -\lambda$ and using the fact that $(\Tr{\hat{B}})^* = \Tr{\hat{B}^\dagger}$:
\begin{eqn}
	& = \frac{1}{\pi^2} \int d^2 \lambda \exp(\lambda^* \alpha - \lambda \alpha^*)
		\Trace{ \hat{A}^\dagger \exp(-\lambda^* \hat{a} + \lambda \hat{a}^\dagger) } \\
	& = W[\hat{A}^\dagger].
\end{eqn}
Therefore, if $\hat{A} = \hat{A}^\dagger$, the conjugate of $W[\hat{A}]$ is equal to itself and therefore is real.
\end{proof}

\begin{definition}
\label{def:wigner:sm:w-function}
	Wigner function is a Wigner transformation of the density matrix:
	\begin{eqn*}
		W(\alpha, \alpha^*) \equiv \mathcal{W}[\hat{\rho}].
	\end{eqn*}
	The Wigner function always exists for any density matrix~\cite{Gardiner2004}.
	Since the density matrix is Hermitian, according to \thmref{wigner:sm:w-real} $W$ is a real function.
\end{definition}

In some cases it will be convenient to use Wigner function in form~\cite{Gardiner2004}
\begin{eqn}
	W (\alpha, \alpha^*)
	= \frac{1}{\pi^2} \int d^2 \lambda \exp(-\lambda \alpha^* + \lambda^* \alpha)
		\chi_W (\lambda, \lambda^*),
\end{eqn}
where $\chi_W (\lambda, \lambda^*)$ is the characteristic function
\begin{eqn}
	\chi_W (\lambda, \lambda^*)	= \Trace{ \hat{\rho} \hat{D}(\lambda, \lambda^*) }.
\end{eqn}

\begin{theorem}[Operator correspondences]
\label{thm:wigner:sm:correspondences}
	For any Hilbert-Schmidt operator $\hat{A}$ \todo{needs definition?}
	\begin{eqn*}
		\mathcal{W} [ \hat{a} \hat{A} ]
			& = \left( \alpha + \frac{1}{2} \frac{\partial}{\partial \alpha^*} \right) \mathcal{W}[\hat{A}],
		\quad
		\mathcal{W} [ \hat{a}^\dagger \hat{A} ]
			= \left( \alpha^* - \frac{1}{2} \frac{\partial}{\partial \alpha} \right) \mathcal{W}[\hat{A}], \\
		\mathcal{W} [ \hat{A} \hat{a} ]
			& = \left( \alpha - \frac{1}{2} \frac{\partial}{\partial \alpha^*} \right) \mathcal{W}[\hat{A}],
		\quad
		\mathcal{W} [ \hat{A} \hat{a}^\dagger ]
			= \left( \alpha^* + \frac{1}{2} \frac{\partial}{\partial \alpha} \right) \mathcal{W}[\hat{A}].
	\end{eqn*}
\end{theorem}
\begin{proof}
We will prove the first correspondence.
First, let us transform the trace using~\eqnref{wigner:sm:displacement-derivatives}:
\begin{eqn}
	\Trace{ \hat{a} \hat{A} \hat{D} }
	= \Trace{ \hat{A} \hat{D} \hat{a}}
	= \Trace{ \hat{A} \left(
		-\frac{\partial}{\partial \lambda^*}
		-\frac{1}{2} \lambda
	\right) \hat{D}}
	= \left(
		-\frac{\partial}{\partial \lambda^*}
		-\frac{1}{2} \lambda
	\right) \Trace{ \hat{A} \hat{D}}
\end{eqn}
Using this:
\begin{eqn}
	\mathcal{W} [ \hat{a} \hat{A} ]
	& = \frac{1}{\pi^2} \int d^2 \lambda \exp(-\lambda \alpha^* + \lambda^* \alpha)
		\Trace{ \hat{a} \hat{A} \hat{D}(\lambda, \lambda^*) } \\
	& = \frac{1}{\pi^2} \int d^2 \lambda \exp(-\lambda \alpha^* + \lambda^* \alpha)
		\left(
			-\frac{\partial}{\partial \lambda^*}
			-\frac{1}{2} \lambda
		\right)
		\Trace{ \hat{A} \hat{D}(\lambda, \lambda^*) } \\
	& = \frac{1}{2} \frac{\partial}{\partial \alpha^*} \mathcal{W} [\hat{A}]
	- \frac{1}{\pi^2} \int d^2 \lambda \exp(-\lambda \alpha^* + \lambda^* \alpha)
		\frac{\partial}{\partial \lambda^*}
		\Trace{ \hat{A} \hat{D}(\lambda, \lambda^*) }.
\end{eqn}
The second term is almost the definition of the Wigner function, except for the partial derivative over $\lambda^*$.
Moving it using integration by parts and \lmmref{c-numbers:zero-integrals} ($\hat{A}$ is Hilbert-Schmidt, which means that $\Trace{\hat{A} \hat{D}}$ is square-integrable~\cite{Cahill1969}):
\begin{eqn}
	= \frac{1}{2} \frac{\partial}{\partial \alpha^*} \mathcal{W} [\hat{A}]
	+ \frac{1}{\pi^2} \int d^2 \lambda \left(
		\frac{\partial}{\partial \lambda^*} \exp(-\lambda \alpha^* + \lambda^* \alpha)
	\right)
	\Trace{ \hat{A} \hat{D}(\lambda, \lambda^*) } \\
	= \left( \alpha + \frac{1}{2} \frac{\partial}{\partial \alpha^*} \right) \mathcal{W} [\hat{A}].
	\qedhere
\end{eqn}
\end{proof}

\begin{lemma}
\label{lmm:wigner:sm:moments-from-chi}
	\begin{eqn*}
		\langle \symprod{ \hat{a}^r (\hat{a}^\dagger)^s } \rangle
		= \left.
			\left( \frac{\partial}{\partial \lambda} \right)^s
			\left( -\frac{\partial}{\partial \lambda^*} \right)^r
			\chi_W (\lambda, \lambda^*)
		\right|_{\lambda=0}.
	\end{eqn*}
\end{lemma}
\begin{proof}
The exponent in the expression for $\chi_W$ can be expanded as
\begin{eqn}
	\exp (\lambda \hat{a}^\dagger - \lambda^* \hat{a})
	= \sum_{r,s}
		\frac{(-\lambda^*)^r \lambda^s}{r!s!}
		\symprod{ \hat{a}^r (\hat{a}^\dagger)^s }.
\end{eqn}
Thus
\begin{eqn}
	\chi_W(\lambda, \lambda^*)
	& = \sum_{r,s}
		\frac{(-\lambda^*)^r \lambda^s}{r!s!}
		\Trace{
			\hat{\rho} \symprod{ \hat{a}^r (\hat{a}^\dagger)^s }
		} \\
	& = \sum_{r,s}
		\frac{(-\lambda^*)^r \lambda^s}{r!s!}
		\langle \symprod{ \hat{a}^r (\hat{a}^\dagger)^s } \rangle
\end{eqn}
Apparently, the application of $(\partial / \partial \lambda)^s$ and $(-\partial / \partial \lambda^*)^r$ will eliminate all lower order moments,
and setting $\lambda = 0$ afterwards will eliminate all higher order moments,
leaving only $\symprod{ \hat{a}^r (\hat{a}^\dagger)^s }$:
\begin{eqn}
	\left.
		\left( \frac{\partial}{\partial \lambda} \right)^s
		\left( -\frac{\partial}{\partial \lambda^*} \right)^r
		\chi_W (\lambda, \lambda^*)
	\right|_{\lambda=0}
	= r! s! \frac{1}{r! s!}
		\langle \symprod{ \hat{a}^r (\hat{a}^\dagger)^s } \rangle
	= \langle \symprod{ \hat{a}^r (\hat{a}^\dagger)^s } \rangle.
	\qedhere
\end{eqn}
\end{proof}

Now we can get the final relation.
\begin{theorem}
\label{thm:wigner:sm:moments}
	\begin{eqn*}
		\int d^2\alpha\, \alpha^r (\alpha^*)^s W(\alpha, \alpha^*)
		= \langle \symprod{ \hat{a}^r (\hat{a}^\dagger)^s } \rangle
	\end{eqn*}
\end{theorem}
\begin{proof}
By definition of the Wigner function:
\begin{eqn}
	\int d^2\alpha\, \alpha^r (\alpha^*)^s W(\alpha, \alpha^*)
	= \frac{1}{\pi^2}
		\int d^2\alpha\, \alpha^r (\alpha^*)^s
		\int d^2\lambda \exp(-\lambda \alpha^* + \lambda^* \alpha)
		\chi_W (\lambda, \lambda^*).
\end{eqn}
Integrating over $\alpha$ using \lmmref{c-numbers:fourier-of-moments}:
\begin{eqn}
	= \int d^2\lambda
		\left(
			\left( \frac{\partial}{\partial \lambda} \right)^s
			\left( -\frac{\partial}{\partial \lambda^*} \right)^r
			\delta(\Real \lambda) \delta(\Imag \lambda)
		\right)
		\chi_W (\lambda, \lambda^*).
\end{eqn}
Integrating by parts and eliminating terms which fit \lmmref{c-numbers:zero-delta-integrals}:
\begin{eqn}
	& = \int d^2\lambda
		\delta(\Real \lambda) \delta(\Imag \lambda)
		\left( \frac{\partial}{\partial \lambda} \right)^s
		\left( -\frac{\partial}{\partial \lambda^*} \right)^r
		\chi_W (\lambda, \lambda^*) \\
	& = \left.
		\left( \frac{\partial}{\partial \lambda} \right)^s
		\left( -\frac{\partial}{\partial \lambda^*} \right)^r
		\chi_W (\lambda, \lambda^*)
	\right|_{\lambda=0}.
\end{eqn}
Now, recognising the final expression as a part of \lmmref{wigner:sm:moments-from-chi}, we immideately get the statement of the theorem.
\end{proof}
