% =============================================================================
\section{Single-mode Wigner transformation}
% =============================================================================

In this section we will work with a single-mode bosonic system with standard creation and annihilation operators
\begin{eqn}
	[ \hat{a}, \hat{a} ] & = [ \hat{a}^\dagger, \hat{a}^\dagger ] = 0, \\
	[ \hat{a}, \hat{a}^\dagger ] & = 1.
\end{eqn}
The essential part of the definition is the displacement operator which was first introduced by Weyl~\cite{Weyl1950} and thoroughly described by Cahill and Glauber~\cite{Cahill1969}:

\begin{definition}
	Displacement operator is an operator-valued function of complex variable
	\begin{eqn*}
	\label{eqn:mm-wigner:sm:displacement-op}
		\hat{D}(\lambda) = \exp(\lambda \hat{a}^\dagger - \lambda^* \hat{a}).
	\end{eqn*}
\end{definition}

Using Baker-Hausdorff theorem to split non-commuting creation and annihilation operator in the exponent, one can find that
\begin{eqn}
\label{eqn:mm-wigner:sm:displacement-derivatives}
	\frac{\cwd}{\cwd \lambda} \hat{D}(\lambda)
	& = \hat{D}(\lambda) (\hat{a}^\dagger + \frac{1}{2} \lambda^*)
	= (\hat{a}^\dagger - \frac{1}{2} \lambda^*) \hat{D}(\lambda), \\
	-\frac{\cwd}{\cwd \lambda^*} \hat{D}(\lambda)
	& = \hat{D}(\lambda) (\hat{a} + \frac{1}{2} \lambda)
	= (\hat{a} - \frac{1}{2} \lambda) \hat{D}(\lambda).
\end{eqn}

Using the displacement operator we can define the Wigner transformation:

\begin{definition}
\label{def:mm-wigner:sm:w-transformation}
	Wigner transformation converts an operator $\hat{A}$ on a Hilbert space to a complex-valued function $\mathcal{W}[\hat{A}](\alpha)$ on phase space.
	\begin{eqn*}
		\mathcal{W}\left[\hat{A}\right]
		= \frac{1}{\pi^2} \int \upd^2 \lambda \exp(-\lambda \alpha^* + \lambda^* \alpha)
			\Trace{ \hat{A} \hat{D}(\lambda) }.
	\end{eqn*}
	The backward transformation (called the Weyl transformation) gives back the operator:
	\begin{eqn*}
		\mathcal{W}^{-1}[f]
		= \frac{1}{\pi} \int \upd^2 \xi \hat{D}^{\dagger}(\xi, \xi^*)
			\int \upd^2 \eta \exp(-\eta \xi^* + \eta^* \xi) f(\eta, \eta^*).
	\end{eqn*}
\end{definition}

It is easy to demonstrate that Wigner and Weyl transformations define a bijection $\mathbb{H} \leftrightarrow (\mathbb{C} \rightarrow \mathbb{C})$ for Hilbert-Schmidt operators and square-integrable functions.
Let us assume $\hat{A} \equiv \mathcal{W}^{-1}[f]$.
Then, using the fact that $\Trace{\hat{D}^{\dagger}(\xi) \hat{D}(\lambda)} = \pi \delta(\Real \xi - \Real \lambda) \delta(\Imag \xi - \Imag \lambda)$~\cite{Cahill1969}:
\begin{eqn}
	\mathcal{W}[\hat{A}]
	={} & \frac{1}{\pi^3} \int \upd^2 \lambda \exp(-\lambda \alpha^* + \lambda^* \alpha) \\
	&	\times \Trace{
			\int \upd^2 \xi \hat{D}^{\dagger}(\xi, \xi^*)
				\int \upd^2 \eta \exp(-\eta \xi^* + \eta^* \xi) f(\eta, \eta^*)
			\hat{D}(\lambda)
		} \\
	={} & \frac{1}{\pi^2} \int \upd^2 \lambda \int \upd^2 \eta
	 	\exp(-\lambda \alpha^* + \lambda^* \alpha)
		\exp(-\eta \lambda^* + \eta^* \lambda) f(\eta, \eta^*).
\end{eqn}
Applying \lmmref{c-numbers:fourier-of-moments} to integrate over $\lambda$:
\begin{eqn}
	& = \frac{1}{\pi^2} \int \upd^2 \eta \int \upd^2 \lambda
	 	\exp(-\lambda (\alpha^* - \eta^*) + \lambda^* (\alpha - \eta)) f(\eta, \eta^*) \\
	& = \int \upd^2 \eta \delta(\Real \alpha - \Real \eta) \delta(\Imag \alpha - \Imag \eta) f(\eta, \eta^*) \\
	& = f(\alpha).
\end{eqn}
The proof for the other direction looks the same.

Although in general the result of the Wigner transformation is a complex-valued function, for Hermitian operators, which are of the main interest for us, it is can be proved to be always real-valued.

\begin{theorem}
\label{thm:mm-wigner:sm:w-real}
	If an operator $\hat{A}$ is Hermitian, its Wigner transformation $W[\hat{A}]$ is a real-valued function.
\end{theorem}
\begin{proof}
Let us calculate the conjugation of $W[\hat{A}]$:
\begin{eqn}
	(W[\hat{A}])^*
	& = \frac{1}{\pi^2} \int \upd^2 \lambda \exp(-\lambda^* \alpha + \lambda \alpha^*)
		(\Trace{ \hat{A} \exp(\lambda \hat{a}^\dagger - \lambda^* \hat{a}) })^*.
\end{eqn}
Changing variables as $\lambda \rightarrow -\lambda$ and using the fact that $(\Tr{\hat{B}})^* = \Tr{\hat{B}^\dagger}$:
\begin{eqn}
	& = \frac{1}{\pi^2} \int \upd^2 \lambda \exp(\lambda^* \alpha - \lambda \alpha^*)
		\Trace{ \hat{A}^\dagger \exp(-\lambda^* \hat{a} + \lambda \hat{a}^\dagger) } \\
	& = W[\hat{A}^\dagger].
\end{eqn}
Therefore, if $\hat{A} = \hat{A}^\dagger$, the conjugate of $W[\hat{A}]$ is equal to itself and therefore is real.
\end{proof}

Earlier in this section we have defined the Wigner transformation for arbitrary operators, but the most often occurring and the most important case is the transformation of a density matrix.

\begin{definition}
\label{def:mm-wigner:sm:w-function}
	Wigner function is the Wigner transformation of a density matrix:
	\begin{eqn*}
		W(\alpha) \equiv \mathcal{W}[\hat{\rho}].
	\end{eqn*}
\end{definition}

The Wigner function always exists for any density matrix~\cite{Gardiner2004}.
Note also that since any density matrix is Hermitian, according to \thmref{mm-wigner:sm:w-real} $W(\alpha)$ is a real-valued function.

In some cases it will be convenient to use Wigner function in form
\begin{eqn}
	W (\alpha)
	= \frac{1}{\pi^2} \int \upd^2 \lambda \exp(-\lambda \alpha^* + \lambda^* \alpha)
		\chi_W (\lambda),
\end{eqn}
where $\chi_W (\lambda)$ is the characteristic function
\begin{eqn}
	\chi_W (\lambda) = \Trace{ \hat{\rho} \hat{D}(\lambda) }.
\end{eqn}

We will finish this section with the proofs of two main theorems which are required for any practical application of the Wigner function.
The first theorem provides a way to transform any master equation written in terms of creation and annihilation operators to a partial differential equation for the Wigner function.

\begin{theorem}[operator correspondences]
\label{thm:mm-wigner:sm:correspondences}
	For any Hilbert-Schmidt operator $\hat{A}$
	\begin{eqn*}
		\mathcal{W} [ \hat{a} \hat{A} ]
			& = \left( \alpha + \frac{1}{2} \frac{\cwd}{\cwd \alpha^*} \right) \mathcal{W}[\hat{A}],
		\quad
		\mathcal{W} [ \hat{a}^\dagger \hat{A} ]
			= \left( \alpha^* - \frac{1}{2} \frac{\cwd}{\cwd \alpha} \right) \mathcal{W}[\hat{A}], \\
		\mathcal{W} [ \hat{A} \hat{a} ]
			& = \left( \alpha - \frac{1}{2} \frac{\cwd}{\cwd \alpha^*} \right) \mathcal{W}[\hat{A}],
		\quad
		\mathcal{W} [ \hat{A} \hat{a}^\dagger ]
			= \left( \alpha^* + \frac{1}{2} \frac{\cwd}{\cwd \alpha} \right) \mathcal{W}[\hat{A}].
	\end{eqn*}
\end{theorem}
\begin{proof}
We will prove the first correspondence.
First, let us transform the trace using~\eqnref{mm-wigner:sm:displacement-derivatives}:
\begin{eqn}
	\Trace{ \hat{a} \hat{A} \hat{D} }
	= \Trace{ \hat{A} \hat{D} \hat{a}}
	= \left(
		-\frac{\cwd}{\cwd \lambda^*}
		-\frac{1}{2} \lambda
	\right) \Trace{ \hat{A} \hat{D}}.
\end{eqn}
Subsituting this into the definition of the Wigner transformation:
\begin{eqn}
	\mathcal{W} [ \hat{a} \hat{A} ]
	& = \frac{1}{\pi^2} \int \upd^2 \lambda \exp(-\lambda \alpha^* + \lambda^* \alpha)
		\Trace{ \hat{a} \hat{A} \hat{D}(\lambda) } \\
	& = \frac{1}{\pi^2} \int \upd^2 \lambda \exp(-\lambda \alpha^* + \lambda^* \alpha)
		\left(
			-\frac{\cwd}{\cwd \lambda^*}
			-\frac{1}{2} \lambda
		\right)
		\Trace{ \hat{A} \hat{D}(\lambda) } \\
	& = \frac{1}{2} \frac{\cwd}{\cwd \alpha^*} \mathcal{W} [\hat{A}]
	- \frac{1}{\pi^2} \int \upd^2 \lambda \exp(-\lambda \alpha^* + \lambda^* \alpha)
		\frac{\cwd}{\cwd \lambda^*}
		\Trace{ \hat{A} \hat{D}(\lambda) }.
\end{eqn}
The second term is almost the definition of the Wigner function, except for the derivative over $\lambda^*$.
Moving it using integration by parts and \lmmref{c-numbers:zero-integrals} ($\hat{A}$ is Hilbert-Schmidt, which means that $\Trace{\hat{A} \hat{D}}$ is square-integrable~\cite{Cahill1969}):
\begin{eqn}
	= \frac{1}{2} \frac{\cwd}{\cwd \alpha^*} \mathcal{W} [\hat{A}]
	+ \frac{1}{\pi^2} \int \upd^2 \lambda \left(
		\frac{\cwd}{\cwd \lambda^*} \exp(-\lambda \alpha^* + \lambda^* \alpha)
	\right)
	\Trace{ \hat{A} \hat{D}(\lambda) } \\
	= \left( \alpha + \frac{1}{2} \frac{\cwd}{\cwd \alpha^*} \right) \mathcal{W} [\hat{A}].
	\qedhere
\end{eqn}
\end{proof}

This theorem is paired with the second theorem, which helps extract observables (again, expressed in terms of creation and annihilation operators) from the known Wigner function.
We will start by proving the auxiliary lemma which connects the expectation of an operator product and the characteristic function.

\begin{lemma}
\label{lmm:mm-wigner:sm:moments-from-chi}
	For a system with the density matrix $\hat{\rho}$ and the corresponding characteristic function $\chi_W$:
	\begin{eqn*}
		\langle \symprod{ \hat{a}^r (\hat{a}^\dagger)^s } \rangle
		= \left.
			\left( \frac{\cwd}{\cwd \lambda} \right)^s
			\left( -\frac{\cwd}{\cwd \lambda^*} \right)^r
			\chi_W (\lambda)
		\right|_{\lambda=0}.
	\end{eqn*}
\end{lemma}
\begin{proof}
The exponent in the expression for $\chi_W$ can be expanded as
\begin{eqn}
	\exp (\lambda \hat{a}^\dagger - \lambda^* \hat{a})
	= \sum_{j,k}
		\frac{(-\lambda^*)^j \lambda^k}{j!k!}
		\symprod{ \hat{a}^j (\hat{a}^\dagger)^k }.
\end{eqn}
Thus
\begin{eqn}
	\chi_W(\lambda)
	& = \sum_{j,k}
		\frac{(-\lambda^*)^j \lambda^k}{j!k!}
		\Trace{
			\hat{\rho} \symprod{ \hat{a}^j (\hat{a}^\dagger)^k }
		} \\
	& = \sum_{j,k}
		\frac{(-\lambda^*)^j \lambda^k}{j!k!}
		\langle \symprod{ \hat{a}^j (\hat{a}^\dagger)^k } \rangle.
\end{eqn}
Apparently, the application of $(\cwd / \cwd \lambda)^s$ and $(-\cwd / \cwd \lambda^*)^r$ will eliminate all moments with $j < r$ and $k < s$, and setting $\lambda = 0$ afterwards will eliminate all moments with $j > r$ and $k > s$,
leaving only the term with $\symprod{ \hat{a}^r (\hat{a}^\dagger)^s }$:
\begin{eqn}
	\left.
		\left( \frac{\cwd}{\cwd \lambda} \right)^s
		\left( -\frac{\cwd}{\cwd \lambda^*} \right)^r
		\chi_W (\lambda)
	\right|_{\lambda=0}
	& = r! s! \frac{1}{r! s!}
		\langle \symprod{ \hat{a}^r (\hat{a}^\dagger)^s } \rangle \\
	& = \langle \symprod{ \hat{a}^r (\hat{a}^\dagger)^s } \rangle.
	\qedhere
\end{eqn}
\end{proof}

Using this lemma we can now prove the main theorem which connects operator expectations and moments of the Wigner function.

\begin{theorem}
\label{thm:mm-wigner:sm:moments}
	For a system with the density matrix $\rho$ and the corresponding Wigner function $W(\alpha)$, and any non-negative integer $r$, $s$:
	\begin{eqn*}
		\langle \symprod{ \hat{a}^r (\hat{a}^\dagger)^s } \rangle
		= \int \upd^2\alpha\, \alpha^r (\alpha^*)^s W(\alpha).
	\end{eqn*}
\end{theorem}
\begin{proof}
By definition of the Wigner function:
\begin{eqn}
	\int \upd^2\alpha\, \alpha^r (\alpha^*)^s W(\alpha)
	= \frac{1}{\pi^2}
		\int \upd^2\alpha\, \alpha^r (\alpha^*)^s
		\int \upd^2\lambda \exp(-\lambda \alpha^* + \lambda^* \alpha)
		\chi_W (\lambda).
\end{eqn}
Integrating over $\alpha$ using \lmmref{c-numbers:fourier-of-moments}:
\begin{eqn}
	= \int \upd^2\lambda
		\left(
			\left( \frac{\cwd}{\cwd \lambda} \right)^s
			\left( -\frac{\cwd}{\cwd \lambda^*} \right)^r
			\delta(\Real \lambda) \delta(\Imag \lambda)
		\right)
		\chi_W (\lambda).
\end{eqn}
Integrating by parts and eliminating terms which fit \lmmref{c-numbers:zero-delta-integrals}:
\begin{eqn}
	& = \int \upd^2\lambda
		\delta(\Real \lambda) \delta(\Imag \lambda)
		\left( \frac{\cwd}{\cwd \lambda} \right)^s
		\left( -\frac{\cwd}{\cwd \lambda^*} \right)^r
		\chi_W (\lambda) \\
	& = \left.
		\left( \frac{\cwd}{\cwd \lambda} \right)^s
		\left( -\frac{\cwd}{\cwd \lambda^*} \right)^r
		\chi_W (\lambda)
	\right|_{\lambda=0}.
\end{eqn}
Now, recognising the final expression as a part of \lmmref{mm-wigner:sm:moments-from-chi}, we immideately get the statement of the theorem.
\end{proof}

With the help of \thmref{mm-wigner:sm:correspondences} and \thmref{mm-wigner:sm:moments} one can straightforwardly transform any master equation written in terms of creation and annihilation operators $\hat{a}^\dagger$ and $\hat{a}$, transform it to an ordinary differential equation, and use its solution to get expectations of any observables.
