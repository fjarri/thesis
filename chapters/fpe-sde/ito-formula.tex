% =============================================================================
\section{It\^o formula}
% =============================================================================

In this section we will follow a procedure similar to the one in the previous section and derive the It\^o formula for the differential of a functional, based on the standard definition for the multi-variable real-valued case.
Again, we will formulate the real-valued correspondence in a way that is convenient for future proofs.

\begin{lemma}
\label{lmm:fpe-sde:ito-formula:ito-f-real}
	Let $\zvec^T \equiv (z_1 \ldots z_M)$ be a set of real variables, and $\Zvec(t)$ be a standard $L$-dimensional Wiener process.
	For the set of \abbrev{sde}s in the It\^o form
	\begin{eqn*}
		\upd\zvec = \avec(\zvec, t) \upd t + B(\zvec, t) \upd\Zvec(t),
	\end{eqn*}
	the differential of a function $f(\zvec)$ is
	\begin{eqn*}
		\upd f(\zvec) =
			\avec \cdot \vcwd_{\zvec} f(\zvec) \upd t
			+ \frac{1}{2} \Trace{ B B^T \vcwd_{\zvec} \vcwd_{\zvec}^T } f(\zvec) \upd t
			+ \Trace{ B \upd\Zvec \vcwd_{\zvec}^T } f(\zvec).
	\end{eqn*}
\end{lemma}
\begin{proof}
For the detailed proof see Gardiner~\cite{Gardiner1997}, section 4.3.3.
\end{proof}

As a next step, we will extend this lemma to operate on vectors of complex variables and \abbrev{sde}s with complex-valued coefficients from \thmref{fpe-sde:corr:fpe-sde-complex}.

\begin{theorem}
\label{thm:fpe-sde:ito-formula:ito-f-complex}
	Let $\balpha^T \equiv (\alpha_1 \ldots \alpha_M)$ be a set of complex variables, and $\Zvec = (\mathbf{X} + i\mathbf{Y}) / \sqrt{2}$ be an $L$-dimensional standard complex-valued Wiener process, containing two standard $L$-dimensional Wiener processes $\mathbf{X}$ and $\mathbf{Y}$.
	For the set of \abbrev{sde}s in the It\^o form
	\begin{eqn*}
		\upd\balpha = \avec(\balpha, t) \upd t + B(\balpha, t) \upd\Zvec(t),
	\end{eqn*}
	the differential of a function $f(\balpha)$ is
	\begin{eqn*}
		\upd f(\balpha) =
			2 \Real (\avec \cdot \vcwd_{\balpha}) f(\balpha) \upd t
			+ \Trace{ B B^H \vcwd_{\balpha^*} \vcwd_{\balpha}^T } f(\balpha) \upd t
			+ 2 \Real \Trace{ B \upd\Zvec \vcwd_{\balpha}^T } f(\balpha).
	\end{eqn*}
\end{theorem}
\begin{proof}
The proof follows the same scheme as \thmref{fpe-sde:corr:fpe-sde-complex}, just in the opposite direction.
Let $f = g + ih$, $\balpha = \mathbf{x} + i \mathbf{y}$, $\avec = \mathbf{u} + i \mathbf{v}$, $B = F + iG$, $\vcwd_{\balpha} = (\vcwd_{\mathbf{x}} - i \vcwd_{\mathbf{y}}) / 2$.
Then the set of \abbrev{sde}s from the statement is equivalent to
\begin{eqn}
	\upd \begin{pmatrix} \mathbf{x} \\ \mathbf{y} \end{pmatrix}
	= \begin{pmatrix} \mathbf{u} \\ \mathbf{v} \end{pmatrix} \upd t
		+ \frac{1}{\sqrt{2}} \begin{pmatrix} F & -G \\ G & F \end{pmatrix}
			\begin{pmatrix} \upd\mathbf{X} \\ \upd\mathbf{Y} \end{pmatrix}.
\end{eqn}
Applying \lmmref{fpe-sde:ito-formula:ito-f-real} for real-valued functions $g(\mathbf{x}, \mathbf{y})$ and $h(\mathbf{x}, \mathbf{y})$ and combining them into $f = g + ih$, we get
\begin{eqn}
	\upd f ={} &
		\begin{pmatrix} \mathbf{x} \\ \mathbf{y} \end{pmatrix} \cdot
			\begin{pmatrix} \vcwd_{\mathbf{x}} \\ \vcwd_{\mathbf{y}} \end{pmatrix} f \upd t
		+ \frac{1}{4} \Trace{
			\begin{pmatrix} F & -G \\ G & F \end{pmatrix}
			\begin{pmatrix} F^T & G^T \\ -G^T & F^T \end{pmatrix}
			\begin{pmatrix} \vcwd_{\mathbf{x}} \\ \vcwd_{\mathbf{y}} \end{pmatrix}
			\begin{pmatrix} \vcwd_{\mathbf{x}} \\ \vcwd_{\mathbf{y}} \end{pmatrix}^T
		} f \upd t  \\
	& + \frac{1}{\sqrt{2}} \Trace{
			\begin{pmatrix} F & -G \\ G & F \end{pmatrix}
			\begin{pmatrix} \upd\mathbf{X} \\ \upd\mathbf{Y} \end{pmatrix}
			\begin{pmatrix} \vcwd_{\mathbf{x}} \\ \vcwd_{\mathbf{y}} \end{pmatrix}^T
		} f.
\end{eqn}
Now let us match this equation and the lemma statement term by term.

First term:
\begin{eqn}
	2 \Real ( \avec \cdot \vcwd_{\balpha} )
	& = \Real \left(
			\left( \mathbf{u} + i\mathbf{v} \right) \cdot \left( \vcwd_{\mathbf{x}} - i \vcwd_{\mathbf{y}} \right)
		\right) \\
	& = \mathbf{u} \cdot \vcwd_{\mathbf{x}} + \mathbf{v} \cdot \vcwd_{\mathbf{y}} \\
	& = \begin{pmatrix} \mathbf{x} \\ \mathbf{y} \end{pmatrix} \cdot
		\begin{pmatrix} \vcwd_{\mathbf{x}} \\ \vcwd_{\mathbf{y}} \end{pmatrix}.
\end{eqn}

Second term:
\begin{eqn}
	\Trace{ B B^H \vcwd_{\balpha^*} \vcwd_{\balpha}^T }
	={} & \frac{1}{4} \Trace{
		(F F^T + G G^T)
		(\vcwd_{\mathbf{x}} \vcwd_{\mathbf{x}}^T
			+ \vcwd_{\mathbf{y}} \vcwd_{\mathbf{y}}^T)
		} \\
	& - \frac{1}{4} \Trace {
		(F G^T - G F^T)
		(\vcwd_{\mathbf{x}} \vcwd_{\mathbf{y}}^T
			- \vcwd_{\mathbf{y}} \vcwd_{\mathbf{x}}^T)
		} \\
	& + \frac{i}{4} \Trace{
		(F G^T - G F^T)
		(\vcwd_{\mathbf{x}} \vcwd_{\mathbf{x}}^T
			+ \vcwd_{\mathbf{y}} \vcwd_{\mathbf{y}}^T)
	} \\
	& + \frac{i}{4} \Trace{
		(G G^T + F F^T)
		(\vcwd_{\mathbf{x}} \vcwd_{\mathbf{y}}^T
			- \vcwd_{\mathbf{y}} \vcwd_{\mathbf{x}}^T)
	}.
\end{eqn}
Same as in \thmref{fpe-sde:corr:fpe-sde-complex} we notice that $F F^T + G G^T$ and $\vcwd_{\mathbf{x}} \vcwd_{\mathbf{x}}^T + \vcwd_{\mathbf{y}} \vcwd_{\mathbf{y}}^T$ are symmetric matrices, and $F G^T - G F^T$ and $\vcwd_{\mathbf{x}} \vcwd_{\mathbf{y}}^T - \vcwd_{\mathbf{y}} \vcwd_{\mathbf{x}}^T$ are antisymmetric ones.
Therefore, the last two terms contain traces of antisymmetric matrices and are, consequently, equal to zero:
\begin{eqn}
	={} & \frac{1}{4} \Trace{
		(F F^T + G G^T) \vcwd_{\mathbf{x}} \vcwd_{\mathbf{x}}^T
		+ (F G^T - G F^T) \vcwd_{\mathbf{y}} \vcwd_{\mathbf{x}}^T)
		} \\
	& + \frac{1}{4} \Trace {
		(G F^T - F G^T) \vcwd_{\mathbf{x}} \vcwd_{\mathbf{y}}^T
		+ (F F^T + G G^T) \vcwd_{\mathbf{y}} \vcwd_{\mathbf{y}}^T)
		} \\
	={} & \frac{1}{4} \Trace {
		\begin{pmatrix}
			F F^T + G G^T & F G^T - G F^T \\
			G F^T - F G^T & F F^T + G G^T
		\end{pmatrix}
		\begin{pmatrix}
			\vcwd_{\mathbf{x}} \vcwd_{\mathbf{x}}^T & \vcwd_{\mathbf{x}} \vcwd_{\mathbf{y}}^T \\
			\vcwd_{\mathbf{y}} \vcwd_{\mathbf{x}}^T & \vcwd_{\mathbf{y}} \vcwd_{\mathbf{y}}^T
		\end{pmatrix}
	} \\
	={} & \frac{1}{4} \Trace{
		\begin{pmatrix} F & -G \\ G & F \end{pmatrix}
		\begin{pmatrix} F^T & G^T \\ -G^T & F^T \end{pmatrix}
		\begin{pmatrix} \vcwd_{\mathbf{x}} \\ \vcwd_{\mathbf{y}} \end{pmatrix}
		\begin{pmatrix} \vcwd_{\mathbf{x}} \\ \vcwd_{\mathbf{y}} \end{pmatrix}^T
	}.
\end{eqn}

Third term:
\begin{eqn}
	2 \Real \Trace{ B \upd\Zvec \vcwd_{\balpha}^T }
	& = \frac{1}{\sqrt{2}} \Real \Trace{
		(F + iG) (\upd\mathbf{X} + i \upd\mathbf{Y}) (\vcwd_{\mathbf{x}} - i\vcwd_{\mathbf{y}})
	} \\
	& = \frac{1}{\sqrt{2}} \Trace{
		F \upd\mathbf{X} \vcwd_{\mathbf{x}} + F \upd\mathbf{Y} \vcwd_{\mathbf{y}}
		- G \upd\mathbf{Y} \vcwd_{\mathbf{x}} + G \upd\mathbf{X} \vcwd_{\mathbf{y}}
	} \\
	& = \frac{1}{\sqrt{2}} \Trace{
			\begin{pmatrix} F & -G \\ G & F \end{pmatrix}
			\begin{pmatrix} \upd\mathbf{X} \\ \upd\mathbf{Y} \end{pmatrix}
			\begin{pmatrix} \vcwd_{\mathbf{x}} \\ \vcwd_{\mathbf{y}} \end{pmatrix}^T
		}.
\end{eqn}

All terms have matched, thus proving the theorem.
\end{proof}

The above theorem can be reformulated for the multi-component \abbrev{sde}s from \thmref{fpe-sde:corr:mc-fpe-sde}.

\begin{theorem}
\label{thm:fpe-sde:ito-formula:mc-ito-f}
	Let $\balpha^{(j)},\, j = 1 \ldots C$ be $C$ sets of complex variables $\balpha^{(j)} \equiv (\alpha_1^{(j)} \ldots \alpha_{M_j}^{(j)})$.
	For the \abbrev{sde} in the It\^o form
	\begin{eqn*}
		\upd\balpha^{(j)} = \avec^{(j)} \upd t + B^{(j)} \upd\Zvec,
	\end{eqn*}
	the differential of a function $f(\balpha^{(1)}, \ldots, \balpha^{(C)})$ is
	\begin{eqn*}
		\upd f ={} &
			2 \sum_{j=1}^C \Real (\avec^{(j)} \cdot \vcwd_{\balpha^{(j)}}) f \upd t
			+ \sum_{j=1}^C \sum_{k=1}^C \Trace{
				B^{(j)} (B^{(k)})^H \vcwd_{(\balpha^{(k)})^*} \vcwd_{\balpha^{(j)}}^T } f \upd t \\
		& + 2 \sum_{j=1}^C \Real \Trace{ B^{(j)} \upd\Zvec \vcwd_{\balpha^{(j)}}^T } f.
	\end{eqn*}
\end{theorem}
\begin{proof}
Proved analogously to \thmref{fpe-sde:corr:mc-fpe-sde}, by combining $\balpha^{(j)}$ into a single vector and applying \thmref{fpe-sde:ito-formula:ito-f-complex}.
\end{proof}

Finally, we can use the multi-component reformulation to derive the It\^o formula in functional form for the set of \abbrev{sde}s from \thmref{fpe-sde:corr:fpe-sde-func}.

\begin{theorem}
\label{thm:fpe-sde:ito-formula:func-ito-f}
	Given the set of functional \abbrev{sde}s in the It\^o form
	\begin{eqn*}
		\upd f_j = \proj{\restbasis{j}} \left[
			\mathcal{A}_j \upd t + \sum_{l=1}^L \mathcal{B}_{jl} \upd Q_l
		\right],
	\end{eqn*}
	the differential of a functional operator $\mathcal{F}[\fvec]$ is
	\begin{eqn*}
		\upd \mathcal{F}[\fvec]
		={} & \int \upd\xvec^\prime \left(
			2 \sum_{j=1}^C \Real \left(
				\mathcal{A}_j^\prime \frac{\fdelta}{\fdelta f_j^\prime}
			\right) \mathcal{F}[\fvec] \upd t \right. \\
		& + \sum_{j=1}^C \sum_{k=1}^C \sum_{l=1}^L
				\mathcal{B}_{jl}^\prime
				\mathcal{B}_{kl}^{\prime *}
				\frac{\fdelta}{\fdelta f_j^\prime}
				\frac{\fdelta}{\fdelta f_k^{\prime *}} \mathcal{F}[\fvec] \upd t \\
		& \left. + 2 \sum_{j=1}^C \sum_{l=1}^L
			\Real \left(
				\mathcal{B}_{jl}^\prime
				\upd Q_l^\prime
				\frac{\fdelta}{\fdelta f_j^\prime}
			\right)
			\mathcal{F}[\fvec]
		\right).
	\end{eqn*}
	Here $f_j$, $\mathcal{A}_j$, $\mathcal{B}_{jl}$, $L$, and $Q_l$ are defined in the same way as in \thmref{fpe-sde:corr:fpe-sde-func}.
\end{theorem}
\begin{proof}
The set of \abbrev{sde}s can be rewritten in terms of complex vectors as
\begin{eqn}
	\upd\alpha_{\mvec}^{(j)}
	= \int \upd\xvec\, \phi_{j,\mvec}^* \mathcal{A}_j \upd t
	+ \sum_{l=1}^L \sum_{\pvec \in \fullbasis}
		\int \upd\xvec\, \phi_{j,\mvec}^* \mathcal{B}_{jl} \phi_{\pvec} \upd Z_{\pvec,l},\quad
	\mvec \in \restbasis_j.
\end{eqn}
Now, treating the functional operator as a function of $C$ complex vectors
\begin{eqn}
	\mathcal{F} \equiv \mathcal{F}[\mathcal{C}_{\restbasis_1}(\balpha^{(1)}), \ldots, \mathcal{C}_{\restbasis_C}(\balpha^{(C)})],
\end{eqn}
we can use \thmref{fpe-sde:ito-formula:mc-ito-f} with the drift vectors
\begin{eqn}
	a_{\mvec}^{(j)} = \int \upd\xvec\, \phi_{j,\mvec}^* \mathcal{A}_j,
\end{eqn}
and the noise matrices
\begin{eqn}
	B_{\mvec,(\pvec,l)}^{(j)}
	= \int \upd\xvec\, \phi_{j,\mvec}^* \mathcal{B}_{jl} \phi_{\pvec}.
\end{eqn}
Applying \thmref{fpe-sde:ito-formula:mc-ito-f}:
\begin{eqn}
	\upd \mathcal{F}
	={} &
		2 \sum_{j=1}^C \sum_{\mvec \in \restbasis_j} \Real \left(
			\int \upd\xvec^\prime \phi_{j,\mvec}^{\prime*} \mathcal{A}_j^\prime
			\frac{\cwd}{\cwd \alpha_{j,\mvec}}
		\right) \mathcal{F} \upd t \\
	& + \sum_{j=1}^C \sum_{k=1}^C
			\sum_{\mvec \in \restbasis_j} \sum_{\nvec \in \restbasis_k}
			\sum_{l=1}^L \sum_{\pvec \in \fullbasis}
			\int \upd\xvec^\prime \phi_{j,\mvec}^{\prime *} \mathcal{B}_{jl}^\prime \phi_{\pvec}^\prime
			\int \upd\xvec^{\prime\prime} \phi_{k,\nvec}^{\prime\prime} \mathcal{B}_{kl}^{\prime\prime *} \phi_{\pvec}^{\prime\prime *}
			\frac{\cwd^2 \mathcal{F}}{\cwd \alpha_{k,\nvec}^* \cwd \alpha_{j,\mvec}}
			\upd t \\
	& + 2 \sum_{j=1}^C \Real \left(
			\sum_{\mvec \in \restbasis_j}
			\sum_{l=1}^L \sum_{\pvec \in \fullbasis}
			\int \upd\xvec^\prime \phi_{j,\mvec}^{\prime*} \mathcal{B}_{jl}^\prime \phi_{\pvec}^\prime
			\upd Z_{\pvec,l}
			\frac{\cwd}{\cwd \alpha_{j,\mvec}}
			\mathcal{F}
	\right).
\end{eqn}
Recognising definitions of the functional differentials, the functional Wiener process, and the delta function, we get
\begin{eqn}
	={} & 2 \sum_{j=1}^C \Real \left(
			\int \upd\xvec^\prime \mathcal{A}_j^\prime
			\frac{\fdelta}{\fdelta f_j^\prime}
		\right) \mathcal{F} \upd t
	+ \sum_{j=1}^C \sum_{k=1}^C \sum_{l=1}^L
			\int \upd\xvec^\prime \mathcal{B}_{jl}^\prime
			\mathcal{B}_{kl}^{\prime *}
			\frac{\fdelta}{\fdelta f_j^\prime}
			\frac{\fdelta}{\fdelta f_k^{\prime *}} \mathcal{F} \upd t \\
	& + 2 \sum_{j=1}^C \sum_{l=1}^L \Real \left(
			\int \upd\xvec^\prime \mathcal{B}_{jl}^\prime
			\upd Q_l^\prime
			\frac{\fdelta}{\fdelta f_j^\prime}
		\right) \mathcal{F},
\end{eqn}
which leads to the statement of the theorem.
\end{proof}

Alternatively, the functional It\^o formula can be written in the matrix form as
\begin{eqn}
	\upd \mathcal{F}[\fvec]
	={} & \int \upd\xvec^\prime \left(
		2 \Real \left(
			\mathbfcal{A}^\prime \cdot \vfdelta_{\bPsi^\prime}
		\right) \mathcal{F}[\fvec] \upd t
		+ \Trace{
			\mathcal{B}^\prime
			(\mathcal{B}^\prime)^H
			\vfdelta_{\fvec^{\prime *}}
			\vfdelta_{\fvec^\prime}^T
		} \mathcal{F}[\fvec] \upd t \right. \\
	& \left. + 2 \Real \Trace{
			\mathcal{B}^\prime
			\upd\mathbf{Q}^\prime
			\vfdelta_{\fvec^\prime}^T
		} \mathcal{F}[\fvec]
	\right).
\end{eqn}
