% =============================================================================
\section{It\^{o} formula}
% =============================================================================

In this section we will derive the It\^{o} formula for the differential of a functional, based on the standard definition for multi-variable real-valued case.

\begin{theorem}
\label{thm:wigner-bec:ito-formula:ito-f-real}
	Let $\zvec^T \equiv (z_1 \ldots z_M)$ be a set of real variables, and $\Zvec(t)$ be a standard $L$-dimensional Wiener process.
	For the SDE in It\^{o} form
	\begin{eqn*}
		d\zvec = \avec(\zvec, t) dt + B(\zvec, t) d\Zvec(t),
	\end{eqn*}
	the differential of a function $f(\zvec)$ is
	\begin{eqn*}
		df(\zvec) = \left(
			\avec \cdot \bpartial_{\zvec} dt
			+ \frac{1}{2} \Trace{ B B^T \bpartial_{\zvec} \bpartial_{\zvec}^T } dt
			+ \Trace{ B d\Zvec \bpartial_{\zvec}^T }
		\right) f(\zvec).
	\end{eqn*}
\end{theorem}
\begin{proof}
This is just a statement from~\cite{Gardiner1997} expressed in matrix form.
\end{proof}

This theorem can be extended to complex variables.

\begin{theorem}
\label{thm:wigner-bec:ito-formula:ito-f-complex}
	Let $\balpha^T \equiv (\alpha_1 \ldots \alpha_M)$ be a set of complex variables, and $\Zvec = (\bm{X} + i\bm{Y}) / \sqrt{2}$ be an $L$-dimensional complex-valued Wiener process, containing two standard $L$-dimensional Wiener processes $\bm{X}$ and $\bm{Y}$.
	For the SDE in It\^{o} form
	\begin{eqn*}
		d\balpha = \avec(\balpha, t) dt + B(\balpha, t) d\Zvec(t),
	\end{eqn*}
	the differential of a function $f(\balpha)$ is
	\begin{eqn*}
		df(\balpha) = \left(
			2 \Real (\avec \cdot \bpartial_{\balpha}) dt
			+ \Trace{ B B^H \bpartial_{\balpha^*} \bpartial_{\balpha}^T } dt
			+ 2 \Real \Trace{ B d\Zvec \bpartial_{\balpha}^T }
		\right) f(\balpha).
	\end{eqn*}
\end{theorem}
\begin{proof}
The proof follows the same scheme as \thmref{wigner-bec:fpe:fpe-sde-complex}, just in the opposite direction.
Let $f = g + ih$, $\balpha = \bm{x} + i \bm{y}$, $\avec = \bm{u} + i \bm{v}$, $B = F + iG$, $\bpartial_{\balpha} = (\bpartial_{\bm{x}} - i \bpartial_{\bm{y}}) / 2$.
Then the set of SDEs from the statement is equivalent to
\begin{eqn}
	d \begin{pmatrix} \bm{x} \\ \bm{y} \end{pmatrix}
	= \begin{pmatrix} \bm{u} \\ \bm{v} \end{pmatrix} dt
		+ \frac{1}{\sqrt{2}} \begin{pmatrix} F & -G \\ G & F \end{pmatrix}
			\begin{pmatrix} d\bm{X} \\ d\bm{Y} \end{pmatrix}.
\end{eqn}
Applying \thmref{wigner-bec:ito-formula:ito-f-real} for real-valued functions $g(\bm{x}, \bm{y})$ and $h(\bm{x}, \bm{y})$ and combining them into $f = g + ih$:
\begin{eqn}
	df ={} &
		\begin{pmatrix} \bm{x} \\ \bm{y} \end{pmatrix} \cdot
			\begin{pmatrix} \bpartial_{\bm{x}} \\ \bpartial_{\bm{y}} \end{pmatrix} f dt
		+ \frac{1}{4} \Trace{
			\begin{pmatrix} F & -G \\ G & F \end{pmatrix}
			\begin{pmatrix} F^T & G^T \\ -G^T & F^T \end{pmatrix}
			\begin{pmatrix} \bpartial_{\bm{x}} \\ \bpartial_{\bm{y}} \end{pmatrix}
			\begin{pmatrix} \bpartial_{\bm{x}} \\ \bpartial_{\bm{y}} \end{pmatrix}^T
		} f dt  \\
	& + \frac{1}{\sqrt{2}} \Trace{
			\begin{pmatrix} F & -G \\ G & F \end{pmatrix}
			\begin{pmatrix} d\bm{X} \\ d\bm{Y} \end{pmatrix}
			\begin{pmatrix} \bpartial_{\bm{x}} \\ \bpartial_{\bm{y}} \end{pmatrix}^T
		} f
\end{eqn}
Now let us match this equation and the lemma statement term by term.

First term:
\begin{eqn}
	2 \Real ( \avec \cdot \bpartial_{\balpha} )
	& = \Real \left(
			\left( \bm{u} + i\bm{v} \right) \cdot \left( \bpartial_{\bm{x}} - i \bpartial_{\bm{y}} \right)
		\right) \\
	& = \bm{u} \cdot \bpartial_{\bm{x}} + \bm{v} \cdot \bpartial_{\bm{y}} \\
	& = \begin{pmatrix} \bm{x} \\ \bm{y} \end{pmatrix} \cdot
		\begin{pmatrix} \bpartial_{\bm{x}} \\ \bpartial_{\bm{y}} \end{pmatrix}
\end{eqn}

Second term:
\begin{eqn}
	\Trace{ B B^H \bpartial_{\balpha^*} \bpartial_{\balpha}^T }
	={} & \frac{1}{4} \Trace{
		(F F^T + G G^T)
		(\bpartial_{\bm{x}} \bpartial_{\bm{x}}^T
			+ \bpartial_{\bm{y}} \bpartial_{\bm{y}}^T)
		} \\
	& - \frac{1}{4} \Trace {
		(F G^T - G F^T)
		(\bpartial_{\bm{x}} \bpartial_{\bm{y}}^T
			- \bpartial_{\bm{y}} \bpartial_{\bm{x}}^T)
		} \\
	& + \frac{i}{4} \Trace{
		(F G^T - G F^T)
		(\bpartial_{\bm{x}} \bpartial_{\bm{x}}^T
			+ \bpartial_{\bm{y}} \bpartial_{\bm{y}}^T)
	} \\
	& + \frac{i}{4} \Trace{
		(G G^T + F F^T)
		(\bpartial_{\bm{x}} \bpartial_{\bm{y}}^T
			- \bpartial_{\bm{y}} \bpartial_{\bm{x}}^T)
	}
\end{eqn}
Same as in \thmref{wigner-bec:fpe:fpe-sde-complex} we notice that $F F^T + G G^T$ and $\bpartial_{\bm{x}} \bpartial_{\bm{x}}^T + \bpartial_{\bm{y}} \bpartial_{\bm{y}}^T$ are symmetric matrices, and $F G^T - G F^T$ and $\bpartial_{\bm{x}} \bpartial_{\bm{y}}^T - \bpartial_{\bm{y}} \bpartial_{\bm{x}}^T$ are antisymmetric.
Therefore the last two terms contain traces of antisymmetric matrices and are equal to zero.
\begin{eqn}
	={} & \frac{1}{4} \Trace{
		(F F^T + G G^T) \bpartial_{\bm{x}} \bpartial_{\bm{x}}^T
		+ (F G^T - G F^T) \bpartial_{\bm{y}} \bpartial_{\bm{x}}^T)
		} \\
	& + \frac{1}{4} \Trace {
		(G F^T - F G^T) \bpartial_{\bm{x}} \bpartial_{\bm{y}}^T
		+ (F F^T + G G^T) \bpartial_{\bm{y}} \bpartial_{\bm{y}}^T)
		} \\
	={} & \frac{1}{4} \Trace {
		\begin{pmatrix}
			F F^T + G G^T & F G^T - G F^T \\
			G F^T - F G^T & F F^T + G G^T
		\end{pmatrix}
		\begin{pmatrix}
			\bpartial_{\bm{x}} \bpartial_{\bm{x}}^T & \bpartial_{\bm{x}} \bpartial_{\bm{y}}^T \\
			\bpartial_{\bm{y}} \bpartial_{\bm{x}}^T & \bpartial_{\bm{y}} \bpartial_{\bm{y}}^T
		\end{pmatrix}
	} \\
	={} & \frac{1}{4} \Trace{
		\begin{pmatrix} F & -G \\ G & F \end{pmatrix}
		\begin{pmatrix} F^T & G^T \\ -G^T & F^T \end{pmatrix}
		\begin{pmatrix} \bpartial_{\bm{x}} \\ \bpartial_{\bm{y}} \end{pmatrix}
		\begin{pmatrix} \bpartial_{\bm{x}} \\ \bpartial_{\bm{y}} \end{pmatrix}^T
	}.
\end{eqn}

Third term:
\begin{eqn}
	2 \Real \Trace{ B d\Zvec \bpartial_{\balpha}^T }
	& = \frac{1}{\sqrt{2}} \Real \Trace{
		(F + iG) (d\bm{X} + id\bm{Y}) (\bpartial_{\bm{x}} - i\bpartial_{\bm{y}})
	} \\
	& = \frac{1}{\sqrt{2}} \Trace{
		F d\bm{X} \bpartial_{\bm{x}} + F d\bm{Y} \bpartial_{\bm{y}}
		- G d\bm{Y} \bpartial_{\bm{x}} + G d\bm{X} \bpartial_{\bm{y}}
	} \\
	& = \frac{1}{\sqrt{2}} \Trace{
			\begin{pmatrix} F & -G \\ G & F \end{pmatrix}
			\begin{pmatrix} d\bm{X} \\ d\bm{Y} \end{pmatrix}
			\begin{pmatrix} \bpartial_{\bm{x}} \\ \bpartial_{\bm{y}} \end{pmatrix}^T
		}.
\end{eqn}

All terms have matched, thus proving the theorem.
\end{proof}

\begin{theorem}
\label{thm:wigner-bec:ito-formula:mc-ito-f}
	Let $\balpha^{(c)},\, c = 1..C$ be $C$ sets of complex variables $\balpha^{(c)} \equiv (\alpha_1^{(c)} \ldots \alpha_{M_c}^{(c)})$.
	For the SDE in It\^{o} form
	\begin{eqn*}
		d\balpha^{(c)} = \avec^{(c)} dt + B^{(c)} d\Zvec,
	\end{eqn*}
	the differential of a function $f(\balpha^{(1)}, \ldots, \balpha^{(C)})$ is
	\begin{eqn*}
		df ={} & \left(
			2 \sum_{c=1}^C \Real (\avec^{(c)} \cdot \bpartial_{\balpha^{(c)}}) dt
			+ \sum_{m=1}^C \sum_{n=1}^C \Trace{
				B^{(m)} (B^{(n)})^H \bpartial_{(\balpha^{(n)})^*} \bpartial_{\balpha^{(m)}}^T } dt \right. \\
		& \left. + 2 \sum_{c=1}^C \Real \Trace{ B^{(c)} d\Zvec \bpartial_{\balpha^{(c)}}^T }
		\right) f.
	\end{eqn*}
\end{theorem}
\begin{proof}
Proved analogously to \thmref{wigner-bec:fpe:mc-fpe-sde}, by combining $\balpha^{(c)}$ into a single vector	and applying \thmref{wigner-bec:ito-formula:ito-f-complex}.
\end{proof}

\begin{theorem}
\label{thm:wigner-bec:ito-formula:func-ito-f}
	Given functional SDEs in It\^{o} form
	\begin{eqn*}
		d\Psi^{(c)} = \mathcal{A}^{(c)} dt + \sum_{\lvec} \mathcal{B}_{\lvec}^{(c)} dQ_{\lvec},
	\end{eqn*}
	the differential of a functional $F[\Psivec]$ is
	\begin{eqn*}
		dF[\Psivec]
		={} & \int d\xvec^\prime \left(
			2 \sum_{c=1}^C \Real \left(
				\mathcal{A}^{(c)\prime} \frac{\delta}{\delta \Psi_c^\prime}
			\right) dt
			+ \sum_{i=1}^C \sum_{j=1}^C \sum_{\lvec}
				\mathcal{B}_{\lvec}^{(i)\prime}
				\mathcal{B}_{\lvec}^{(j)\prime *}
				\frac{\delta}{\delta \Psi_i^\prime}
				\frac{\delta}{\delta \Psi_j^{\prime *}} dt
			\right. \\
		& \left. + 2 \sum_{c=1}^C \sum_{\lvec} \Real \left(
				\mathcal{B}_{\lvec}^{(i)\prime}
				dQ_{\lvec}^\prime
				\frac{\delta}{\delta \Psi_c^\prime}
			\right)
		\right) F[\Psivec]
	\end{eqn*}
	\todo{Consider rewriting it as
	\begin{eqn*}
		dF[\Psivec]
		= \int d\xvec^\prime \left(
			2 \Real \left(
				\vec{\mathcal{A}}^\prime \cdot \bdelta_{\bPsi^\prime}
			\right) dt
			+ \Trace{
				\mathcal{B}^\prime
				(\mathcal{B}^\prime)^H
				\bdelta_{\Psivec^{\prime *}}
				\bdelta_{\Psivec^\prime}^T
			} dt
			+ 2 \Real \Trace{
				\mathcal{B}^\prime
				d\vec{Q}^\prime
				\bdelta_{\Psivec^\prime}^T
			}
		\right) F[\Psivec].
	\end{eqn*}
	}
\end{theorem}
\begin{proof}
In terms of complex vectors SDEs can be rewritten as
\begin{eqn}
	d\alpha_{\mvec}^{(c)}
	= \int d\xvec \phi_{c,\mvec}^* \mathcal{A}^{(c)} dt
	+ \sum_{\pvec \in \fullbasis, \lvec}
		\int d\xvec \phi_{c,\mvec}^* \mathcal{B}_{\lvec}^{(c)} \phi_{\pvec} dZ_{\pvec,\lvec}.
\end{eqn}
Now, treating the functional as a function of complex vector $F \equiv F(\balpha^{(1)}, \ldots, \balpha^{(C)})$, we can use \thmref{wigner-bec:ito-formula:mc-ito-f} with
\begin{eqn}
	(\bm{a}^{(c)})_{\mvec} = \int d\xvec \phi_{c,\mvec}^* \mathcal{A}^{(c)},
\end{eqn}
and
\begin{eqn}
	(B^{(c)})_{\mvec,(\pvec,\lvec)}
	= \int d\xvec \phi_{c,\mvec}^* \mathcal{B}_{\lvec}^{(c)} \phi_{\pvec}.
\end{eqn}
This gives us
\begin{eqn}
	dF
	={} & \left(
		2 \sum_{c=1}^C \sum_{\mvec \in \restbasis_c} \Real \left(
			\int d\xvec^\prime \phi_{c,\mvec}^{\prime*} \mathcal{A}^{(c)\prime}
			\frac{\partial}{\partial \alpha_{c,\mvec}}
		\right) \right. \\
	& \left. + \sum_{i=1}^C \sum_{j=1}^C
			\sum_{\mvec \in \restbasis_i} \sum_{\nvec \in \restbasis_j}
			\sum_{\pvec \in \fullbasis, \lvec}
			\int d\xvec^\prime \phi_{i,\mvec}^{\prime *} \mathcal{B}_{\lvec}^{(i)\prime} \phi_{\pvec}^\prime
			\int d\xvec^{\prime\prime} \phi_{j,\nvec}^{\prime\prime} \mathcal{B}_{\lvec}^{(j)\prime\prime *} \phi_{\pvec}^{\prime\prime *}
			\frac{\partial}{\partial_{\alpha_{j,\nvec}^*}}
			\frac{\partial}{\partial_{\alpha_{i,\mvec}}} \right. \\
	& \left. + 2 \sum_{c=1}^C \Real \left(
			\sum_{\mvec \in \restbasis_c}
			\sum_{\pvec \in \fullbasis, \lvec}
			\int d\xvec^\prime \phi_{i,\mvec}^{\prime*} \mathcal{B}_{\lvec}^{(i)\prime} \phi_{\pvec}^\prime
			dZ_{\pvec,\lvec}
			\partial_{\alpha_{c,\mvec}}
		\right)
	\right) F
\end{eqn}
Recognizing definitions of functional differentials, functional Wiener process, and the delta function, we get
\begin{eqn}
	={} & \left(
		2 \sum_{c=1}^C \Real \left(
			\int d\xvec^\prime \mathcal{A}^{(c)\prime}
			\frac{\delta}{\delta \Psi_c^\prime}
		\right) \right. \\
	& \left. + \sum_{i=1}^C \sum_{j=1}^C \sum_{\lvec}
			\int d\xvec^\prime \mathcal{B}_{\lvec}^{(i)\prime}
			\mathcal{B}_{\lvec}^{(j)\prime *}
			\frac{\delta}{\delta \Psi_i^\prime}
			\frac{\delta}{\delta \Psi_j^{\prime *}}
		\right. \\
	& \left. + 2 \sum_{c=1}^C \sum_{\lvec} \Real \left(
			\int d\xvec^\prime \mathcal{B}_{\lvec}^{(i)\prime}
			dQ_{\lvec}^\prime
			\frac{\delta}{\delta \Psi_c^\prime}
		\right)
	\right) F
\end{eqn}
Which leads to the statement of the theorem.
\end{proof}
