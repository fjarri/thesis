\chapter{Multimode field operators formalism}
\label{cha:appendix:multimode-formalism}


% =============================================================================
\section{Single-mode operators}
% =============================================================================

We start from the set of single-mode operators $\hat{a}_j$, which obey bosonic commutation relations:
\begin{equation}
\label{eqn:multimode-formalism:single-mode-commutators}
\begin{split}
	[ \hat{a}_j, \hat{a}_k ] & = [ \hat{a}_j^\dagger, \hat{a}_k^\dagger ] = 0, \\
	[ \hat{a}_j, \hat{a}_k^\dagger ] & = \delta_{jk}.
\end{split}
\end{equation}

In order to work with the moments of multimode operators we will need the equations for commutators of arbitrary single-mode operator products $[ \hat{a}_n, \hat{a}_{m_1}^\dagger \ldots \hat{a}_{m_k}^\dagger ]$ and $[ \hat{a}_n^\dagger, \hat{a}_{m_1} \ldots \hat{a}_{m_k} ]$.
Let us find the expression for the first commutator by induction.
Providing that we know the expression for $[ \hat{a}_n, \hat{a}_{m_1}^\dagger \ldots \hat{a}_{m_{k-1}}^\dagger ]$,
commutator of order $k$ can be expanded as:
\[
	[ \hat{a}_n, \hat{a}_{m_1}^\dagger \ldots \hat{a}_{m_k}^\dagger ]
	= (1 - \delta_{n m_k})
		[ \hat{a}_n, \hat{a}_{m_1}^\dagger \ldots \hat{a}_{m_{k-1}}^\dagger ] \hat{a}_{m_k}
	+ \delta_{n m_k} (
		\hat{a}_n \hat{a}_{m_1}^\dagger \ldots \hat{a}_{m_{k-1}}^\dagger \hat{a}_n^\dagger
		- \hat{a}_{m_1}^\dagger \ldots \hat{a}_{m_{k-1}}^\dagger \hat{a}_n^\dagger \hat{a}_n
	)
	= (*).
\]
Here we have split the initial commutator into two possible outcomes, depending on whether $n = m_k$.
First term, corresponding to $n \ne m_k$, contains the known commutator of lower order.
In the second term we have substituted $\hat{a}_n$ for $\hat{a}_{m_k}$,
since the delta function outside the parentheses ensures that $n = m_k$.
Swapping $\hat{a}_n^\dagger$ and $\hat{a}_n$ in the last term and, again, recognising the known commutator:
\begin{equation*}
\begin{split}
	(*)
	& = (1 - \delta_{n m_k})
		[ \hat{a}_n, \hat{a}_{m_1}^\dagger \ldots \hat{a}_{m_{k-1}}^\dagger ] \hat{a}_{m_k}
	+ \delta_{n m_k} (
		[ \hat{a}_n, \hat{a}_{m_1}^\dagger \ldots \hat{a}_{m_{k-1}}^\dagger ] \hat{a}_n^\dagger
		+ \hat{a}_{m_1}^\dagger \ldots \hat{a}_{m_{k-1}}^\dagger
	) \\
	& = [ \hat{a}_n, \hat{a}_{m_1}^\dagger \ldots \hat{a}_{m_{k-1}}^\dagger ] \hat{a}_{m_k}
	+ \delta_{n m_k} \hat{a}_{m_1}^\dagger \ldots \hat{a}_{m_{k-1}}^\dagger.
\end{split}
\end{equation*}
Now, starting from the first-order relation $[ \hat{a}_n, \hat{a}_{m_1}^\dagger ] = \delta_{n m_1}$, we can obtain the relation for any order:
\begin{equation*}
\begin{split}
	[ \hat{a}_n, \hat{a}_{m_1}^\dagger \hat{a}_{m_2}^\dagger ]
	& = \delta_{n m_1} \hat{a}_{m_2}^\dagger + \delta_{n m_2} \hat{a}_{m_1}^\dagger, \\
	[ \hat{a}_n, \hat{a}_{m_1}^\dagger \hat{a}_{m_2}^\dagger \hat{a}_{m_3}^\dagger ]
	& = \delta_{n m_1} \hat{a}_{m_2}^\dagger \hat{a}_{m_3}^\dagger
	+ \delta_{n m_2} \hat{a}_{m_1}^\dagger \hat{a}_{m_3}^\dagger
	+ \delta_{n m_3} \hat{a}_{m_1}^\dagger \hat{a}_{m_2}^\dagger, \\
	& \ldots
\end{split}
\end{equation*}
or, in generalised form:
\begin{equation}
\label{eqn:multimode-formalism:single-mode-high-order-commutators}
	[ \hat{a}_n, \hat{a}_{m_1}^\dagger \ldots \hat{a}_{m_k}^\dagger ]
	= \sum\limits_{i=1}^k \delta_{n m_i}
		\prod\limits_{j=1,j \ne i}^k \hat{a}_{m_j}^\dagger.
\end{equation}
Note that if $n = m_1 = \ldots = m_k$, this boils down to the well-known relation from~\cite{Louisell1990}:
\[
	[ \hat{a}, (\hat{a}^\dagger)^k ] = k (\hat{a}^\dagger)^{k-1}.
\]
The general form for the second commutator can be found using the exact same procedure:
\begin{equation}
	[ \hat{a}_n^\dagger, \hat{a}_{m_1} \ldots \hat{a}_{m_k} ]
	= - \sum\limits_{i=1}^k \delta_{n m_i}
		\prod\limits_{j=1,j \ne i}^k \hat{a}_{m_j}.
\end{equation}


% =============================================================================
\section{Restricted basis field}
% =============================================================================

Multimode fields are described by operators $\Psiop_j^{\dagger}(\xvec)$ and $\Psiop_j(\xvec)$,
where $\Psiop_j^{\dagger}(\xvec)$ creates a bosonic atom of spin $j$ at location $\xvec$,
and $\Psiop_j(\xvec)$ destroys one;
the commutators are
\begin{equation}
\label{eqn:multimode-formalism:multimode-commutators}
	[ \Psiop_j(\xvec), \Psiop_k^{\dagger}(\xvec^\prime) ]
	= \delta_{jk} \delta(\xvec-\xvec^\prime).
\end{equation}
Field operators can be decomposed using a single-particle basis \todo{explanation needed?}:
\[
	\Psiop_j(\xvec) = \sum\limits_{\nvec} \phi_{\nvec}(\xvec) \hat{a}_{j,\nvec},
\]
where $\phi_{\nvec}$ is some orthonormal basis,
$\nvec$ is a state vector with $D$ elements.
Single mode operators $\hat{a}_{j,\nvec}$ obey commutation relations \eqnref{multimode-formalism:single-mode-commutators},
the pair $j,\nvec$ serving as a mode identifier.
Orthonormality condition for basis functions is
\[
	\int\limits_A \phi_{\nvec}^*(\xvec) \phi_{\mvec}(\xvec) d\xvec = \delta_{\nvec\mvec},
\]
where the exact nature of integration area $A$ depends on the basis set.
Hereinafter we assume that the integration is always performed over $A$.

Now suppose we want to consider only modes from some subset $L$.
Corresponding projection operator can be written as
\[
	P \equiv \sum\limits_{\nvec \in L} \lvert \nvec \rangle \langle \nvec \rvert,
\]
Or, in coordinate form:
\[
	P [f(\xvec)]
	= \sum\limits_{\nvec \in L} \phi_{\nvec} (\xvec) \int
		d\xvec^\prime\, \phi_{\nvec}^*(\xvec^\prime) f(\xvec^\prime).
\]
Being applied to the field operator $\Psiop_j$, this operator returns the restricted field operator
\[
	P [\Psiop_j]
	= \sum\limits_{\nvec \in L} \phi_{\nvec} (\xvec) \hat{a}_{j,\nvec}
	= \Psiop_{jP} (\xvec),
\]
containing only modes from subset $L$.
If $L$ is the whole mode space, then obviously $P \equiv \mathds{1}$.
In order to simplify equations, we will consider all field operators in this chapter to be restricted and omit the index $P$.

Because of the restricted nature of the operator, commutation relations~\eqnref{multimode-formalism:multimode-commutators} no longer apply.
The following ones should be used instead:
\begin{equation}
\label{eqn:multimode-formalism:restricted-commutators}
\begin{split}
	\left[ \Psiop_j(\xvec), \Psiop_k(\xvec^\prime) \right]
	& = \left[ \Psiop_j^\dagger(\xvec), \Psiop_k^\dagger(\xvec^\prime) \right] = 0, \\
	\left[ \Psiop_j(\xvec), \Psiop_k^\dagger(\xvec^\prime) \right]
	& = \delta_{jk} \delta_P(\xvec - \xvec^\prime),
\end{split}
\end{equation}
where the restricted delta function $\delta_P$ is defined as
\begin{equation}
\label{eqn:multimode-formalism:restricted-delta}
	\delta_P(\xvec - \xvec^\prime)
	= \sum\limits_{\nvec \in L} \phi_{\nvec}^* (\xvec) \phi_{\nvec} (\xvec^\prime).
\end{equation}
Note that conjugation operator swaps variables in $\delta_P$: $\delta_P^*(\xvec - \xvec^\prime) = \delta_P(\xvec^\prime - \xvec)$.

Restricted delta function can be used to rewrite equation for projection operator $P$:
\[
	P [f(\xvec)] = \int d\xvec^\prime \delta_P(\xvec^\prime - \xvec) f(\xvec^\prime).
\]
The Hermitian conjugate of $P$ is thus defined as
\[
	(P [f(\xvec)])^\dagger
	= \int d\xvec^\prime \delta_P^*(\xvec^\prime - \xvec) f^\dagger(\xvec^\prime)
	= P^\dagger [f^\dagger(\xvec)].
\]


% =============================================================================
\section{High order commutators}
% =============================================================================

Let us find the expression for high-order commutators of restricted field operators, analogous to equation~\eqnref{multimode-formalism:single-mode-high-order-commutators} for single-mode operators.
It can be done using the similar recursive procedure.
Given that we know the expression for $\left[ \Psiop, ( \Psiop^{\prime\dagger} )^{l-1} \right]$,
the commutator of higher order can be expanded as
\begin{equation*}
\begin{split}
	\left[ \Psiop, ( \Psiop^{\prime\dagger} )^l \right]
	& = \Psiop ( \Psiop^{\prime\dagger} )^l - ( \Psiop^{\prime\dagger} )^l \Psiop \\
	& = (
		\delta_P (\xvec - \xvec^\prime) + \Psiop^{\prime\dagger} \Psiop
	) ( \Psiop^{\prime\dagger} )^{l-1}
	- ( \Psiop^{\prime\dagger} )^l \Psiop \\
	& = \delta_P (\xvec - \xvec^\prime) ( \Psiop^{\prime\dagger} )^{l-1}
	+ \Psiop^{\prime\dagger} (
		\Psiop ( \Psiop^{\prime\dagger} )^{l-1}
		- ( \Psiop^{\prime\dagger} )^{l-1} \Psiop
	) \\
	& = \delta_P (\xvec - \xvec^\prime) ( \Psiop^{\prime\dagger} )^{l-1}
	+ \Psiop^{\prime\dagger} [
		\Psiop, ( \Psiop^{\prime\dagger} )^{l-1}
	].
\end{split}
\end{equation*}
Now we can get the commutator of any order starting from the known relation~\eqnref{multimode-formalism:restricted-commutators}:
\[
	\left[ \Psiop, ( \Psiop^{\prime\dagger} )^l \right]
	= l \delta_P (\xvec - \xvec^\prime) ( \Psiop^{\prime\dagger} )^{l-1}.
\]
Accompanying conjugated relation:
\[
	\left[ \Psiop^\dagger, ( \Psiop^\prime )^l \right]
	= - l \delta_P^* (\xvec - \xvec^\prime) ( \Psiop^\prime )^{l-1}.
\]

A further generalisation of these relations is
\begin{equation}
\label{eqn:multimode-formalism:functional-commutators}
\begin{split}
	\left[ \Psiop, f( \Psiop^\prime, \Psiop^{\prime\dagger} ) \right]
	& = \delta_P (\xvec - \xvec^\prime) \frac{\partial f}{\partial \Psiop^{\prime\dagger}} \\
	\left[ \Psiop^\dagger, f( \Psiop^\prime, \Psiop^{\prime\dagger} ) \right]
	& = -\delta_P^* (\xvec - \xvec^\prime) \frac{\partial f}{\partial \Psiop^\prime},
\end{split}
\end{equation}
where $f(x, y)$ is a function that can be expanded in the power series of $x$ and $y$.
Let us prove the first relation; the procedure for the second one is the same.
Without loss of generality, we can assume that $f(\Psiop^\prime, \Psiop^{\prime\dagger})$ can be expanded in power series of normally ordered operators (otherwise we can just use commutation relations).
Thus
\begin{equation*}
\begin{split}
	\left[ \Psiop, f( \Psiop^\prime, \Psiop^{\prime\dagger} ) \right]
	& = \sum\limits_{r,s} f_{rs} [ \Psiop, (\Psiop^{\prime\dagger})^r (\Psiop^\prime)^s ] \\
	& = \sum\limits_{r,s} f_{rs} [ \Psiop, (\Psiop^{\prime\dagger})^r ] (\Psiop^\prime)^s \\
	& = \sum\limits_{r,s} f_{rs} r \delta_P(\xvec - \xvec^\prime)
		(\Psiop^{\prime\dagger})^{r-1} (\Psiop^\prime)^s \\
	& = \delta_P (\xvec - \xvec^\prime) \frac{\partial f}{\partial \Psiop^{\prime\dagger}}
\end{split}
\end{equation*}
