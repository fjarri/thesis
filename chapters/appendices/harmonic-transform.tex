\chapter{Transformation to harmonic oscillator basis}
\label{cha:appendix:harmonic-transform}


This chapter briefly describes transformations to and from harmonic basis,
which were explained in detail in~\cite{Dion2003}.


% =============================================================================
\section{One-dimensional oscillator}
% =============================================================================

Consider one-dimensional harmonic oscillator with the Hamiltonian
\[
	H = -\frac{\hbar^2 \nabla^2}{2 m} + \frac{m \omega^2 x^2}{2},
\]
where $m$ is the mass of a particle and $\omega$ is the oscillator frequency.
Characteristic length for this oscillator is
\[
	l_x = \sqrt{\frac{\hbar}{m \omega}}.
\]
Then the eigenfunctions of the Hamiltonian are
\begin{equation}
\label{eqn:harmonic-transform:harmonic-modes}
	\phi_n = \frac{1}{\sqrt{2^n n! l_x} \sqrt[4]{\pi}} H_n(x / l_x)
		\exp \left( -\frac{x^2}{2 l_x^2} \right),
\end{equation}
where $H_n$ is ``physicists'\,'' Hermite polynomial of order $n$.
Corresponding eigenvalues are
\[
	E_n = \hbar \omega (n + \frac{1}{2}).
\]
One can easily check that this set is orthonormal:
\[
	\int\limits_{-\infty}^{\infty} \phi_m(x) \phi_n(x) dx = \delta_{mn}.
\]

Given some function $f(x)$, one can expand it into harmonic oscillator basis as
\[
	C_n = \int f(x) \phi_n(x) dx,
\]
and the backward transformation is, obviously,
\[
	f(x) = \sum_{n} C_n \phi_n(x).
\]
In general, when we do not know anything about $f(x)$,
the value of the integral can be calculated only approximately.
But we can obtain exact results for functions of a certain type,
and by choosing points where we want to sample $f(x)$.

The method is based on a Gauss-Hermite quadrature \todo{citation needed?},
which states that the value of the integral can be approximated as
\[
	\int\limits_{-\infty}^{\infty} g(x) e^{-x^2} dx
	= \sum_{i=1}^N w_i g(r_i) + E,
\]
where $N$ is the number of sample points.
Weights $w_i$ are calculated as
\[
	w_i = \frac{2^{N-1} N! \sqrt{\pi}}{N^2 (H_{N-1}(r_i))^2},
\]
and sample points $r_i$ are roots of Hermite polynomial $H_N$.
The error term is
\[
	E = \frac{N! \sqrt{\pi}}{2^N (2N)!} g^{(2N)}(\xi).
\]
Therefore if $g(x)$ is a polynomial of order $M$,
one can eliminate the error term by choosing $N$ so that $2N \ge M + 1$,
thus making the integration exact.

Now let us say we want to find population of the first $M$ modes for $f(x) = \Psi(x)^s$,
where $\Psi(x) = \sum_{m=0}^{M-1} C_m \phi_m(x)$ and $s$ is a natural number.
This means that wee need to integrate
\[
	F_m = \int \Psi(x)^s \phi_m(x) dx.
\]
By definition of mode functions~\eqnref{harmonic-transform:harmonic-modes},
\[
	\Psi(x)^s \phi_m(x) = P(x / l_x) \exp \left( -\frac{(s+1) x^2}{2 l_x^2} \right),
\]
where $P(x)$ is the polynomial of order less than $(M-1)s + m$.
Since we want to have the same set of sample points for any $m \in [0, M-1]$,
we will consider the worst case $m = M-1$,
which makes the order of $P(x)$ limited by $(M-1)(s+1)$.
The integral can now be transformed to the form necessary to apply Gauss-Hermite quadrature:
\begin{equation*}
\begin{split}
	F_m
	& = \int P(x / l_x) \exp \left( -\frac{(s+1) x^2}{2 l_x^2} \right) dx \\
	& = \int l_x \sqrt{\frac{2}{s+1}} P \left( y \sqrt{\frac{2}{s+1}} \right) e^{-y^2} dy \\
	& = \sum_{i=1}^N w_i P \left( r_i \sqrt{\frac{2}{s+1}} \right) l_x \sqrt{\frac{2}{s+1}} \\
	& = \sum_{i=1}^N w_i
		\Psi \left( l_x r_i \sqrt{\frac{2}{s+1}} \right)^s
		\phi_m \left( l_x r_i \sqrt{\frac{2}{s+1}} \right)
		\exp(r_i^2) l_x \sqrt{\frac{2}{s+1}} \\
	& = \sum_{i=1}^N \tilde{w}_i \phi_m(\tilde{x}_i) f(\tilde{x}_i).
\end{split}
\end{equation*}
Here the sample points are
\[
	\tilde{x}_i = l_x r_i \sqrt{\frac{2}{s+1}},
\]
and modified weights are
\[
	\tilde{w}_i = w_i l_x \sqrt{\frac{2}{s+1}} \exp(r_i^2).
\]
The number of sampling points is determined by the order of $P(x)$:
\[
	N \ge \frac{(M - 1)(s + 1)}{2}.
\]

Since we usually need population for all modes at once,
it is more convenient to use the decomposition in matrix form:
\[
	\bm{F} = \Phi\,\mathrm{diag}(\tilde{\bm{w}}) \bm{f},
\]
where $\Phi_{mi} = \phi_m(\tilde{x}_i)$,
$\tilde{\bm{w}}$ is a vector of elements $\tilde{w}_i$,
and $\bm{f}$ is a vector of elements $f(\tilde{x}_i)$.
Corresponding backward transform is then expressed as
\[
	\bm{f} = \Phi^T \bm{F}.
\]

Note that the same method is applicable for $f(x) = \Psi(x)^a (\Psi^*(x))^b$,
in which case $s = a + b$.
