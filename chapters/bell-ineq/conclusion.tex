% =============================================================================
\section{Conclusion}
% =============================================================================

The results in this chapter show that probabilistic sampling can demonstrate the violation of Bell inequalities, due to the extended range of functions of phase-space variables in a quasiprobability representation.
Generally, both low- and high-order correlations can be obtained.
The low-order correlations are the simplest as the corresponding sampling error does not grow with the system size.
Correlations with order comparable to the system size can also be simulated, but with exponentially large number of samples (although this does not imply exponentially large memory resources).
Universal digital quantum computers, if sufficiently large, are expected to overcome these problems~\cite{Lloyd1996}.
However, these are currently limited in size to 6 qubits or less~\cite{Lanyon2011}.
Regardless of whether these are scalable to larger sizes, it is still useful to obtain theoretical predictions with methods that only require present technology.

Other examples of quantum correlations have been treated using phase-space methods, including quantum soliton dynamics~\cite{Drummond1993a}, interacting quantum fields~\cite{Deuar2007} equivalent to $\sim10^{6}$ qubits and the Dicke superfluorescence model~\cite{Altland2012}.
These simulated correlations are in general agreement with experimental observations~\cite{Jaskula2010}, showing the range of potential applicability of these techniques.
