% =============================================================================
\section{Conclusion}
% =============================================================================

\copypaste{copypaste begins}

Our results show that low order correlations are the simplest to obtain with probabilistic sampling.
Higher order correlations in GHZM states can also be simulated, but with greater difficulty.
While these require exponentially many samples to reduce errors to acceptable levels, they do not require exponentially large memory resources.
Universal digital quantum computers, if sufficiently large, are expected to overcome these problems~\cite{Lloyd1996}.
However, these are currently limited in size to 6 qubits or less~\cite{Lanyon2011}.
Whether or not these are scalable to large sizes, it remains fundamentally useful to obtain theoretical predictions with methods that don't require quantum technology, for scientific reasons.
If we wish to test quantum theory on mesoscopic scales, we cannot assume that our hardware strictly obeys quantum mechanics on these scales.

Other examples of quantum correlations have been treated using phase-space methods, including quantum soliton dynamics~\cite{Drummond1993a}, interacting quantum fields~\cite{Deuar2007} equivalent to $\sim10^{6}$ qubits and the Dicke superfluorescence model~\cite{Altland2012}.
These simulated correlations are in general agreement with experimental observations~\cite{Jaskula2010}, showing the range of potential applicability of these techniques.

In summary, probabilistic digital algorithms can simulate mesoscopic quantum paradoxes up to sizes much larger than those found in current experiments.
Such technologies are proposed for quantum secret-sharing~\cite{Hillery1999} and high-precision atom interferometers~\cite{He2011}, amongst others.
Digital simulations therefore have a direct application both to fundamental physics, and to the design and implementation of these devices.
Our demonstration that mesoscopic Bell violations can be treated probabilistically will lead to further developments.
The fact that our simulations can be exponentially faster than direct measurements is a surprising consequence of the classical parallelism inherent in this computational strategy.

\copypaste{copypaste ends}
