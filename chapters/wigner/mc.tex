% =============================================================================
\section{Multi-component functional Wigner representation}
% =============================================================================

The definition of Wigner transformation, along with \thmref{wigner:func:correspondences} and \thmref{wigner:func:moments} can be further extended to the case of several components.

\begin{definition}
	\begin{eqn*}
		\int \delta^2 \bLambda \equiv \int \delta^2 \Lambda_1 \ldots \int \delta^2 \Lambda_C
	\end{eqn*}
\end{definition}

\begin{definition}
	\begin{eqn*}
		& \hat{D}_j :: \mathbb{F}_{\restbasis_j} \rightarrow \mathbb{H}_{\restbasis_j} \\
		& \hat{D}_j[\Lambda, \Lambda^*] = \exp \int d\xvec \left(
			\Lambda \Psiop_j^\dagger - \Lambda^* \Psiop_j
		\right),
	\end{eqn*}
\end{definition}

\begin{definition}
\label{def:wigner:mc:w-transformation}
	Multi-component functional Wigner transformation $\mathcal{W}$ is defined as
	\begin{eqn*}
		& \mathcal{W} :: \left( \mathbb{R}^D \rightarrow \prod_{j=1}^C \mathbb{H}_{\restbasis_j} \right)
			\rightarrow \prod_{j=1}^C \mathbb{F}_{\restbasis_j}
			\rightarrow \mathbb{C} \\
		& \mathcal{W}[\hat{A}]
		= \frac{1}{\pi^{2 \sum|\restbasis_j|}} \int \delta^2 \bLambda
			\left( \prod_{j=1}^C D[\Lambda_j, \Lambda_j^*, \Psi_j, \Psi_j^*] \right)
			\Trace{ \hat{A} \prod_{j=1}^C \hat{D}_j[\Lambda_j, \Lambda_j^*] },
	\end{eqn*}
	where $\Lambda_j \in \mathbb{F}_{\restbasis_j}$.
	It transforms an operator $\hat{A}$ on a restricted subset of a Hilbert space to a functional $(\mathcal{W}[\hat{A}])[\bPsi, \bPsi^*]$.
	It can be proved, same as in \thmref{mm-wigner:sm:w-real} for the single-mode case, that $\mathcal{W}[\hat{A}]$ is real if $\hat{A}$ is Hermitian.
	Corresponding Weyl transformation is
	\begin{eqn*}
		\mathcal{W}^{-1}[F]
		= \frac{1}{\pi^{\sum |\restbasis_j|}} \int \delta^2 \bXi
			\left( \prod_{j=1}^C \hat{D}_j^{\dagger}[\Xi_j, \Xi_j^*] \right)
			\int \delta^2 \bPhi
				\left( \prod_{j=1}^C D[\Phi_j, \Phi_j^*, \Xi_j, \Xi_j^*] \right)
				F[\bPhi, \bPhi^*].
	\end{eqn*}
\end{definition}

\begin{definition}
\label{def:wigner:mc:w-functional}
	The Wigner functional is
	\begin{eqn*}
		& W :: \prod_{j=1}^C \mathbb{F}_{\restbasis_j} \rightarrow \mathbb{R} \\
		& W [\bPsi, \bPsi^*]
		\equiv \mathcal{W}[\hat{\rho}]
		= \frac{1}{\pi^{2 \sum|\restbasis_j|}} \int \delta^2 \bLambda
			\left( \prod_{j=1}^C D[\Lambda_j, \Lambda_j^*, \Psi_j, \Psi_j^*] \right)
			\chi_W [\bLambda, \bLambda^*],
	\end{eqn*}
	where $\chi_W [\Lambda, \Lambda^*]$ is the characteristic functional
	\begin{eqn*}
		& \chi_W :: \prod_{j=1}^C \mathbb{F}_{\restbasis_j} \rightarrow \mathbb{C} \\
		& \chi_W [\bLambda, \bLambda^*]
		= \Trace{ \hat{\rho} \prod_{j=1}^C \hat{D}_j[\Lambda_j, \Lambda_j^*] }.
	\end{eqn*}
\end{definition}

\begin{theorem}[Multi-component extension of \thmref{wigner:func:correspondences}]
\label{thm:wigner:mc:correspondences}
	\begin{eqn*}
		\mathcal{W} [ \Psiop_j \hat{A} ]
			& = \left( \Psi_j + \frac{1}{2} \frac{\delta}{\delta \Psi_j^*} \right) \mathcal{W}[\hat{A}],
		\quad
		\mathcal{W} [ \Psiop_j^\dagger \hat{A} ]
			= \left( \Psi_j^* - \frac{1}{2} \frac{\delta}{\delta \Psi_j} \right) \mathcal{W}[\hat{A}], \\
		\mathcal{W} [ \hat{A} \Psiop_j ]
			& = \left( \Psi_j - \frac{1}{2} \frac{\delta}{\delta \Psi_j^*} \right) \mathcal{W}[\hat{A}],
		\quad
		\mathcal{W} [ \hat{A} \Psiop_j^\dagger ]
			= \left( \Psi_j^* + \frac{1}{2} \frac{\delta}{\delta \Psi_j} \right) \mathcal{W}[\hat{A}].
	\end{eqn*}
\end{theorem}
\begin{proof}
Proved in exactly the same way as \thmref{wigner:func:correspondences}.
\end{proof}

\begin{theorem}[Multi-component extension of \thmref{wigner:func:moments}]
\label{thm:wigner:mc:moments}
	For any non-negative integer $r_j$, $s_j$
	\begin{eqn*}
		\int \delta^2 \bPsi\,
			\left( \prod_{j=1}^C \Psi_j^{r_j} (\Psi_j^*)^{s_j} \right) W[\bPsi, \bPsi^*]
		= \langle \symprod{ \prod_{j=1}^C \Psiop_j^{r_j} (\Psiop_j^\dagger)^{s_j} } \rangle.
	\end{eqn*}
\end{theorem}
\begin{proof}
Proved in exactly the same way as \thmref{wigner:func:moments}, processing each component successively.
\end{proof}
