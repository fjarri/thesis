% =============================================================================
\chapter{Functional calculus}
\label{cha:appendix:func-calculus}
% =============================================================================

The definitions of differentiation and integration from~\appref{c-numbers} can be extended to operate on functions and functionals.
This proved to be a useful tool in the derivation of the functional Wigner transformation and expressing accompanied results, as it helps encapsulate bases and mode populations inside wave functions and field operators.
It has been introduced (among other places) in most of the papers treating functional extensions of quasiprobability with varying level of detail.
The most extensive description was made by Dalton~\cite{Dalton2011}.
Although, while these papers cover the foundations quite well, they are missing several important results which are essential for this thesis, and which will be therefore proved in this Appendix.


% =============================================================================
\section{Functional spaces and projections}
% =============================================================================

We will assume we a provided with an arbitrary orthonormal basis $\fullbasis$, consisting of functions $\phi_{\nvec}(\xvec)$, where $\xvec \in \mathbb{R}^D$ is a coordinate vector, and $\nvec \in \fullbasis$ is a mode identifier.
The exact nature of a mode identifier is irrelevant; we only require it to have an equivalence relation defined, and be enumerable.
For example, for a three-dimensional harmonic potential, the mode identifier will be a tuple $(k,l,m)$ of three non-negative integers.

Orthonormality and completeness conditions for basis functions are, respectively,
\begin{eqns}
	\int\limits_A \phi_{\nvec}^*(\xvec) \phi_{\mvec}(\xvec) \upd\xvec & = \delta_{\nvec\mvec}, \\
	\sum_{\nvec} \phi_{\nvec}^*(\xvec) \phi_{\nvec}(\xvec^\prime) & = \delta(\xvec^\prime - \xvec),
\end{eqns}
where the integration area $A$ depends on the basis set (for example, $A$ is the whole space for harmonic oscillator modes, and a box for plane waves).
Hereinafter we assume that the integration $\int \upd\xvec$ is always performed over $A$, unless explicitly stated otherwise. In addition, to avoid clutter, for functions of coordinates the argument list will be omitted (i.e. $f(\xvec) \equiv f$ and $f(\xvec^\prime) \equiv f^\prime$) except where it is necessary for clarity.

Various functions can be combined from all, or the certain subset of basis modes by means of a composition:

\begin{definition}
	For some subset of the full basis $\restbasis \subseteq \fullbasis$, composition transformation creates a function from a vector of mode populations:
	\begin{eqn*}
		\mathcal{C}_{\restbasis}(\balpha)
		= \sum_{\nvec \in \restbasis} \phi_{\nvec} \alpha_{\nvec}.
	\end{eqn*}
	Decomposition transformation, correspondingly, creates a vector of populations out of a function:
	\begin{eqn*}
		(\mathcal{C}_{\restbasis}^{-1}[f])_{\nvec}
		= \int \upd\xvec \phi_{\nvec}^* f,\,{\nvec} \in \restbasis.
	\end{eqn*}
\end{definition}

For any subset of the full basis, there is a certain subspace of functions that can be obtained by composing modes from this subset only.

\begin{definition}
	For some subset of the full basis $\restbasis \subseteq \fullbasis$, $\mathbb{F}_{\restbasis} \equiv (\mathbb{R}^D \rightarrow \mathbb{C})_{\restbasis}$ is a space of all functions of coordinates, which can be obtained from $\mathcal{C}_{\restbasis}$.
	In other words, such functions consist only of modes from $\restbasis$.
	We denote $\mathbb{F}_{\fullbasis} \equiv \mathbb{F}$.
\end{definition}

Using this definition, the type of composition and decomposition transformations can be written as
\begin{eqn}
		\mathcal{C}_{\restbasis} \in \mathbb{C}^{|\restbasis|} \rightarrow \mathbb{F}_{\restbasis},\quad
		\mathcal{C}_{\restbasis}^{-1} \in \mathbb{F} \rightarrow \mathbb{C}^{|\restbasis|}
\end{eqn}
Note also that for any $f \in \mathbb{F}_{\restbasis}$ corresponding composition and decomposition are reversible, i.e. $\mathcal{C}_{\restbasis}(\mathcal{C}_{\restbasis}^{-1}[f]) \equiv f$.
We will refer such functions as restricted functions.

The result of any non-linear transformation of a function $f \in \mathbb{F}_{\restbasis}$ is not guaranteed to belong to $\mathbb{F}_{\restbasis}$ and requires explicit projection to be used with other restricted functions from the same subspace $\mathbb{F}_{\restbasis}$:

\begin{definition}
\label{def:func-calculus:projector}
	An arbitrary function can be projected to $\mathbb{F}_{\restbasis}$ using the projection transformation:
	\begin{eqn*}
		& \proj{\restbasis} \in \mathbb{F} \rightarrow \mathbb{F}_{\restbasis} \\
		& \proj{\restbasis}[f](\xvec)
		\equiv (\mathcal{C}_{\restbasis}(\mathcal{C}_{\restbasis}^{-1}[f])) (\xvec)
		= \sum_{\nvec \in \restbasis} \phi_{\nvec} \int
			d\xvec^\prime\, \phi_{\nvec}^{\prime*} f^\prime.
	\end{eqn*}
	Obviously, $\proj{\fullbasis} \equiv \mathds{1}$.
\end{definition}

Being applied to the delta function, the projection function produces a restricted delta function:

\begin{definition}
\label{def:func-calculus:restricted-delta}
	The restricted delta function $\delta_{\restbasis} \in \mathbb{F}_{\restbasis}$ is defined as
	\begin{eqn*}
		\delta_{\restbasis}(\xvec^\prime, \xvec)
		= \proj{\restbasis}[\delta]
		= \sum_{\nvec \in \restbasis} \phi_{\nvec}^{\prime*} \phi_{\nvec}.
	\end{eqn*}
	Note that in general $\delta_{\restbasis}^*(\xvec^\prime, \xvec) = \delta_{\restbasis}(\xvec, \xvec^\prime)$, so the order of variables is important.
	For a full basis, this definition coincides with the standard delta function: $\delta_{\fullbasis}(\xvec^\prime, \xvec) \equiv \delta(\xvec^\prime - \xvec)$.
\end{definition}

Projection transformation can be, in turn, expressed using the restricted delta function:
\begin{eqn}
	\proj{\restbasis}[f](\xvec) = \int d\xvec^\prime \delta_{\restbasis}(\xvec^\prime, \xvec) f^\prime.
\end{eqn}
The conjugate of $\proj{\restbasis}$ is therefore
\begin{eqn}
	(\proj{\restbasis}[f](\xvec))^*
	= \int d\xvec^\prime \delta_{\restbasis}^*(\xvec^\prime, \xvec) f^{\prime*}
	= \proj{\restbasis}^* [f^*](\xvec).
\end{eqn}


% =============================================================================
\section{Functional differentiation}
% =============================================================================

First we must introduce some operations on functions, which will replace common differentials and integrals used in single and multi-mode cases

Let $\mathcal{F}[f] :: \mathbb{F}_{\restbasis} \rightarrow \mathbb{F}$ be some transformation (note that the result is not guaranteed to belong to the restricted basis).
Because of the bijection between $\mathbb{F}_{\restbasis}$ and $\mathbb{C}^{|\restbasis|}$, $\mathcal{F}$ can be alternatively treated as a function of a vector of complex numbers:
\begin{eqn}
	& \mathcal{F} :: \mathbb{C}^{|\restbasis|} \rightarrow \mathbb{C}^\infty \\
	& \mathcal{F}(\balpha) \equiv \mathcal{C}_{\restbasis}^{-1}[\mathcal{F}[\mathcal{C}_{\restbasis}(\balpha)]].
\end{eqn}
Using this correspondence, we can define the functional differentiation.

\begin{definition}
\label{def:func-calculus:func-diff}
	Functional derivative is defined as
	\begin{eqn*}
		& \frac{\delta}{\delta f^\prime} ::
		\left(
			\mathbb{F}_{\restbasis} \rightarrow \mathbb{F}
		\right)
		\rightarrow
		\left(
			\mathbb{R}^D \rightarrow \mathbb{F}_{\restbasis} \rightarrow \mathbb{F}
		\right) \\
		& \frac{\delta \mathcal{F}[f]}{\delta f^\prime}
		= \sum_{\nvec \in \restbasis} \phi_{\nvec}^{\prime*}
			\frac{\partial \mathcal{F}(\balpha)}{\partial \alpha_{\nvec}}.
	\end{eqn*}
\end{definition}

Note that the transformation being returned differs from the one which was taken:
the result of new transformation is a function depending on two variables from $\mathbb{R}^D$, not one.
The second variable comes from the function we are differentiating by.

Functional derivative definition behaves in many ways similar to common derivative.
\begin{lemma}
	Functional differentiation from~\defref{func-calculus:func-diff} obeys sum, product, quotient, and chain differentiation rules.
\end{lemma}
\begin{proof}
\todo{Sum, product and quotient are more or less obvious; but should we prove chain differentiation?}
\end{proof}

\begin{lemma}
	If $g(t)$ is a function that can be expanded into power series, and functional $\mathcal{F}[f] \equiv g(f)$, $\mathcal{F} \in \mathbb{F}_{\restbasis} \rightarrow \mathbb{F}$, then
	\begin{eqn*}
		\frac{\delta \mathcal{F}[f]}{\delta f(\xvec^\prime)} (\xvec)
		= \delta_{\restbasis}(\xvec^\prime - \xvec)
			\left. \frac{\partial g(t)}{\partial t} \right|_{t = f(\xvec)}
	\end{eqn*}
\end{lemma}
\begin{proof}
We will consider $g(t) = t^k$ case first, which will straightforwardly lead to the statement of the lemma.
For $k = 1$, obviously,
\begin{eqn}
	\frac{\delta f}{\delta f(\xvec^\prime)} (\xvec)
	= \delta_{\restbasis}(\xvec^\prime, \xvec)
\end{eqn}
Then for other values of $k$:
\begin{eqn}
	\frac{\delta \mathcal{F}[f]}{\delta f(\xvec^\prime)} (\xvec)
	& = \frac{\delta f^k}{\delta f(\xvec^\prime)} (\xvec)
	= \sum_{\nvec \in \restbasis} \phi_{\nvec}^{\prime*}
		\frac{\partial f^k}{\partial \alpha_{\nvec}} \\
	& = \sum_{\nvec \in \restbasis} \phi_{\nvec}^{\prime*}
		\frac{\partial f^k}{\partial f}
		\frac{\partial f}{\partial \alpha_{\nvec}}
	= k f^{k-1}
		\sum_{\nvec \in \restbasis} \phi_{\nvec}^{\prime*}
		\frac{\partial f}{\partial \alpha_{\nvec}} \\
	& = k \delta_{\restbasis}(\xvec^\prime, \xvec) f^{k-1}(\xvec)
	= \delta_{\restbasis}(\xvec^\prime, \xvec)
		\left. \frac{\partial t^k}{\partial t} \right|_{t = f(\xvec)}.
	\qedhere
\end{eqn}
\end{proof}

\begin{lemma}
	If $g(z, z^*)$ can be expanded into series of $z^n (z^*)^m$, and functional $\mathcal{F}[f, f^*] \equiv g(f, f^*)$, $\mathcal{F} \in \mathbb{F}_{\restbasis} \rightarrow \mathbb{F}$, then $\delta \mathcal{F} / \delta f^\prime$ and $\delta \mathcal{F} / \delta f^{\prime*}$ can be treated as a partial differentiation of the functional of two independent variables $f$ and $f^*$.
	In other words:
	\begin{eqn*}
		\frac{\delta \mathcal{F}}{\delta f^\prime}
		= \delta_P(\xvec^\prime, \xvec) \left.
			\frac{\partial g(z, z^*)}{\partial z}
		\right|_{z=f(x)},
		\quad
		\frac{\delta \mathcal{F}}{\delta f^{\prime*}}
		= \delta_P^*(\xvec^\prime, \xvec) \left.
			\frac{\partial g(z, z^*)}{\partial z^*}
		\right|_{z=f(x)}
	\end{eqn*}
\end{lemma}
\begin{proof}
Proof is similar to \thmref{c-numbers:independent-vars}.
\end{proof}


% =============================================================================
\section{Functional integration}
% =============================================================================

Functional integration is defined as

\begin{definition}
	\begin{eqn*}
		& \int \delta^2 f :: (\mathbb{F}_{\restbasis} \rightarrow \mathbb{F}) \rightarrow \mathbb{C} \\
		& \int \delta^2 f \mathcal{F}[f]
		= \int d^2\balpha \mathcal{F}(\balpha)
		= \left(
			\prod_{\nvec \in \restbasis} \int d^2\alpha_{\nvec}
		\right) \mathcal{F}(\balpha),
	\end{eqn*}
	where the product of integrals stands for their successive application.
    If the basis contains an infinite number of modes, the integral is treated as a limit $|\restbasis| \rightarrow \infty$.
	\todo{\cite{Dalton2011} has detailed explanation, do we need it here?}
\end{definition}

Functional integration has the Fourier-like property analogous to \lmmref{c-numbers:fourier-of-moments}, but its statement requires the definition of the delta functional:

\begin{definition}
\label{def:func-calculus:delta-functional}
	For a function $\Lambda \in \mathbb{F}_{\restbasis}$ the delta functional is
	\begin{eqn*}
		\Delta_{\restbasis}[\Lambda]
		\equiv \prod_{\nvec \in \restbasis} \delta(\Real \lambda_{\nvec}) \delta(\Imag \lambda_{\nvec}),
	\end{eqn*}
	where $\blambda = \mathcal{C}_{\restbasis}^{-1}[\Lambda]$.
\end{definition}

The delta functional has the same property as the common delta function:
\begin{eqn}
	\int \delta^2 \Lambda \mathcal{F}[\Lambda] \Delta_{\restbasis}[\Lambda]
	& = \left(
			\prod_{\nvec \in \restbasis} \int d^2\lambda_{\nvec}
		\right)
		\mathcal{F}(\blambda)
		\prod_{\nvec \in \restbasis} \delta(\Real \lambda_{\nvec}) \delta(\Imag \lambda_{\nvec}) \\
	& = \left. \mathcal{F}(\blambda) \right|_{\forall \nvec \in \restbasis\, \lambda_{\nvec} = 0} \\
	& = \left. \mathcal{F}[\Lambda] \right|_{\Lambda \equiv 0}
\end{eqn}

\begin{lemma}[Functional extension of \lmmref{c-numbers:fourier-of-moments}]
\label{lmm:func-calculus:fourier-of-moments}
	For $\Psi \in \mathbb{F}_{\restbasis}$ and $\Lambda \in \mathbb{F}_{\restbasis}$, and for any non-negative integers $r$ and $s$:
	\begin{eqn*}
		\int \delta^2\Psi\, \Psi^r (\Psi^*)^s \exp
			\int d\xvec \left( -\Lambda \Psi^* + \Lambda^* \Psi \right)
		= \pi^{2|\restbasis|}
			\left( -\frac{\delta}{\delta \Lambda^*} \right)^r
			\left( \frac{\delta}{\delta \Lambda} \right)^s
			\Delta_{\restbasis}[\Lambda]
	\end{eqn*}
\end{lemma}
\begin{proof}
\begin{eqn}
	& \int \delta^2\Psi\, \Psi^r (\Psi^*)^s \exp
		\int d\xvec \left( -\Lambda \Psi^* + \Lambda^* \Psi \right) \\
	& = \left(
			\prod_{\nvec \in \restbasis} \int d^2\alpha_{\nvec}
		\right)
		\left( \sum_{\nvec \in \restbasis} \phi_{\nvec} \alpha_{\nvec} \right)^r
		\left( \sum_{\nvec \in \restbasis} \phi^*_{\nvec} \alpha_{\nvec}^* \right)^s
		\prod_{\nvec \in \restbasis} \exp(-\lambda_{\nvec} \alpha_{\nvec}^* + \lambda_{\nvec}^* \alpha_{\nvec}).
\end{eqn}
Expanding powers of $\Psi$ and $\Psi^*$ using multinomial theorem:
\begin{eqn2}
	& ={} && \left(
			\prod_{\nvec \in \restbasis} \int d^2\alpha_{\nvec}
		\right)
		\left(
			\sum_{\sum u_{\mvec} = r} \binom{r}{ \left\{ u_{\mvec} \right\} }
			\prod_{\nvec \in \restbasis} \phi_{\nvec}^{u_{\nvec}} \alpha_{\nvec}^{u_{\nvec}}
		\right) \\
	& && \left(
			\sum_{\sum v_{\mvec} = s} \binom{s}{ \left\{ v_{\mvec} \right\} }
			\prod_{\nvec \in \restbasis} (\phi_{\nvec}^*)^{v_{\nvec}} (\alpha_{\nvec}^*)^{v_{\nvec}}
		\right)
		\prod_{\nvec \in \restbasis} \exp(-\lambda_{\nvec} \alpha_{\nvec}^* + \lambda_{\nvec}^* \alpha_{\nvec}),
\end{eqn2}
where $\binom{r}{ \left\{ u_{\mvec} \right\} } \equiv r! / (\prod u_{\mvec}!)$ are multinomial coefficients.
Splitting variables:
\begin{eqn2}
	& ={} && \sum_{ \sum u_{\mvec} = r,\, \sum v_{\mvec} = s }
		\binom{r}{ \left\{ u_{\mvec} \right\} }
		\binom{s}{ \left\{ v_{\mvec} \right\} } \\
	& && \prod_{\nvec \in \restbasis}
			\phi_{\nvec}^{u_{\nvec}} (\phi_{\nvec}^*)^{v_{\nvec}}
			\int d^2\alpha_{\nvec}
				\alpha_{\nvec}^{u_{\nvec}}
				(\alpha_{\nvec}^*)^{v_{\nvec}}
				\exp(-\lambda_{\nvec} \alpha_{\nvec}^* + \lambda_{\nvec}^* \alpha_{\nvec}).
\end{eqn2}
Applying \lmmref{c-numbers:fourier-of-moments}, collapsing sums, and recognizing \defref{func-calculus:func-diff} and \defref{func-calculus:delta-functional}:
\begin{eqn2}
	& ={} && \sum_{\sum u_{\mvec} = r,\, \sum v_{\mvec} = s}
		\binom{r}{ \left\{ u_{\mvec} \right\} }
		\binom{s}{ \left\{ v_{\mvec} \right\} }
		\pi^{2|\restbasis|} \\
	& && \prod_{\nvec \in \restbasis}
			\phi_{\nvec}^{u_{\nvec}} (\phi_{\nvec}^*)^{v_{\nvec}}
			\left( -\frac{\partial}{\partial \lambda_{\nvec}^*} \right)^{u_{\nvec}}
			\left( \frac{\partial}{\partial \lambda_{\nvec}} \right)^{v_{\nvec}}
			\delta(\Real \lambda_{\nvec}) \delta(\Imag \lambda_{\nvec}) \\
	& ={} && \pi^{2|\restbasis|}
		\left( -\sum_{\nvec \in \restbasis} \phi_{\nvec} \frac{\partial}{\partial \lambda_{\nvec}^*} \right)^r
		\left( \sum_{\nvec \in \restbasis} \phi_{\nvec}^* \frac{\partial}{\partial \lambda_{\nvec}} \right)^s
		\prod_{\nvec \in \restbasis} \delta(\Real \lambda_{\nvec}) \delta(\Imag \lambda_{\nvec}) \\
	& ={} && \pi^{2|\restbasis|}
		\left( -\frac{\delta}{\delta \Lambda^*} \right)^r
		\left( \frac{\delta}{\delta \Lambda} \right)^s
		\Delta_{\restbasis}[\Lambda]
	\qedhere
\end{eqn2}
\end{proof}

\begin{definition}
	Displacement functional is
	\begin{eqn}
		& D :: \mathbb{F}_{\restbasis} \rightarrow \mathbb{F}_{\restbasis} \rightarrow \mathbb{C} \\
		& D[\Lambda, \Lambda^*, \Psi, \Psi^*] = \exp \int d\xvec \left(
			-\Lambda \Psi^* + \Lambda^* \Psi
		\right).
	\end{eqn}
\end{definition}

\begin{lemma}[Functional extension of \lmmref{c-numbers:zero-integrals}]
\label{lmm:func-calculus:zero-integrals}
	For a bounded functional $F(\blambda, \blambda^*)$
	\begin{eqn*}
		\int \delta^2\Lambda
			\frac{\delta}{\delta \Lambda^\prime} \left(
				D[\Lambda, \Lambda^*, \Psi, \Psi^*]
				F[\Lambda, \Lambda^*]
			\right)
		& = 0 \\
		\int \delta^2\Lambda
			\frac{\delta}{\delta \Lambda^{\prime*}}
			\left(
				D[\Lambda, \Lambda^*, \Psi, \Psi^*]
				F[\Lambda, \Lambda^*]
			\right)
		& = 0.
	\end{eqn*}
\end{lemma}
\begin{proof}
We will prove the first equation.
Let $\Lambda = \mathcal{C}_{\restbasis}(\blambda)$ and $\Psi = \mathcal{C}_{\restbasis}(\balpha)$.
Displacement functional can be represented as a function of mode vector:
\begin{eqn}
	D[\Lambda, \Lambda^*, \Psi, \Psi^*]
	& = \exp \int dx \sum_{\nvec \in \restbasis,\mvec \in \restbasis} \left(
		- \phi_{\nvec} \phi_{\mvec}^* \lambda_{\nvec} \alpha_{\mvec}^*
		+ \phi_{\nvec}^* \phi_{\mvec} \lambda_{\nvec}^* \alpha_{\mvec}
	\right) \\
	& = \exp \sum_{\nvec \in \restbasis,\mvec \in \restbasis} \left(
		- \delta_{\nvec \mvec} \lambda_{\nvec} \alpha_{\nvec}^*
		+ \delta_{\nvec \mvec} \lambda_{\nvec}^* \alpha_{\nvec}
	\right) \\
	& = \exp \sum_{\nvec \in \restbasis} \left(
		-\lambda_{\nvec} \alpha_{\nvec}^* + \lambda_{\nvec}^* \alpha_{\nvec}
	\right).
\end{eqn}

We introduce the special notation for this lemma to indicate the subset of $\restbasis$ used by operators and functionals.
With this notation, for fixed $\nvec$:
\begin{eqn}
	D[\Lambda, \Lambda^*, \Psi, \Psi^*]
	& = \prod_{\mvec \in \restbasis} \exp \left(
		- \lambda_{\mvec} \alpha_{\mvec}^* + \lambda_{\mvec}^* \alpha_{\mvec}
	\right) \\
	& = \exp \left(
		- \lambda_{\nvec} \alpha_{\nvec}^* + \lambda_{\nvec}^* \alpha_{\nvec}
	\right)
	\prod_{\mvec \in \restbasis, \mvec \ne \nvec} \exp \left(
		- \lambda_{\mvec} \alpha_{\mvec}^* + \lambda_{\mvec}^* \alpha_{\mvec}
	\right) \\
	& = D_{\lnot \nvec} D_{\nvec},
\end{eqn}
and, similarly,
\begin{eqn}
	\Lambda & = \Lambda_{\lnot \nvec} + \Lambda_{\nvec}, \\
	\int d^2 \blambda & = \int d^2 \blambda_{\lnot \nvec} \int d^2 \lambda_{\nvec}.
\end{eqn}

With this notation:
\begin{eqn}
	& \int \delta^2\Lambda
		\frac{\delta}{\delta \Lambda^\prime} \left(
			D[\Lambda, \Lambda^*, \Psi, \Psi^*]
			F[\Lambda, \Lambda^*]
		\right) \\
	& = \int d^2 \blambda
		\sum_{\nvec \in \restbasis} \phi_{\nvec}^{\prime*} \frac{\partial}{\partial \lambda_{\nvec}}
			D_{\lnot \nvec} D_{\nvec}
			F[\Lambda, \Lambda^*] \\
	& = \sum_{\nvec \in \restbasis} \phi_{\nvec}^{\prime*}
		\int d^2 \blambda_{\lnot \nvec} D_{\lnot \nvec}
		\int d^2 \lambda_{\nvec} \frac{\partial}{\partial \lambda_{\nvec}} D_{\nvec} F(\blambda, \blambda^*).
\end{eqn}
For each term the internal is equal to zero because of \lmmref{c-numbers:zero-integrals}, therefore the whole sum is zero.
\end{proof}

\begin{lemma}[Functional extension of \lmmref{c-numbers:zero-delta-integrals}]
\label{lmm:func-calculus:zero-delta-integrals}
	For $\Lambda \in \mathbb{F}_{\restbasis}$ \todo{Again, any limitations on $F$?}
	\begin{eqn*}
		\int \delta^2\Lambda
			\frac{\delta}{\delta \Lambda} \left(
				\left(
					\left( \frac{\delta}{\delta \Lambda} \right)^s
					\left( -\frac{\delta}{\delta \Lambda^*} \right)^r
					\Delta_{\restbasis}[\Lambda]
				\right)
				F[\lambda, \lambda^*]
			\right)
		& = 0 \\
		\int \delta^2\Lambda
			\frac{\delta}{\delta \Lambda^*} \left(
				\left(
					\left( \frac{\delta}{\delta \Lambda} \right)^s
					\left( -\frac{\delta}{\delta \Lambda^*} \right)^r
					\Delta_{\restbasis}[\Lambda]
				\right)
				F[\lambda, \lambda^*]
			\right)
		& = 0 \\
	\end{eqn*}
\end{lemma}
\begin{proof}
Proved by expanding functional integration and differentials into modes and recognizing \lmmref{c-numbers:zero-delta-integrals}.
\end{proof}

In order to perform transformations of master equations in the future, we will need a lemma, which justifies certain operation with Laplacian (which is a part of kinetic term in Hamiltonian).

\begin{lemma}
\label{lmm:func-calculus:move-laplacian}
	If $\forall \nvec \in \restbasis\, \xvec \in \partial A$ $\phi_n(\xvec) = 0$, then for any $\mathcal{F} \in \mathbb{F}_{\restbasis} \rightarrow \mathbb{F}$
	\begin{eqn*}
		\int\limits_A d\xvec \left(
			\nabla^2 \frac{\delta}{\delta \Psi}
		\right) \Psi \mathcal{F}[\Psi, \Psi^*]
		= \int\limits_A d\xvec \frac{\delta}{\delta \Psi}
		( \nabla^2 \Psi ) \mathcal{F}[\Psi, \Psi^*]
	\end{eqn*}
\end{lemma}
\begin{proof}
Integration limits play an important role in this proof, so we will write them explicitly.
\begin{eqn}
	\int\limits_A d\xvec \left(
		\nabla^2 \frac{\delta}{\delta \Psi}
	\right) \Psi
	= \sum_{\nvec \in \restbasis, \mvec \in \restbasis} \left(
			\int\limits_A d\xvec ( \nabla^2 \phi_{\nvec}^* ) \phi_{\mvec}
		\right)
		\frac{\partial}{\partial \alpha_{\nvec}} \alpha_{\mvec} \mathcal{F}(\mathbf{\alpha})
	= (*)
\end{eqn}
Using Green's first identity and the fact that eigenfunctions are equal to zero at the boundary of $A$:
\begin{eqn}
	\int\limits_A d\xvec ( \nabla^2 \phi_{\nvec}^* ) \phi_{\mvec}
	& = \oint\limits_{\partial A} \phi_{\mvec} (\nabla \phi_{\nvec}^* \cdot \mathbf{v}) dS
	- \int\limits_A d\xvec ( \nabla \phi_{\nvec}^* ) ( \nabla \phi_{\mvec} ) \\
	& = 0 - \int\limits_A d\xvec ( \nabla \phi_{\nvec}^* ) ( \nabla \phi_{\mvec} ) \\
	& = \oint\limits_{\partial A} \phi_{\nvec}^* (\nabla \phi_{\mvec} \cdot \mathbf{v}) dS
	- \int\limits_A d\xvec ( \nabla \phi_{\nvec}^* ) ( \nabla \phi_{\mvec} ) \\
	& = \int\limits_A d\xvec \phi_{\nvec}^* ( \nabla^2 \phi_{\mvec} ),
\end{eqn}
where $\mathbf{v}$ is the outward pointing unit normal of surface element $dS$.
Thus
\begin{eqn}
	(*)
	= \sum_{\nvec \in \restbasis, \mvec \in \restbasis} \left(
			\int\limits_A d\xvec \phi_{\nvec}^* ( \nabla^2 \phi_{\mvec} )
		\right)
		\frac{\partial}{\partial \alpha_{\nvec}} \alpha_{\mvec} \mathcal{F}(\mathbf{\alpha})
	= \int\limits_A d\xvec \frac{\delta}{\delta \Psi}
		( \nabla^2 \Psi ) \mathcal{F}[\Psi, \Psi^*].
	\qedhere
\end{eqn}
\end{proof}

Note that this lemma imposes additional requirement for basis functions, but in practical applications it is always satisfied.
For example, in plane wave basis eigenfunctions are equal to zero at the border of the bounding box, and in harmonic oscillator basis they are equal to zero on the infinity (which can be considered the boundary of their integration area).
Hereinafter we will assume that this condition is true for any basis we work with.
