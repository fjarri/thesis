% =============================================================================
\chapter{Wirtinger differentiation}
\label{cha:appendix:c-numbers}
% =============================================================================

Formally, a function of complex variable has to be holomorphic in order to be complex differentiable.
In many cases it is enough to have less strict ``physicists'\,'' complex differentiation rules.
These rules were developed by Wirtinger in~\cite{Wirtinger1927};
a very good review and a thorough description of their application can be found in~\cite{Kreutz-Delgado2009}.
Further extension of these rules to vectors and matrices is done in~\cite{Hjorungnes2007}.
This section will outline these rules and provide some lemmas based on them, which will be used further.

\begin{definition}
\label{def:c-numbers:wirtinger}
	For a complex variable $z = x + iy$ and a function $f(z) = u(x, y) + iv(x, y)$
	\begin{eqn*}
		\left( \frac{\partial f(z)}{\partial z} \right)
		= \frac{1}{2} \left(
			\frac{\partial f}{\partial x} - i \frac{\partial f}{\partial y}
		\right).
	\end{eqn*}
\end{definition}

\begin{lemma}
	If $f(z)$ is holomorphic, then the Wirtinger differentiation is equivalent to the standard definition.
\end{lemma}

\begin{lemma}
	For any ``good'' (even non-holomorphic) $f(z)$, Wirtinger differentiation obeys sum, product, quotient, and chain differentiation rules
	(the former one is applied as if $f(z) \equiv f(z, z^*)$).
\end{lemma}

Hereinafter we will use Wirtinger differentiation unless explicitly stated otherwise,
because some important functions we will encounter are not holomorphic.
This differentiation has all intuitively assumed properties, along with some not quite obvious ones.

\begin{lemma}
	For any nonnegative integers $a$ and $b$.
	\begin{eqn*}
		\frac{\partial}{\partial z} (z^a (z^*)^b) = a z^{a-1} (z^*)^b,
		\quad
		\frac{\partial}{\partial z^*} (z^a (z^*)^b) = b z^a (z^*)^{b-1},
	\end{eqn*}
\end{lemma}
\begin{proof}
Let us assume that the statement of the lemma is valid for some $a$ and $b$, then using product rule
\begin{eqn}
	\frac{\partial}{\partial z} (z^{a+1} (z^*)^b)
	& = \frac{\partial}{\partial z} (z z^a (z^*)^b)
	= z^a (z^*)^b + z \frac{\partial}{\partial z} (z^a (z^*)^b) \\
	& = z^a (z^*)^b + a z z^{a-1} (z^*)^b
	= (a + 1) z^a (z^*)^b.
\end{eqn}
The part for $\partial/\partial z^*$ can is proved in the same way.
One can easily prove (by transition to real values) that $\partial(z z^*)/\partial z = z^*$ and $\partial (z z^*)/\partial z^* = z$.
By induction, the statement is true for any natural $a$ and $b$,
and it is obviously true if $a = 0$ or $b = 0$, which proves the lemma.
\todo{This can be proved for any real $a$ and $b$, if necessary.}
\end{proof}

This is straightforwardly followed by
\begin{lemma}
\label{lmm:c-numbers:independent-vars}
	If $f(z)$ can be expanded into series of $z^n (z^*)^m$, $\partial f(z)/\partial z$ and $\partial f(z)/\partial z^*$ can be treated as partial differentiation of the function of two independent variables $z$ and $z^*$.
	In other words:
	\begin{eqn*}
		\frac{\partial}{\partial z} f(z, z^*) \equiv \frac{\partial}{\partial u} f(u, v),
		\quad
		\frac{\partial}{\partial z^*} f(z, z^*) \equiv \frac{\partial}{\partial v} f(u, v).
	\end{eqn*}
\end{lemma}

Therefore, although technically $f(z, z^*)$ depends on one complex variable, when differentiated it behaves as if it was a function of two complex variables.
Now we can prove two lemmas which will help us deal with some integrals.

\begin{definition}
	For a complex variable $z = x + iy$ the integral
	\begin{eqn*}
		\int d^2 z \equiv \int_{-\infty}^{\infty} \int_{-\infty}^{\infty} dx\, dy,
	\end{eqn*}
	or, in other words, stands for the two-dimensional integral over the complex plane.
\end{definition}

\begin{lemma}
\label{lmm:c-numbers:fourier-of-moments}
	For complex $\lambda$ and any non-negative integers $r$ and $s$:
	\begin{eqn*}
		\int d^2\alpha\, \alpha^r (\alpha^*)^s \exp(-\lambda \alpha^* + \lambda^* \alpha)
		= \pi^2
			\left( -\frac{\partial}{\partial \lambda^*} \right)^r
			\left( \frac{\partial}{\partial \lambda} \right)^s
			\delta(\Real \lambda) \delta(\Imag \lambda)
	\end{eqn*}
\end{lemma}
\begin{proof}
First, changing the variables in the integrals and using known Fourier transform relations, we can prove that for real $x$ and $v$, and non-negative integer $n$
\begin{eqn}
\label{eqn:c-numbers:fourier-real}
	\int\limits_{-\infty}^{\infty} dv\, v^n \exp(\pm 2 i x v)
	= \pi (\mp i / 2)^n \delta^{(n)}(x),
\end{eqn}
Note that we have explicitly written integration limits here;
they are swapped when we change the variable in the first integral.

Denoting $\alpha = u + iv$ and $\lambda = x + iy$, we can expand the initial expression as
\begin{eqn}
	& \int d^2\alpha\, \alpha^r (\alpha^*)^s \exp(-\lambda \alpha^* + \lambda^* \alpha) \\
	& = \int du dv \exp(2ivx - 2iuy)
		\sum_{l=0}^r \binom{r}{l} u^l (iv)^{r-l}
		\sum_{m=0}^s \binom{s}{m} u^m (-iv)^{s-m} \\
	& = \sum_{l=0}^r \sum_{m=0}^s \binom{r}{l} \binom{s}{m}
		i^{r-l} (-i)^{s-m}
		\int du\, u^{l+m} \exp(2ivx)
		\int dv\, v^{r-l+s-m} \exp(-2iuy).
\end{eqn}
Applying~\eqnref{c-numbers:fourier-real} and grouping differentials:
\begin{eqn}
	& = \pi^2 \sum_{l=0}^r \sum_{m=0}^s \binom{r}{l} \binom{s}{m}
		i^{r-l} (-i)^{s-m}
		(-i/2)^{l+m} \delta^{(l+m)}(y)
		(i/2)^{r-l+s-m} \delta^{(r-l+s-m)}(x) \\
	& = \pi^2
		\sum_{l=0}^r \binom{r}{l}
			\frac{1}{2^r}
			(-i \partial / \partial y)^l
			(-\partial / \partial x)^{r-l}
		\sum_{m=0}^s \binom{s}{m}
			\frac{1}{2^s}
			(-i \partial / \partial y)^m
			(\partial / \partial x)^{s-m}
		\delta(y) \delta(x).
\end{eqn}
Collapsing sums and recognizing \defref{c-numbers:wirtinger}:
\begin{eqn}
	& = \pi^2
		\left( \frac{1}{2} (-i \partial / \partial y - \partial / \partial x) \right)^r
		\left( \frac{1}{2} (-i \partial / \partial y + \partial / \partial x) \right)^s
		\delta(y) \delta(x) \\
	& = \pi^2
		\left( -\frac{\partial}{\partial \lambda^*} \right)^r
		\left( \frac{\partial}{\partial \lambda} \right)^s
		\delta(\Real \lambda) \delta(\Imag \lambda).
		\qedhere
\end{eqn}
\end{proof}

\begin{lemma}
\label{lmm:c-numbers:zero-integrals}
	If $f(\lambda, \lambda^*)$ is square-integrable, then for any complex $\alpha$:
	\begin{eqn*}
		\int d^2\lambda
			\frac{\partial}{\partial \lambda} \left(
				\exp(-\lambda \alpha^* + \lambda^* \alpha)
				f(\lambda, \lambda^*)
			\right)
		& = 0 \\
		\int d^2\lambda
			\frac{\partial}{\partial \lambda^*}
			\left(
				\exp(-\lambda \alpha^* + \lambda^* \alpha)
				f(\lambda, \lambda^*)
			\right)
		& = 0.
	\end{eqn*}
\end{lemma}
\begin{proof}
Square-integrability of $f$ means $\lim_{\Real \lambda \rightarrow \infty} = 0$ and $\lim_{\Imag \lambda \rightarrow \infty} = 0$, so the statement of the lemma can be proved by transforming to real variables and integrating.
\end{proof}

\begin{lemma}
\label{lmm:c-numbers:zero-delta-integrals}
	\todo{Any limitations on $f$? It is bounded (on account of being $\chi_W$), for one.}
	\begin{eqn*}
		\int d^2\lambda
			\frac{\partial}{\partial \lambda} \left(
				\left(
					\left( \frac{\partial}{\partial \lambda} \right)^s
					\left( -\frac{\partial}{\partial \lambda^*} \right)^r
					\delta(\Real \lambda) \delta(\Imag \lambda)
				\right)
				f(\lambda, \lambda^*)
			\right)
		& = 0 \\
		\int d^2\lambda
			\frac{\partial}{\partial \lambda^*}
			\left(
				\left(
					\left( \frac{\partial}{\partial \lambda} \right)^s
					\left( -\frac{\partial}{\partial \lambda^*} \right)^r
					\delta(\Real \lambda) \delta(\Imag \lambda)
				\right)
				f(\lambda, \lambda^*)
			\right)
		& = 0.
	\end{eqn*}
\end{lemma}
\begin{proof}
Proved straightforwardly by expanding integrals in real values, separating variables and integrating, using the fact that any derivative of the delta function is zero on the infinity.
\end{proof}
