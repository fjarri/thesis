% =============================================================================
\section{Choice of grid}
\label{sec:bec-noise:grid}
% =============================================================================

In the previous section we mentioned that the spatial grid size for interferometry simulations was chosen to be $8\times8\times64$.
This is the point of equilibrium for two conflicting requirements.
First, the number of grid points (which corresponds to the number of modes) must be low enough to satisfy the Wigner truncation condition~\eqnref{wigner-bec:truncation:population-condition}.
To quantify this, we set the threshold $N > 10 M$, where $N$ is the total population ($55,000$ atoms in our case) and $M$ is the number of grid points.
Second, the number of grid points must be high enough (or, alternatively, the grid spacing must be low enough) to describe the dynamics of the condensate.
As a quantitative threshold we will use the condition that the difference in the quantities of interest should be lower than $1$\% when the grid spacing is halved.

For our tests, we initially picked the grid
\begin{eqn}
    \mathbf{G}_{\mathrm{ref}}
    \equiv G_{\mathrm{ref}}^x \times G_{\mathrm{ref}}^y \times G_{\mathrm{ref}}^z
    = 8\times8\times64
\end{eqn}
as the starting point for several reasons:
\begin{itemize}
\item it has $4096$ points, which is just below the Wigner truncation threshold defined above;
\item its shape corresponds to the shape of the trap (in other words, the spacings in every direction are close to each other);
\item the number of points in each dimension is a power of $2$, which is convenient for numerical calculations.
\end{itemize}
The box for this grid was set to be $1.2$ times wider in all directions than the Thomas-Fermi ground state~\eqnref{bec-noise:mean-field:tf-gs} for the target population.
This resulted in the box with measures
\begin{eqn}
    \mathbf{B}_{\mathrm{ref}}
    \equiv B_{\mathrm{ref}}^x \times B_{\mathrm{ref}}^y \times B_{\mathrm{ref}}^z
    = 9.10\un{\mu m}\times8.83\un{\mu m}\times75.48\un{\mu m},
\end{eqn}
which was used as the reference point along with the grid $\mathbf{G}_{\mathrm{ref}}$.

The tests consisted of running the simulation for the Ramsey sequence with the time $t=1.3\un{s}$ and comparing resulting vectors $\mathbfcal{V}$, containing $100$ sampled values of $\mathcal{V}(t)$ for times from $0\un{s}$ to $1.3\un{s}$, as
\begin{eqn}
    \Delta \mathcal{V}_{\mathrm{grid, box}}
    =
        \left\|
            \mathbfcal{V}_{\mathrm{grid, box}}
            - \mathbfcal{V}_{G_{\mathrm{ref}}, B_{\mathrm{ref}}}
        \right\|_2 /
        \left\|
            \mathbfcal{V}_{G_{\mathrm{ref}}, B_{\mathrm{ref}}}
        \right\|_2,
\end{eqn}
where $\|\ldots\|_2$ is the $2$-norm.

In different runs we varied axial and radial grid spacing (box length divided by the number of points in the corresponding dimension) separately.
Since the integration algorithm performed best with numbers of grid points being powers of $2$, the required spacing was achieved by changing both box length and grid size (for example, the axial spacing $0.75$ of the reference one was obtained by using $2 G_{\mathrm{ref}}^z$ grid points in the $z$ direction, and $1.5 B_{\mathrm{ref}}^z$ box length in that direction).
To verify this approach we ran two additional tests with doubled box length and number of grid points in both directions, which resulted in two additional points for the relative spacing $1.0$ in each pane of \figref{bec-noise:grid:spatial-convergence}.
These points are very close to the reference points (which have $y$-coordinate equal to zero), which proves that the $\Delta \mathcal{V}$ value is only sensitive to the grid spacing and not to the box or grid sizes separately.

\begin{figure}
    \centerline{%
    \includegraphics{figures_generated/test/grid_check_gpe.pdf}%
    \includegraphics{figures_generated/test/grid_check_wigner.pdf}}

    \caption[Spatial convergence tests]{
    Spatial convergence tests for the Ramsey sequence.
    The compound difference in visibility $\Delta \mathcal{V}$ is plotted against the axial (blue circles) and radial (red triangles) spacing, normalized by the spacing for the reference grid $\mathbf{G}_{\mathrm{ref}}$ and box $\mathbf{B}_{\mathrm{ref}}$.
    Simulation parameters are the same as in \figref{bec-noise:visibility:ramsey-visibility}.
    All points were obtained with $160,000$ time steps, the truncated Wigner simulations were run with $64$ trajectories.
    }%endcaption

    \label{fig:bec-noise:grid:spatial-convergence}
\end{figure}

In order to check only the effect of spatial resolution without being influenced by changing number of modes, we performed a batch of tests for \abbrev{cgpe}s~\eqnref{bec-noise:mean-field:cgpes-simplified} in addition to the truncated Wigner \abbrev{sde}s~\eqnref{bec-noise:wigner:sde}.
The results for \abbrev{cgpe}s are plotted in \figref{bec-noise:grid:spatial-convergence},~(a).
It is evident that further decrease of spacing after the reference point both in radial and axial directions does not change the results much.
The change caused by decreased radial spacing is less than our $1$\% threshold, and the effect of axial spacing is too small to be noticeable in the plot.
On the other hand, when the radial spacing is increased, $\Delta \mathcal{V}$ starts to grow.
This means that the reference grid $G_{\mathrm{ref}}$ provides enough spatial resolution for our purposes.

For the truncated Wigner \abbrev{sde}s, \figref{bec-noise:grid:spatial-convergence},~(b) shows that the decrease of radial or axial spacing as compared to the reference values still changes the results significantly.
But since it is not observed in \abbrev{cgpe}s case, we must conclude that this is the result of the criterion~\eqnref{wigner-bec:truncation:population-condition} being violated.
Even at $t=0\un{s}$ the number of modes for the reference grid is only ten times smaller than $N=55,000$, and at $t=1.3\un{s}$ the total population becomes $2$ times smaller due to losses (see \figref{bec-noise:visibility:echo-visibility},~(a)).
Therefore, the further increase of the number of grid points and, correspondingly, modes, makes the truncated Wigner results unreliable.

In view of the results described in this section, we have chosen the grid $\mathbf{G}_{\mathrm{ref}}$ and the box $\mathbf{B}_{\mathrm{ref}}$ for all the interferometry simulations in \charef{bec-noise} and \charef{bec-squeezing}.
The only exception to this is \secref{bec-squeezing:separation}, where we used the same grid, but the box size was scaled down according to the lower total number of atoms (we used the same criterion of $1.2$ times the size of the Thomas-Fermi ground state).

It must be noted that it is possible to decouple the grid size and the number of modes by applying the projector in~\eqnref{bec-noise:wigner:sde} explicitly instead of relying on the implicit cutoff enforced by the grid.
This requires grid padding to avoid aliasing~\cite{Norrie2006}, and thus significantly increases simulation time.
On the other hand it does not give much in terms of spatial resolution, since, with higher frequency modes being projected out in mode space, the corresponding fine details in the coordinate space disappear.
Therefore in this thesis we have settled with using only the implicit cutoff.
