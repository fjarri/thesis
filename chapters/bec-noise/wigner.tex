% =============================================================================
\section{Quasiprobability approach}
\label{sec:bec-noise:wigner}
% =============================================================================

Having discussed the classical mean-field approach in the previous section, we will now apply the functional truncated Wigner method described in \charef{wigner-bec}.
While the mean-field approximation gives a fast way to get a qualitative picture of the \abbrev{bec} dynamics, it ignores many important effects that originate from the inherent quantum nature of the system.
We will see how the initial master equation produces \abbrev{cgpe}s very similar to~\eqnref{bec-noise:mean-field:cgpes-simplified} with the loss terms similar to~\eqnref{bec-noise:mean-field:losses}.

Since we assume that the \abbrev{bec} has $s$-wave interactions, we can use the effective Hamiltonian~\eqnref{wigner-bec:hamiltonian:effective-H} expressed using quantum field creation and annihilation operators $\Psiop_j^{\dagger}(\xvec)$ and $\Psiop_j(\xvec)$:
\begin{eqn}
\label{eqn:bec-noise:wigner:master-eqn}
    \hat{H} = \int \upd \xvec \sum_{j=1}^2 \sum_{k=1}^2 \left\{
        \Psiop_j^{\dagger} K_{jk} \Psiop_k
        + \frac{g_{jk}}{2} \Psiop_j^\dagger \Psiop_k^\dagger \Psiop_j \Psiop_k
    \right\},
\end{eqn}
where $g_{jk}$ are nonlinear interaction coefficients~\eqnref{bec-noise:system:g}.
The single-particle Hamiltonian $K_{jk}$ includes the electromagnetic coupling terms, added by analogy with \abbrev{cgpe}s~\eqnref{bec-noise:mean-field:cgpes-simplified}:
\begin{eqn}
    K_{jk}
    = \left(
            -\frac{\hbar^2}{2m} \nabla^2 + V_j(\xvec)
        \right) \delta_{jk}
        + \hbar \tilde{\Omega}_{jk}(t).
\end{eqn}
Here, same as in the classical \abbrev{cgpe}s, $V_j(\xvec)$ are external potentials~\eqnref{bec-noise:system:V}, and $\tilde{\Omega}$ is the linear coupling matrix
\begin{eqn}
    \tilde{\Omega}
    = \frac{\hbar \Omega}{2}
        \begin{pmatrix}
            0 & e^{-i (\delta t + \alpha)} \\
            e^{i (\delta t + \alpha)} & 0
        \end{pmatrix}.
\end{eqn}

Taking the nonlinear losses into account, the evolution of the system's density matrix $\hat{\rho}$ is described by the Markovian master equation~\eqnref{wigner-bec:master-eqn:master-eqn}:
\begin{eqn}
    \frac{\upd\hat{\rho}}{\upd t} =
        - \frac{i}{\hbar} \left[ \hat{H}, \hat{\rho} \right]
        + \sum_{\lvec \in L} \kappa_{\lvec} \int \upd \xvec
            \mathcal{L}_{\lvec} \left[ \hat{\rho} \right],
\end{eqn}
where $\mathcal{L}$ are loss operators~\eqnref{wigner-bec:master-eqn:loss-op}, and $L$ is the set of loss processes that are experimentally shown to affect the trapped \Rb{} \abbrev{bec}~\cite{Mertes2007,Egorov2011}:
\begin{itemize}
    \item three-body recombination $\hat{O}_{111}=\Psiop_{1}^3$,
    \item two-body interspecies loss $\hat{O}_{12}=\Psiop_{1}\Psiop_{2}$, and
    \item two-body intraspecies loss $\hat{O}_{22}=\Psiop_{2}^2$,
\end{itemize}
with the coefficients depending on the components used in the particular experiment.

Applying the general formulas from \charef{wigner-bec}, we can transform the master equation~\eqnref{bec-noise:wigner:master-eqn} to a \abbrev{fpe} for the truncated Wigner functional, and further to a set of \abbrev{sde}s in the It\^{o} form~\eqnref{wigner-bec:fpe-bec:sde}
\begin{eqn}
\label{eqn:bec-noise:wigner:sde}
    \upd\Psi_1 & = \mathcal{P}_{\restbasis_1} \left[
        \mathcal{A}_1 \upd t
        + \sum_{\lvec \in L} \mathcal{B}_{1,\lvec} \upd Q_{\lvec}
    \right], \\
    \upd\Psi_2 & = \mathcal{P}_{\restbasis_2} \left[
        \mathcal{A}_2 \upd t
        + \sum_{\lvec \in L} \mathcal{B}_{2,\lvec} \upd Q_{\lvec}
    \right].
\end{eqn}
with the drift terms
\begin{eqn}
    \mathcal{A}_1
    ={} & - \frac{i}{\hbar} \left(
            \sum_{k=1}^2 K_{1k} \Psi_k
            + \Psi_1 \sum_{k=1}^2 U_{1k} \left(
                |\Psi_k|^2 - \frac{\delta_{1k} + 1}{2} \delta_{\restbasis_k}(\xvec, \xvec)
            \right)
        \right) \\
    & - 3\kappa_{111} \left( |\Psi_1|^2
        - 3 \delta_{\restbasis_1}(\xvec, \xvec) \right) |\Psi_1|^2 \Psi_1
        - \kappa_{12} \left( |\Psi_{2}|^2
        - \frac{\delta_{\restbasis_2}(\xvec, \xvec)}{2} \right) \Psi_1, \\
    \mathcal{A}_2
    ={} & - \frac{i}{\hbar} \left(
            \sum_{k=1}^2 K_{2k} \Psi_k
            + \Psi_2 \sum_{k=1}^2 U_{2k} \left(
                |\Psi_{k}|^2 - \frac{\delta_{2k} + 1}{2} \delta_{\restbasis_k}(\xvec, \xvec)
            \right)
        \right) \\
    & - \kappa_{12} \left(|\Psi_1|^2 - \frac{\delta_{\restbasis_1}(\xvec, \xvec)}{2} \right) \Psi_2
    - 2\kappa_{22} \left(|\Psi_2|^2 - \delta_{\restbasis_2}(\xvec, \xvec) \right)\Psi_2,
\end{eqn}
and the noise terms
\begin{eqn}
    \mathcal{B}_{1,111} = 3 \sqrt{\kappa_{111}} \left( \Psi_1^* \right)^2,\quad
    \mathcal{B}_{1,12} = \sqrt{\kappa_{12}} \Psi_2^*,\quad
    \mathcal{B}_{1,22} = 0,
\end{eqn}
\begin{eqn}
    \mathcal{B}_{2,111} = 0,\quad
    \mathcal{B}_{2,12} = \sqrt{\kappa_{12}} \Psi_1^*,\quad
    \mathcal{B}_{2,22} = 2\sqrt{\kappa_{22}} \Psi_2^*.
\end{eqn}
Note that the main order of the loss components in the drift terms coincides with the phenomenological loss components of \abbrev{cgpe}s~\eqnref{bec-noise:mean-field:cgpes}.

\copypaste{
For initial conditions in interferometry it is usually sufficient to consider a coherent state amplitude $\Psi_j^{(c)}$, corresponding to a typical initial state with Poissonian number fluctuations, as produced by a beam splitter.
For greater accuracy, the initial state can be modified to account for initial  correlations, thermal noise, or additional fluctuations.
If normal ordered correlations are measured, one has to express them as a sum of symmetrically ordered terms.
}

\copypaste{
This includes all the known nonlinear quantum noise effects of quantum dynamics, like phase diffusion, entanglement and quantum squeezing, in the limit of large particle number.
The initial noise terms do not occur in the semi-classical Gross-Pitaevskii approximation, which is therefore unable to predict these effects.
Thus, while the lossless equations are identical to the Gross-Pitaevskii equations, the inclusion of initial noise terms together with nonlinear interactions leads to quantum phase-diffusion.
Such methods can be used for either free-space or trapped atom interferometry, provided there is an appropriate mode truncation.
Additional quantum noise enters from the effects of damping and losses, due to the fluctuation-dissipation theorem.
These effects are important at high densities in atomic traps.
}

$\zeta_{\lvec}^{(n)}(\xvec, t)$ is a corresponding complex,
stochastic delta-correlated Gaussian noise with
\begin{equation}
    \left\langle
        \zeta_{\lvec}^{(n)} (\xvec,t) \zeta_{\kvec}^{(m)*}(\xvec^\prime, t^\prime)
    \right\rangle =
    \delta_{\lvec \kvec} \delta^{nm} \delta^{D} \left(
        \xvec - \xvec^\prime
    \right)
    \delta \left( t - t^\prime \right).
\end{equation}
The multiplicative noise coefficient
\begin{equation}
    \beta_{\lvec,j}^{(n)} \left( \Psivec \right) =
    \sqrt{\kappa_{\lvec}^{(n)}}
    \frac{\partial O_{\lvec}^{(n)}}{\partial\Psi_j}
\end{equation}
is a fluctuation-dissipation term,
so that the Wigner variables remain equivalent to the corresponding operators.

The loss coefficients in~\eqnref{bec-noise:wigner:sde} can be converted to the conventional form,
which is defined using atom number losses:
\begin{equation}
    \dot{n}_j = - \gamma^{(n)}_{\lvec,j} n^{m_1}_1 n^{m_2}_2 \ldots ,
\end{equation}
where $n_j$ is the density of component $j$ and $m_j$
is the number of spin-$j$ atoms lost in the collision.
The conversion can be carried out as $\gamma^{(n)}_{\lvec,j} = 2 m_j \kappa^{(n)}_{\lvec}$.

In this work we use a basis of plane waves in the volume $V$,
and the density of component $j$ is calculated as a probabilistic average:
\begin{equation}
\label{eqn:wigner-density}
    n_j (\xvec)
        = \langle \Psi^*_j (\xvec) \Psi_j (\xvec) \rangle_{\mathrm{paths}} - \frac{M}{2V}.
\end{equation}
Here we use the fact that the approximate Wigner function is a probability distribution
equivalent to an averaged sum over different simulation paths.

\begin{figure}
    %\begin{tabular}{l l}
    %\imagetop{\hspace*{0.44in}\includegraphics[width=0.72\columnwidth]{ramsey_sequence.eps}} & \imagetop{(a)} \\
    %\imagetop{\includegraphics[width=0.85\columnwidth]{long_ramsey_visibility.eps}} & \imagetop{(b)} \\
    %\imagetop{\hspace*{0.44in}\includegraphics[width=0.72\columnwidth]{echo_sequence.eps}} & \imagetop{(c)} \\
    %\imagetop{\includegraphics[width=0.85\columnwidth]{long_rephasing_visibility.eps}} & \imagetop{(d)}
    %\end{tabular}

    \caption{
    Timeline of the experiment for Ramsey (a) and Ramsey with spin echo (c); (b) and (d) are the simulated plots of interferometric visibility.
    Classical GPE (red dashed lines) and Wigner calculations (blue solid lines) are shown.
    $N = 5.5 \times 10^4$,
    $\omega_x = \omega_y = 2 \pi \times 97.0\un{Hz}$,
    $\omega_z = 2 \pi \times 11.69\un{Hz}$,
    $a_{11} = 100.4\,a_0$, $a_{12} = 97.993\,a_0$, $a_{22} = 95.57\,a_0$~\cite{Egorov2011},
    $a_0$ is the Bohr radius.
    Nonlinear atomic losses:
    $\gamma^{(3)}_{111} = 5.4 \times 10^{-30}\un{cm^6/s}$~\cite{Mertes2007},
    $\gamma^{(2)}_{12} = 1.51 \times 10^{-14}\un{cm^3/s}$,
    $\gamma^{(2)}_{22} = 8.1 \times 10^{-14}\un{cm^3/s}$~\cite{Egorov2011}.}

    \label{fig:visibility}
\end{figure}

To illustrate the applications of this method we consider recent interferometry
experiments with a two-component BEC involving two hyperfine states
${\ket{F=1,\, m_F=-1}}$ and ${\ket{F=2,\, m_F=+1}}$ in \Rb~\cite{Egorov2011}.
A conventional Ramsey sequence (\figref{visibility},~(a)) has been used
with a BEC confined in a cigar-shaped magnetic trap with the frequencies $(97.0, 97.0, 11.69)\un{Hz}$
in a bias magnetic field of $3.23\un{G}$, so that magnetic field dephasing is largely eliminated~\cite{Hall1998}.
The first $\pi/2$ pulse prepares a non-equilibrium superposition of states ${\ket{1,-1}}$ and ${\ket{2,+1}}$
and the spatial modes of two components periodically separate and merge again~\cite{Mertes2007}.
The spatially-separated spin components evolve differently, as they have
different scattering lengths.
As a result, these collective oscillations lead to periodic dephasing and
self-rephasing of the BEC components, clearly visible in both GPE and Wigner
simulations of interference fringe visibility
$\mathcal{V}$ (\figref{visibility},~(b)).
Asymmetric losses of two states are one cause of the contrast decay.
This can be partially compensated by the application of a spin echo pulse
mid-way through the evolution (\figref{visibility},~(c)).
The GPE simulations wrongly predict (dashed lines) that visibility is largely
recovered at long evolution times using the spin echo method.
However, the addition of quantum noise (solid line) via the Wigner simulations
noticeably speeds up the visibility decay even with a spin echo pulse present.
This is in agreement with experimental observations, and shows that these
effects play a significant part in the decay of visibility, even for
large particle numbers.

The important feature of these quantum dynamical simulations
is that they are able to treat large numbers of atoms (55,000 in this case),
while correctly tracking all the quantum noise sources, and also extending the simulations to long time-scales.
Both of these features, large atom numbers and long time-scales,
are essential ingredients to accurate interferometric measurements.
The simulations give accurate predictions despite large, multi-mode dynamical motion in three dimensions
and substantial losses of most of the condensate atoms~\cite{Egorov2011}.
On longer time-scales, the experimental accuracy is limited by technical noises, and we have no data for comparisons.

