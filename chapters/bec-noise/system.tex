% =============================================================================
\section{Two-component trapped BEC}
% =============================================================================

The system we are interested in is a harmonically trapped \abbrev{bec} such as the one in~\cite{Riedel2010,Egorov2011,Egorov2013}.
The \abbrev{bec} in question has two components with a significant nonlinear repulsive interaction (both inter- and intra-component), nonlinear losses and perhaps some additional linear unitary effects, such as electromagnetic coupling between components.
The number of components is only set to two in order to simplify resulting equations for the experiments described in this chapter; it can be easily increased if necessary.


\copypaste{
We start by assuming that the BEC has $s$-wave interactions, together with Markovian losses due to $n$-body collisions.
We employ a master equation together with the Wigner-Moyal quantum phase-space representation~\cite{Gardiner2004} and a truncation of third- and higher-order derivatives in the equations of motion.
If we regard the commonly used Gross-Pitaevskii equation as a classical, first approximation to mean-field condensate dynamics, the truncated Wigner approach is best thought of as the second term in an expansion in inverse particle number.
}

In the present Letter, we treat an ultra-cold,
interacting multi-component spinor Bose gas in $D$ effective dimensions.
The basic Hamiltonian is easily expressed using quantum fields
$\Psiop_j^{\dagger}(\xvec)$ and $\Psiop_j(\xvec)$,
where $\Psiop_j^{\dagger}(\xvec)$ creates a bosonic atom of spin $j$
at location $\xvec$, and $\Psiop_j(\xvec)$ destroys one;
the commutators are
$[\Psiop_j(\xvec),\Psiop_k^{\dagger}(\xvec^\prime)] =
\delta^{(D)}(\xvec-\xvec^\prime)\delta_{jk}.$
The resulting physics of a dilute, low-temperature Bose gas
is well-described in the $s$-wave scattering limit by an effective Hamiltonian
with contact interactions and external potentials:
\begin{equation}
    \hat{H} / \hbar = \int \upd^{D}\xvec \left\{
        \Psiop_j^{\dagger} K_{jk} \Psiop_k +
        \frac{U_{jk}}{2} \Psiop_j^{\dagger} \Psiop_k^{\dagger}
        \Psiop_k \Psiop_j
    \right\}.
\end{equation}
Here we omit the field argument $(\xvec)$ for brevity,
and use the Einstein summation convention of summing over repeated indices.
$K_{jk}$ is the single-particle Hamiltonian:
\begin{equation}
    K_{jk} = \left( -\frac{\hbar}{2m} \nabla^2 + \omega_j + V_j(\xvec) / \hbar \right) \delta_{jk} +
        \tilde{\Omega}_{jk}(t),
\end{equation}
where $m$ is the atomic mass, $V_j$ is the external trapping potential for spin $j$,
$\omega_j$ is the internal energy of spin $j$,
$\tilde{\Omega}_{jk}$ represents a time-dependent coupling
that is used to rotate one spin projection into another,
and $U_{jk}$ is the atom-atom interaction term.
Thus, $n_j = \langle \Psiop_j^{\dagger} \Psiop_j \rangle$
is the spin-$j$ atomic density.
For a dilute gas at low enough temperatures,
$U_{jk}=4\pi\hbar a_{jk} / m$, where $a_{jk}$ is the $s$-wave scattering length in three dimensions.
Here we assume a momentum cutoff $k_{c} \ll 1 / a_{jk}$,
otherwise the couplings must be renormalized~\cite{Sinatra2002}.
