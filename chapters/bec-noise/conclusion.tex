% =============================================================================
\section{Conclusion}
% =============================================================================

The examples in this chapter demonstrate that the truncated Wigner approach gives long-time predictions of quantum effects for the system comprised of a large number of atoms.
Both of these features, large atom numbers and long time-scales, are essential to accurate interferometric measurements.

We show that current experiments are limited primarily by technical noise, and an accurate inclusion of these factors in the simulations can improve the predictions significantly.
The uncertainty in the initial atom number has a large effect on phase noise and accounts for part of the difference of our simulations with the experimental results.
A quantitative estimation of its effect on visibility requires it to be formally included in the
model and remains for future work.

Another possible source of discrepancies are finite temperature effects.
These can be included in the model by using an initial thermalized state (see the end of \secref{wigner-bec:initial-states} for references).
This factor may account for the noticeable difference in the minimum of the first visibility fringe (\figref{bec-noise:visibility:ramsey-visibility}), which cannot be explained by technical noise.
