% =============================================================================
\section{Phase noise}
% =============================================================================

In this work we use a basis of plane waves in the volume $V$,
and the density of component $j$ is calculated as a probabilistic average:
\begin{equation}
\label{eqn:wigner-density}
    n_j (\xvec)
        = \langle \Psi^*_j (\xvec) \Psi_j (\xvec) \rangle_{\mathrm{paths}} - \frac{M}{2V}.
\end{equation}
Here we use the fact that the approximate Wigner function is a probability distribution
equivalent to an averaged sum over different simulation paths.

\begin{figure}
    %\begin{tabular}{l l}
    %\imagetop{\hspace*{0.44in}\includegraphics[width=0.72\columnwidth]{ramsey_sequence.eps}} & \imagetop{(a)} \\
    %\imagetop{\includegraphics[width=0.85\columnwidth]{long_ramsey_visibility.eps}} & \imagetop{(b)} \\
    %\imagetop{\hspace*{0.44in}\includegraphics[width=0.72\columnwidth]{echo_sequence.eps}} & \imagetop{(c)} \\
    %\imagetop{\includegraphics[width=0.85\columnwidth]{long_rephasing_visibility.eps}} & \imagetop{(d)}
    %\end{tabular}

    \caption{
    Timeline of the experiment for Ramsey (a) and Ramsey with spin echo (c); (b) and (d) are the simulated plots of interferometric visibility.
    Classical GPE (red dashed lines) and Wigner calculations (blue solid lines) are shown.
    $N = 5.5 \times 10^4$,
    $\omega_x = \omega_y = 2 \pi \times 97.0\un{Hz}$,
    $\omega_z = 2 \pi \times 11.69\un{Hz}$,
    $a_{11} = 100.4\,a_0$, $a_{12} = 97.993\,a_0$, $a_{22} = 95.57\,a_0$~\cite{Egorov2011},
    $a_0$ is the Bohr radius.
    Nonlinear atomic losses:
    $\gamma^{(3)}_{111} = 5.4 \times 10^{-30}\un{cm^6/s}$~\cite{Mertes2007},
    $\gamma^{(2)}_{12} = 1.51 \times 10^{-14}\un{cm^3/s}$,
    $\gamma^{(2)}_{22} = 8.1 \times 10^{-14}\un{cm^3/s}$~\cite{Egorov2011}.}

    \label{fig:visibility}
\end{figure}

To illustrate the applications of this method we consider recent interferometry
experiments with a two-component BEC involving two hyperfine states
${\ket{F=1,\, m_F=-1}}$ and ${\ket{F=2,\, m_F=+1}}$ in \Rb~\cite{Egorov2011}.
A conventional Ramsey sequence (\figref{visibility},~(a)) has been used
with a BEC confined in a cigar-shaped magnetic trap with the frequencies $(97.0, 97.0, 11.69)\un{Hz}$
in a bias magnetic field of $3.23\un{G}$, so that magnetic field dephasing is largely eliminated~\cite{Hall1998}.
The first $\pi/2$ pulse prepares a non-equilibrium superposition of states ${\ket{1,-1}}$ and ${\ket{2,+1}}$
and the spatial modes of two components periodically separate and merge again~\cite{Mertes2007}.
The spatially-separated spin components evolve differently, as they have
different scattering lengths.
As a result, these collective oscillations lead to periodic dephasing and
self-rephasing of the BEC components, clearly visible in both GPE and Wigner
simulations of interference fringe visibility
$\mathcal{V}$ (\figref{visibility},~(b)).
Asymmetric losses of two states are one cause of the contrast decay.
This can be partially compensated by the application of a spin echo pulse
mid-way through the evolution (\figref{visibility},~(c)).
The GPE simulations wrongly predict (dashed lines) that visibility is largely
recovered at long evolution times using the spin echo method.
However, the addition of quantum noise (solid line) via the Wigner simulations
noticeably speeds up the visibility decay even with a spin echo pulse present.
This is in agreement with experimental observations, and shows that these
effects play a significant part in the decay of visibility, even for
large particle numbers.

The important feature of these quantum dynamical simulations
is that they are able to treat large numbers of atoms (55,000 in this case),
while correctly tracking all the quantum noise sources, and also extending the simulations to long time-scales.
Both of these features, large atom numbers and long time-scales,
are essential ingredients to accurate interferometric measurements.
The simulations give accurate predictions despite large, multi-mode dynamical motion in three dimensions
and substantial losses of most of the condensate atoms~\cite{Egorov2011}.
On longer time-scales, the experimental accuracy is limited by technical noises, and we have no data for comparisons.
