% =============================================================================
\chapter{Functional FPE to SDE correspondences}
\label{cha:appendix:fpe-sde}
% =============================================================================

Wigner transformation, to which the majority of this thesis is dedicated to, produces a \abbrev{fpe}, or its functional equivalent, from an initial master equation.
\abbrev{fpe} is an equation in partial derivatives, and, in general, is not easy to solve~--- even numerically.
The major part of the usefullness of the Wigner transformation paired with the Wigner truncation is that it produces \abbrev{fpe} in a special form, which can be further transformed to a set of \abbrev{sde}s, with the Wigner function playing the role of a probability distribution.
Algorithms of solving such equations numerically are much more straightforward.

The actual correspondence between \abbrev{fpe} and \abbrev{sde}s is formulated and proved for real-valued coefficients in literature~\cite{Risken1996}.
In this thesis we will need to transform \abbrev{fpe}s with complex coefficients, or even functional operator ones.
While it is always possible to express them in real-valued form, it is much more convenient to derive correspondence theorems that work directly on such \abbrev{fpe}s.
In this Appendix we will do that by proceeding successively from the initial real-valued theorem to complex-valued and functional correspondences.

In addition, we will do the same for the It\^o formula, which provides the expression for the time derivative of any function of transverse variables.
This formula is useful, among other cases, if one wants to derive the time dependence of some integral observable (for instance, population), without solving \abbrev{sde}s themselves.
Alternatively, it can serve as an additional test of a numerical algorithm used to propagate \abbrev{sde}s in time.

% =============================================================================
\section{Correspondences}
% =============================================================================

We will start by formulating the known real-valued correspondence in a form which is more convenient for further proofs in this section, and is also closer to the results one obtains from the Wigner transformation.

\begin{lemma}[real-valued \abbrev{fpe}--\abbrev{sde}s correspondence in a convenient form.]
\label{lmm:fpe-sde:corr:fpe-sde-real}
	Let $\zvec^T \equiv (z_1 \ldots z_M)$ be a set of real variables.
	Then the \abbrev{fpe}
	\begin{eqn*}
		\frac{\upd W}{\upd t}
		= -\vcwd_{\zvec}^T \cdot \avec W
		+ \frac{1}{2} \Trace{ \vcwd_{\zvec} \vcwd_{\zvec}^T B B^T } W
	\end{eqn*}
	is equivalent to the set of \abbrev{sde}s in the It\^o form
	\begin{eqn*}
		\upd\zvec = \avec \upd t + B \upd\Zvec
	\end{eqn*}
	and to the set of \abbrev{sde}s in the Stratonovich form
	\begin{eqn*}
		\upd\zvec = (\avec - \svec)\upd t + B \upd\Zvec,
	\end{eqn*}
	where the noise-induced (Stratonovich) drift vector $\svec$ has elements
	\begin{eqn*}
		s_j
		= \frac{1}{2} \sum_{k,i} B_{ki} \frac{\cwd}{\cwd z_k} B_{ji}
		= \frac{1}{2} \Trace{B^T \vcwd_{\zvec} \evec_j^T B},
	\end{eqn*}
	with $\evec_j$ being a unit vector with elements $(\evec_j)_i = \delta_{ij}$.
	Here $W \equiv W(\zvec)$ is a probability distribution, $\avec \equiv \avec(\zvec)$ is a vector function, $B \equiv B(\zvec)$ is a matrix function ($B$ having the size $M \times L$, where $L$ is the number of noise sources), $\vcwd_{\zvec}^T \equiv (\upd/\upd z_1 \ldots \upd/\upd z_M)$ is a cogradient vector, and $\Zvec$ is a standard $L$-dimensional Wiener process with $\langle \upd Z_j^2 \rangle = \upd t$.
\end{lemma}
\begin{proof}
For the detailed proof see Risken~\cite{Risken1996}, sections 3.3 and 3.4.
\end{proof}

The above theorem can be extended to work with complex Wirtinger derivatives and complex-valued coefficients.
Of course, in order to produce a real-valued $\upd W/\upd t$ in the left-hand part, an \abbrev{fpe} must have a particular form.

\begin{theorem}
\label{thm:fpe-sde:corr:fpe-sde-complex}
	Let $\balpha^T \equiv (\alpha_1 \ldots \alpha_M)$ be a set of complex variables.
	Then the \abbrev{fpe}
	\begin{eqn*}
		\frac{\upd W}{\upd t}
		= -\vcwd_{\balpha}^T \avec W - \vcwd_{\balpha^*}^T \avec^* W
		+ \Trace{ \vcwd_{\balpha^*} \vcwd_{\balpha}^T B B^H } W
	\end{eqn*}
	is equivalent to the set of \abbrev{sde}s in the It\^o form
	\begin{eqn*}
		\upd\balpha = \avec \upd t + B \upd\Zvec,
	\end{eqn*}
	and to the set of \abbrev{sde}s in the Stratonovich form
	\begin{eqn*}
		\upd\balpha = (\avec - \svec) \upd t + B \upd\Zvec,
	\end{eqn*}
	where the Stratonovich term has elements
	\begin{eqn*}
		s_j = \frac{1}{2} \Trace{ B^H \vcwd_{\balpha^*} \evec_j^T B },
	\end{eqn*}
	and $\Zvec = (\mathbf{X} + i\mathbf{Y}) / \sqrt{2}$ is an $L$-dimensional standard complex-valued Wiener process (with $\langle \upd Z_j \upd Z_k^* \rangle = \delta_{jk} \upd t$), containing two standard $L$-dimensional Wiener processes $\mathbf{X}$ and $\mathbf{Y}$.
\end{theorem}
\begin{proof}
Let us expand the \abbrev{fpe} using real variables: $\balpha = \xvec + i \yvec$, $\avec = \mathbf{u} + i \mathbf{v}$, $B = F + iG$, $\vcwd_{\balpha} = (\vcwd_{\xvec} - i \vcwd_{\yvec}) / 2$.
This results in
\begin{eqn}
	\frac{\upd W}{\upd t}
	={} & - \vcwd_{\xvec}^T \mathbf{u} W
	- \vcwd_{\yvec}^T \mathbf{v} W
	+ \frac{1}{4} \Trace{
		(\vcwd_{\xvec} \vcwd_{\xvec}^T
			+ \vcwd_{\yvec} \vcwd_{\yvec}^T)
		(F F^T + G G^T) \right. \\
	& \left. - (\vcwd_{\xvec} \vcwd_{\yvec}^T
			- \vcwd_{\yvec} \vcwd_{\xvec}^T)
		(F G^T - G F^T)
	} W \\
	& + \frac{i}{4} \Trace{
		(\vcwd_{\xvec} \vcwd_{\xvec}^T
			+ \vcwd_{\yvec} \vcwd_{\yvec}^T)
		(F G^T - G F^T)
	} W \\
	& + \frac{i}{4} \Trace{
		(\vcwd_{\xvec} \vcwd_{\yvec}^T
			- \vcwd_{\yvec} \vcwd_{\xvec}^T)
		(F F^T + G G^T)
	} W.
\end{eqn}
Since $F F^T + G G^T$ and $\vcwd_{\xvec} \vcwd_{\xvec}^T + \vcwd_{\yvec} \vcwd_{\yvec}^T$ are symmetric matrices, and $F G^T - G F^T$ and $\vcwd_{\xvec} \vcwd_{\yvec}^T - \vcwd_{\yvec} \vcwd_{\xvec}^T$ are antisymmetric ones, the corresponding traces are equal to zero, which gives us the \abbrev{fpe} in real variables
\begin{eqn}
	\frac{\upd W}{\upd t}
	={} & - \vcwd_{\xvec}^T \mathbf{u} W
	- \vcwd_{\yvec}^T \mathbf{v} W
	+ \frac{1}{4} \Trace{
		(\vcwd_{\xvec} \vcwd_{\xvec}^T
			+ \vcwd_{\yvec} \vcwd_{\yvec}^T)
		(F F^T + G G^T) \right. \\
	& \left. - (\vcwd_{\xvec} \vcwd_{\yvec}^T
			- \vcwd_{\yvec} \vcwd_{\xvec}^T)
		(F G^T - G F^T)
	} W.
\end{eqn}

In order to use \lmmref{fpe-sde:corr:fpe-sde-real}, we need to merge variables $\xvec$ and $\yvec$ into one variable vector $\zvec \equiv \xvec \oplus \yvec$.
This will give us an equation in the form identical to that from the lemma, with drift vector $\tilde{\avec} \equiv \mathbf{u} \oplus \mathbf{v}$ and diffusion matrix
\begin{eqn}
	\tilde{B} \tilde{B}^T \equiv \frac{1}{2} \begin{pmatrix}
		F F^T + G G^T & F G^T - G F^T \\
		G F^T - F G^T & F F^T + G G^T
	\end{pmatrix},
\end{eqn}
which gives the noise matrix
\begin{eqn}
	\tilde{B} = \frac{1}{\sqrt{2}} \begin{pmatrix}
		F & -G \\
		G & F
	\end{pmatrix}.
\end{eqn}
Therefore, the equivalent \abbrev{sde}s in the It\^o form are
\begin{eqn}
	d\zvec = \tilde{\avec} dt + \tilde{B} d\tilde{\Zvec},
\end{eqn}
where $d\tilde{\Zvec} \equiv d\mathbf{X} \oplus d\mathbf{Y}$.
Returning to the previous variables:
\begin{eqn}
	d\xvec & = \mathbf{u} dt + \frac{1}{\sqrt{2}} F d\mathbf{X} - \frac{1}{\sqrt{2}} G d\mathbf{Y}, \\
	d\yvec & = \mathbf{v} dt + \frac{1}{\sqrt{2}} G d\mathbf{X} + \frac{1}{\sqrt{2}} F d\mathbf{Y}.
\end{eqn}
Multiplying the second equation by $i$ and adding it to the first one:
\begin{eqn}
	d\balpha = \avec dt + \frac{1}{\sqrt{2}} (F + iG) (d\mathbf{X} + id\mathbf{Y}),
\end{eqn}
which leads to the It\^o part of the theorem statement
\begin{eqn}
	d\balpha = \avec dt + B d\Zvec.
\end{eqn}

The noise-induced drift term in the Stratonovich case can be calculated by substituting $\tilde{B}$ into the expression for $s_j$ from \lmmref{fpe-sde:corr:fpe-sde-real}.
We will calculate $s_j$ with $j$ belonging to the $\xvec$ and the $\yvec$ part of the coordinate space separately.
Starting from the $\xvec$ part:
\begin{eqn}
	s_j^{(x)}
	= \frac{1}{4} \Trace{
		\begin{pmatrix}
			F^T & G^T \\ -G^T & F^T
		\end{pmatrix}
		\begin{pmatrix}
			\vcwd_{\xvec} \\
			\vcwd_{\yvec}
		\end{pmatrix}
		\begin{pmatrix}
			\evec_j^T & 0
		\end{pmatrix}
		\begin{pmatrix}
			F & -G \\ G & F
		\end{pmatrix}
	}.
\end{eqn}
Multiplying the matrices, we get:
\begin{eqn}
	={} & \frac{1}{4} \Trace{
		\begin{pmatrix}
			F^T & G^T \\ -G^T & F^T
		\end{pmatrix}
		\begin{pmatrix}
			\vcwd_{\xvec} \\
			\vcwd_{\yvec}
		\end{pmatrix}
		\begin{pmatrix}
			\evec_j^T F & - \evec_j^T G
		\end{pmatrix}
	} \\
	={} & \frac{1}{4} \Trace{
		\begin{pmatrix}
			F^T & G^T \\ -G^T & F^T
		\end{pmatrix}
		\begin{pmatrix}
			\vcwd_{\xvec} \evec_j^T F & - \vcwd_{\xvec} \evec_j^T G \\
			\vcwd_{\yvec} \evec_j^T F & - \vcwd_{\yvec} \evec_j^T G
		\end{pmatrix}
	} \\
	={} & \frac{1}{4} \left(
		\Trace{ F^T \vcwd_{\xvec} \evec_j^T F }
		+ \Trace{ G^T \vcwd_{\yvec} \evec_j^T F } \right. \\
	& \left. + \Trace{ G^T \vcwd_{\xvec} \evec_j^T G }
		- \Trace{ F^T \vcwd_{\yvec} \evec_j^T G }
	\right).
\end{eqn}
Similarly for the $\yvec$ part,
\begin{eqn}
	s_j^{(y)}
	={} & \frac{1}{4} \left(
		\Trace{ F^T \vcwd_{\xvec} \evec_j^T G }
		+ \Trace{ G^T \vcwd_{\yvec} \evec_j^T G } \right. \\
	& \left. - \Trace{ G^T \vcwd_{\xvec} \evec_j^T F }
		+ \Trace{ F^T \vcwd_{\yvec} \evec_j^T F }
	\right).
\end{eqn}
Therefore, the final term in the complex-valued \abbrev{sde}s is
\begin{eqn}
	s_j
	= s_j^{(x)} + i s_j^{(y)}
	= \frac{1}{2} \Trace{ B^H \vcwd_{\balpha^*} \evec_j^T B },
\end{eqn}
which finishes the proof.
\end{proof}

Note the asymmetry in the expression for the Stratonovich term: if $B = B(\alpha)$, then $\mathbf{s} \equiv 0$.
This is initially caused by the asymmetry in the target \abbrev{sde}s.
A truly general form of an \abbrev{fpe} would be
\begin{eqn}
	\frac{\upd W}{\upd t}
	={} & - 2 \Real \left( \vcwd_{\balpha}^T \avec \right) W
	+ \Trace{ \vcwd_{\balpha^*} \vcwd_{\balpha}^T B_1 B_1^H } W
	+ \Trace{ \vcwd_{\balpha^*} \vcwd_{\balpha}^T B_2 B_2^H } W \\
	& + 2 \Real \left(
		\Trace{ \vcwd_{\balpha} \vcwd_{\balpha}^T B_1 B_2^T }
		+ \Trace{ \vcwd_{\balpha} \vcwd_{\balpha}^T B_2 B_1^T }
	\right) W,
\end{eqn}
which corresponds to the system of \abbrev{sde}s
\begin{eqn}
	\upd\balpha = (\avec - \svec) \upd t + B_1 \upd\Zvec + B_2 \upd\Zvec^*,
\end{eqn}
where the Stratonovich term has elements
\begin{eqn}
	s_j ={} & \frac{1}{2} \left(
		\Trace{ B_1^H \vcwd_{\balpha^*} \evec_j^T B_1 }
		+ \Trace{ B_2^H \vcwd_{\balpha^*} \evec_j^T B_2 } \right. \\
		& \left. + \Trace{ B_1^T \vcwd_{\balpha} \evec_j^T B_2 }
		+ \Trace{ B_2^T \vcwd_{\balpha} \evec_j^T B_1 }
	\right).
\end{eqn}
In the theorem above we limited the space of possible \abbrev{sde}s to those with $B_2 \equiv 0$, leading to the observed asymmetry.

In many applications (some of which are discussed in this thesis), it is advantageous to enumerate the state vector of a system using two variables instead of one, namely the mode identifier and the component number.
This helps to describe particles which can occupy the same set of modes, but are otherwise distinguishable.
We will now reformulate the previous theorem, including this component distinction.

\begin{theorem}[multi-component reformulation of \thmref{fpe-sde:corr:fpe-sde-complex}]
\label{thm:fpe-sde:corr:mc-fpe-sde}
	Let $\balpha^{(j)},\, j = 1 \ldots C$ be $C$ sets of complex variables $\balpha^{(j)} \equiv (\alpha_1^{(j)} \ldots \alpha_{M_j}^{(j)})$.
	Then the \abbrev{fpe}
	\begin{eqn*}
		\frac{\upd W}{\upd t}
		={} & - \sum_{j=1}^C \vcwd_{\balpha^{(j)}}^T \avec^{(j)} W
		- \sum_{j=1}^C \vcwd_{(\balpha^{(j)})^*}^T (\avec^{(j)})^* W \\
		& + \sum_{j=1}^C \sum_{k=1}^C
			\Trace{
				\vcwd_{(\balpha^{(j)})^*}
				\vcwd_{\balpha^{(k)}}^T
				B^{(k)} (B^{(j)})^H
			} W
	\end{eqn*}
	is equivalent to the set of \abbrev{sde}s in the It\^o form
	\begin{eqn*}
		\upd\balpha^{(j)} = \avec^{(j)} \upd t + B^{(j)} \upd\Zvec,
	\end{eqn*}
	or to the set of \abbrev{sde}s in the Stratonovich form
	\begin{eqn*}
		\upd\balpha^{(j)} = (\avec^{(j)} - \svec^{(j)}) \upd t + B^{(j)} \upd\Zvec,
	\end{eqn*}
	where the Stratonovich term has elements
	\begin{eqn*}
		s_i^{(j)} = \frac{1}{2} \sum_{k=1}^C
			\Trace{ (B^{(k)})^H \vcwd_{(\balpha^{(k)})^*} \evec_i^T B^{(j)} }.
	\end{eqn*}
	Here $\Zvec$ is an $L$-dimensional standard complex-valued Wiener process, and the noise matrices $B^{(j)}$ have sizes $M_j \times L$.
\end{theorem}
\begin{proof}
Let us join all variable sets $\balpha^{(j)}$ into a single set
\begin{eqn}
	\balpha \equiv \bigoplus_{j=1}^C \balpha^{(j)}.
\end{eqn}
Then we can use \thmref{fpe-sde:corr:fpe-sde-complex} with the drift vector
\begin{eqn}
	\avec = \bigoplus_{j=1}^C \avec^{(j)},
\end{eqn}
the cogradient vector
\begin{eqn}
	\vcwd_{\balpha} = \bigoplus_{j=1}^C \vcwd_{\balpha^{(j)}},
\end{eqn}
and the noise matrix
\begin{eqn}
	B = \begin{pmatrix}
		B^{(1)} \\ \vdots \\ B^{(C)}
	\end{pmatrix}.
\end{eqn}
This gives us \abbrev{sde}s in the It\^o form
\begin{eqn}
	\upd\balpha = \avec \upd t + B \upd\Zvec,
\end{eqn}
where $d\Zvec$ is an $L$-dimensional standard complex-valued Wiener process.
Splitting this equation for different components, we get the It\^o part of the theorem statement.
Substituting $B$ into the expression for the Stratonovich term:
\begin{eqn}
	s_i^{(j)}
	& = \frac{1}{2} \Trace{
		\begin{pmatrix} (B^{(1)})^H & \cdots & (B^{(C)})^H \end{pmatrix}
		\begin{pmatrix}
			\vcwd_{(\balpha^{(1)})^*} \\
			\vdots \\
			\vcwd_{(\balpha^{(C)})^*}
		\end{pmatrix}
		\begin{pmatrix} 0 & \cdots & \evec_i^T & \cdots & 0 \end{pmatrix}
		\begin{pmatrix}
			B^{(1)} \\
			\vdots \\
			B^{(C)}
		\end{pmatrix}
	}.
\end{eqn}
Multiplying the matrices successively, we get
\begin{eqn}
	& = \frac{1}{2} \Trace{
		\begin{pmatrix} (B^{(1)})^H & \cdots & (B^{(C)})^H \end{pmatrix}
		\begin{pmatrix}
			\vcwd_{(\balpha^{(1)})^*} \evec_i^T B^{(j)} \\
			\vdots \\
			\vcwd_{(\balpha^{(C)})^*} \evec_i^T B^{(j)}
		\end{pmatrix}
	} \\
	& = \frac{1}{2} \sum_{k=1}^C \Trace{
		(B^{(k)})^H
		\vcwd_{(\balpha^{(k)})^*}
		\evec_i^T
		B^{(j)}
	},
\end{eqn}
which is the expression from the theorem statement.
\end{proof}

Most of the time we will deal with \abbrev{fpe}s in a functional form, so we will reformulate the correspondence once again, now using functional derivatives.

\begin{theorem}
\label{thm:fpe-sde:corr:fpe-sde-func}
	The \abbrev{fpe} in a functional form
	\begin{eqn*}
		\frac{\upd W}{\upd t}
		={} & \int \upd\xvec \left(
			- \sum_{j=1}^C \frac{\fdelta}{\fdelta f_j} \mathcal{A}_j W
			- \sum_{j=1}^C \frac{\fdelta}{\fdelta f_j^*} \mathcal{A}_j^* W \right. \\
		& \left. + \sum_{j=1}^C \sum_{k=1}^C \frac{\fdelta^2}{\fdelta f_j^* \fdelta f_k}
				\sum_{l=1}^L \mathcal{B}_{kl} \mathcal{B}_{jl}^* W
		\right)
	\end{eqn*}
	is equivalent to the set of \abbrev{sde}s in the It\^o form
	\begin{eqn*}
		\upd f_j = \mathcal{P}_{\restbasis_j} \left[
			\mathcal{A}_j \upd t
			+ \sum_{l=1}^L \mathcal{B}_{jl} \upd Q_l
		\right],
	\end{eqn*}
	or the set of \abbrev{sde}s in the Stratonovich form
	\begin{eqn*}
		\upd f_j = \mathcal{P}_{\restbasis_j} \left[
			(\mathcal{A}_j - \mathcal{S}_j) \upd t
			+ \sum_{l=1}^L \mathcal{B}_{jl} \upd Q_l
		\right],
	\end{eqn*}
	where
	\begin{eqn*}
		\mathcal{S}_j = \frac{1}{2} \sum_{k=1}^C \sum_{l=1}^L
			\mathcal{B}_{kl}^*
			\frac{\fdelta}{\fdelta f_k^*}
			\mathcal{B}_{jl}.
	\end{eqn*}
	Here $f_j \in \mathbb{F}_{\mathbb{M}_j}$, $\mathcal{A}_j \equiv \mathcal{A}_j[\fvec]$ and $\mathcal{B}_{jl} \equiv \mathcal{B}_{jl}[\fvec]$ are functional operators, $W \equiv W[\fvec]$ is a probability functional, and $L$ is the number of noise sources.
	The standard functional Wiener processes $Q_l$ are the compositions of standard complex-valued Wiener processes:
	\begin{eqn*}
		Q_l = \sum_{\nvec \in \fullbasis} \phi_{\nvec} Z_{l,\nvec}.
	\end{eqn*}
\end{theorem}
\begin{proof}
Expanding the functional derivatives according to \defref{func-calculus:func-diff} with $f_j = \sum_{\nvec \in \restbasis_j} \phi_{j,\nvec} \alpha_{j,\nvec}$:
\begin{eqn}
	\frac{\upd W}{\upd t}
	={} & \left(
		- \sum_{j=1}^C \sum_{\nvec \in \restbasis_j}
			\frac{\cwd}{\cwd \alpha_{j,\nvec}}
			\int \upd\xvec\, \phi_{j,\nvec}^* \mathcal{A}_j W
		- \sum_{j=1}^C \sum_{\nvec \in \restbasis_j}
			\frac{\cwd}{\cwd \alpha_{j,\nvec}^*}
			\int \upd\xvec\, \phi_{j,\nvec} \mathcal{A}_j^* W
		\right. \\
	&	\left. + \sum_{j=1}^C \sum_{k=1}^C
			\sum_{\mvec \in \restbasis_j, \nvec \in \restbasis_k}
			\frac{\cwd}{\cwd \alpha_{j,\mvec}^*}
			\frac{\cwd}{\cwd \alpha_{k,\nvec}}
			\int \upd\xvec
			\phi_{j,\mvec} \phi_{k,\nvec}^*
			\sum_{l=1}^L \mathcal{B}_{jl}^* \mathcal{B}_{kl} W
	\right).
\end{eqn}
The diffusion term has to be transformed in order to conform to \thmref{fpe-sde:corr:mc-fpe-sde}:
\begin{eqn}
	\int \upd\xvec \phi_{j,\mvec} \phi_{k,\nvec}^* \sum_{l=1}^L
		\mathcal{B}_{kl} \mathcal{B}_{jl}^*
	& = \int \upd\xvec \int \upd\xvec^\prime
			\phi_{j,\mvec}^\prime \phi_{k,\nvec}^*
			\sum_{l=1}^L \mathcal{B}_{jl}^{\prime *} \mathcal{B}_{kl}
			\delta(\xvec - \xvec^\prime) \\
	& = \int \upd\xvec \int \upd\xvec^\prime
			\phi_{j,\mvec}^\prime \phi_{k,\nvec}^*
			\sum_{l=1}^L \mathcal{B}_{jl}^{\prime *} \mathcal{B}_{kl}
			\sum_{\pvec \in \fullbasis} \phi_{\pvec}^{\prime*} \phi_{\pvec} \\
	& = \sum_{l=1}^L \sum_{\pvec \in \fullbasis}
		\int \upd\xvec\,
			\phi_{j,\mvec} \mathcal{B}_{jl}^* \phi_{\pvec}^*
		\int \upd\xvec\,
			\phi_{k,\nvec}^* \mathcal{B}_{kl} \phi_{\pvec}.
\end{eqn}
Note that we did not specify the index of the full basis used to expand the delta function.
It can be any orthonormal and complete basis, in particular one of $\fullbasis_j$~--- this will not change the result.

Now we have the \abbrev{fpe} in the form required by \thmref{fpe-sde:corr:mc-fpe-sde} with
\begin{eqn}
	a_{\mvec}^{(j)}
	= \int \upd\xvec\, \phi_{j,\mvec}^* \mathcal{A}_j,\,
	\mvec \in \restbasis_j,
\end{eqn}
and
\begin{eqn}
\label{eqn:fpe-sde:corr:func-noise-matrix}
	B_{\mvec,(\pvec,l)}^{(j)}
	= \int \upd\xvec\, \phi_{j,\mvec}^* \mathcal{B}_{jl} \phi_{\pvec},\,
	\mvec \in \restbasis_j, \pvec \in \fullbasis, l \in [1 \ldots L].
\end{eqn}
Note that the columns of $B$ are enumerated using the compound index $\pvec,l$.

Therefore, the initial \abbrev{fpe} is equivalent to the set of \abbrev{sde}s in the It\^o form
\begin{eqn}
	\upd\alpha_{\mvec}^{(j)}
	= \int \upd\xvec\, \phi_{j,\mvec}^* \mathcal{A}_j \upd t
	+ \sum_{\pvec \in \fullbasis, l \in [1 \ldots L]}
		\int \upd\xvec\, \phi_{j,\mvec}^* \mathcal{B}_{jl} \phi_{\pvec} \upd Z_{\pvec,l}.
\end{eqn}
Multiplying by $\phi_{j,\mvec}^\prime$ and grouping by component:
\begin{eqn}
	\sum_{\mvec \in \restbasis_j} \phi_{j,\mvec}^\prime \upd\alpha_{\mvec}^{(j)}
	={} & \sum_{\mvec \in \restbasis_j} \phi_{j,\mvec}^\prime \int \upd\xvec\, \phi_{j,\mvec}^* \mathcal{A}_j \upd t \\
	& + \sum_{\mvec \in \restbasis_j} \phi_{j,\mvec}^\prime \int \upd\xvec\, \phi_{j,\mvec}^*
		\sum_{l=1}^L \sum_{\pvec \in \fullbasis}
			\mathcal{B}_{jl} \phi_{\pvec} \upd Z_{\pvec,l}.
\end{eqn}
Recognising \defref{func-calculus:projector} of the projection operator:
\begin{eqn}
	\upd f_j
	= \proj{\restbasis_j} \left[
		\mathcal{A}_j \upd t
		+ \sum_{l=1}^L \mathcal{B}_{jl}
			\sum_{\pvec \in \fullbasis} \phi_{\pvec} \upd Z_{\pvec,l}
	\right].
\end{eqn}
Defining the standard functional Wiener process as $Q_l = \sum_{\pvec \in \fullbasis} \phi_{\pvec} Z_{\pvec,l}$:
\begin{eqn}
	\upd f_c
	= \proj{\restbasis_j} \left[
		\mathcal{A}_j \upd t
		+ \sum_{l=1}^L \mathcal{B}_{jl} \upd Q_l
	\right].
\end{eqn}

Performing the same multiplication and summation on the Stratonovich term from \thmref{fpe-sde:corr:mc-fpe-sde}:
\begin{eqn}
	\mathcal{S}_j
	= \sum_{\mvec \in \restbasis_j} \phi_{j,\mvec}^\prime s_{\mvec}^{(j)}
	= \frac{1}{2} \sum_{\mvec \in \restbasis_j} \phi_{j,\mvec}^\prime \sum_{k=1}^C \Trace{
		(B^{(k)})^H \vcwd_{(\balpha^{(k)})^*} \evec_{\mvec}^T B^{(j)}
	}.
\end{eqn}
Transforming the trace to a summation:
\begin{eqn}
	\mathcal{S}_j
	= \frac{1}{2} \sum_{\mvec \in \restbasis_c} \phi_{j,\mvec}^\prime \sum_{k=1}^C
		\sum_{\nvec \in \restbasis_k} \sum_{l=1}^L \sum_{\pvec \in \fullbasis}
			(B_{\nvec (\pvec,l)}^{(k)})^*
			\frac{\cwd}{\cwd (\alpha_{\nvec}^{(k)})^*}
			B_{\mvec (\pvec,l)}^{(j)}.
\end{eqn}
Using the multimode form~\eqnref{fpe-sde:corr:func-noise-matrix} of the noise matrix:
\begin{eqn}
	\mathcal{S}_j
	= \frac{1}{2} \sum_{\mvec \in \restbasis_j} \phi_{j,\mvec}^\prime \sum_{k=1}^C
		\sum_{\nvec \in \restbasis_k} \sum_{l=1}^L \sum_{\pvec \in \fullbasis}
			\int \upd\xvec\, \phi_{k,\nvec} \mathcal{B}_{kl}^* \phi_{\pvec}^*
			\int \upd\xvec\, \phi_{j,\mvec}^*
				\frac{\cwd}{\cwd (\alpha_{\nvec}^{(k)})^*}
				\mathcal{B}_{jl} \phi_{\pvec}.
\end{eqn}
Substituting $\sum_{\pvec \in \fullbasis} \phi_{\pvec}^* \phi_{\pvec} = \delta(\xvec - \xvec^\prime)$:
\begin{eqn}
	\mathcal{S}_j
	= \frac{1}{2} \sum_{\mvec \in \restbasis_j} \phi_{j,\mvec}^\prime
		\sum_{k=1}^C \sum_{\nvec \in \restbasis_k} \sum_{l=1}^L
			\int \upd\xvec\,
				\phi_{k,\nvec} \mathcal{B}_{kl}^*
				\phi_{j,\mvec}^* \frac{\cwd}{\cwd (\alpha_{\nvec}^{(k)})^*}
				\mathcal{B}_{jl}.
\end{eqn}
Recognising the projection transformation and the functional differential:
\begin{eqn}
	\mathcal{S}_j
	& = \proj{\restbasis_j} \left[
		\frac{1}{2} \sum_{k=1}^C \sum_{\nvec \in \restbasis_k} \sum_{l=1}^L
			\phi_{k,\nvec} \mathcal{B}_{kl}^*
			\frac{\cwd}{\cwd (\alpha_{\nvec}^{(k)})^*}
			\mathcal{B}_{jl}
	\right] \\
	& = \proj{\restbasis_j} \left[
		\frac{1}{2} \sum_{k=1}^C \sum_{l=1}^L
		\mathcal{B}_{kl}^*
		\frac{\fdelta}{\fdelta f_k^*}
		\mathcal{B}_{jl}
	\right].
	\qedhere
\end{eqn}
\end{proof}

Alternatively, the \abbrev{fpe} from the above theorem can be expressed in a short matrix form.
The \abbrev{fpe} in this case is
\begin{eqn}
	\frac{dW}{dt}
	= \int d\xvec \left(
		- \vfdelta_{\fvec} \cdot \mathbfcal{A} W
		- \vfdelta_{\fvec^*} \cdot \mathbfcal{A}^* W
		+ \Trace{ \vfdelta_{\fvec^*} \vfdelta_{\fvec}^T \mathcal{B} \mathcal{B}^H } W
	\right),
\end{eqn}
where the functional cogradient $\vfdelta_{\fvec} = \left( \fdelta/\fdelta f_1 \ldots \fdelta/\fdelta f_C \right)$, $\mathbf{\mathcal{A}}$ is a vector of $C$ functional operators, and $\mathcal{B}$ is a matrix of $C \times L$ functional operators.
Such \abbrev{fpe} is equivalent to the matrix \abbrev{sde} in the Stratonovich form:
\begin{eqn}
	\upd \fvec = \mathbfcal{P} \left[
		\left( \mathbfcal{A} - \mathbfcal{S} \right) \upd t
		+ \mathcal{B} \upd \Qvec
	\right],
\end{eqn}
with the Stratonovich term
\begin{eqn}
	\mathbfcal{S}_j
	= \frac{1}{2} \Trace{ \mathcal{B}^H \vfdelta_{\fvec^*} \evec_j^T \mathcal{B} },
\end{eqn}
where $\mathbfcal{P}^T \equiv (\proj{\restbasis_1} \ldots \proj{\restbasis_C})$ is a vector of projection operators, and $\Qvec^T \equiv (Q_1 \ldots Q_C)$ is a vector of standard functional Wiener processes.

% =============================================================================
\section{It\^o formula}
% =============================================================================

In this section we will follow a procedure similar to the one in the previous section and derive the It\^o formula for the differential of a functional, based on the standard definition for the multi-variable real-valued case.
Again, we will formulate the real-valued correspondence in a way that is convenient for future proofs.

\begin{lemma}
\label{lmm:fpe-sde:ito-formula:ito-f-real}
	Let $\zvec^T \equiv (z_1 \ldots z_M)$ be a set of real variables, and $\Zvec(t)$ be a standard $L$-dimensional Wiener process.
	For the set of \abbrev{sde}s in the It\^o form
	\begin{eqn*}
		\upd\zvec = \avec(\zvec, t) \upd t + B(\zvec, t) \upd\Zvec(t),
	\end{eqn*}
	the differential of a function $f(\zvec)$ is
	\begin{eqn*}
		\upd f(\zvec) =
			\avec \cdot \vcwd_{\zvec} f(\zvec) \upd t
			+ \frac{1}{2} \Trace{ B B^T \vcwd_{\zvec} \vcwd_{\zvec}^T } f(\zvec) \upd t
			+ \Trace{ B \upd\Zvec \vcwd_{\zvec}^T } f(\zvec).
	\end{eqn*}
\end{lemma}
\begin{proof}
For the detailed proof see Gardiner~\cite{Gardiner1997}, section 4.3.3.
\end{proof}

As a next step, we will extend this lemma to operate on vectors of complex variables and \abbrev{sde}s with complex-valued coefficients from \thmref{fpe-sde:corr:fpe-sde-complex}.

\begin{theorem}
\label{thm:fpe-sde:ito-formula:ito-f-complex}
	Let $\balpha^T \equiv (\alpha_1 \ldots \alpha_M)$ be a set of complex variables, and $\Zvec = (\mathbf{X} + i\mathbf{Y}) / \sqrt{2}$ be an $L$-dimensional standard complex-valued Wiener process, containing two standard $L$-dimensional Wiener processes $\mathbf{X}$ and $\mathbf{Y}$.
	For the set of \abbrev{sde}s in the It\^o form
	\begin{eqn*}
		\upd\balpha = \avec(\balpha, t) \upd t + B(\balpha, t) \upd\Zvec(t),
	\end{eqn*}
	the differential of a function $f(\balpha)$ is
	\begin{eqn*}
		\upd f(\balpha) =
			2 \Real (\avec \cdot \vcwd_{\balpha}) f(\balpha) \upd t
			+ \Trace{ B B^H \vcwd_{\balpha^*} \vcwd_{\balpha}^T } f(\balpha) \upd t
			+ 2 \Real \Trace{ B \upd\Zvec \vcwd_{\balpha}^T } f(\balpha).
	\end{eqn*}
\end{theorem}
\begin{proof}
The proof follows the same scheme as \thmref{fpe-sde:corr:fpe-sde-complex}, just in the opposite direction.
Let $f = g + ih$, $\balpha = \mathbf{x} + i \mathbf{y}$, $\avec = \mathbf{u} + i \mathbf{v}$, $B = F + iG$, $\vcwd_{\balpha} = (\vcwd_{\mathbf{x}} - i \vcwd_{\mathbf{y}}) / 2$.
Then the set of \abbrev{sde}s from the statement is equivalent to
\begin{eqn}
	\upd \begin{pmatrix} \mathbf{x} \\ \mathbf{y} \end{pmatrix}
	= \begin{pmatrix} \mathbf{u} \\ \mathbf{v} \end{pmatrix} \upd t
		+ \frac{1}{\sqrt{2}} \begin{pmatrix} F & -G \\ G & F \end{pmatrix}
			\begin{pmatrix} \upd\mathbf{X} \\ \upd\mathbf{Y} \end{pmatrix}.
\end{eqn}
Applying \lmmref{fpe-sde:ito-formula:ito-f-real} for real-valued functions $g(\mathbf{x}, \mathbf{y})$ and $h(\mathbf{x}, \mathbf{y})$ and combining them into $f = g + ih$, we get
\begin{eqn}
	\upd f ={} &
		\begin{pmatrix} \mathbf{x} \\ \mathbf{y} \end{pmatrix} \cdot
			\begin{pmatrix} \vcwd_{\mathbf{x}} \\ \vcwd_{\mathbf{y}} \end{pmatrix} f \upd t
		+ \frac{1}{4} \Trace{
			\begin{pmatrix} F & -G \\ G & F \end{pmatrix}
			\begin{pmatrix} F^T & G^T \\ -G^T & F^T \end{pmatrix}
			\begin{pmatrix} \vcwd_{\mathbf{x}} \\ \vcwd_{\mathbf{y}} \end{pmatrix}
			\begin{pmatrix} \vcwd_{\mathbf{x}} \\ \vcwd_{\mathbf{y}} \end{pmatrix}^T
		} f \upd t  \\
	& + \frac{1}{\sqrt{2}} \Trace{
			\begin{pmatrix} F & -G \\ G & F \end{pmatrix}
			\begin{pmatrix} \upd\mathbf{X} \\ \upd\mathbf{Y} \end{pmatrix}
			\begin{pmatrix} \vcwd_{\mathbf{x}} \\ \vcwd_{\mathbf{y}} \end{pmatrix}^T
		} f.
\end{eqn}
Now let us match this equation and the lemma statement term by term.

First term:
\begin{eqn}
	2 \Real ( \avec \cdot \vcwd_{\balpha} )
	& = \Real \left(
			\left( \mathbf{u} + i\mathbf{v} \right) \cdot \left( \vcwd_{\mathbf{x}} - i \vcwd_{\mathbf{y}} \right)
		\right) \\
	& = \mathbf{u} \cdot \vcwd_{\mathbf{x}} + \mathbf{v} \cdot \vcwd_{\mathbf{y}} \\
	& = \begin{pmatrix} \mathbf{x} \\ \mathbf{y} \end{pmatrix} \cdot
		\begin{pmatrix} \vcwd_{\mathbf{x}} \\ \vcwd_{\mathbf{y}} \end{pmatrix}.
\end{eqn}

Second term:
\begin{eqn}
	\Trace{ B B^H \vcwd_{\balpha^*} \vcwd_{\balpha}^T }
	={} & \frac{1}{4} \Trace{
		(F F^T + G G^T)
		(\vcwd_{\mathbf{x}} \vcwd_{\mathbf{x}}^T
			+ \vcwd_{\mathbf{y}} \vcwd_{\mathbf{y}}^T)
		} \\
	& - \frac{1}{4} \Trace {
		(F G^T - G F^T)
		(\vcwd_{\mathbf{x}} \vcwd_{\mathbf{y}}^T
			- \vcwd_{\mathbf{y}} \vcwd_{\mathbf{x}}^T)
		} \\
	& + \frac{i}{4} \Trace{
		(F G^T - G F^T)
		(\vcwd_{\mathbf{x}} \vcwd_{\mathbf{x}}^T
			+ \vcwd_{\mathbf{y}} \vcwd_{\mathbf{y}}^T)
	} \\
	& + \frac{i}{4} \Trace{
		(G G^T + F F^T)
		(\vcwd_{\mathbf{x}} \vcwd_{\mathbf{y}}^T
			- \vcwd_{\mathbf{y}} \vcwd_{\mathbf{x}}^T)
	}.
\end{eqn}
Same as in \thmref{fpe-sde:corr:fpe-sde-complex} we notice that $F F^T + G G^T$ and $\vcwd_{\mathbf{x}} \vcwd_{\mathbf{x}}^T + \vcwd_{\mathbf{y}} \vcwd_{\mathbf{y}}^T$ are symmetric matrices, and $F G^T - G F^T$ and $\vcwd_{\mathbf{x}} \vcwd_{\mathbf{y}}^T - \vcwd_{\mathbf{y}} \vcwd_{\mathbf{x}}^T$ are antisymmetric ones.
Therefore, the last two terms contain traces of antisymmetric matrices and are, consequently, equal to zero:
\begin{eqn}
	={} & \frac{1}{4} \Trace{
		(F F^T + G G^T) \vcwd_{\mathbf{x}} \vcwd_{\mathbf{x}}^T
		+ (F G^T - G F^T) \vcwd_{\mathbf{y}} \vcwd_{\mathbf{x}}^T)
		} \\
	& + \frac{1}{4} \Trace {
		(G F^T - F G^T) \vcwd_{\mathbf{x}} \vcwd_{\mathbf{y}}^T
		+ (F F^T + G G^T) \vcwd_{\mathbf{y}} \vcwd_{\mathbf{y}}^T)
		} \\
	={} & \frac{1}{4} \Trace {
		\begin{pmatrix}
			F F^T + G G^T & F G^T - G F^T \\
			G F^T - F G^T & F F^T + G G^T
		\end{pmatrix}
		\begin{pmatrix}
			\vcwd_{\mathbf{x}} \vcwd_{\mathbf{x}}^T & \vcwd_{\mathbf{x}} \vcwd_{\mathbf{y}}^T \\
			\vcwd_{\mathbf{y}} \vcwd_{\mathbf{x}}^T & \vcwd_{\mathbf{y}} \vcwd_{\mathbf{y}}^T
		\end{pmatrix}
	} \\
	={} & \frac{1}{4} \Trace{
		\begin{pmatrix} F & -G \\ G & F \end{pmatrix}
		\begin{pmatrix} F^T & G^T \\ -G^T & F^T \end{pmatrix}
		\begin{pmatrix} \vcwd_{\mathbf{x}} \\ \vcwd_{\mathbf{y}} \end{pmatrix}
		\begin{pmatrix} \vcwd_{\mathbf{x}} \\ \vcwd_{\mathbf{y}} \end{pmatrix}^T
	}.
\end{eqn}

Third term:
\begin{eqn}
	2 \Real \Trace{ B \upd\Zvec \vcwd_{\balpha}^T }
	& = \frac{1}{\sqrt{2}} \Real \Trace{
		(F + iG) (\upd\mathbf{X} + i \upd\mathbf{Y}) (\vcwd_{\mathbf{x}} - i\vcwd_{\mathbf{y}})
	} \\
	& = \frac{1}{\sqrt{2}} \Trace{
		F \upd\mathbf{X} \vcwd_{\mathbf{x}} + F \upd\mathbf{Y} \vcwd_{\mathbf{y}}
		- G \upd\mathbf{Y} \vcwd_{\mathbf{x}} + G \upd\mathbf{X} \vcwd_{\mathbf{y}}
	} \\
	& = \frac{1}{\sqrt{2}} \Trace{
			\begin{pmatrix} F & -G \\ G & F \end{pmatrix}
			\begin{pmatrix} \upd\mathbf{X} \\ \upd\mathbf{Y} \end{pmatrix}
			\begin{pmatrix} \vcwd_{\mathbf{x}} \\ \vcwd_{\mathbf{y}} \end{pmatrix}^T
		}.
\end{eqn}

All terms have matched, thus proving the theorem.
\end{proof}

The above theorem can be reformulated for the multi-component \abbrev{sde}s from \thmref{fpe-sde:corr:mc-fpe-sde}.

\begin{theorem}
\label{thm:fpe-sde:ito-formula:mc-ito-f}
	Let $\balpha^{(j)},\, j = 1 \ldots C$ be $C$ sets of complex variables $\balpha^{(j)} \equiv (\alpha_1^{(j)} \ldots \alpha_{M_j}^{(j)})$.
	For the \abbrev{sde} in the It\^o form
	\begin{eqn*}
		\upd\balpha^{(j)} = \avec^{(j)} \upd t + B^{(j)} \upd\Zvec,
	\end{eqn*}
	the differential of a function $f(\balpha^{(1)}, \ldots, \balpha^{(C)})$ is
	\begin{eqn*}
		\upd f ={} &
			2 \sum_{j=1}^C \Real (\avec^{(j)} \cdot \vcwd_{\balpha^{(j)}}) f \upd t
			+ \sum_{j=1}^C \sum_{k=1}^C \Trace{
				B^{(j)} (B^{(k)})^H \vcwd_{(\balpha^{(k)})^*} \vcwd_{\balpha^{(j)}}^T } f \upd t \\
		& + 2 \sum_{j=1}^C \Real \Trace{ B^{(j)} \upd\Zvec \vcwd_{\balpha^{(j)}}^T } f.
	\end{eqn*}
\end{theorem}
\begin{proof}
Proved analogously to \thmref{fpe-sde:corr:mc-fpe-sde}, by combining $\balpha^{(j)}$ into a single vector and applying \thmref{fpe-sde:ito-formula:ito-f-complex}.
\end{proof}

Finally, we can use the multi-component reformulation to derive the It\^o formula in functional form for the set of \abbrev{sde}s from \thmref{fpe-sde:corr:fpe-sde-func}.

\begin{theorem}
\label{thm:fpe-sde:ito-formula:func-ito-f}
	Given the set of functional \abbrev{sde}s in the It\^o form
	\begin{eqn*}
		\upd f_j = \proj{\restbasis{j}} \left[
			\mathcal{A}_j \upd t + \sum_{l=1}^L \mathcal{B}_{jl} \upd Q_l
		\right],
	\end{eqn*}
	the differential of a functional operator $\mathcal{F}[\fvec]$ is
	\begin{eqn*}
		\upd \mathcal{F}[\fvec]
		={} & \int \upd\xvec^\prime \left(
			2 \sum_{j=1}^C \Real \left(
				\mathcal{A}_j^\prime \frac{\fdelta}{\fdelta f_j^\prime}
			\right) \mathcal{F}[\fvec] \upd t \right. \\
		& + \sum_{j=1}^C \sum_{k=1}^C \sum_{l=1}^L
				\mathcal{B}_{jl}^\prime
				\mathcal{B}_{kl}^{\prime *}
				\frac{\fdelta}{\fdelta f_j^\prime}
				\frac{\fdelta}{\fdelta f_k^{\prime *}} \mathcal{F}[\fvec] \upd t \\
		& \left. + 2 \sum_{j=1}^C \sum_{l=1}^L
			\Real \left(
				\mathcal{B}_{jl}^\prime
				\upd Q_l^\prime
				\frac{\fdelta}{\fdelta f_j^\prime}
			\right)
			\mathcal{F}[\fvec]
		\right).
	\end{eqn*}
	Here $f_j$, $\mathcal{A}_j$, $\mathcal{B}_{jl}$, $L$, and $Q_l$ are defined in the same way as in \thmref{fpe-sde:corr:fpe-sde-func}.
\end{theorem}
\begin{proof}
The set of \abbrev{sde}s can be rewritten in terms of complex vectors as
\begin{eqn}
	\upd\alpha_{\mvec}^{(j)}
	= \int \upd\xvec\, \phi_{j,\mvec}^* \mathcal{A}_j \upd t
	+ \sum_{l=1}^L \sum_{\pvec \in \fullbasis}
		\int \upd\xvec\, \phi_{j,\mvec}^* \mathcal{B}_{jl} \phi_{\pvec} \upd Z_{\pvec,l},\quad
	\mvec \in \restbasis_j.
\end{eqn}
Now, treating the functional operator as a function of $C$ complex vectors
\begin{eqn}
	\mathcal{F} \equiv \mathcal{F}[\mathcal{C}_{\restbasis_1}(\balpha^{(1)}), \ldots, \mathcal{C}_{\restbasis_C}(\balpha^{(C)})],
\end{eqn}
we can use \thmref{fpe-sde:ito-formula:mc-ito-f} with the drift vectors
\begin{eqn}
	a_{\mvec}^{(j)} = \int \upd\xvec\, \phi_{j,\mvec}^* \mathcal{A}_j,
\end{eqn}
and the noise matrices
\begin{eqn}
	B_{\mvec,(\pvec,l)}^{(j)}
	= \int \upd\xvec\, \phi_{j,\mvec}^* \mathcal{B}_{jl} \phi_{\pvec}.
\end{eqn}
Applying \thmref{fpe-sde:ito-formula:mc-ito-f}:
\begin{eqn}
	\upd \mathcal{F}
	={} &
		2 \sum_{j=1}^C \sum_{\mvec \in \restbasis_j} \Real \left(
			\int \upd\xvec^\prime \phi_{j,\mvec}^{\prime*} \mathcal{A}_j^\prime
			\frac{\cwd}{\cwd \alpha_{j,\mvec}}
		\right) \mathcal{F} \upd t \\
	& + \sum_{j=1}^C \sum_{k=1}^C
			\sum_{\mvec \in \restbasis_j} \sum_{\nvec \in \restbasis_k}
			\sum_{l=1}^L \sum_{\pvec \in \fullbasis}
			\int \upd\xvec^\prime \phi_{j,\mvec}^{\prime *} \mathcal{B}_{jl}^\prime \phi_{\pvec}^\prime
			\int \upd\xvec^{\prime\prime} \phi_{k,\nvec}^{\prime\prime} \mathcal{B}_{kl}^{\prime\prime *} \phi_{\pvec}^{\prime\prime *}
			\frac{\cwd^2 \mathcal{F}}{\cwd \alpha_{k,\nvec}^* \cwd \alpha_{j,\mvec}}
			\upd t \\
	& + 2 \sum_{j=1}^C \Real \left(
			\sum_{\mvec \in \restbasis_j}
			\sum_{l=1}^L \sum_{\pvec \in \fullbasis}
			\int \upd\xvec^\prime \phi_{j,\mvec}^{\prime*} \mathcal{B}_{jl}^\prime \phi_{\pvec}^\prime
			\upd Z_{\pvec,l}
			\frac{\cwd}{\cwd \alpha_{j,\mvec}}
			\mathcal{F}
	\right).
\end{eqn}
Recognising definitions of the functional differentials, the functional Wiener process, and the delta function, we get
\begin{eqn}
	={} & 2 \sum_{j=1}^C \Real \left(
			\int \upd\xvec^\prime \mathcal{A}_j^\prime
			\frac{\fdelta}{\fdelta f_j^\prime}
		\right) \mathcal{F} \upd t
	+ \sum_{j=1}^C \sum_{k=1}^C \sum_{l=1}^L
			\int \upd\xvec^\prime \mathcal{B}_{jl}^\prime
			\mathcal{B}_{kl}^{\prime *}
			\frac{\fdelta}{\fdelta f_j^\prime}
			\frac{\fdelta}{\fdelta f_k^{\prime *}} \mathcal{F} \upd t \\
	& + 2 \sum_{j=1}^C \sum_{l=1}^L \Real \left(
			\int \upd\xvec^\prime \mathcal{B}_{jl}^\prime
			\upd Q_l^\prime
			\frac{\fdelta}{\fdelta f_j^\prime}
		\right) \mathcal{F},
\end{eqn}
which leads to the statement of the theorem.
\end{proof}

Alternatively, the functional It\^o formula can be written in the matrix form as
\begin{eqn}
	\upd \mathcal{F}[\fvec]
	={} & \int \upd\xvec^\prime \left(
		2 \Real \left(
			\mathbfcal{A}^\prime \cdot \vfdelta_{\bPsi^\prime}
		\right) \mathcal{F}[\fvec] \upd t
		+ \Trace{
			\mathcal{B}^\prime
			(\mathcal{B}^\prime)^H
			\vfdelta_{\fvec^{\prime *}}
			\vfdelta_{\fvec^\prime}^T
		} \mathcal{F}[\fvec] \upd t \right. \\
	& \left. + 2 \Real \Trace{
			\mathcal{B}^\prime
			\upd\mathbf{Q}^\prime
			\vfdelta_{\fvec^\prime}^T
		} \mathcal{F}[\fvec]
	\right).
\end{eqn}

