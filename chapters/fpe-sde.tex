% =============================================================================
\chapter{Functional FPE to SDE correspondences}
\label{cha:appendix:fpe-sde}
% =============================================================================

Wigner transformation, to which the majority of this thesis is dedicated to, produces a Fokker-Planck equation (\abbrev{fpe}), or its functional equivalent, from an initial master equation.
FPE is an equation in partial derivatives, and, in general, is not easy to solve --- even numerically.
The major part of the usefullness of the Wigner transformation paired with the Wigner truncation is that it produces \abbrev{fpe} in a special form, which can be further transformed to a set of stochastic differential equations (\abbrev{sde}s), with the Wigner function playing the role of a probability distribution.
Algorithms of solving such equations numerically are much more straightforward.

The actual correspondence between \abbrev{fpe} and \abbrev{sde}s is formulated and proved for real-valued coefficients in literature~\cite{Risken1996}.
In this thesis we will need to transform \abbrev{fpe}s with complex coefficients, or even functional operator ones.
While it is always possible to express them in real-valued form, it is much more convenient to derive correspondence theorems that work directly on such \abbrev{fpe}s.
In this Appendix we will do that by proceeding successively from the initial real-valued theorem to complex-valued and functional correspondences.

In addition, we will do the same for the It\^o formula, which provides the expression for the time derivative of any function of transverse variables.
This formula is useful, among other cases, if one wants to derive the time dependence of some integral observable (for instance, population), without solving \abbrev{sde}s themselves.
Alternatively, it can serve as an additional test of a numerical algorithm used to propagate \abbrev{sde}s in time.

% =============================================================================
\section{Correspondences}
% =============================================================================

We will start by formulating the known real-valued correspondence in a form which is more convenient for further proofs in this section, and also close to the results one obtains from the Wigner transformation.

\begin{lemma}[real-valued \abbrev{fpe}--\abbrev{sde}s correspondence in convenient form.]
\label{lmm:fpe-sde:corr:fpe-sde-real}
	Let $\zvec^T \equiv (z_1 \ldots z_M)$ be a set of real variables.
	Then the \abbrev{fpe}
	\begin{eqn*}
		\frac{\upd W}{\upd t}
		= -\vcwd_{\zvec}^T \cdot \avec W
		+ \frac{1}{2} \Trace{ \vcwd_{\zvec} \vcwd_{\zvec}^T B B^T } W
	\end{eqn*}
	is equivalent to the set of \abbrev{sde}s in It\^{o} form
	\begin{eqn*}
		\upd\zvec = \avec \upd t + B \upd\Zvec
	\end{eqn*}
	and to the set of \abbrev{sde}s in Stratonovich form
	\begin{eqn*}
		\upd\zvec = (\avec - \svec)\upd t + B \upd\Zvec,
	\end{eqn*}
	where the noise-induced (Stratonovich) drift vector $\svec$ has elements
	\begin{eqn*}
		s_j
		= \frac{1}{2} \sum_{k,i} B_{ki} \frac{\cwd}{\cwd z_k} B_{ji}
		= \frac{1}{2} \Trace{B^T \vcwd_{\zvec} \evec_j^T B},
	\end{eqn*}
	$\evec_i$ being the unit vector with elements $(\evec_j)_i = \delta_{ij}$.
	Here $W \equiv W(\zvec)$ is a probability distribution, $\avec \equiv \avec(\zvec)$ is a vector function, $B \equiv B(\zvec)$ is a matrix function ($B$ having size $M \times L$, where $L$ corresponds to the number of noise sources), $\vcwd_{\zvec}^T \equiv (\upd/\upd z_1 \ldots \upd/\upd z_M)$ is a cogradient vector, and $\Zvec$ is a standard $L$-dimensional Wiener process with $\langle \upd Z_j^2 \rangle = \upd t$.
\end{lemma}
\begin{proof}
For the detailed proof see~\cite{Risken1996}, sections 3.3 and 3.4.
\end{proof}

The above theorem can be extended to work with complex Wirtinger derivatives and complex-valued coefficients.
Of couse, in order to produce the real-valued $\upd W/\upd t$ in the left part, a \abbrev{fpe} must have a particular form.

\begin{theorem}
\label{thm:fpe-sde:corr:fpe-sde-complex}
	Let $\balpha^T \equiv (\alpha_1 \ldots \alpha_M)$ be a set of complex variables.
	Then the \abbrev{fpe}
	\begin{eqn*}
		\frac{\upd W}{\upd t}
		= -\vcwd_{\balpha}^T \avec W - \vcwd_{\balpha^*}^T \avec^* W
		+ \Trace{ \vcwd_{\balpha^*} \vcwd_{\balpha}^T B B^H } W
	\end{eqn*}
	is equivalent to the set of \abbrev{sde}s in It\^{o} form
	\begin{eqn*}
		\upd\balpha = \avec \upd t + B \upd\Zvec,
	\end{eqn*}
	and to the set of \abbrev{sde}s in Stratonovich form
	\begin{eqn*}
		\upd\balpha = (\avec - \svec) \upd t + B \upd\Zvec,
	\end{eqn*}
	where the Stratonovich term has elements
	\begin{eqn*}
		s_j = \frac{1}{2} \Trace{ B^H \vcwd_{\balpha^*} \evec_j^T B },
	\end{eqn*}
	and $\Zvec = (\mathbf{X} + i\mathbf{Y}) / \sqrt{2}$ is an $L$-dimensional standard complex-valued Wiener process (with $\langle \upd Z_j \upd Z_k^* \rangle = \delta_{jk} \upd t$), containing two standard $L$-dimensional Wiener processes $\mathbf{X}$ and $\mathbf{Y}$.
\end{theorem}
\begin{proof}
Let us expand the \abbrev{fpe} using real values $\balpha = \xvec + i \yvec$, $\avec = \mathbf{u} + i \mathbf{v}$, $B = F + iG$, $\vcwd_{\balpha} = (\vcwd_{\xvec} - i \vcwd_{\yvec}) / 2$.
Thus
\begin{eqn}
	\frac{\upd W}{\upd t}
	={} & - \vcwd_{\xvec}^T \mathbf{u} W
	- \vcwd_{\yvec}^T \mathbf{v} W
	+ \frac{1}{4} \Trace{
		(\vcwd_{\xvec} \vcwd_{\xvec}^T
			+ \vcwd_{\yvec} \vcwd_{\yvec}^T)
		(F F^T + G G^T) \right. \\
	& \left. - (\vcwd_{\xvec} \vcwd_{\yvec}^T
			- \vcwd_{\yvec} \vcwd_{\xvec}^T)
		(F G^T - G F^T)
	} W \\
	& + \frac{i}{4} \Trace{
		(\vcwd_{\xvec} \vcwd_{\xvec}^T
			+ \vcwd_{\yvec} \vcwd_{\yvec}^T)
		(F G^T - G F^T)
	} W \\
	& + \frac{i}{4} \Trace{
		(\vcwd_{\xvec} \vcwd_{\yvec}^T
			- \vcwd_{\yvec} \vcwd_{\xvec}^T)
		(F F^T + G G^T)
	} W.
\end{eqn}
Since $F F^T + G G^T$ and $\vcwd_{\xvec} \vcwd_{\xvec}^T + \vcwd_{\yvec} \vcwd_{\yvec}^T$ are symmetric matrices, and $F G^T - G F^T$ and $\vcwd_{\xvec} \vcwd_{\yvec}^T - \vcwd_{\yvec} \vcwd_{\xvec}^T$ are antisymmetric, corresponding traces are equal to zero, which gives us \abbrev{fpe} in real variables
\begin{eqn}
	\frac{\upd W}{\upd t}
	={} & - \vcwd_{\xvec}^T \mathbf{u} W
	- \vcwd_{\yvec}^T \mathbf{v} W
	+ \frac{1}{4} \Trace{
		(\vcwd_{\xvec} \vcwd_{\xvec}^T
			+ \vcwd_{\yvec} \vcwd_{\yvec}^T)
		(F F^T + G G^T) \right. \\
	& \left. - (\vcwd_{\xvec} \vcwd_{\yvec}^T
			- \vcwd_{\yvec} \vcwd_{\xvec}^T)
		(F G^T - G F^T)
	} W.
\end{eqn}

In order to use \lmmref{fpe-sde:corr:fpe-sde-real},
we need to join variables $\xvec$ and $\yvec$ into one variable vector $\zvec \equiv \xvec \oplus \yvec$.
This will give us the equation in the form identical to that from the lemma, with the drift vector $\tilde{\avec} \equiv \mathbf{u} \oplus \mathbf{v}$ and the diffusion matrix
\begin{eqn}
	\tilde{B} \tilde{B}^T \equiv \frac{1}{2} \begin{pmatrix}
		F F^T + G G^T & F G^T - G F^T \\
		G F^T - F G^T & F F^T + G G^T
	\end{pmatrix},
\end{eqn}
which gives the noise matrix
\begin{eqn}
	\tilde{B} = \frac{1}{\sqrt{2}} \begin{pmatrix}
		F & -G \\
		G & F
	\end{pmatrix}.
\end{eqn}
Therefore the equivalent \abbrev{sde}s in It\^{o} form are
\begin{eqn}
	d\zvec = \tilde{\avec} dt + \tilde{B} d\tilde{\Zvec},
\end{eqn}
where $d\tilde{\Zvec} \equiv d\mathbf{X} \oplus d\mathbf{Y}$.
Returning to our previous variables:
\begin{eqn}
	d\xvec & = \mathbf{u} dt + \frac{1}{\sqrt{2}} F d\mathbf{X} - \frac{1}{\sqrt{2}} G d\mathbf{Y}, \\
	d\yvec & = \mathbf{v} dt + \frac{1}{\sqrt{2}} G d\mathbf{X} + \frac{1}{\sqrt{2}} F d\mathbf{Y}.
\end{eqn}
Multiplying the second equation by $i$ and adding it to the first one:
\begin{eqn}
	d\balpha = \avec dt + \frac{1}{\sqrt{2}} (F + iG) (d\mathbf{X} + id\mathbf{Y}),
\end{eqn}
which leads to the It\^{o} part of the theorem statement.
\begin{eqn}
	d\balpha = \avec dt + B d\Zvec.
\end{eqn}

Noise-induced drift term in Stratonovich case can be calculated by substituting $\tilde{B}$ into the expression for $s_j$ from \lmmref{fpe-sde:corr:fpe-sde-real}.
We will calculate $s_j$ with $j$ belonging to $\xvec$ and $\yvec$ part of coordinate space separately.
\begin{eqn}
	s_j^{(x)}
	= \frac{1}{4} \Trace{
		\begin{pmatrix}
			F^T & G^T \\ -G^T & F^T
		\end{pmatrix}
		\begin{pmatrix}
			\vcwd_{\xvec} \\
			\vcwd_{\yvec}
		\end{pmatrix}
		\begin{pmatrix}
			\evec_j^T & 0
		\end{pmatrix}
		\begin{pmatrix}
			F & -G \\ G & F
		\end{pmatrix}
	}
\end{eqn}
Multiplying matrices:
\begin{eqn}
	={} & \frac{1}{4} \Trace{
		\begin{pmatrix}
			F^T & G^T \\ -G^T & F^T
		\end{pmatrix}
		\begin{pmatrix}
			\vcwd_{\xvec} \\
			\vcwd_{\yvec}
		\end{pmatrix}
		\begin{pmatrix}
			\evec_j^T F & - \evec_j^T G
		\end{pmatrix}
	} \\
	={} & \frac{1}{4} \Trace{
		\begin{pmatrix}
			F^T & G^T \\ -G^T & F^T
		\end{pmatrix}
		\begin{pmatrix}
			\vcwd_{\xvec} \evec_j^T F & - \vcwd_{\xvec} \evec_j^T G \\
			\vcwd_{\yvec} \evec_j^T F & - \vcwd_{\yvec} \evec_j^T G
		\end{pmatrix}
	} \\
	={} & \frac{1}{4} \left(
		\Trace{ F^T \vcwd_{\xvec} \evec_j^T F }
		+ \Trace{ G^T \vcwd_{\yvec} \evec_j^T F } \right. \\
	& \left. + \Trace{ G^T \vcwd_{\xvec} \evec_j^T G }
		- \Trace{ F^T \vcwd_{\yvec} \evec_j^T G }
	\right).
\end{eqn}
Similarly for the $\yvec$ part,
\begin{eqn}
	s_j^{(y)}
	={} & \frac{1}{4} \left(
		\Trace{ F^T \vcwd_{\xvec} \evec_j^T G }
		+ \Trace{ G^T \vcwd_{\yvec} \evec_j^T G } \right. \\
	& \left. - \Trace{ G^T \vcwd_{\xvec} \evec_j^T F }
		+ \Trace{ F^T \vcwd_{\yvec} \evec_j^T F }
	\right).
\end{eqn}
Therefore the final term in complex-valued \abbrev{sde}s is
\begin{eqn}
	s_j
	= s_j^{(x)} + i s_j^{(y)}
	= \frac{1}{2} \Trace{ B^H \vcwd_{\balpha^*} \evec_j^T B },
\end{eqn}
which finishes the proof.
\end{proof}

Note the asymmetry in the expression for Stratonovich term: if $B = B(\alpha)$, then $\mathbf{s} \equiv 0$.
It is initially caused by the asymmetry in the target \abbrev{sde}s.
Truly general form of \abbrev{fpe} would be
\begin{eqn}
	\frac{\upd W}{\upd t}
	={} & - 2 \Real \left( \vcwd_{\balpha}^T \avec \right) W
	+ \Trace{ \vcwd_{\balpha^*} \vcwd_{\balpha}^T B_1 B_1^H } W
	+ \Trace{ \vcwd_{\balpha^*} \vcwd_{\balpha}^T B_2 B_2^H } W \\
	& + 2 \Real \left(
		\Trace{ \vcwd_{\balpha} \vcwd_{\balpha}^T B_1 B_2^T }
		+ \Trace{ \vcwd_{\balpha} \vcwd_{\balpha}^T B_2 B_1^T }
	\right) W,
\end{eqn}
which corresponds to the system of \abbrev{sde}s
\begin{eqn}
	\upd\balpha = (\avec - \svec) \upd t + B_1 \upd\Zvec + B_2 \upd\Zvec^*,
\end{eqn}
where the Stratonovich term has elements
\begin{eqn}
	s_j ={} & \frac{1}{2} \left(
		\Trace{ B_1^H \vcwd_{\balpha^*} \evec_j^T B_1 }
		+ \Trace{ B_2^H \vcwd_{\balpha^*} \evec_j^T B_2 } \right. \\
		& \left. + \Trace{ B_1^T \vcwd_{\balpha} \evec_j^T B_2 }
		+ \Trace{ B_2^T \vcwd_{\balpha} \evec_j^T B_1 }
	\right).
\end{eqn}
In the theorem above we limited the space of possible \abbrev{sde}s to those with $B_2 \equiv 0$, leading to the observed asymmetry.

In many applications (some of which are discussed in this thesis), it is advantageous to enumerate state vector of the system with two variables instead of one: mode numbers and component numbers.
This helps to describe particles which can occupy the same set of modes, but are otherwise distinguishable.
We will now reformulate the previous theorem, including this component distinction.

\begin{theorem}[Multi-component reformulation of \thmref{fpe-sde:corr:fpe-sde-complex}]
\label{thm:fpe-sde:corr:mc-fpe-sde}
	Let $\balpha^{(j)},\, j = 1 \ldots C$ be $C$ sets of complex variables $\balpha^{(j)} \equiv (\alpha_1^{(j)} \ldots \alpha_{M_j}^{(j)})$.
	Then the \abbrev{fpe}
	\begin{eqn*}
		\frac{\upd W}{\upd t}
		={} & - \sum_{j=1}^C \vcwd_{\balpha^{(j)}}^T \avec^{(j)} W
		- \sum_{j=1}^C \vcwd_{(\balpha^{(j)})^*}^T (\avec^{(j)})^* W \\
		& + \sum_{j=1}^C \sum_{k=1}^C
			\Trace{
				\vcwd_{(\balpha^{(j)})^*}
				\vcwd_{\balpha^{(k)}}^T
				B^{(k)} (B^{(j)})^H
			} W
	\end{eqn*}
	is equivalent to the set of \abbrev{sde}s in It\^{o} form
	\begin{eqn*}
		\upd\balpha^{(j)} = \avec^{(j)} \upd t + B^{(j)} \upd\Zvec,
	\end{eqn*}
	or to the set of \abbrev{sde}s in Stratonovich form
	\begin{eqn*}
		\upd\balpha^{(j)} = (\avec^{(j)} - \svec^{(j)}) \upd t + B^{(j)} \upd\Zvec,
	\end{eqn*}
	where the Stratonovich term has elements
	\begin{eqn*}
		s_i^{(j)} = \frac{1}{2} \sum_{k=1}^C
			\Trace{ (B^{(k)})^H \vcwd_{(\balpha^{(k)})^*} \evec_i^T B^{(j)} }.
	\end{eqn*}
	Here $\upd\Zvec$ is an $L$-dimensional standard complex-valued Wiener process, and noise matrices $B^{(j)}$ have sizes $M_j \times L$.
\end{theorem}
\begin{proof}
Let us join all variable sets $\balpha^{(j)}$ into a single set
\begin{eqn}
	\balpha \equiv \bigoplus_{j=1}^C \balpha^{(j)}.
\end{eqn}
Then we can use \thmref{fpe-sde:corr:fpe-sde-complex} with the drift vector
\begin{eqn}
	\avec = \bigoplus_{j=1}^C \avec^{(j)},
\end{eqn}
the cogradient vector
\begin{eqn}
	\vcwd_{\balpha} = \bigoplus_{j=1}^C \vcwd_{\balpha^{(j)}},
\end{eqn}
and the noise matrix
\begin{eqn}
	B = \begin{pmatrix}
		B^{(1)} \\ \vdots \\ B^{(C)}
	\end{pmatrix}.
\end{eqn}
This gives us \abbrev{sde}s in It\^{o} form
\begin{eqn}
	\upd\balpha = \avec \upd t + B \upd\Zvec,
\end{eqn}
where $d\Zvec$ is an $L$-dimensional standard complex-valued Wiener process.
Splitting this equation for different components, we get the It\^{o} part of the theorem statement.
Subsituting $B$ into the expression for the Stratonovich term:
\begin{eqn}
	s_i^{(j)}
	& = \frac{1}{2} \Trace{
		\begin{pmatrix} (B^{(1)})^H & \cdots & (B^{(C)})^H \end{pmatrix}
		\begin{pmatrix}
			\vcwd_{(\balpha^{(1)})^*} \\
			\vdots \\
			\vcwd_{(\balpha^{(C)})^*}
		\end{pmatrix}
		\begin{pmatrix} 0 & \cdots & \evec_i^T & \cdots & 0 \end{pmatrix}
		\begin{pmatrix}
			B^{(1)} \\
			\vdots \\
			B^{(C)}
		\end{pmatrix}
	}.
\end{eqn}
Multiplying matrices successively:
\begin{eqn}
	& = \frac{1}{2} \Trace{
		\begin{pmatrix} (B^{(1)})^H & \cdots & (B^{(C)})^H \end{pmatrix}
		\begin{pmatrix}
			\vcwd_{(\balpha^{(1)})^*} \evec_i^T B^{(j)} \\
			\vdots \\
			\vcwd_{(\balpha^{(C)})^*} \evec_i^T B^{(j)}
		\end{pmatrix}
	} \\
	& = \frac{1}{2} \sum_{k=1}^C \Trace{
		(B^{(k)})^H
		\vcwd_{(\balpha^{(k)})^*}
		\evec_i^T
		B^{(j)}
	},
\end{eqn}
which is the expression from the theorem statement.
\end{proof}

Most of the time we will deal with \abbrev{fpe}s in functional form, so we will reformulate the correspondence once again, now using functional derivatives.

\begin{theorem}
\label{thm:fpe-sde:corr:fpe-sde-func}
	The \abbrev{fpe} in functional form
	\begin{eqn*}
		\frac{\upd W}{\upd t}
		={} & \int \upd\xvec \left(
			- \sum_{j=1}^C \frac{\fdelta}{\fdelta \Psi_j} \mathcal{A}_j W
			- \sum_{j=1}^C \frac{\fdelta}{\fdelta \Psi_j^*} \mathcal{A}_j^* W \right. \\
		& \left. + \sum_{j=1}^C \sum_{k=1}^C \frac{\fdelta^2}{\fdelta \Psi_j^* \fdelta \Psi_k}
				\sum_{\lvec \in \mathbb{L}} \mathcal{B}_{k,\lvec} \mathcal{B}_{j,\lvec}^* W
		\right)
	\end{eqn*}
	is equivalent to the set of \abbrev{sde}s in It\^{o} form
	\begin{eqn*}
		\upd\Psi_j = \mathcal{P}_{\restbasis_j} \left[
			\mathcal{A}_j \upd t
			+ \sum_{\lvec \in \mathbb{L}} \mathcal{B}_{j,\lvec} \upd Q_{\lvec}
		\right],
	\end{eqn*}
	or the set of \abbrev{sde}s in It\^{o} form
	\begin{eqn*}
		\upd\Psi_j = \mathcal{P}_{\restbasis_j} \left[
			(\mathcal{A}_j - \mathcal{S}_j) \upd t
			+ \sum_{\lvec \in \mathbb{L}} \mathcal{B}_{j,\lvec} \upd Q_{\lvec}
		\right],
	\end{eqn*}
	where
	\begin{eqn*}
		\mathcal{S}_j = \frac{1}{2} \sum_{k=1}^C \sum_{\lvec \in \mathbb{L}}
			\mathcal{B}_{k,\lvec}^*
			\frac{\fdelta}{\fdelta \Psi_k^*}
			\mathcal{B}_{j,\lvec}.
	\end{eqn*}
	Here $\Psi_j \in \mathbb{F}_{\mathbb{M}_j}$, $\mathcal{A}_j \equiv \mathcal{A}_j[\bPsi]$ and $\mathcal{B}_{j,\lvec} \equiv \mathcal{B}_{j,\lvec}[\bPsi]$ are functional operators, $W \equiv W[\bPsi]$ is a probability functional, and $\mathbb{L}$ is a set of noise sources.
	Standard functional Wiener processes $Q_{\lvec}$ are compositions of standard complex-valued Wiener processes:
	\begin{eqn*}
		Q_{\lvec} = \sum_{\nvec \in \fullbasis} \phi_{\nvec} Z_{\lvec,\nvec}.
	\end{eqn*}
\end{theorem}
\begin{proof}
Expanding functional derivatives according to \defref{func-calculus:func-diff} with $\Psi_j = \sum_{\nvec \in \restbasis_j} \phi_{j,\nvec} \alpha_{j,\nvec}$:
\begin{eqn}
	\frac{\upd W}{\upd t}
	={} & \left(
		- \sum_{j=1}^C \sum_{\nvec \in \restbasis_j}
			\frac{\cwd}{\cwd \alpha_{j,\nvec}}
			\int \upd\xvec\, \phi_{j,\nvec}^* \mathcal{A}_j W
		- \sum_{j=1}^C \sum_{\nvec \in \restbasis_j}
			\frac{\cwd}{\cwd \alpha_{j,\nvec}^*}
			\int \upd\xvec\, \phi_{j,\nvec} \mathcal{A}_j^* W
		\right. \\
	&	\left. + \sum_{j=1}^C \sum_{k=1}^C
			\sum_{\mvec \in \restbasis_j, \nvec \in \restbasis_k}
			\frac{\cwd}{\cwd \alpha_{j,\mvec}^*}
			\frac{\cwd}{\cwd \alpha_{k,\nvec}}
			\int \upd\xvec
			\phi_{j,\mvec} \phi_{k,\nvec}^*
			\sum_{\lvec \in \mathbb{L}} \mathcal{B}_{j,\lvec}^* \mathcal{B}_{k,\lvec} W
	\right).
\end{eqn}
The diffusion term has to be transformed in order to conform to \thmref{fpe-sde:corr:mc-fpe-sde}:
\begin{eqn}
	\int \upd\xvec \phi_{j,\mvec} \phi_{k,\nvec}^* \sum_{\lvec \in \mathbb{L}}
		\mathcal{B}_{k,\lvec} \mathcal{B}_{j,\lvec}^*
	& = \int \upd\xvec \int \upd\xvec^\prime
			\phi_{j,\mvec}^\prime \phi_{k,\nvec}^*
			\sum_{\lvec \in \mathbb{L}} \mathcal{B}_{j,\lvec}^{\prime *} \mathcal{B}_{k,\lvec}
			\delta(\xvec - \xvec^\prime) \\
	& = \int \upd\xvec \int \upd\xvec^\prime
			\phi_{j,\mvec}^\prime \phi_{k,\nvec}^*
			\sum_{\lvec \in \mathbb{L}} \mathcal{B}_{j,\lvec}^{\prime *} \mathcal{B}_{k,\lvec}
			\sum_{\pvec \in \fullbasis} \phi_{\pvec}^{\prime*} \phi_{\pvec} \\
	& = \sum_{\lvec \in \mathbb{L}} \sum_{\pvec \in \fullbasis}
		\int \upd\xvec\,
			\phi_{j,\mvec} \mathcal{B}_{j,\lvec}^* \phi_{\pvec}^*
		\int \upd\xvec\,
			\phi_{k,\nvec}^* \mathcal{B}_{k,\lvec} \phi_{\pvec}.
\end{eqn}
Note that we did not specify the index of the full basis used to expand the delta function.
It can be any orthonormal and complete basis, in particular one of $\fullbasis_j$ --- this will not change the result.

Now we have the \abbrev{fpe} in the from required by \thmref{fpe-sde:corr:mc-fpe-sde} with
\begin{eqn}
	a_{\mvec}^{(j)}
	= \int \upd\xvec\, \phi_{j,\mvec}^* \mathcal{A}_j,\,
	\mvec \in \restbasis_j,
\end{eqn}
and
\begin{eqn}
\label{eqn:fpe-sde:corr:func-noise-matrix}
	B_{\mvec,(\pvec,\lvec)}^{(j)}
	= \int \upd\xvec\, \phi_{j,\mvec}^* \mathcal{B}_{j,\lvec} \phi_{\pvec},\,
	\mvec \in \restbasis_j, \pvec \in \fullbasis, \lvec \in \mathbb{L}.
\end{eqn}
Note that columns of $B$ are enumerated using the compound index $\pvec,\lvec$.

Therefore the initial \abbrev{fpe} is equivalent to the set of \abbrev{sde}s in It\^{o} form
\begin{eqn}
	\upd\alpha_{\mvec}^{(j)}
	= \int \upd\xvec\, \phi_{j,\mvec}^* \mathcal{A}_j \upd t
	+ \sum_{\pvec \in \fullbasis, \lvec \in \mathbb{L}}
		\int \upd\xvec\, \phi_{j,\mvec}^* \mathcal{B}_{j,\lvec} \phi_{\pvec} \upd Z_{\pvec,\lvec}.
\end{eqn}
Multiplying by $\phi_{j,\mvec}^\prime$ and grouping by component:
\begin{eqn}
	\sum_{\mvec \in \restbasis_j} \phi_{j,\mvec}^\prime \upd\alpha_{\mvec}^{(j)}
	={} & \sum_{\mvec \in \restbasis_j} \phi_{j,\mvec}^\prime \int \upd\xvec\, \phi_{j,\mvec}^* \mathcal{A}_j \upd t \\
	& + \sum_{\mvec \in \restbasis_j} \phi_{j,\mvec}^\prime \int \upd\xvec\, \phi_{j,\mvec}^*
		\sum_{\lvec \in \mathbb{L}} \sum_{\pvec \in \fullbasis}
			\mathcal{B}_{j,\lvec} \phi_{\pvec} \upd Z_{\pvec,\lvec}.
\end{eqn}
Recognizing \defref{func-calculus:projector} of the projection operator:
\begin{eqn}
	\upd\Psi_j
	= \proj{\restbasis_j} \left[
		\mathcal{A}_j \upd t
		+ \sum_{\lvec \in \mathbb{L}} \mathcal{B}_{j,\lvec}
			\sum_{\pvec \in \fullbasis} \phi_{\pvec} \upd Z_{\pvec,\lvec}
	\right].
\end{eqn}
Defining the standard functional Wiener process as $Q_{\lvec} = \sum_{\pvec \in \fullbasis} \phi_{\pvec} \upd Z_{\pvec,\lvec}$:
\begin{eqn}
	\upd\Psi_c
	= \proj{\restbasis_j} \left[
		\mathcal{A}_j \upd t
		+ \sum_{\lvec \in \mathbb{L}} \mathcal{B}_{j,\lvec} \upd Q_{\lvec}
	\right].
\end{eqn}

Performing the same multiplication and summation on the Stratonovich term from \thmref{fpe-sde:corr:mc-fpe-sde}:
\begin{eqn}
	\mathcal{S}_j
	= \sum_{\mvec \in \restbasis_j} \phi_{j,\mvec}^\prime s_{\mvec}^{(j)}
	= \frac{1}{2} \sum_{\mvec \in \restbasis_j} \phi_{j,\mvec}^\prime \sum_{k=1}^C \Trace{
		(B^{(k)})^H \vcwd_{(\balpha^{(k)})^*} \evec_{\mvec}^T B^{(j)}
	}.
\end{eqn}
Transforming the trace to a summation:
\begin{eqn}
	= \frac{1}{2} \sum_{\mvec \in \restbasis_c} \phi_{j,\mvec}^\prime \sum_{k=1}^C
		\sum_{\nvec \in \restbasis_k} \sum_{\lvec \in \mathbb{L}} \sum_{\pvec \in \fullbasis}
			(B_{\nvec (\pvec,\lvec)}^{(k)})^*
			\frac{\cwd}{\cwd (\alpha_{\nvec}^{(k)})^*}
			B_{\mvec (\pvec,\lvec)}^{(j)}.
\end{eqn}
Using the multimode form~\eqnref{fpe-sde:corr:func-noise-matrix} of the noise matrix:
\begin{eqn}
	= \frac{1}{2} \sum_{\mvec \in \restbasis_j} \phi_{j,\mvec}^\prime \sum_{k=1}^C
		\sum_{\nvec \in \restbasis_k} \sum_{\lvec \in \mathbb{L}} \sum_{\pvec \in \fullbasis}
			\int \upd\xvec\, \phi_{k,\nvec} \mathcal{B}_{k,\lvec}^* \phi_{\pvec}^*
			\int \upd\xvec\, \phi_{j,\mvec}^*
				\frac{\cwd}{\cwd (\alpha_{\nvec}^{(k)})^*}
				\mathcal{B}_{j,\lvec} \phi_{\pvec}.
\end{eqn}
Substituting $\sum_{\pvec \in \fullbasis} \phi_{\pvec}^* \phi_{\pvec} = \delta(\xvec - \xvec^\prime)$:
\begin{eqn}
	= \frac{1}{2} \sum_{\mvec \in \restbasis_j} \phi_{j,\mvec}^\prime
		\sum_{k=1}^C \sum_{\nvec \in \restbasis_k} \sum_{\lvec \in \mathbb{L}}
			\int \upd\xvec\,
				\phi_{k,\nvec} \mathcal{B}_{k,\lvec}^*
				\phi_{j,\mvec}^* \frac{\cwd}{\cwd (\alpha_{\nvec}^{(k)})^*}
				\mathcal{B}_{j,\lvec}.
\end{eqn}
Recognizing the projection transformation and the functional differential:
\begin{eqn}
	& = \proj{\restbasis_j} \left[
		\frac{1}{2} \sum_{k=1}^C \sum_{\nvec \in \restbasis_k} \sum_{\lvec \in \mathbb{L}}
			\phi_{k,\nvec} \mathcal{B}_{k,\lvec}^*
			\frac{\cwd}{\cwd (\alpha_{\nvec}^{(k)})^*}
			\mathcal{B}_{j,\lvec}
	\right] \\
	& = \proj{\restbasis_j} \left[
		\frac{1}{2} \sum_{k=1}^C \sum_{\lvec \in \mathbb{L}}
		\mathcal{B}_{k,\lvec}^*
		\frac{\fdelta}{\fdelta \Psi_k^*}
		\mathcal{B}_{j,\lvec}
	\right].
	\qedhere
\end{eqn}
\end{proof}

Alternatively, the \abbrev{fpe} from the above theorem can be expressed in a short matrix form:
\begin{eqn}
	\frac{dW}{dt}
	= \int d\xvec \left(
		- \vfdelta_{\bPsi} \cdot \mathbfcal{A} W
		- \vfdelta_{\bPsi^*} \cdot \mathbfcal{A}^* W
		+ \Trace{ \vfdelta_{\bPsi^*} \vfdelta_{\bPsi}^T \mathcal{B} \mathcal{B}^H } W
	\right),
\end{eqn}
where the functional cogradient $\vfdelta_{\bPsi} = \left( \fdelta/\fdelta \Psi_1 \ldots \fdelta/\fdelta \Psi_C \right)$, $\mathbf{\mathcal{A}}$ is a vector of $C$ functional operators, and $\mathcal{B}$ is a matrix of $C \times |\mathbb{L}|$ functional operators.


% =============================================================================
\section{It\^{o} formula}
% =============================================================================

In this section we will derive the It\^{o} formula for the differential of a functional, based on the standard definition for multi-variable real-valued case.

\begin{theorem}
\label{thm:fpe-sde:ito-formula:ito-f-real}
	Let $\zvec^T \equiv (z_1 \ldots z_M)$ be a set of real variables, and $\Zvec(t)$ be a standard $L$-dimensional Wiener process.
	For the SDE in It\^{o} form
	\begin{eqn*}
		d\zvec = \avec(\zvec, t) dt + B(\zvec, t) d\Zvec(t),
	\end{eqn*}
	the differential of a function $f(\zvec)$ is
	\begin{eqn*}
		df(\zvec) = \left(
			\avec \cdot \vcwd_{\zvec} dt
			+ \frac{1}{2} \Trace{ B B^T \vcwd_{\zvec} \vcwd_{\zvec}^T } dt
			+ \Trace{ B d\Zvec \vcwd_{\zvec}^T }
		\right) f(\zvec).
	\end{eqn*}
\end{theorem}
\begin{proof}
This is just a statement from~\cite{Gardiner1997} expressed in matrix form.
\end{proof}

This theorem can be extended to complex variables.

\begin{theorem}
\label{thm:fpe-sde:ito-formula:ito-f-complex}
	Let $\balpha^T \equiv (\alpha_1 \ldots \alpha_M)$ be a set of complex variables, and $\Zvec = (\mathbf{X} + i\mathbf{Y}) / \sqrt{2}$ be an $L$-dimensional complex-valued Wiener process, containing two standard $L$-dimensional Wiener processes $\mathbf{X}$ and $\mathbf{Y}$.
	For the SDE in It\^{o} form
	\begin{eqn*}
		d\balpha = \avec(\balpha, t) dt + B(\balpha, t) d\Zvec(t),
	\end{eqn*}
	the differential of a function $f(\balpha)$ is
	\begin{eqn*}
		df(\balpha) = \left(
			2 \Real (\avec \cdot \vcwd_{\balpha}) dt
			+ \Trace{ B B^H \vcwd_{\balpha^*} \vcwd_{\balpha}^T } dt
			+ 2 \Real \Trace{ B d\Zvec \vcwd_{\balpha}^T }
		\right) f(\balpha).
	\end{eqn*}
\end{theorem}
\begin{proof}
The proof follows the same scheme as \thmref{fpe-sde:corr:fpe-sde-complex}, just in the opposite direction.
Let $f = g + ih$, $\balpha = \mathbf{x} + i \mathbf{y}$, $\avec = \mathbf{u} + i \mathbf{v}$, $B = F + iG$, $\vcwd_{\balpha} = (\vcwd_{\mathbf{x}} - i \vcwd_{\mathbf{y}}) / 2$.
Then the set of SDEs from the statement is equivalent to
\begin{eqn}
	d \begin{pmatrix} \mathbf{x} \\ \mathbf{y} \end{pmatrix}
	= \begin{pmatrix} \mathbf{u} \\ \mathbf{v} \end{pmatrix} dt
		+ \frac{1}{\sqrt{2}} \begin{pmatrix} F & -G \\ G & F \end{pmatrix}
			\begin{pmatrix} d\mathbf{X} \\ d\mathbf{Y} \end{pmatrix}.
\end{eqn}
Applying \thmref{fpe-sde:ito-formula:ito-f-real} for real-valued functions $g(\mathbf{x}, \mathbf{y})$ and $h(\mathbf{x}, \mathbf{y})$ and combining them into $f = g + ih$:
\begin{eqn}
	df ={} &
		\begin{pmatrix} \mathbf{x} \\ \mathbf{y} \end{pmatrix} \cdot
			\begin{pmatrix} \vcwd_{\mathbf{x}} \\ \vcwd_{\mathbf{y}} \end{pmatrix} f dt
		+ \frac{1}{4} \Trace{
			\begin{pmatrix} F & -G \\ G & F \end{pmatrix}
			\begin{pmatrix} F^T & G^T \\ -G^T & F^T \end{pmatrix}
			\begin{pmatrix} \vcwd_{\mathbf{x}} \\ \vcwd_{\mathbf{y}} \end{pmatrix}
			\begin{pmatrix} \vcwd_{\mathbf{x}} \\ \vcwd_{\mathbf{y}} \end{pmatrix}^T
		} f dt  \\
	& + \frac{1}{\sqrt{2}} \Trace{
			\begin{pmatrix} F & -G \\ G & F \end{pmatrix}
			\begin{pmatrix} d\mathbf{X} \\ d\mathbf{Y} \end{pmatrix}
			\begin{pmatrix} \vcwd_{\mathbf{x}} \\ \vcwd_{\mathbf{y}} \end{pmatrix}^T
		} f
\end{eqn}
Now let us match this equation and the lemma statement term by term.

First term:
\begin{eqn}
	2 \Real ( \avec \cdot \vcwd_{\balpha} )
	& = \Real \left(
			\left( \mathbf{u} + i\mathbf{v} \right) \cdot \left( \vcwd_{\mathbf{x}} - i \vcwd_{\mathbf{y}} \right)
		\right) \\
	& = \mathbf{u} \cdot \vcwd_{\mathbf{x}} + \mathbf{v} \cdot \vcwd_{\mathbf{y}} \\
	& = \begin{pmatrix} \mathbf{x} \\ \mathbf{y} \end{pmatrix} \cdot
		\begin{pmatrix} \vcwd_{\mathbf{x}} \\ \vcwd_{\mathbf{y}} \end{pmatrix}
\end{eqn}

Second term:
\begin{eqn}
	\Trace{ B B^H \vcwd_{\balpha^*} \vcwd_{\balpha}^T }
	={} & \frac{1}{4} \Trace{
		(F F^T + G G^T)
		(\vcwd_{\mathbf{x}} \vcwd_{\mathbf{x}}^T
			+ \vcwd_{\mathbf{y}} \vcwd_{\mathbf{y}}^T)
		} \\
	& - \frac{1}{4} \Trace {
		(F G^T - G F^T)
		(\vcwd_{\mathbf{x}} \vcwd_{\mathbf{y}}^T
			- \vcwd_{\mathbf{y}} \vcwd_{\mathbf{x}}^T)
		} \\
	& + \frac{i}{4} \Trace{
		(F G^T - G F^T)
		(\vcwd_{\mathbf{x}} \vcwd_{\mathbf{x}}^T
			+ \vcwd_{\mathbf{y}} \vcwd_{\mathbf{y}}^T)
	} \\
	& + \frac{i}{4} \Trace{
		(G G^T + F F^T)
		(\vcwd_{\mathbf{x}} \vcwd_{\mathbf{y}}^T
			- \vcwd_{\mathbf{y}} \vcwd_{\mathbf{x}}^T)
	}
\end{eqn}
Same as in \thmref{fpe-sde:corr:fpe-sde-complex} we notice that $F F^T + G G^T$ and $\vcwd_{\mathbf{x}} \vcwd_{\mathbf{x}}^T + \vcwd_{\mathbf{y}} \vcwd_{\mathbf{y}}^T$ are symmetric matrices, and $F G^T - G F^T$ and $\vcwd_{\mathbf{x}} \vcwd_{\mathbf{y}}^T - \vcwd_{\mathbf{y}} \vcwd_{\mathbf{x}}^T$ are antisymmetric.
Therefore the last two terms contain traces of antisymmetric matrices and are equal to zero.
\begin{eqn}
	={} & \frac{1}{4} \Trace{
		(F F^T + G G^T) \vcwd_{\mathbf{x}} \vcwd_{\mathbf{x}}^T
		+ (F G^T - G F^T) \vcwd_{\mathbf{y}} \vcwd_{\mathbf{x}}^T)
		} \\
	& + \frac{1}{4} \Trace {
		(G F^T - F G^T) \vcwd_{\mathbf{x}} \vcwd_{\mathbf{y}}^T
		+ (F F^T + G G^T) \vcwd_{\mathbf{y}} \vcwd_{\mathbf{y}}^T)
		} \\
	={} & \frac{1}{4} \Trace {
		\begin{pmatrix}
			F F^T + G G^T & F G^T - G F^T \\
			G F^T - F G^T & F F^T + G G^T
		\end{pmatrix}
		\begin{pmatrix}
			\vcwd_{\mathbf{x}} \vcwd_{\mathbf{x}}^T & \vcwd_{\mathbf{x}} \vcwd_{\mathbf{y}}^T \\
			\vcwd_{\mathbf{y}} \vcwd_{\mathbf{x}}^T & \vcwd_{\mathbf{y}} \vcwd_{\mathbf{y}}^T
		\end{pmatrix}
	} \\
	={} & \frac{1}{4} \Trace{
		\begin{pmatrix} F & -G \\ G & F \end{pmatrix}
		\begin{pmatrix} F^T & G^T \\ -G^T & F^T \end{pmatrix}
		\begin{pmatrix} \vcwd_{\mathbf{x}} \\ \vcwd_{\mathbf{y}} \end{pmatrix}
		\begin{pmatrix} \vcwd_{\mathbf{x}} \\ \vcwd_{\mathbf{y}} \end{pmatrix}^T
	}.
\end{eqn}

Third term:
\begin{eqn}
	2 \Real \Trace{ B d\Zvec \vcwd_{\balpha}^T }
	& = \frac{1}{\sqrt{2}} \Real \Trace{
		(F + iG) (d\mathbf{X} + id\mathbf{Y}) (\vcwd_{\mathbf{x}} - i\vcwd_{\mathbf{y}})
	} \\
	& = \frac{1}{\sqrt{2}} \Trace{
		F d\mathbf{X} \vcwd_{\mathbf{x}} + F d\mathbf{Y} \vcwd_{\mathbf{y}}
		- G d\mathbf{Y} \vcwd_{\mathbf{x}} + G d\mathbf{X} \vcwd_{\mathbf{y}}
	} \\
	& = \frac{1}{\sqrt{2}} \Trace{
			\begin{pmatrix} F & -G \\ G & F \end{pmatrix}
			\begin{pmatrix} d\mathbf{X} \\ d\mathbf{Y} \end{pmatrix}
			\begin{pmatrix} \vcwd_{\mathbf{x}} \\ \vcwd_{\mathbf{y}} \end{pmatrix}^T
		}.
\end{eqn}

All terms have matched, thus proving the theorem.
\end{proof}

\begin{theorem}
\label{thm:fpe-sde:ito-formula:mc-ito-f}
	Let $\balpha^{(c)},\, c = 1..C$ be $C$ sets of complex variables $\balpha^{(c)} \equiv (\alpha_1^{(c)} \ldots \alpha_{M_c}^{(c)})$.
	For the SDE in It\^{o} form
	\begin{eqn*}
		d\balpha^{(c)} = \avec^{(c)} dt + B^{(c)} d\Zvec,
	\end{eqn*}
	the differential of a function $f(\balpha^{(1)}, \ldots, \balpha^{(C)})$ is
	\begin{eqn*}
		df ={} & \left(
			2 \sum_{c=1}^C \Real (\avec^{(c)} \cdot \vcwd_{\balpha^{(c)}}) dt
			+ \sum_{m=1}^C \sum_{n=1}^C \Trace{
				B^{(m)} (B^{(n)})^H \vcwd_{(\balpha^{(n)})^*} \vcwd_{\balpha^{(m)}}^T } dt \right. \\
		& \left. + 2 \sum_{c=1}^C \Real \Trace{ B^{(c)} d\Zvec \vcwd_{\balpha^{(c)}}^T }
		\right) f.
	\end{eqn*}
\end{theorem}
\begin{proof}
Proved analogously to \thmref{fpe-sde:corr:mc-fpe-sde}, by combining $\balpha^{(c)}$ into a single vector	and applying \thmref{fpe-sde:ito-formula:ito-f-complex}.
\end{proof}

\begin{theorem}
\label{thm:fpe-sde:ito-formula:func-ito-f}
	Given functional SDEs in It\^{o} form
	\begin{eqn*}
		d\Psi^{(c)} = \mathcal{A}^{(c)} dt + \sum_{\lvec} \mathcal{B}_{\lvec}^{(c)} dQ_{\lvec},
	\end{eqn*}
	the differential of a functional $F[\Psivec]$ is
	\begin{eqn*}
		dF[\Psivec]
		={} & \int d\xvec^\prime \left(
			2 \sum_{c=1}^C \Real \left(
				\mathcal{A}^{(c)\prime} \frac{\delta}{\delta \Psi_c^\prime}
			\right) dt
			+ \sum_{i=1}^C \sum_{j=1}^C \sum_{\lvec}
				\mathcal{B}_{\lvec}^{(i)\prime}
				\mathcal{B}_{\lvec}^{(j)\prime *}
				\frac{\delta}{\delta \Psi_i^\prime}
				\frac{\delta}{\delta \Psi_j^{\prime *}} dt
			\right. \\
		& \left. + 2 \sum_{c=1}^C \sum_{\lvec} \Real \left(
				\mathcal{B}_{\lvec}^{(i)\prime}
				dQ_{\lvec}^\prime
				\frac{\delta}{\delta \Psi_c^\prime}
			\right)
		\right) F[\Psivec]
	\end{eqn*}
	\todo{Consider rewriting it as
	\begin{eqn*}
		dF[\Psivec]
		= \int d\xvec^\prime \left(
			2 \Real \left(
				\vec{\mathcal{A}}^\prime \cdot \bdelta_{\bPsi^\prime}
			\right) dt
			+ \Trace{
				\mathcal{B}^\prime
				(\mathcal{B}^\prime)^H
				\bdelta_{\Psivec^{\prime *}}
				\bdelta_{\Psivec^\prime}^T
			} dt
			+ 2 \Real \Trace{
				\mathcal{B}^\prime
				d\vec{Q}^\prime
				\bdelta_{\Psivec^\prime}^T
			}
		\right) F[\Psivec].
	\end{eqn*}
	}
\end{theorem}
\begin{proof}
In terms of complex vectors SDEs can be rewritten as
\begin{eqn}
	d\alpha_{\mvec}^{(c)}
	= \int d\xvec \phi_{c,\mvec}^* \mathcal{A}^{(c)} dt
	+ \sum_{\pvec \in \fullbasis, \lvec}
		\int d\xvec \phi_{c,\mvec}^* \mathcal{B}_{\lvec}^{(c)} \phi_{\pvec} dZ_{\pvec,\lvec}.
\end{eqn}
Now, treating the functional as a function of complex vector $F \equiv F(\balpha^{(1)}, \ldots, \balpha^{(C)})$, we can use \thmref{fpe-sde:ito-formula:mc-ito-f} with
\begin{eqn}
	(\mathbf{a}^{(c)})_{\mvec} = \int d\xvec \phi_{c,\mvec}^* \mathcal{A}^{(c)},
\end{eqn}
and
\begin{eqn}
	(B^{(c)})_{\mvec,(\pvec,\lvec)}
	= \int d\xvec \phi_{c,\mvec}^* \mathcal{B}_{\lvec}^{(c)} \phi_{\pvec}.
\end{eqn}
This gives us
\begin{eqn}
	dF
	={} & \left(
		2 \sum_{c=1}^C \sum_{\mvec \in \restbasis_c} \Real \left(
			\int d\xvec^\prime \phi_{c,\mvec}^{\prime*} \mathcal{A}^{(c)\prime}
			\frac{\partial}{\partial \alpha_{c,\mvec}}
		\right) \right. \\
	& \left. + \sum_{i=1}^C \sum_{j=1}^C
			\sum_{\mvec \in \restbasis_i} \sum_{\nvec \in \restbasis_j}
			\sum_{\pvec \in \fullbasis, \lvec}
			\int d\xvec^\prime \phi_{i,\mvec}^{\prime *} \mathcal{B}_{\lvec}^{(i)\prime} \phi_{\pvec}^\prime
			\int d\xvec^{\prime\prime} \phi_{j,\nvec}^{\prime\prime} \mathcal{B}_{\lvec}^{(j)\prime\prime *} \phi_{\pvec}^{\prime\prime *}
			\frac{\partial}{\partial_{\alpha_{j,\nvec}^*}}
			\frac{\partial}{\partial_{\alpha_{i,\mvec}}} \right. \\
	& \left. + 2 \sum_{c=1}^C \Real \left(
			\sum_{\mvec \in \restbasis_c}
			\sum_{\pvec \in \fullbasis, \lvec}
			\int d\xvec^\prime \phi_{i,\mvec}^{\prime*} \mathcal{B}_{\lvec}^{(i)\prime} \phi_{\pvec}^\prime
			dZ_{\pvec,\lvec}
			\partial_{\alpha_{c,\mvec}}
		\right)
	\right) F
\end{eqn}
Recognizing definitions of functional differentials, functional Wiener process, and the delta function, we get
\begin{eqn}
	={} & \left(
		2 \sum_{c=1}^C \Real \left(
			\int d\xvec^\prime \mathcal{A}^{(c)\prime}
			\frac{\delta}{\delta \Psi_c^\prime}
		\right) \right. \\
	& \left. + \sum_{i=1}^C \sum_{j=1}^C \sum_{\lvec}
			\int d\xvec^\prime \mathcal{B}_{\lvec}^{(i)\prime}
			\mathcal{B}_{\lvec}^{(j)\prime *}
			\frac{\delta}{\delta \Psi_i^\prime}
			\frac{\delta}{\delta \Psi_j^{\prime *}}
		\right. \\
	& \left. + 2 \sum_{c=1}^C \sum_{\lvec} \Real \left(
			\int d\xvec^\prime \mathcal{B}_{\lvec}^{(i)\prime}
			dQ_{\lvec}^\prime
			\frac{\delta}{\delta \Psi_c^\prime}
		\right)
	\right) F
\end{eqn}
Which leads to the statement of the theorem.
\end{proof}

