% =============================================================================
\section{Hamiltonian}
% =============================================================================

In order to take into account quantum effects, we must start from the master equation.
The basic Hamiltonian is easily expressed using quantum fields $\Psiop_j^{\dagger}(\xvec)$ and $\Psiop_j(\xvec)$, where $\xvec$ is a $D$-dimensional coordinate vector, $\Psiop_j^{\dagger}(\xvec)$ creates a bosonic atom of spin $j$ at location $\xvec$, and $\Psiop_j(\xvec)$ destroys one; the commutators are defined by~\eqnref{wigner:op-calculus:commutators}.
Second-quantized Hamiltonian for the system looks like:
\begin{eqn}
\label{eqn:wigner-bec:hamiltonian:H}
	\hat{H} / \hbar = \int d\xvec \left\{
		\Psiop_j^{\dagger} K_{jk} \Psiop_k
		+ \frac{1}{2} \int d\xvec^\prime
			\Psiop_j^\dagger \Psiop_k^{\prime\dagger}
			U_{jk}(\xvec - \xvec^\prime)
			\Psiop_j^\prime \Psiop_k
	\right\}.
\end{eqn}
Here we use the Einstein summation convention of summing over repeated indices.
$U_{jk}$ is the two-body scattering potential, and $K_{jk}$ is the single-particle Hamiltonian:
\begin{eqn}
	K_{jk} = \left(
			-\frac{\hbar}{2m} \nabla^2 + \omega_j + V_j(\xvec) / \hbar
		\right) \delta_{jk}
		+ \tilde{\Omega}_{jk}(t),
\end{eqn}
where $V_j$ is the external trapping potential for spin $j$,
$\omega_j$ is the internal energy of spin $j$,
and $\tilde{\Omega}_{jk}$ represents a time-dependent coupling that is used to rotate one spin projection into another.
In the subspace of two coupled components $\tilde{\Omega}_{jk}$ can be defined as:
\begin{eqn}
	\tilde{\Omega} = \frac{\Omega}{2} \begin{pmatrix}
		0 & e^{i(\omega t + \alpha)} + e^{-i(\omega t + \alpha)} \\
		e^{i (\omega t + \alpha)} + e^{-i(\omega t + \alpha)} & 0
	\end{pmatrix},
\end{eqn}
where $\omega$ and $\alpha$ are frequency and phase of the oscillator,
and $\Omega$ is Rabi frequency (cf. equation~\eqnref{mean-field:rotation-matrix}).

If we impose an energy cutoff $\ecut$ and only take into account low-energy modes,
the general scattering potential $U_{jk}(\xvec - \xvec^\prime)$ can be replaced by contact potential $U_{jk} \delta(\xvec - \xvec^\prime)$~\cite{Morgan2000}, giving the effective Hamiltonian
\begin{eqn}
\label{eqn:wigner-bec:hamiltonian:effective-H}
	\hat{H} / \hbar = \int d\xvec \left\{
		\Psiop_j^{\dagger} K_{jk} \Psiop_k
		+ \frac{U_{jk}}{2} \Psiop_j^\dagger \Psiop_k^\dagger \Psiop_j \Psiop_k
	\right\}.
\end{eqn}

For $s$-wave scattering in three dimensions the coefficient is $U_{jk} = 4 \pi \hbar a_{jk} / m$,
where $a_{jk}$ is the scattering length.
Note that in general case the coefficient must be renormalised depending on the grid~\cite{Sinatra2002},
but the change is small if $dx \gg a_{jk}$.
