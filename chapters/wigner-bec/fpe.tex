% =============================================================================
\section{Fokker-Planck equation}
% =============================================================================

The general approach to numerical solution of the Fokker-Planck equation~\eqnref{wigner-bec:truncation:fpe} is to transform it to the equivalent set of stochastic differential equations (SDEs) for $\Psi_j$.
Since the transformation is defined for real-valued variables only \todo{citation needed},
we have to modify the equation.

First, noticing that $K_{jk}$, $U_{jk}$ and $\kappa_{\lvec}$ are real-valued
(which is important for the further transformations),
we can rewrite equation~\eqnref{wigner-bec:truncation:fpe} as
\[
	\frac{dW}{dt}
	= \int d\xvec \left(
		- \sum_{j=1}^C \frac{\delta}{\delta \Psi_j} A_j
		- \sum_{j=1}^C \frac{\delta}{\delta \Psi_j^*} A_j^*
		+ \sum_{j=1}^C \sum_{k=1}^C \frac{\delta^2}{\delta \Psi_j^* \delta \Psi_k} D_{jk}
	\right) W,
\]
where
\[
	A_j = -\frac{i}{\hbar} \left(
			\sum_{k=1}^C K_{jk} \Psi_k
			+ \sum_{k=1}^C U_{jk} \Psi_j \Psi_k \Psi_k^*
		\right)
		- \sum_{\lvec} \kappa_{\lvec} \frac{\partial O_{\lvec}^*}{\partial \Psi_j^*} O_{\lvec},
\]
and
\[
	D_{jk} = \sum_{\lvec} \kappa_{\lvec}
		\frac{\partial O_{\lvec}}{\partial \Psi_j}
		\frac{\partial O_{\lvec}^*}{\partial \Psi_k^*}.
\]
Considering $\Psi_j = \sum_{\nvec \in L} \phi_{\nvec} \alpha_{j,\nvec}$ and replacing functional derivatives with ordinary ones:
\[
	\frac{dW}{dt}
	= \left(
		- \sum_{j=1}^C \sum_{\nvec \in L}
			\frac{\partial}{\partial \alpha_{j,\nvec}}
			\int d\xvec \phi_{\nvec}^* A_j
		- \sum_{j=1}^C \sum_{\nvec \in L}
			\frac{\partial}{\partial \alpha_{j,\nvec}^*}
			\int d\xvec \phi_{\nvec} A_j^*
		+ \sum_{j=1}^C \sum_{k=1}^C
			\sum_{\mvec,\nvec \in L}
			\frac{\partial}{\partial \alpha_{j,\mvec}^*}
			\frac{\partial}{\partial \alpha_{k,\nvec}}
			\int d\xvec
			\phi_{\mvec} \phi_{\nvec}^* D_{jk}
	\right) W.
\]

\begin{lemma}[FPE--SDEs correspondence in convenient form.]
If $\bm{z}^T \equiv (z_1 \ldots z_M)$ is a set of real-valued variables,
Fokker-Planck equation
\[
	\frac{dW}{dt}
	= -\bm{\partial}_{\bm{z}}^T \bm{a} W
	+ \frac{1}{2} \Trace{ \bm{\partial}_{\bm{z}} \bm{\partial}_{\bm{z}}^T B B^T }  W
\]
is equivalent to a set of stochastic differential equations in It\^{o} form
\[
	d\bm{z} = \bm{a} dt + B d\bm{Z}
\]
and to a set of stochastic differential equations in Stratonovich form
\[
	d\bm{z} = (\bm{a} - \bm{c})dt + B d\bm{Z},
\]
where the noise-induced (or spurious) drift vector $\bm{c}$ has elements
\[
	c_i
	= \sum_{k,j} B_{kj} \frac{\partial}{\partial z_k} B_{ij}
	= \Trace{B^T \bm{\partial}_z \bm{e}_i^T B},
\]
$\bm{e}_i$ being the unit vector with elements $(\bm{e}_i)_j = \delta_{ij}$.
\todo{Is there a better way to express $\bm{c}$ in terms of matrices?}
Here $W \equiv W(\bm{z})$ is a probability distribution,
$\bm{a} \equiv \bm{a}(\bm{z})$ is a vector function,
$B \equiv B(\bm{z})$ is a matrix function ($B$ having size $M \times L$, where $L$ corresponds to the number of noise sources),
$\partial_{\bm{z}}^T \equiv (\partial_{z_1} \ldots \partial_{z_M})$ is a vector differential,
and $d\bm{Z}$ is a standard $L$-dimensional Wiener process.
\end{lemma}
\begin{proof}
For details see~\cite{Risken1996}, sections 3.3 and 3.4.
\todo{Consider the case of colored noise (\cite{Risken1996}, 3.1 and appendix A).}
\end{proof}

Theorem 1. FPE with ordinary differentials -> SDEs
Theorem 2. FPE with functional differentials -> SDEs
