% =============================================================================
\section{Initial states}
% =============================================================================

Before integrating the evolution equations~\eqnref{wigner-bec:fpe-bec:sde} or~\eqnref{wigner-bec:fpe-bec:sde-stratonovich}, the initial value of $\Psi_j$ at $t=0$ has to be sampled.
The general procedure is to take the density matrix of the desired initial state and find its Wigner transformation using \defref{wigner:mc:w-transformation}.
The resulting Wigner function is then sampled.


% =============================================================================
\subsection{Coherent state}
% =============================================================================

The simplest case of initial state is a coherent state.

\begin{theorem}
\label{thm:wigner-bec:initial-state:coherent-state}
	The Wigner distribution for a multimode coherent state with with the expectation value $\Psi^{(0)} \equiv \sum_{\nvec \in restbasis} \alpha_{\nvec}^{(0)} \phi_{\nvec}$ is
	\begin{eqn*}
		W_{\mathrm{coh}} [\Psi]
		= \left( \frac{2}{\pi} \right)^{|\restbasis|} \prod_{\nvec \in \restbasis}
			\exp(-2 |\alpha_{\nvec} - \alpha_{\nvec}^{(0)}|^2).
	\end{eqn*}
\end{theorem}
\begin{proof}
The density matrix of the state is
\begin{eqn}
	\hat{\rho}
	= \vert \alpha_{\nvec}^{(0)},\, \nvec \in \restbasis \rangle
		\langle \alpha_{\nvec}^{(0)},\, \nvec \in \restbasis \vert
	= \left( \prod_{\nvec \in \restbasis} \vert \alpha_{\nvec}^{(0)} \rangle \right)
		\left( \prod_{\nvec \in \restbasis} \langle \alpha_{\nvec}^{(0)} \vert \right).
\end{eqn}
Then the characteristic functional for this state can be expressed as
\begin{eqn}
	\chi_W [\Lambda]
	& = \Trace{
		\left( \prod_{\nvec \in \restbasis} \vert \alpha_{\nvec}^{(0)} \rangle \right)
		\left( \prod_{\nvec \in \restbasis} \langle \alpha_{\nvec}^{(0)} \vert \right)
		\left( \prod_{\nvec \in \restbasis} \hat{D}_{\nvec} (\lambda_{\nvec}, \lambda_{\nvec}^*) \right)
	} \\
	& = \prod_{\nvec \in \restbasis}
		\langle \alpha_{\nvec}^{(0)} \vert
		\hat{D}_{\nvec} (\lambda_{\nvec}, \lambda_{\nvec}^*)
		\vert \alpha_{\nvec}^{(0)} \rangle,
\end{eqn}
where $\Lambda = \sum_{\nvec \in \restbasis} \lambda_{\nvec} \phi_{\nvec}$.
Displacement operators obey the multiplication law~\cite{Cahill1969}
\begin{eqn}
	\hat{D}(\lambda, \lambda^*) \hat{D}(\alpha, \alpha^*)
	= \hat{D}(\lambda + \alpha, \lambda^* + \alpha^*)
		\exp(\frac{1}{2} (\lambda \alpha^* - \lambda^* \alpha)),
\end{eqn}
and the scalar product of two coherent state is calculated as~\cite{Cahill1969}
\begin{eqn}
	\langle \beta \vert \alpha \rangle
	= \exp(-\frac{1}{2} |\alpha|^2 - \frac{1}{2} |\beta|^2 + \beta^* \alpha).
\end{eqn}
Therefore
\begin{eqn}
	\hat{D}(\lambda, \lambda^*) \vert \alpha \rangle
	& = \hat{D}(\lambda, \lambda^*) \hat{D}(\alpha, \alpha^*) \vert 0 \rangle \\
	& = \exp(\frac{1}{2} (\lambda \alpha^* - \lambda^* \alpha))
		\vert \lambda + \alpha \rangle,
\end{eqn}
and the characteristic functional can be simplified as:
\begin{eqn}
	\chi_W [\Lambda]
	& = \prod_{\nvec \in \restbasis}
		\exp(\frac{1}{2} (\lambda_{\nvec} (\alpha_{\nvec}^{(0)})^*
			- \lambda_{\nvec}^* \alpha_{\nvec}^{(0)}))
		\langle \alpha_{\nvec}^{(0)} \vert \lambda_{\nvec} + \alpha_{\nvec}^{(0)} \rangle \\
	& = \prod_{\nvec \in \restbasis}
		\exp(
			- \lambda_{\nvec}^* \alpha_{\nvec}^{(0)}
			+ \lambda_{\nvec} (\alpha_{\nvec}^{(0)})^*
			- \frac{1}{2} |\lambda|^2
		).
\end{eqn}

Finally, the Wigner functional is
\begin{eqn}
	W_c [\Psi]
	& = \frac{1}{\pi^{2|\restbasis|}} \prod_{\nvec \in \restbasis} \left(
		\int \upd^2\lambda_{\nvec}
			\exp(
				- \lambda_{\nvec} (\alpha_{\nvec}^* - (\alpha_{\nvec}^{(0)})^*)
				+ \lambda_{\nvec}^* (\alpha_{\nvec} - \alpha_{\nvec}^{(0)})
				- \frac{1}{2} |\lambda|^2
			)
	\right) \\
	& = \left( \frac{2}{\pi} \right)^{|\restbasis|} \prod_{\nvec \in \restbasis}
		\exp(-2 |\alpha_{\nvec} - \alpha_{\nvec}^{(0)}|^2).
	\qedhere
\end{eqn}
\end{proof}

The resulting Wigner distribution is a product of independent complex-valued Gaussian distributions for each mode, with an expectation value equal to the expectation value of the mode, and the variance equal to $\frac{1}{2}$.
Therefore the initial state can be sampled as
\begin{eqn}
	\alpha_{\nvec} = \alpha_{\nvec}^{(0)} + \frac{1}{\sqrt{2}} \eta_{\nvec},
\end{eqn}
where $\eta_{\nvec}$ are normally distributed complex random numbers with zero mean, $\langle \eta_{\mvec} \eta_{\nvec} \rangle = 0$ and $\langle \eta_{\mvec} \eta_{\nvec}^* \rangle = \delta_{\mvec,\nvec}$, or, in other words, with components distributed independently with variance $\frac{1}{2}$.
This looks like adding half a ``vacuum particle'' to each mode.
In functional form this can be written as
\begin{eqn}
	\Psi_j(\xvec, 0)
	= \Psi_j^{(0)}(\xvec, 0)
		+ \sum_{\nvec \in \restbasis} \frac{\eta_{j,\nvec}}{\sqrt{2}} \phi_{\nvec}(\xvec),
\end{eqn}
where $\Psi_j^{(0)}(\xvec, 0)$ is the ``classical'' ground state of the system.


% =============================================================================
\subsection{Other cases}
% =============================================================================

\copypaste{
More involved examples, including thermalized states and Bogoliubov states, are reviewed by Blakie~\textit{et~al}~\cite{Blakie2008}, and Ruostekoski and Martin~\cite{Ruostekoski2010}.
In particular, a numerically efficient way to sample a Wigner distribution for Bogoliubov states was developed by Sinatra~\textit{et~al}~\cite{Sinatra2002}.
}
