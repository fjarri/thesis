% =============================================================================
\section{Usage examples}
\label{sec:wigner-bec:examples}
% =============================================================================

The general theorems in the sections above may seem complicated, so in this section we will see into two examples of how they can be applied to real systems.

% =============================================================================
\subsection{Single-component example}
% =============================================================================

First, we will first consider a simple case with a single component \abbrev{bec}, with 3-body loss and no unitary evolution (the same as described by Norrie~\textit{et~al}~\cite{Norrie2006a}).
For this system we have $K \equiv 0$, $U \equiv 0$ and $\hat{O} = \Psiop^3$ (and, consequently, $O = \Psi^3$), and we also denote $\gamma = 6\kappa$.
The \abbrev{fpe} for this system is therefore
\begin{eqn}
    \frac{\upd W}{\upd t}
    ={} & -\frac{\fdelta}{\fdelta\Psi} \left(
            - \frac{\gamma}{2} |\Psi|^4 \Psi
            + \frac{3\gamma}{2} |\Psi|^2 \Psi \delta_{\restbasis}(\xvec, \xvec)
            - \frac{3\gamma}{4} \Psi \delta_{\restbasis}^2(\xvec, \xvec)
        \right) \\
    & - \frac{\fdelta}{\fdelta \Psi^*} \left(
            - \frac{\gamma}{2} |\Psi|^4 \Psi^*
            + \frac{3\gamma}{2} |\Psi|^2 \Psi^* \delta_{\restbasis}(\xvec, \xvec)
            - \frac{3\gamma}{4} \Psi^* \delta_{\restbasis}^2(\xvec, \xvec)
        \right) \\
    & + \frac{\fdelta^2}{\fdelta\Psi^* \fdelta\Psi} \left(
            \frac{3\gamma}{2} |\Psi|^4
            - 3\gamma |\Psi|^2 \delta_{\restbasis}(\xvec, \xvec)
            + \frac{3\gamma}{4} \delta_{\restbasis}^2(\xvec, \xvec)
        \right) \\
    & + \frac{\fdelta^3}{\fdelta\Psi^* \fdelta\Psi^2} \left(
            \frac{3\gamma}{8} |\Psi|^2 \Psi
            - \frac{3\gamma}{8} \Psi \delta_{\restbasis}(\xvec, \xvec)
        \right) \\
    & + \frac{\fdelta^3}{\fdelta(\Psi^*)^2 \fdelta\Psi} \left(
            \frac{3\gamma}{8} |\Psi|^2 \Psi^*
            - \frac{3\gamma}{8} \Psi^* \delta_{\restbasis}(\xvec, \xvec)
        \right) \\
    & + \frac{\fdelta}{\fdelta\Psi^3} \left(
            \frac{\gamma}{24} \Psi^3
        \right)
        + \frac{\fdelta^3}{\fdelta(\Psi^*)^3} \left(
            \frac{\gamma}{24} (\Psi^*)^3
        \right)
        + \mathcal{O}\left[ \frac{1}{N^4} \right].
\end{eqn}
After the truncation, the resulting stochastic equation describing
the system is
\begin{eqn}
    \upd\Psi
    & = \mathcal{P}_{\restbasis} \left[
            - \frac{\gamma}{6} \left(
                \frac{\upp O^*}{\upp \Psi^*} O
                - \frac{1}{2} \delta_{\restbasis}(\xvec, \xvec) \frac{\upp^2 O^*}{\upp(\Psi^*)^2}
                    \frac{\upp O}{\upp \Psi}
            \right) \upd t
            + \sqrt{\frac{\gamma}{6}} \frac{\upp O^*}{\upp \Psi^*} \upd Q(\xvec,t)
        \right] \\
    & = \mathcal{P}_{\restbasis} \left[
            - \left(
                \frac{\gamma}{2} |\Psi|^4 \Psi
                - \frac{3\gamma}{2} \delta_{\restbasis}(\xvec, \xvec) |\Psi|^2 \Psi
            \right) \upd t
            + \sqrt{\frac{3\gamma}{2}}(\Psi^*)^2 \upd Q(\xvec,t)
        \right].
\end{eqn}
The equation coincides with the one given by Norrie \textit{et al}, except for the additional correction to the drift term, which is of order $1/N$ and therefore cannot be omitted.

If we calculate the rate population change over time using \thmref{wigner-bec:fpe-bec:population-change}, we obtain
\begin{eqn}
    \frac{\upd N}{\upd t}
    = -\gamma \int \upd\xvec \left(
        \pathavg{|\Psi|^6}
        - \frac{9}{2} \delta_{\restbasis}(\xvec, \xvec) \pathavg{|\Psi|^4}
    \right).
\end{eqn}
This can be transformed further to more conventional form.
Using the equivalence~\eqnref{wigner-bec:fpe-bec:moments} and the ordering transformation formula~\eqnref{wigner-bec:fpe-bec:ordering-transformation} for field operators, we get
\begin{eqn}
    \pathavg{|\Psi|^4}
    & = g^{(2)} n^2
        + 2 \delta_{\restbasis}(\xvec, \xvec) n
        + \frac{1}{2} \delta_{\restbasis}^2(\xvec, \xvec), \\
    \pathavg{|\Psi|^6}
    & = g^{(3)} n^3
        + \frac{9}{2} \delta_{\restbasis}(\xvec, \xvec) g^{(2)} n^2
        + \frac{9}{2} \delta_{\restbasis}^2(\xvec, \xvec) n
        + \frac{3}{4} \delta_{\restbasis}^3(\xvec, \xvec).
\end{eqn}
Here $n = \langle \Psiop^\dagger \Psiop \rangle$ is the particle density, and $g^{(k)}$ are correlation factors.
Substituting above expressions into the equation for the population change rate:
\begin{eqn}
    \frac{\upd N}{\upd t}
    = - \gamma \int \upd\xvec \left(
        g^{(3)} n^3
        - \frac{9}{2} \delta_{\restbasis}^2(\xvec, \xvec) n
        - \frac{3}{2} \delta_{\restbasis}^3(\xvec, \xvec)
    \right).
\end{eqn}
We see that the second highest term in the expression is canceled, which agrees with the expansion being correct up to the order $1/N$.
If the quantum correction term to the drift is omitted, one finds that a physically incorrect quadratic nonlinear term proportional to $n^2$ is obtained, which is inconsistent with an exact short-time solution to the master equation~\cite{Norrie2006a}.


% =============================================================================
\subsection{Two-component example}
% =============================================================================

As a more involved example, let us consider a two component $^{87}$Rb \abbrev{bec} from recent experiments~\cite{Egorov2011,Opanchuk2012}.
In this case we have both unitary evolution (including nonlinear interaction)~\eqnref{wigner-bec:hamiltonian:effective-H}, and three sources of losses: three-body recombination $\hat{O}_{111}=\Psiop_{1}^3$, two-body interspecies loss $\hat{O}_{12}=\Psiop_{1}\Psiop_{2}$ and two-body intraspecies loss $\hat{O}_{22}=\Psiop_{2}^2$.
This gives us \abbrev{sde}s~\eqnref{wigner-bec:fpe-bec:sde} with drift terms
\begin{eqn}
    \mathcal{A}_1
    ={} & - \frac{i}{\hbar} \left(
            \sum_{k=1}^2 K_{1k} \Psi_k
            + \Psi_1 \sum_{k=1}^2 U_{1k} \left(
                |\Psi_k|^2 - \frac{\delta_{1k} + 1}{2} \delta_{\restbasis_k}(\xvec, \xvec)
            \right)
        \right) \\
    & - 3\kappa_{111} \left( |\Psi_1|^2
        - 3 \delta_{\restbasis_1}(\xvec, \xvec) \right) |\Psi_1|^2 \Psi_1
        - \kappa_{12} \left( |\Psi_{2}|^2
        - \frac{\delta_{\restbasis_2}(\xvec, \xvec)}{2} \right) \Psi_1, \\
    \mathcal{A}_2
    ={} & - \frac{i}{\hbar} \left(
            \sum_{k=1}^2 K_{2k} \Psi_k
            + \Psi_2 \sum_{k=1}^2 U_{2k} \left(
                |\Psi_{k}|^2 - \frac{\delta_{2k} + 1}{2} \delta_{\restbasis_k}(\xvec, \xvec)
            \right)
        \right) \\
    & - \kappa_{12} \left(|\Psi_1|^2 - \frac{\delta_{\restbasis_1}(\xvec, \xvec)}{2} \right) \Psi_2
    - 2\kappa_{22} \left(|\Psi_2|^2 - \delta_{\restbasis_2}(\xvec, \xvec) \right)\Psi_2,
\end{eqn}
and noise terms
\begin{eqn}
    \mathcal{B}_{1,111} = 3 \sqrt{\kappa_{111}} \left( \Psi_1^* \right)^2,\quad
    \mathcal{B}_{1,12} = \sqrt{\kappa_{12}} \Psi_2^*,\quad
    \mathcal{B}_{1,22} = 0,
\end{eqn}
\begin{eqn}
    \mathcal{B}_{2,111} = 0,\quad
    \mathcal{B}_{2,12} = \sqrt{\kappa_{12}} \Psi_1^*,\quad
    \mathcal{B}_{2,22} = 2\sqrt{\kappa_{22}} \Psi_2^*.
\end{eqn}

This type of stochastic equation is needed to treat coherent \abbrev{bec} interferometry in the presence of nonlinear loss terms caused by two and three body collisions.
