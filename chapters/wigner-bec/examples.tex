% =============================================================================
\section{Usage example}
\label{sec:wigner-bec:examples}
% =============================================================================

General theorems from the previous sections may seem complicated, so in this section we will look into a simple example of how they can be applied to real systems.

We will consider a simple case of a single-component \abbrev{bec} with a 3-body loss process and no unitary evolution (the same as described by Norrie \textit{et~al}~\cite{Norrie2006a}).
The master equation for this system is
\begin{eqn}
    \frac{\upd \hat{\rho}}{\upd t}
    & = \frac{\gamma}{6} \int \upd\xvec \mathcal{L}_{111}[\hat{\rho}] \\
    & \equiv \frac{\gamma}{6} \int \upd\xvec \left(
            2 \Psiop^3 \hat{\rho} (\Psiop^\dagger)^3
            - (\Psiop^\dagger)^3 \Psiop^3 \hat{\rho}
            - \hat{\rho} (\Psiop^\dagger)^3 \Psiop^3
        \right),
\end{eqn}
so we have $K \equiv 0$, $U \equiv 0$ and a single source of losses with $\lvec=(1,1,1)$ and $\hat{O}_{111} = \Psiop^3$ (and, consequently, $O_{111} = \Psi^3$).
The coefficient $\gamma = 6 \kappa_{111}$ corresponds to the ``experimental'' normalization of loss rates; see~\eqnref{bec-noise:wigner:loss-rates} and the surrounding paragraph for details.
The \abbrev{fpe} for this system is, therefore,
\begin{eqn}
    \frac{\upd W}{\upd t}
    ={} & -\frac{\fdelta}{\fdelta\Psi} \left(
            - \frac{\gamma}{2} |\Psi|^4 \Psi
            + \frac{3\gamma}{2} |\Psi|^2 \Psi \delta_{\restbasis}(\xvec, \xvec)
            - \frac{3\gamma}{4} \Psi \delta_{\restbasis}^2(\xvec, \xvec)
        \right) \\
    & - \frac{\fdelta}{\fdelta \Psi^*} \left(
            - \frac{\gamma}{2} |\Psi|^4 \Psi^*
            + \frac{3\gamma}{2} |\Psi|^2 \Psi^* \delta_{\restbasis}(\xvec, \xvec)
            - \frac{3\gamma}{4} \Psi^* \delta_{\restbasis}^2(\xvec, \xvec)
        \right) \\
    & + \frac{\fdelta^2}{\fdelta\Psi^* \fdelta\Psi} \left(
            \frac{3\gamma}{2} |\Psi|^4
            - 3\gamma |\Psi|^2 \delta_{\restbasis}(\xvec, \xvec)
            + \frac{3\gamma}{4} \delta_{\restbasis}^2(\xvec, \xvec)
        \right) \\
    & + \frac{\fdelta^3}{\fdelta\Psi^* \fdelta\Psi^2} \left(
            \frac{3\gamma}{8} |\Psi|^2 \Psi
            - \frac{3\gamma}{8} \Psi \delta_{\restbasis}(\xvec, \xvec)
        \right) \\
    & + \frac{\fdelta^3}{\fdelta(\Psi^*)^2 \fdelta\Psi} \left(
            \frac{3\gamma}{8} |\Psi|^2 \Psi^*
            - \frac{3\gamma}{8} \Psi^* \delta_{\restbasis}(\xvec, \xvec)
        \right) \\
    & + \frac{\fdelta}{\fdelta\Psi^3} \left(
            \frac{\gamma}{24} \Psi^3
        \right)
        + \frac{\fdelta^3}{\fdelta(\Psi^*)^3} \left(
            \frac{\gamma}{24} (\Psi^*)^3
        \right)
        + \mathcal{O}\left[ \frac{1}{N^4} \right].
\end{eqn}
After truncation, the resulting stochastic equation describing the system is
\begin{eqn}
    \upd\Psi
    & = \mathcal{P}_{\restbasis} \left[
            - \frac{\gamma}{6} \left(
                \frac{\upp O^*}{\upp \Psi^*} O
                - \frac{1}{2} \delta_{\restbasis}(\xvec, \xvec) \frac{\upp^2 O^*}{\upp(\Psi^*)^2}
                    \frac{\upp O}{\upp \Psi}
            \right) \upd t
            + \sqrt{\frac{\gamma}{6}} \frac{\upp O^*}{\upp \Psi^*} \upd Q(\xvec,t)
        \right] \\
    & = \mathcal{P}_{\restbasis} \left[
            - \left(
                \frac{\gamma}{2} |\Psi|^4 \Psi
                - \frac{3\gamma}{2} \delta_{\restbasis}(\xvec, \xvec) |\Psi|^2 \Psi
            \right) \upd t
            + \sqrt{\frac{3\gamma}{2}}(\Psi^*)^2 \upd Q(\xvec,t)
        \right].
\end{eqn}
The equation above coincides with the one given by Norrie \textit{et~al}, except for the additional correction to the drift term, which is of order $1/N$ and, therefore, cannot be omitted.

If we calculate the rate population change over time using \thmref{wigner-bec:fpe-bec:population-change}, we obtain
\begin{eqn}
    \frac{\upd N}{\upd t}
    = -\gamma \int \upd\xvec \left(
        \pathavg{|\Psi|^6}
        - \frac{9}{2} \delta_{\restbasis}(\xvec, \xvec) \pathavg{|\Psi|^4}
    \right).
\end{eqn}
This can be transformed further to a more conventional form.
Using the equivalence~\eqnref{wigner-bec:fpe-bec:moments} and the ordering transformation formula~\eqnref{wigner-bec:fpe-bec:ordering-transformation} for field operators, we get
\begin{eqn}
    \pathavg{|\Psi|^4}
    & = g^{(2)} n^2
        + 2 \delta_{\restbasis}(\xvec, \xvec) n
        + \frac{1}{2} \delta_{\restbasis}^2(\xvec, \xvec), \\
    \pathavg{|\Psi|^6}
    & = g^{(3)} n^3
        + \frac{9}{2} \delta_{\restbasis}(\xvec, \xvec) g^{(2)} n^2
        + \frac{9}{2} \delta_{\restbasis}^2(\xvec, \xvec) n
        + \frac{3}{4} \delta_{\restbasis}^3(\xvec, \xvec).
\end{eqn}
Here $n = \langle \Psiop^\dagger \Psiop \rangle$ is the particle density, and $g^{(k)}$ are correlation factors.
Substituting above expressions into the equation for the population change rate:
\begin{eqn}
    \frac{\upd N}{\upd t}
    = - \gamma \int \upd\xvec \left(
        g^{(3)} n^3
        - \frac{9}{2} \delta_{\restbasis}^2(\xvec, \xvec) n
        - \frac{3}{2} \delta_{\restbasis}^3(\xvec, \xvec)
    \right).
\end{eqn}
We see that the second highest term in the expression is canceled, which agrees with the expansion being correct up to the order $1/N$.
If the quantum correction term to the drift is omitted, one finds that a physically incorrect quadratic nonlinear term proportional to $n^2$ is obtained, which is inconsistent with the exact short-time solution to the master equation~\cite{Norrie2006a}.
The last two terms in this result are artefacts of the truncated Wigner method.
They give rise to loss processes that occur even when $N=0$, which, of course, is physically impossible.
