% =============================================================================
\section{Wigner truncation}
% =============================================================================

In order to solve operator equation~\eqnref{wigner-bec:master-eqn:master-eqn} numerically,
we will transform it to ordinary differential equation using Wigner transformation~\eqnref{formalism:func-wigner:w-transformation}.

Namely, the term with $K_j$ is transformed using \thmref{formalism:transformations:w-commutator1} and \thmref{formalism:transformations:w-laplacian-commutator1}
(since $K_j$ is basically a sum of Laplacian operator and functions of $\xvec$):
\[
	\mathcal{W} \left[ [ \int d\xvec \Psiop_j^\dagger K_{jk} \Psiop_k, \hat{\rho} ] \right]
	= \int d\xvec \left(
			- \frac{\delta}{\delta \Psi_j} K_{jk} \Psi_k
			+ \frac{\delta}{\delta \Psi_k^*} K_{jk} \Psi_j^*
		\right)
		W,
\]
where Wigner function $W = \mathcal{W}[\hat{\rho}]$.
Nonlinear term is transformed with \thmref{formalism:transformations:w-commutator2}
(minding the locality of interaction and integrating over $\xvec$):
\begin{equation*}
\begin{split}
	\mathcal{W} \left[
		[
			\int d\xvec \frac{U_{jk}}{2}
				\Psiop_j^\dagger K_{jk} \Psiop_k^\dagger \Psiop_j \Psiop_k,
			\hat{\rho}
		]
	\right]
	& = \int d\xvec \frac{U_{jk}}{2} \left(
		\frac{\delta}{\delta \Psi_j} \left(
			- \Psi_j \Psi_k \Psi_k^*
			+ \frac{\delta_P(\xvec, \xvec)}{2} ( \delta_{jk} \Psi_k + \Psi_j )
		\right) \right. \\
	&	\left. + \frac{\delta}{\delta \Psi_j^*} \left(
			\Psi_j^* \Psi_k \Psi_k^*
			- \frac{\delta_P(\xvec, \xvec)}{2} ( \delta_{jk} \Psi_k^* + \Psi_j^* )
		\right) \right. \\
	&	\left. + \frac{\delta}{\delta \Psi_k} \left(
			- \Psi_j \Psi_j^* \Psi_k
			+ \frac{\delta_P(\xvec, \xvec)}{2} ( \delta_{jk} \Psi_j + \Psi_k )
		\right) \right. \\
	&	\left. + \frac{\delta}{\delta \Psi_k^*} \left(
			\Psi_j \Psi_j^* \Psi_k^*
			- \frac{\delta_P(\xvec, \xvec)}{2} ( \delta_{jk} \Psi_j^* + \Psi_k^* )
		\right) \right. \\
	&	\left.
			+ \frac{\delta}{\delta \Psi_j}
			\frac{\delta}{\delta \Psi_j^*}
			\frac{\delta}{\delta \Psi_k}
			\frac{1}{4} \Psi_k
			- \frac{\delta}{\delta \Psi_j}
			\frac{\delta}{\delta \Psi_j^*}
			\frac{\delta}{\delta \Psi_k^*}
			\frac{1}{4} \Psi_k^*
		\right. \\
	&	\left.
			+ \frac{\delta}{\delta \Psi_k}
			\frac{\delta}{\delta \Psi_k^*}
			\frac{\delta}{\delta \Psi_j}
			\frac{1}{4} \Psi_j
			- \frac{\delta}{\delta \Psi_k}
			\frac{\delta}{\delta \Psi_k^*}
			\frac{\delta}{\delta \Psi_j^*}
			\frac{1}{4} \Psi_j^*
		\right) W.
\end{split}
\end{equation*}
Loss operator is transformed with \thmref{formalism:transformations:w-losses}.
\todo{Not writing the resulting expression here, because with the absence of truncation it is too long,
and is practically the same as in theorem statement.}

We can only solve the resulting differential equation if it does not have functional derivatives of order more than 2 \todo{citation needed}.
One can see that


