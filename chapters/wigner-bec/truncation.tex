% =============================================================================
\section{Wigner truncation}
\label{sec:wigner-bec:truncation}
% =============================================================================

In order to solve operator equation~\eqnref{wigner-bec:master-eqn:master-eqn} with the Hamiltonian~\eqnref{wigner-bec:hamiltonian:effective-H} numerically, we will transform it to an ordinary differential equation using the Wigner transformation from \defref{wigner:mc:w-transformation}.

Namely, the term with $K_{jk}$ is transformed using \thmref{wigner-spec:w-commutator1} and \thmref{wigner-spec:w-laplacian-commutator1} (since $K_{jk}$ is basically a sum of the Laplacian operator and functions of $\xvec$):
\begin{eqn}
	\mathcal{W} \left[ [ \int \upd\xvec \Psiop_j^\dagger K_{jk} \Psiop_k, \hat{\rho} ] \right]
	= \int \upd\xvec \left(
			- \frac{\delta}{\delta \Psi_j} K_{jk} \Psi_k
			+ \frac{\delta}{\delta \Psi_k^*} K_{jk} \Psi_j^*
		\right)
		W,
\end{eqn}
where Wigner function $W = \mathcal{W}[\hat{\rho}]$.
Nonlinear term is transformed with \thmref{wigner-spec:w-commutator2} (assuming the locality of the interaction, and $U_{kj} = U_{jk}$):
\begin{eqn}
\label{eqn:wigner-bec:truncation:full-nonlinear}
	\mathcal{W} \left[
		[
			\int \upd\xvec \frac{U_{jk}}{2}
				\Psiop_j^\dagger \Psiop_k^\dagger \Psiop_j \Psiop_k,
			\hat{\rho}
		]
	\right]
	& = \int \upd\xvec U_{jk} \left(
		\frac{\delta}{\delta \Psi_j} \left(
			- \Psi_j \Psi_k \Psi_k^*
			+ \frac{\delta_{\restbasis_k}(\xvec, \xvec)}{2} ( \delta_{jk} \Psi_k + \Psi_j )
		\right) \right. \\
	&	\left. + \frac{\delta}{\delta \Psi_j^*} \left(
			\Psi_j^* \Psi_k \Psi_k^*
			- \frac{\delta_{\restbasis_k}(\xvec, \xvec)}{2} ( \delta_{jk} \Psi_k^* + \Psi_j^* )
		\right) \right. \\
	&	\left.
			+ \frac{\delta}{\delta \Psi_j}
			\frac{\delta}{\delta \Psi_j^*}
			\frac{\delta}{\delta \Psi_k}
			\frac{1}{4} \Psi_k
			- \frac{\delta}{\delta \Psi_j}
			\frac{\delta}{\delta \Psi_j^*}
			\frac{\delta}{\delta \Psi_k^*}
			\frac{1}{4} \Psi_k^*
		\right) W.
\end{eqn}
Loss operator is transformed with \thmref{wigner-spec:w-losses} and result in a similar equation, with a finite number of differential terms up to order $2n$ for $n-$body collisional losses.

\copypaste{Copypaste begins here.}
Assuming that $K_{jk}$, $U_{jk}$, and $\kappa_{\lvec}$ are real-valued, all the transformations described above result in a partial differential equation for $W$ of the form
\begin{eqn}
\label{eqn:wigner-bec:truncation:untruncated-fpe}
	\frac{\upd W}{\upd t}
	= \int \upd\xvec \left\{
		- \sum_{j=1}^{C} \frac{\fdelta}{\fdelta \Psi_j} \mathcal{A}_j
		- \sum_{j=1}^{C} \frac{\fdelta}{\fdelta \Psi_j^*} \mathcal{A}_j^*
		+ \sum_{j=1}^{C} \sum_{k=1}^{C}
			\frac{\fdelta^2}{\fdelta \Psi_j^* \fdelta \Psi_k} \mathcal{D}_{jk}
		+ \mathcal{O} \left[ \frac{\fdelta^3}{\fdelta \Psi_j^3} \right]
	\right\} W.
\end{eqn}
Terms of order higher than $2$ are produced both by the nonlinear term in the Hamiltonian and loss terms.
Such an equation could be solved perturbatively if there were only orders up to $3$ (which means an absence of nonlinear losses)~\cite{Polkovnikov2003}, but in most cases all terms except for first- and second-order ones are truncated.
In order to justify this truncation in a consistent way, we develop an order-by-order expansion in $1/N_j$, where $N_j$ is a characteristic particle number in a physical interaction volume, and truncate terms of formal order $1/N_j^2$.
This is achieved~\cite{Drummond1993} by use of the formal definition of a scaled Wigner function $W^{\psi}$, satisfying a scaled equation in terms of dimensionless scaled fields $\psi$, with:
\begin{eqn}
	\psi_{j} & = \Psi_{j}\sqrt{\ell_c^D / N_j} \\
	\mathcal{A}_j^{\psi} & = t_j \sqrt{\ell_c^D / N_j} \mathcal{A}_j
		+ \mathcal{O} \left( 1 / N_j^2 \right) \\
	\mathcal{D}_{jk}^{\psi} & = t_j \left( \ell_j^D / N_j \right) \mathcal{D}_{jk}
		+ \mathcal{O} \left( 1 / N_j^2 \right).
\end{eqn}
Here $t_j$ is a characteristic interaction time and $\ell_j$ is a characteristic interaction length.
These would normally be chosen as the healing time and healing length respectively in a \abbrev{bec} calculation.
Typically the cell size is chosen as proportional to the healing length, for optimum accuracy in resolving spatial detail.
Using this expansion, a consistent order-by-order expansion in $(1/N_c)$ can be obtained,
of form:
\begin{eqn}
	\frac{\upd W^{\psi}}{\upd \tau}
	= \int \upd\xvec \left\{
		- \sum_{j=1}^C \frac{\fdelta}{\fdelta \psi_j} \mathcal{A}_j^{\psi}
		- \sum_{j=1}^C \frac{\fdelta}{\fdelta \psi_j^*} \mathcal{A}_j^{\psi*}
		+ \sum_{j=1}^C \sum_{k=1}^C \frac{\fdelta^2}{\fdelta \psi_j^* \fdelta \psi_k}
			\mathcal{D}_{jk}^{\psi}
		+ \mathcal{O} \left[ \frac{1}{N_j^2} \right]
	\right\} W^{\psi}.
\end{eqn}

With the assumption of the state being coherent, the simple condition for truncation --- i.e., omitting terms of $O(1/N_j^2)$ --- can be shown to be~\cite{Sinatra2002}
\begin{eqn}
	N_j \gg |\restbasis_j|,
\end{eqn}
where $N_j$ is the total number of atoms of the component $j$.
The inclusion of the mode factor is caused by the fact that the number of additional terms increases as the number of modes increases, which may be needed to treat convergence of the method for large momentum cutoff.
We see immediately that there are subtleties involved if one wishes to include larger numbers of high-momentum modes, since this increases the mode number while leaving the numbers unchanged.
In other words, the truncation technique is inherently restricted in its ability to resolve fine spatial details in the high-momentum cutoff limit.

The $1/N_c$ is equivalent to an expansion in the inverse density, which requires the inequality~\cite{Norrie2006}
\begin{eqn}
\label{eqn:wigner-bec:truncation:delta-condition}
	\delta_{\restbasis_j}(\xvec, \xvec)
	\ll |\Psi_j|^2.
\end{eqn}
The coherency assumption does not, of course, encompass all possible states that can be produced during evolution, which means that the condition above is more of a guide than a restriction.
For certain systems the truncation was shown to work even when~\eqnref{wigner-bec:truncation:delta-condition} is violated~\cite{Ruostekoski2005}.
The validity may also depend on the simulation time~\cite{Javanainen2013}, and other physically
relevant factors.

A common example of such relevant factors is that there can be a large difference in the size of the original parameters.
To illustrate this issue, one may have a situation where $\kappa_1 \approx \kappa_2 N_j$ even though $N_j \gg 1$.
Under these conditions, it is essential to include a scaling of the parameters in calculating the formal order, so that the scaled parameters have comparable sizes.
This allows one to identify correctly which terms are negligible in a given physical problem, and which terms must be included.

In general, one can estimate the validity of truncation for the particular problem and the particular observable by calculating the quantum correction~\cite{Polkovnikov2010}.
Other techniques for estimating validity include comparison with the exact positive-P simulation method~\cite{Drummond1993}, and examining results for unphysical behaviour such as negative occupation numbers~\cite{Deuar2007}.
It is generally the case for unitary evolution that errors caused by truncation grow in time, leading to a finite time horizon for applicability, as explained in the introduction.

The use of this Wigner truncation allows us to simplify the results of \thmref{wigner-spec:w-commutator2} and \thmref{wigner-spec:w-losses}.
Wigner truncation is an expansion up to the order $1/N_j$, so during the simplification, along with the higher order derivatives, we drop all components with $\delta_{\restbasis_j}$ of order higher than $1$ in the drift terms, and of order higher than $0$ in the diffusion terms.
\copypaste{Copypaste ends here.}

\begin{theorem}
Assuming the conditions for Wigner truncation are satisfied, the result of the Wigner transformation of the nonlinear term is
\begin{eqn*}
	\mathcal{W} \left[
		[
			\int \upd\xvec \frac{U_{jk}}{2}
				\Psiop_j^\dagger \Psiop_k^\dagger \Psiop_j \Psiop_k,
			\hat{\rho}
		]
	\right]
	\approx{} & \int \upd\xvec U_{jk} \left(
		\frac{\fdelta}{\fdelta \Psi_j} \left(
			- \Psi_j \Psi_k \Psi_k^*
			+ \frac{\delta_{\restbasis_k}(\xvec, \xvec)}{2} ( \delta_{jk} \Psi_k + \Psi_j )
		\right) \right. \\
	&	\left. + \frac{\fdelta}{\fdelta \Psi_j^*} \left(
			\Psi_j^* \Psi_k \Psi_k^*
			- \frac{\delta_{\restbasis_k}(\xvec, \xvec)}{2} ( \delta_{jk} \Psi_k^* + \Psi_j^* )
		\right) \right) W.
\end{eqn*}
\end{theorem}
\begin{proof}
A straigtforward result of neglecting high-order terms in~\eqnref{wigner-bec:truncation:full-nonlinear}.
\end{proof}

\begin{theorem}
Assuming the conditions for Wigner truncation are satisfied, the result of the Wigner transformation of the loss term is
\begin{eqn*}
	\mathcal{W}[\mathcal{L}_{\lvec}[\hat{\rho}]]
	\approx{} & \left(
		\sum_{m=1}^C \frac{\fdelta}{\fdelta\Psi_m^*}
		\left(
			\frac{\upp O_{\lvec}}{\upp \Psi_m} O_{\lvec}^*
			- \frac{1}{2} \sum_{n=1}^{C} \delta_{\restbasis_n}(\xvec, \xvec)
				\frac{\upp^2 O_{\lvec}}{\upp\Psi_m \upp\Psi_n}
				\frac{\upp O_{\lvec}^*}{\upp\Psi_n^*}
		\right)
	\right. \\
	& + \sum_{m=1}^C \frac{\fdelta}{\fdelta\Psi_m}
	\left(
		\frac{\upp O_{\lvec}^*}{\upp \Psi_m^*} O_{\lvec}
		- \frac{1}{2}\sum_{n=1}^C \delta_{\restbasis_n}(\xvec, \xvec)
			\frac{\upp^2 O_{\lvec}^*}{\upp\Psi_m^* \upp\Psi_n^*}
			\frac{\upp O_{\lvec}}{\upp\Psi_n}
	\right) \\
	& \left. + \sum_{m=1}^C \sum_{n=1}^C
		\frac{\fdelta^2}{\fdelta\Psi_m^* \fdelta\Psi_n}
		\frac{\upp O_{\lvec}}{\upp\Psi_m} \frac{\upp O_{\lvec}^*}{\upp\Psi_n^*}
	\right) W,
	\end{eqn*}
where coupling functionals $O_{\lvec} \equiv O_{\lvec}[\Psivec] = \prod_{c=1}^C \Psi_c^{l_c}$.
\end{theorem}
\begin{proof}
The proof is basically a simplification of the result of \thmref{wigner-spec:w-losses} under two conditions following from neglecting the terms smaller than $1 / N$.
First, we are dropping all terms with high order differentials, which can be expressed as limiting $\sum j_c + \sum k_c \le 2$.
Second, we are only considering the terms with the restricted delta function of up to first order in the drift part (containing the functional differentials of order $1$), and terms with no restricted delta functions in the diffusion part (containing the functional differentials of order $2$).

The only combinations of $j_c$ and $k_c$ for which $Z_{\lvec,\jvec,\kvec}$ is not zero are thus $\{ j_c = \delta_{cm}, k_c = 0, m \in [1, C] \}$, $\{ j_c = 0, k_c = \delta_{cm}, m \in [1, C] \}$ and $\{ j_c = \delta_{cm}, k_c = \delta_{cn}, m \in [1, C], n \in [1, C] \}$.
These combinations produce terms with $\frac{\delta}{\delta \Psi_n^*}$, $\frac{\delta}{\delta \Psi_n}$ and $\frac{\delta^2}{\delta \Psi_p \delta \Psi_n^*}$ respectively:
\begin{eqn}
\label{eqn:wigner-bec:truncation:truncated-losses}
	\mathcal{W}[\mathcal{L}_{\lvec}[\hat{\rho}]]
	\approx{} & \left(
		\sum_{m=1}^C \frac{\fdelta}{\fdelta \Psi_m^*} H[l_m - 1] Z_{\lvec, \evec_m, \mathbf{0}}
		+ \sum_{m=1}^C \frac{\fdelta}{\fdelta \Psi_m} H[l_m - 1] Z_{\lvec, \mathbf{0}, \evec_m}
	\right. \\
	& \left. + \sum_{m=1}^C \sum_{n=1}^C \frac{\fdelta^2}{\fdelta \Psi_m^* \fdelta \Psi_n^*}
			H[l_m - 1] H[l_n - 1] Z_{\lvec, \evec_m, \evec_n}
	\right) W,
\end{eqn}
where $\evec_n$ is a vector consisting of zeros and a single $1$ at the $n$-th position, and $H[n]$ is the discrete Heavyside function.

Evaluating the partially parametrized $Z$ function in the first term using the expression in \thmref{wigner-spec:w-losses}:
\begin{eqn}
	Z_{\lvec, \evec_m, \mathbf{0}}
	= l_m \prod_{c=1}^C
		\exp \left(
			-\frac{\delta_{\restbasis_c}(\xvec, \xvec)}{2}
			\frac{\upp^2}{\upp \Psi_c \upp \Psi_c^*}
		\right)
		\Psi_c^{l_c - \delta_{cm}} (\Psi_c^*)^{l_c}.
\end{eqn}
Expanding the exponent in series and discarding the terms with more than one restricted delta function:
\begin{eqn}
	Z_{\lvec, \evec_m, \mathbf{0}}
	& \approx l_m \left(
		\prod_{c=1}^C \Psi_c^{l_c - \delta_{cm}} (\Psi_c^*)^{l_c}
		- \frac{1}{2} \sum_{n=1}^C
			\delta_{\restbasis_n}(\xvec, \xvec)
			\frac{\upp^2}{\upp \Psi_n \upp \Psi_p^*}
			\prod_{c=1}^C
				\Psi_c^{l_c - \delta_{cm}} (\Psi_c^*)^{l_c}
	\right).
\end{eqn}
Multiplied by the Heavyside function, this can be expressed using derivatives of the coupling functional $O_{\lvec}$:
\begin{eqn}
	H[l_m - 1] Z_{\lvec, \evec_m, \mathbf{0}}
	\approx \frac{\upp O_{\lvec}}{\upp \Psi_m} O_{\lvec}^*
		- \frac{1}{2} \sum_{n=1}^C
			\delta_{\restbasis_n}(\xvec, \xvec)
			\frac{\upp^2 O_{\lvec}}{\upp \Psi_m \upp \Psi_n}
			\frac{\upp O_{\lvec}^*}{\upp \Psi_n^*}.
\end{eqn}

Analogously, for the second term in~\eqnref{wigner-bec:truncation:truncated-losses} we get
\begin{eqn}
	H[l_m - 1] Z_{\lvec, \mathbf{0}, \evec_m}
	\approx \frac{\upp O_{\lvec}^*}{\upp \Psi_m^*} O_{\lvec}
		- \frac{1}{2} \sum_{n=1}^C
			\delta_{\restbasis_n}(\xvec, \xvec)
			\frac{\upp^2 O_{\lvec}^*}{\upp \Psi_m^* \upp \Psi_n^*}
			\frac{\upp O_{\lvec}}{\upp \Psi_n},
\end{eqn}
and for the third term (discarding all terms with the restricted delta function in the expansion of the exponent):
\begin{eqn}
	H[l_m - 1] H[l_n - 1] Z_{\lvec, \evec_m, \evec_n}
	\approx \frac{\upp O_{\lvec}}{\upp\Psi_m} \frac{\upp O_{\lvec}^*}{\upp\Psi_n^*}.
\end{eqn}
Substituting these expressions back into~\eqnref{wigner-bec:truncation:truncated-losses}, we get the statement of the theorem.
\end{proof}
