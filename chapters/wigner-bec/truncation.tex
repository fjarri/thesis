% =============================================================================
\section{Wigner truncation}
% =============================================================================

In order to solve operator equation~\eqnref{wigner-bec:master-eqn:master-eqn} numerically,
we will transform it to ordinary differential equation using Wigner transformation~\eqnref{formalism:func-wigner:w-transformation}.

Namely, the term with $K_j$ is transformed using \thmref{formalism:transformations:w-commutator1} and \thmref{formalism:transformations:w-laplacian-commutator1}
(since $K_j$ is basically a sum of Laplacian operator and functions of $\xvec$):
\[
	\mathcal{W} \left[ [ \int d\xvec \Psiop_j^\dagger K_{jk} \Psiop_k, \hat{\rho} ] \right]
	= \int d\xvec \left(
			- \frac{\delta}{\delta \Psi_j} K_{jk} \Psi_k
			+ \frac{\delta}{\delta \Psi_k^*} K_{jk} \Psi_j^*
		\right)
		W,
\]
where Wigner function $W = \mathcal{W}[\hat{\rho}]$.
Nonlinear term is transformed with \thmref{formalism:transformations:w-commutator2}
(minding the locality of interaction and integrating over $\xvec$):
\begin{equation*}
\begin{split}
	\mathcal{W} \left[
		[
			\int d\xvec \frac{U_{jk}}{2}
				\Psiop_j^\dagger K_{jk} \Psiop_k^\dagger \Psiop_j \Psiop_k,
			\hat{\rho}
		]
	\right]
	& = \int d\xvec \frac{U_{jk}}{2} \left(
		\frac{\delta}{\delta \Psi_j} \left(
			- \Psi_j \Psi_k \Psi_k^*
			+ \frac{\delta_P(\xvec, \xvec)}{2} ( \delta_{jk} \Psi_k + \Psi_j )
		\right) \right. \\
	&	\left. + \frac{\delta}{\delta \Psi_j^*} \left(
			\Psi_j^* \Psi_k \Psi_k^*
			- \frac{\delta_P(\xvec, \xvec)}{2} ( \delta_{jk} \Psi_k^* + \Psi_j^* )
		\right) \right. \\
	&	\left. + \frac{\delta}{\delta \Psi_k} \left(
			- \Psi_j \Psi_j^* \Psi_k
			+ \frac{\delta_P(\xvec, \xvec)}{2} ( \delta_{jk} \Psi_j + \Psi_k )
		\right) \right. \\
	&	\left. + \frac{\delta}{\delta \Psi_k^*} \left(
			\Psi_j \Psi_j^* \Psi_k^*
			- \frac{\delta_P(\xvec, \xvec)}{2} ( \delta_{jk} \Psi_j^* + \Psi_k^* )
		\right) \right. \\
	&	\left.
			+ \frac{\delta}{\delta \Psi_j}
			\frac{\delta}{\delta \Psi_j^*}
			\frac{\delta}{\delta \Psi_k}
			\frac{1}{4} \Psi_k
			- \frac{\delta}{\delta \Psi_j}
			\frac{\delta}{\delta \Psi_j^*}
			\frac{\delta}{\delta \Psi_k^*}
			\frac{1}{4} \Psi_k^*
		\right. \\
	&	\left.
			+ \frac{\delta}{\delta \Psi_k}
			\frac{\delta}{\delta \Psi_k^*}
			\frac{\delta}{\delta \Psi_j}
			\frac{1}{4} \Psi_j
			- \frac{\delta}{\delta \Psi_k}
			\frac{\delta}{\delta \Psi_k^*}
			\frac{\delta}{\delta \Psi_j^*}
			\frac{1}{4} \Psi_j^*
		\right) W.
\end{split}
\end{equation*}
Assuming $U_{kj} = U_{jk}$, it simplifies to
\begin{equation*}
\begin{split}
	\mathcal{W} \left[
		[
			\int d\xvec \frac{U_{jk}}{2}
				\Psiop_j^\dagger K_{jk} \Psiop_k^\dagger \Psiop_j \Psiop_k,
			\hat{\rho}
		]
	\right]
	& = \int d\xvec U_{jk} \left(
		\frac{\delta}{\delta \Psi_j} \left(
			- \Psi_j \Psi_k \Psi_k^*
			+ \frac{\delta_P(\xvec, \xvec)}{2} ( \delta_{jk} \Psi_k + \Psi_j )
		\right) \right. \\
	&	\left. + \frac{\delta}{\delta \Psi_j^*} \left(
			\Psi_j^* \Psi_k \Psi_k^*
			- \frac{\delta_P(\xvec, \xvec)}{2} ( \delta_{jk} \Psi_k^* + \Psi_j^* )
		\right) \right. \\
	&	\left.
			+ \frac{\delta}{\delta \Psi_j}
			\frac{\delta}{\delta \Psi_j^*}
			\frac{\delta}{\delta \Psi_k}
			\frac{1}{4} \Psi_k
			- \frac{\delta}{\delta \Psi_j}
			\frac{\delta}{\delta \Psi_j^*}
			\frac{\delta}{\delta \Psi_k^*}
			\frac{1}{4} \Psi_k^*
		\right) W.
\end{split}
\end{equation*}

Loss operator is transformed with \thmref{formalism:transformations:w-losses}.
\todo{Not writing the resulting expression here, because with the absence of truncation it is too long,
and is practically the same as in theorem statement.}

We can only solve the resulting differential equation if it does not have functional derivatives of order more than 2 \todo{citation needed}.
\todo{Need to formally derive truncation condition.
For now I am assuming that we can drop third- and higher-order derivatives and consider $\delta_P(\xvec, \xvec) \ll | \Psi_j |^2$ for any $j$.}

\begin{lemma}
Assuming the conditions for Wigner truncation are satisfied,
the result of Wigner transformation of the nonlinear term can be written as
\[
	\mathcal{W} \left[
		[
			\frac{U_{jk}}{2}
				\Psiop_j^\dagger K_{jk} \Psiop_k^\dagger \Psiop_j \Psiop_k,
			\hat{\rho}
		]
	\right]
	= U_{jk} \left(
		\frac{\delta}{\delta \Psi_j^*} \Psi_j^* \Psi_k \Psi_k^*
		- \frac{\delta}{\delta \Psi_j} \Psi_j \Psi_k \Psi_k^*
	\right) W.
\]
\end{lemma}

\begin{theorem}
Assuming the conditions for Wigner truncation are satisfied,
i.e. we can drop differentials higher than the second order and consider $\delta_P(\xvec, \xvec) \ll | \Psi_j |^2$,
the result of Wigner transformation of the loss term can be written as
\begin{equation*}
\begin{split}
	\mathcal{W}[\mathcal{L}_{\lvec}[\hat{\rho}]]
	= \sum_{n=1}^C
			\frac{\delta}{\delta \Psi_n^*} \frac{\partial O_{\lvec}}{\partial \Psi_n} O_{\lvec}^*
	+ \sum_{n=1}^C
		\frac{\delta}{\delta \Psi_n} \frac{\partial O_{\lvec}^*}{\partial \Psi_n^*} O_{\lvec}
	+ \sum_{n=1}^C \sum_{p=1}^C
		\frac{\delta^2}{\delta \Psi_n^* \delta \Psi_p}
		\frac{\partial O_{\lvec}}{\partial \Psi_n}
		\frac{\partial O_{\lvec}^*}{\partial \Psi_p^*}.
\end{split}
\end{equation*}
where $O_{\lvec} \equiv O_{\lvec}(\Psivec) = \prod_{c=1}^C \Psi_c^{l_c}$.
\end{theorem}
\begin{proof}
The proof is basically a simplification of the result of \thmref{formalism:transformations:w-losses} under certain conditions.
First, we are neglecting all occurrences of $\delta_P$, which means setting $m_c = 0$ for every $c$.
Second, we are dropping all terms with high order differentials,
which can be expressed as limiting $\sum j_c + \sum k_c \le 2$.
The only combinations of $j_c$ and $k_c$ for which $Z(\jvec, \kvec)$ is not zero are thus
$\{ j_c = \delta_{cn}, k_c = 0, n \in [1, C] \}$,
$\{ j_c = 0, k_c = \delta_{cn}, n \in [1, C] \}$ and
$\{ j_c = \delta_{cn}, k_c = \delta_{cp}, n \in [1, C], p \in [1, C] \}$.
These combinations produce terms with $\frac{\delta}{\delta \Psi_n^*}$,
$\frac{\delta}{\delta \Psi_n}$ and
$\frac{\delta^2}{\delta \Psi_p \delta \Psi_n^*}$ respectively:
\begin{equation*}
\begin{split}
	\mathcal{W}[\mathcal{L}_{\lvec}[\hat{\rho}]]
	& = \sum_{n=1}^C \frac{\delta}{\delta \Psi_n^*}
		2 H[l_n - 1] Q_n(1, 0, 0) \Psi_n^{l_n - 1} (\Psi_n^*)^{l_n}
		\prod_{c=1, c \ne n}^C Q_c(0, 0, 0) \Psi_c^{l_c} (\Psi_c^*)^{l_c} \\
	& + \sum_{n=1}^C \frac{\delta}{\delta \Psi_n}
		2 H[l_n - 1] Q_n(0, 1, 0) \Psi_n^{l_n} (\Psi_n^*)^{l_n - 1}
		\prod_{c=1, c \ne n}^C Q_c(0, 0, 0) \Psi_c^{l_c} (\Psi_c^*)^{l_c} \\
	& + \sum_{n=1}^C \sum_{p=1, p \ne n}^C \frac{\delta^2}{\delta \Psi_n^* \delta \Psi_p}
		4 H[l_n - 1] Q_n(1, 0, 0) \Psi_n^{l_n - 1} (\Psi_n^*)^{l_n}
		H[l_p - 1] Q_p(0, 1, 0) \Psi_p^{l_p} (\Psi_p^*)^{l_p - 1} \\
	&	\prod_{c=1, c \ne n, c \ne p}^C Q_c(0, 0, 0) \Psi_c^{l_c} (\Psi_c^*)^{l_c} \\
	& + \sum_{n=1}^C \frac{\delta^2}{\delta \Psi_n^* \delta \Psi_n}
		4 H[l_n - 1] Q_n(1, 1, 0) \Psi_n^{l_n - 1} (\Psi_n^*)^{l_n - 1}
		\prod_{c=1, c \ne n}^C Q_c(0, 0, 0) \Psi_c^{l_c} (\Psi_c^*)^{l_c},
\end{split}
\end{equation*}
where $H[n]$ is the discrete Heavyside function.

One can easily find that $Q_n(1, 0, 0) = Q_n(0, 1, 0) = l_n / 2$, $Q_n(0, 0, 0) = 1$ and $Q_n(1, 1, 0) =
l_n^2 / 4$.
Therefore:
\begin{equation*}
\begin{split}
	& = \sum_{n=1}^C \frac{\delta}{\delta \Psi_n^*}
		\left( H[l_n - 1] l_n \Psi_n^{l_n - 1} \prod_{c=1, c \ne n}^C \Psi_c^{l_c} \right)
		O_{\lvec}^* \\
	& + \sum_{n=1}^C \frac{\delta}{\delta \Psi_n}
		\left( H[l_n - 1] l_n (\Psi_n^*)^{l_n - 1} \prod_{c=1, c \ne n}^C (\Psi_c^*)^{l_c} \right)
		O_{\lvec} \\
	& + \sum_{n=1}^C \sum_{p=1, p \ne n}^C \frac{\delta^2}{\delta \Psi_n^* \delta \Psi_p}
		\left( H[l_n - 1] l_n \Psi_n^{l_n - 1} \prod_{c=1, c \ne n}^C \Psi_c^{l_c} \right)
		\left( H[l_p - 1] l_p (\Psi_p^*)^{l_p - 1} \prod_{c=1, c \ne p}^C (\Psi_c^*)^{l_c} \right) \\
	& + \sum_{n=1}^C \frac{\delta^2}{\delta \Psi_n^* \delta \Psi_n}
		\left( H[l_n - 1] l_n \Psi_n^{l_n - 1} \prod_{c=1, c \ne n}^C \Psi_c^{l_c} \right)
		\left( H[l_n - 1] l_n (\Psi_n^*)^{l_n - 1} \prod_{c=1, c \ne n}^C (\Psi_c^*)^{l_c} \right) \\
	& = \sum_{n=1}^C \frac{\delta}{\delta \Psi_n^*}
		\frac{\partial O_{\lvec}}{\partial \Psi_n} O_{\lvec}^*
	+ \sum_{n=1}^C \frac{\delta}{\delta \Psi_n}
		\frac{\partial O_{\lvec}^*}{\partial \Psi_n^*} O_{\lvec}
	+ \sum_{n=1}^C \sum_{p=1, p \ne n}^C \frac{\delta^2}{\delta \Psi_n^* \delta \Psi_p}
		\frac{\partial O_{\lvec}^*}{\partial \Psi_p^*}
		\frac{\partial O_{\lvec}}{\partial \Psi_n}
	+ \sum_{n=1}^C \frac{\delta^2}{\delta \Psi_n^* \delta \Psi_n}
		\frac{\partial O_{\lvec}^*}{\partial \Psi_n^*}
		\frac{\partial O_{\lvec}}{\partial \Psi_n} \\
	& = \sum_{n=1}^C \frac{\delta}{\delta \Psi_n^*}
		\frac{\partial O_{\lvec}}{\partial \Psi_n} O_{\lvec}^*
	+ \sum_{n=1}^C \frac{\delta}{\delta \Psi_n}
		\frac{\partial O_{\lvec}^*}{\partial \Psi_n^*} O_{\lvec}
	+ \sum_{n=1}^C \sum_{p=1}^C \frac{\delta^2}{\delta \Psi_n^* \delta \Psi_p}
		\frac{\partial O_{\lvec}}{\partial \Psi_n}
		\frac{\partial O_{\lvec}^*}{\partial \Psi_p^*}
	\qedhere
\end{split}
\end{equation*}
\end{proof}

