% =============================================================================
\section{Fokker-Planck equation for the BEC}
% =============================================================================

Resulting Wigner-transformed master equation is
\begin{eqn}
\label{eqn:wigner-bec:truncation:fpe}
	\frac{dW}{dt}
	& = \int d\xvec \left(
		- \sum_{j=1}^C \frac{\delta}{\delta \Psi_j} \left(
			-\frac{i}{\hbar} \left(
				\sum_{k=1}^C K_{jk} \Psi_k
				+ \sum_{k=1}^C U_{jk} \Psi_j \Psi_k \Psi_k^*
			\right)
			- \sum_{\lvec} \kappa_{\lvec} \frac{\partial O_{\lvec}^*}{\partial \Psi_j^*} O_{\lvec}
		\right)
	\right. \\
	& \left.
		+ \sum_{j=1}^C \frac{\delta}{\delta \Psi_j^*} \left(
			-\frac{i}{\hbar} \left(
				\sum_{k=1}^C K_{jk} \Psi_k^*
				+ \sum_{k=1}^C U_{jk} \Psi_j^* \Psi_k \Psi_k^*
			\right)
			+ \sum_{\lvec} \kappa_{\lvec} \frac{\partial O_{\lvec}}{\partial \Psi_j} O_{\lvec}^*
		\right)
	\right. \\
	& \left. + \sum_{j=1}^C \sum_{k=1}^C \frac{\delta^2}{\delta \Psi_j^* \delta \Psi_k}
		\sum_{\lvec} \kappa_{\lvec}
		\frac{\partial O_{\lvec}}{\partial \Psi_j}
		\frac{\partial O_{\lvec}^*}{\partial \Psi_k^*}
	\right) W
\end{eqn}

The general approach to numerical solution of the Fokker-Planck equation~\eqnref{wigner-bec:truncation:fpe} is to transform it to the equivalent set of stochastic differential equations (SDEs) for $\Psi_j$.
Since the transformation is defined for real-valued variables only \todo{citation needed}, we have to modify the equation.

First, noticing that $K_{jk}$, $U_{jk}$ and $\kappa_{\lvec}$ are real-valued (which is important for the further transformations), we can rewrite equation~\eqnref{wigner-bec:truncation:fpe} as
\begin{eqn}
	\frac{dW}{dt}
	= \int d\xvec \left(
		- \sum_{j=1}^C \frac{\delta}{\delta \Psi_j} \mathcal{A}_j
		- \sum_{j=1}^C \frac{\delta}{\delta \Psi_j^*} \mathcal{A}_j^*
		+ \sum_{j=1}^C \sum_{k=1}^C \frac{\delta^2}{\delta \Psi_j^* \delta \Psi_k} D_{jk}
	\right) W,
\end{eqn}
where
\begin{eqn}
	\mathcal{A}_j = -\frac{i}{\hbar} \left(
			\sum_{k=1}^C K_{jk} \Psi_k
			+ \sum_{k=1}^C U_{jk} \Psi_j \Psi_k \Psi_k^*
		\right)
		- \sum_{\lvec} \kappa_{\lvec} \frac{\partial O_{\lvec}^*}{\partial \Psi_j^*} O_{\lvec},
\end{eqn}
and
\begin{eqn}
	D_{jk} = \sum_{\lvec} \kappa_{\lvec}
		\frac{\partial O_{\lvec}}{\partial \Psi_j}
		\frac{\partial O_{\lvec}^*}{\partial \Psi_k^*}.
\end{eqn}

This allows us to apply \thmref{fpe-sde:corr:fpe-sde-func} with
\begin{eqn}
	\mathcal{B}_{\lvec}^{(j)}
	= \sqrt{\kappa_{\lvec}} \frac{\partial O_{\lvec}^*}{\partial \Psi_j^*}.
\end{eqn}
to get the equivalent set of SDEs in It\^{o} form
\begin{eqn}
\label{eqn:fpe-sde:corr-bec:sde}
	d\Psi_j = \mathcal{P}_{\restbasis_j} \left[
		\mathcal{A}^{(j)} dt + \sum_{\lvec} \mathcal{B}_{\lvec}^{(j)} dQ_{\lvec}
	\right],
\end{eqn}
or, alternatively, in Stratonovich form
\begin{eqn}
	d\Psi_j = \mathcal{P}_{\restbasis_j} \left[
		(\mathcal{A}^{(j)} - \mathcal{S}^{(j)}) dt + \sum_{\lvec} \mathcal{B}_{\lvec}^{(j)} dQ_{\lvec}
	\right],
\end{eqn}
where the Stratonovich term is
\begin{eqn}
	\mathcal{S}^{(j)}
	= \sum_{n=1}^C \sum_{\lvec}
		\sqrt{\kappa_{\lvec}} \frac{\partial O_{\lvec}}{\partial \Psi_n}
		\frac{\delta}{\delta \Psi_n^*}
		\sqrt{\kappa_{\lvec}} \frac{\partial O_{\lvec}^*}{\partial \Psi_j^*}
	= \sum_{n=1}^C \sum_{\lvec} \kappa_{\lvec}
		\frac{\partial O_{\lvec}}{\partial \Psi_n}
		\left(\frac{\partial^2 O_{\lvec}}{\partial \Psi_n \partial \Psi_j} \right)^*
		\delta_{\restbasis_n} (\xvec, \xvec).
\end{eqn}

% =============================================================================
\subsection{Integral averages}
% =============================================================================

It is interesting to get an expression for time dependence of some simple observables using SDEs~\eqnref{fpe-sde:corr-bec:sde} and It\^{o}'s formula (\thmref{fpe-sde:ito-formula:func-ito-f}).
Namely, we are interested in population $N_i = \int d\xvec \langle \Psiop_i^\dagger \Psiop_i \rangle$ and interferometric contrast in two-component BEC $V = 2 \left| \int d\xvec \langle \Psiop_1^\dagger \Psiop_2 \rangle \right| / (N_1 + N_2)$.

\begin{theorem}
	Given a set of SDEs~\eqnref{fpe-sde:corr-bec:sde}, the population changes in time as
	\begin{eqn*}
		\frac{dN_i}{dt}
		=
	\end{eqn*}
\end{theorem}
\begin{proof}
First let us find the expression for $\int d\xvec \langle dF[\Psivec] \rangle$ using It\^{o}'s formula with $F = \Psi_i^* \Psi_i$.
The differentials are evaluated as $\delta F / \delta \Psi_c^\prime = \delta_{ic} \Psi_i^* \delta_{\restbasis_i}(\xvec^\prime, \xvec)$, $\delta F / \delta \Psi_c^{\prime *} = \delta_{ic} \Psi_i \delta_{\restbasis_i}^*(\xvec^\prime, \xvec)$, $\delta^2 F / \delta \Psi_j^\prime \delta \Psi_k^{\prime *} = \delta_{ij} \delta_{ik} \delta_{\restbasis_i}(\xvec^\prime, \xvec) \delta_{\restbasis_i}^*(\xvec^\prime, \xvec)$.
Averages $\langle dQ_{\lvec} \rangle$ are equal to zero, so the average of the whole third term in It\^{o} formula is equal to zero.
Also we can safely simplify drift terms to
\begin{eqn}
	\mathcal{A}^{(i)}
	= - \sum_{\lvec} \kappa_{\lvec} \frac{\partial O_{\lvec}^*}{\partial \Psi_i^*} O_{\lvec},
\end{eqn}
since the unitary evolution part obviously does not affect the total population.

Thus:
\begin{eqn}
	\langle dF[\Psivec] \rangle
	= \langle \int d\xvec^\prime
		\mathcal{A}^{(i)\prime} \Psi_i^* \delta_{\restbasis_i}(\xvec^\prime, \xvec)
		+ \mathcal{A}^{(i)\prime *} \Psi_i \delta_{\restbasis_i}^*(\xvec^\prime, \xvec)
		+ \sum_{\lvec} \mathcal{B}_{\lvec}^{(i)\prime} \mathcal{B}_{\lvec}^{(i)\prime *}
			\delta_{\restbasis_i}(\xvec^\prime, \xvec) \delta_{\restbasis_i}^*(\xvec^\prime, \xvec)
	\rangle
\end{eqn}
Integrating by $d\xvec$, and expanding $\Psi_i$ and restricted delta functions:
\begin{eqn}
	\iint d\xvec d\xvec^\prime \mathcal{A}^{(i)\prime} \Psi_i^* \delta_{\restbasis_i}(\xvec^\prime, \xvec)
	& = \iint d\xvec d\xvec^\prime \mathcal{A}^{(i)\prime}
		\sum_{\nvec \in \restbasis_i} \phi_{i,\nvec}^* \alpha_{i,\nvec}^*
		\sum_{\mvec \in \restbasis_i} \phi_{i,\mvec} \phi_{i,\mvec}^{\prime *} \\
	& = \int d\xvec^\prime \mathcal{A}^{(i)\prime}
		\sum_{\mvec \in \restbasis_i} \sum_{\nvec \in \restbasis_i} \delta_{\mvec,\nvec} \alpha_{i,\nvec}^*
		\phi_{i,\mvec}^{\prime *} \\
	& = \int d\xvec^\prime \mathcal{A}^{(i)\prime}
		\sum_{\mvec \in \restbasis_i} \alpha_{i,\mvec}^* \phi_{i,\mvec}^{\prime *} \\
	& = \int d\xvec \mathcal{A}^{(i)} \Psi_i^*.
\end{eqn}
Similarly,
\begin{eqn}
	\iint d\xvec d\xvec^\prime \mathcal{A}^{(i)\prime *} \Psi_i \delta_{\restbasis_i}^*(\xvec^\prime, \xvec)
	= \int d\xvec \mathcal{A}^{(i)*} \Psi_i,
\end{eqn}
and
\begin{eqn}
	\iint d\xvec d\xvec^\prime \mathcal{B}_{\lvec}^{(i)\prime} \mathcal{B}_{\lvec}^{(i)\prime *}
		\delta_{\restbasis_i}(\xvec^\prime, \xvec) \delta_{\restbasis_i}^*(\xvec^\prime, \xvec)
	= \int d\xvec \mathcal{B}_{\lvec}^{(i)} \mathcal{B}_{\lvec}^{(i)*} \delta_{\restbasis_i}^*(\xvec, \xvec).
\end{eqn}

The whole expression is therefore
\begin{eqn}
	\frac{d}{dt} \int d\xvec \langle F[\Psivec] \rangle
	& = \int d\xvec \langle
		\mathcal{A}^{(i)} \Psi_i^*
		+ \mathcal{A}^{(i)*} \Psi_i
		+ \sum_{\lvec} \mathcal{B}_{\lvec}^{(i)} \mathcal{B}_{\lvec}^{(i)*} \delta_{\restbasis_i}^*(\xvec, \xvec)
	\rangle \\
	& = \sum_{\lvec} \kappa_{\lvec} \int d\xvec \langle
		- \frac{\partial O_{\lvec}^*}{\partial \Psi_i^*} O_{\lvec} \Psi_i^*
		- \frac{\partial O_{\lvec}}{\partial \Psi_i} O_{\lvec}^* \Psi_i
		+ \frac{\partial O_{\lvec}}{\partial \Psi_i} \frac{\partial O_{\lvec}^*}{\partial \Psi_i^*}
			\delta_{\restbasis_i}^*(\xvec, \xvec)
	\rangle
\end{eqn}

In order to rewrite the expression above in terms of more intuitive quantities, like component densities, we have to employ the relation \todo{change variable name for coefficient to not clash with nonlinear interactions?}
\begin{eqn}
	\langle \prod_{c=1}^C (\Psiop_c^\dagger)^{l_c} \Psiop_c^{l_c} \rangle
	= g_{\lvec} \prod_{c=1}^C \langle (\Psiop_c^\dagger) \Psiop_c \rangle^{l_c}.
\end{eqn}
In other words, coefficients $g_{\lvec}$ define the degree of high-order correlations.
In the ideal condensate they are equal to 1, and increase with the temperature (the so called photon bunching, or atom bunching).
For example, the theoretical maximum value for $g_{111}$ is $3!$~\cite{Kagan1985}, which was confirmed experimentally~\cite{Burt1997}.

\end{proof}
