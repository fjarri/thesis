% =============================================================================
\section{Fokker-Planck equation for the BEC}
% =============================================================================

With the Wigner truncation applied, we can get rid of the third- and higher-order functional derivatives in~\eqnref{wigner-bec:truncation:untruncated-fpe}.
Under the reasonable assumption of $K_{jk}$, $U_{jk}$ and $\kappa_{\lvec}$ being real-valued,
this results in the functional equation
\begin{eqn}
\label{eqn:wigner-bec:truncation:fpe}
	\frac{\upd W}{\upd t}
	= \int \upd\xvec \left(
		- \sum_{j=1}^C \frac{\fdelta}{\fdelta \Psi_j} \mathcal{A}_j
		- \sum_{j=1}^C \frac{\fdelta}{\fdelta \Psi_j^*} \mathcal{A}_j^*
		+ \sum_{j=1}^C \sum_{k=1}^C \frac{\fdelta^2}{\fdelta \Psi_j^* \fdelta \Psi_k}
			\mathcal{D}_{jk}
	\right) W,
\end{eqn}
with the drift terms
\begin{eqn}
\label{eqn:wigner-bec:truncation:drift-term}
	\mathcal{A}_j
	={} & -\frac{i}{\hbar} \left(
		\sum_{k=1}^C K_{jk} \Psi_k
		+ \Psi_j \sum_{k=1}^C U_{jk} \left(
			|\Psi_k|^2 - \frac{\delta_{jk} + 1}{2} \delta_{\restbasis_k}(\xvec, \xvec)
		\right)
	\right) \\
	& - \sum_{\lvec \in L} \kappa_{\lvec} \left(
		\frac{\upp O_{\lvec}^*}{\upp \Psi_j^*} O_{\lvec}
		- \frac{1}{2} \sum_{k=1}^C \delta_{\restbasis_k}(\xvec, \xvec)
			\frac{\upp^2 O_{\lvec}^*}{\upp \Psi_j^* \upp \Psi_k^*}
			\frac{\upp O_{\lvec}}{\upp \Psi_k}
	\right),
\end{eqn}
and the diffusion matrix
\begin{eqn}
	\mathcal{D}_{jk} = \sum_{\lvec \in L} \kappa_{\lvec}
		\frac{\upp O_{\lvec}}{\upp \Psi_j}
		\frac{\upp O_{\lvec}^*}{\upp \Psi_k^*}.
\end{eqn}

It can be shown (see \appref{fpe-sde} for details) that this truncated equation has a positive-definite diffusion matrix $\mathcal{D}$ and is therefore a Fokker-Planck equation (\abbrev{fpe}), with its solution $W(t)$ being a probability distribution (provided that $W(0)$ was a probability distribution).
This differs from the original Wigner function(al), which can be negative, and is the result of the truncation.
Thus the solution of this equation may not be equivalent to the solution of the orignal master equation~\eqnref{wigner-bec:master-eqn:master-eqn} (if the corresponding density matrices have non-positive Wigner functions), but, given that the truncation conditions are satisfied, the effect is negligible.

Direct solution of the above \abbrev{fpe} is generally impractical, and a Monte Carlo or sampled calculation is called for. The equation can be further transformed to the equivalent set of stochastic differential equations using \thmref{fpe-sde:corr:fpe-sde-func} with
\begin{eqn}
	\mathcal{B}_{j\lvec}
	= \sqrt{\kappa_{\lvec}} \frac{\partial O_{\lvec}^*}{\partial \Psi_j^*}.
\end{eqn}
This results in the set of functional \abbrev{sde}s in It\^{o} form
\begin{eqn}
\label{eqn:fpe-sde:corr-bec:sde}
	\upd\Psi_j = \mathcal{P}_{\restbasis_j} \left[
		\mathcal{A}_j \upd t
		+ \sum_{\lvec \in L} \mathcal{B}_{j\lvec} \upd Q_{\lvec}
	\right],
\end{eqn}
or, alternatively, in Stratonovich form
\begin{eqn}
	\upd\Psi_j = \mathcal{P}_{\restbasis_j} \left[
		(\mathcal{A}_j - \mathcal{S}_j) \upd t
		+ \sum_{\lvec \in L} \mathcal{B}_{j\lvec} \upd Q_{\lvec}
	\right],
\end{eqn}
where the Stratonovich term is
\begin{eqn}
	\mathcal{S}_j
	& = \frac{1}{2} \sum_{k=1}^C \sum_{\lvec \in L}
		\mathcal{B}_{k\lvec}^*
		\frac{\fdelta}{\fdelta \Psi_k^*}
		\mathcal{B}_{j\lvec} \\
	& = \frac{1}{2} \sum_{k=1}^C \sum_{\lvec \in L}
		\delta_{\restbasis_k}(\xvec, \xvec)
		\frac{\upp^2 O_{\lvec}^*}{\upp \Psi_k^* \upp \Psi_j^*}
		\frac{\upp O_{\lvec}}{\upp \Psi_k}.
\end{eqn}
Note that the Stratonovich term is exactly equal to the correction in the loss-induced part of the drift term~\eqnref{wigner-bec:truncation:drift-term}.
This means that in Stratonovich form the \abbrev{sde}s are actually simpler.


% =============================================================================
\subsection{Integral averages}
% =============================================================================

It is interesting to get an expression for time dependence of some simple observables using SDEs~\eqnref{fpe-sde:corr-bec:sde} and It\^{o}'s formula (\thmref{fpe-sde:ito-formula:func-ito-f}).
Namely, we are interested in population $N_i = \int \upd\xvec \langle \Psiop_i^\dagger \Psiop_i \rangle$ and interferometric contrast in two-component BEC $V = 2 \left| \int \upd\xvec \langle \Psiop_1^\dagger \Psiop_2 \rangle \right| / (N_1 + N_2)$.

\begin{theorem}
	Given a set of SDEs~\eqnref{fpe-sde:corr-bec:sde}, the population changes in time as
	\begin{eqn*}
		\frac{dN_i}{dt}
		=
	\end{eqn*}
\end{theorem}
\begin{proof}
First let us find the expression for $\int \upd\xvec \langle dF[\Psivec] \rangle$ using It\^{o}'s formula with $F = \Psi_i^* \Psi_i$.
The differentials are evaluated as $\delta F / \delta \Psi_c^\prime = \delta_{ic} \Psi_i^* \delta_{\restbasis_i}(\xvec^\prime, \xvec)$, $\delta F / \delta \Psi_c^{\prime *} = \delta_{ic} \Psi_i \delta_{\restbasis_i}^*(\xvec^\prime, \xvec)$, $\delta^2 F / \delta \Psi_j^\prime \delta \Psi_k^{\prime *} = \delta_{ij} \delta_{ik} \delta_{\restbasis_i}(\xvec^\prime, \xvec) \delta_{\restbasis_i}^*(\xvec^\prime, \xvec)$.
Averages $\langle dQ_{\lvec} \rangle$ are equal to zero, so the average of the whole third term in It\^{o} formula is equal to zero.
Also we can safely simplify drift terms to
\begin{eqn}
	\mathcal{A}^{(i)}
	= - \sum_{\lvec} \kappa_{\lvec} \frac{\partial O_{\lvec}^*}{\partial \Psi_i^*} O_{\lvec},
\end{eqn}
since the unitary evolution part obviously does not affect the total population.

Thus:
\begin{eqn}
	\langle dF[\Psivec] \rangle
	= \langle \int \upd\xvec^\prime
		\mathcal{A}^{(i)\prime} \Psi_i^* \delta_{\restbasis_i}(\xvec^\prime, \xvec)
		+ \mathcal{A}^{(i)\prime *} \Psi_i \delta_{\restbasis_i}^*(\xvec^\prime, \xvec)
		+ \sum_{\lvec} \mathcal{B}_{\lvec}^{(i)\prime} \mathcal{B}_{\lvec}^{(i)\prime *}
			\delta_{\restbasis_i}(\xvec^\prime, \xvec) \delta_{\restbasis_i}^*(\xvec^\prime, \xvec)
	\rangle
\end{eqn}
Integrating by $d\xvec$, and expanding $\Psi_i$ and restricted delta functions:
\begin{eqn}
	\iint d\xvec d\xvec^\prime \mathcal{A}^{(i)\prime} \Psi_i^* \delta_{\restbasis_i}(\xvec^\prime, \xvec)
	& = \iint d\xvec d\xvec^\prime \mathcal{A}^{(i)\prime}
		\sum_{\nvec \in \restbasis_i} \phi_{i,\nvec}^* \alpha_{i,\nvec}^*
		\sum_{\mvec \in \restbasis_i} \phi_{i,\mvec} \phi_{i,\mvec}^{\prime *} \\
	& = \int \upd\xvec^\prime \mathcal{A}^{(i)\prime}
		\sum_{\mvec \in \restbasis_i} \sum_{\nvec \in \restbasis_i} \delta_{\mvec,\nvec} \alpha_{i,\nvec}^*
		\phi_{i,\mvec}^{\prime *} \\
	& = \int \upd\xvec^\prime \mathcal{A}^{(i)\prime}
		\sum_{\mvec \in \restbasis_i} \alpha_{i,\mvec}^* \phi_{i,\mvec}^{\prime *} \\
	& = \int \upd\xvec \mathcal{A}^{(i)} \Psi_i^*.
\end{eqn}
Similarly,
\begin{eqn}
	\iint d\xvec d\xvec^\prime \mathcal{A}^{(i)\prime *} \Psi_i \delta_{\restbasis_i}^*(\xvec^\prime, \xvec)
	= \int \upd\xvec \mathcal{A}^{(i)*} \Psi_i,
\end{eqn}
and
\begin{eqn}
	\iint d\xvec d\xvec^\prime \mathcal{B}_{\lvec}^{(i)\prime} \mathcal{B}_{\lvec}^{(i)\prime *}
		\delta_{\restbasis_i}(\xvec^\prime, \xvec) \delta_{\restbasis_i}^*(\xvec^\prime, \xvec)
	= \int \upd\xvec \mathcal{B}_{\lvec}^{(i)} \mathcal{B}_{\lvec}^{(i)*} \delta_{\restbasis_i}^*(\xvec, \xvec).
\end{eqn}

The whole expression is therefore
\begin{eqn}
	\frac{d}{dt} \int \upd\xvec \langle F[\Psivec] \rangle
	& = \int \upd\xvec \langle
		\mathcal{A}^{(i)} \Psi_i^*
		+ \mathcal{A}^{(i)*} \Psi_i
		+ \sum_{\lvec} \mathcal{B}_{\lvec}^{(i)} \mathcal{B}_{\lvec}^{(i)*} \delta_{\restbasis_i}^*(\xvec, \xvec)
	\rangle \\
	& = \sum_{\lvec} \kappa_{\lvec} \int \upd\xvec \langle
		- \frac{\partial O_{\lvec}^*}{\partial \Psi_i^*} O_{\lvec} \Psi_i^*
		- \frac{\partial O_{\lvec}}{\partial \Psi_i} O_{\lvec}^* \Psi_i
		+ \frac{\partial O_{\lvec}}{\partial \Psi_i} \frac{\partial O_{\lvec}^*}{\partial \Psi_i^*}
			\delta_{\restbasis_i}^*(\xvec, \xvec)
	\rangle
\end{eqn}

In order to rewrite the expression above in terms of more intuitive quantities, like component densities, we have to employ the relation \todo{change variable name for coefficient to not clash with nonlinear interactions?}
\begin{eqn}
	\langle \prod_{c=1}^C (\Psiop_c^\dagger)^{l_c} \Psiop_c^{l_c} \rangle
	= g_{\lvec} \prod_{c=1}^C \langle (\Psiop_c^\dagger) \Psiop_c \rangle^{l_c}.
\end{eqn}
In other words, coefficients $g_{\lvec}$ define the degree of high-order correlations.
In the ideal condensate they are equal to 1, and increase with the temperature (the so called photon bunching, or atom bunching).
For example, the theoretical maximum value for $g_{111}$ is $3!$~\cite{Kagan1985}, which was confirmed experimentally~\cite{Burt1997}.

\end{proof}
