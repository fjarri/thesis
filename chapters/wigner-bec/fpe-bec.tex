% =============================================================================
\section{Fokker-Planck equation for the BEC}
% =============================================================================

With the Wigner truncation applied, we can remove the third- and higher-order functional derivatives in~\eqnref{wigner-bec:truncation:untruncated-fpe}.
Under the reasonable assumption of $K_{jk}$, $U_{jk}$ and $\kappa_{\lvec}$ being real-valued, this results in the functional equation
\begin{eqn}
\label{eqn:wigner-bec:truncation:fpe}
	\frac{\upd W}{\upd t}
	= \int \upd\xvec \left(
		- \sum_{j=1}^C \frac{\fdelta}{\fdelta \Psi_j} \mathcal{A}_j
		- \sum_{j=1}^C \frac{\fdelta}{\fdelta \Psi_j^*} \mathcal{A}_j^*
		+ \sum_{j=1}^C \sum_{k=1}^C \frac{\fdelta^2}{\fdelta \Psi_j^* \fdelta \Psi_k}
			\mathcal{D}_{jk}
	\right) W,
\end{eqn}
with drift terms
\begin{eqn}
\label{eqn:wigner-bec:truncation:drift-term}
	\mathcal{A}_j
	={} & -\frac{i}{\hbar} \left(
		\sum_{k=1}^C K_{jk} \Psi_k
		+ \Psi_j \sum_{k=1}^C U_{jk} \left(
			|\Psi_k|^2 - \frac{\delta_{jk} + 1}{2} \delta_{\restbasis_k}(\xvec, \xvec)
		\right)
	\right) \\
	& - \sum_{\lvec \in L} \kappa_{\lvec} \left(
		\frac{\upp O_{\lvec}^*}{\upp \Psi_j^*} O_{\lvec}
		- \frac{1}{2} \sum_{k=1}^C \delta_{\restbasis_k}(\xvec, \xvec)
			\frac{\upp^2 O_{\lvec}^*}{\upp \Psi_j^* \upp \Psi_k^*}
			\frac{\upp O_{\lvec}}{\upp \Psi_k}
	\right),
\end{eqn}
and diffusion matrix
\begin{eqn}
\label{eqn:wigner-bec:truncation:diffusion-term}
	\mathcal{D}_{jk} = \sum_{\lvec \in L} \kappa_{\lvec}
		\frac{\upp O_{\lvec}}{\upp \Psi_j}
		\frac{\upp O_{\lvec}^*}{\upp \Psi_k^*}.
\end{eqn}

It can be shown (see \appref{fpe-sde} for details) that this truncated equation has a positive-definite diffusion matrix $\mathcal{D}$ and is, therefore, a Fokker-Planck equation (\abbrev{fpe}), with its solution $W(t)$ being a probability distribution (provided that $W(0)$ was a probability distribution).
This differs from the original Wigner function(al), which can be negative, and is the result of the truncation.
Thus, the solution of this equation may not be equivalent to the solution of the original master equation~\eqnref{wigner-bec:master-eqn:master-eqn}, if the corresponding density matrices have non-positive Wigner functions.
If the truncation condition is satisfied, the difference is small (see \charef{exact} for a simple comparison, and references therein).

A direct solution of the above \abbrev{fpe} is generally impractical, and a Monte Carlo or a sampled calculation is called for.
The equation can be further transformed to an equivalent set of stochastic differential equations (\abbrev{sde}s) using \thmref{fpe-sde:corr:fpe-sde-func} with
\begin{eqn}
	\mathcal{B}_{j\lvec}
	= \sqrt{\kappa_{\lvec}} \frac{\partial O_{\lvec}^*}{\partial \Psi_j^*}.
\end{eqn}
This results in the set of functional \abbrev{sde}s in the It\^o form
\begin{eqn}
\label{eqn:wigner-bec:fpe-bec:sde}
	\upd\Psi_j = \mathcal{P}_{\restbasis_j} \left[
		\mathcal{A}_j \upd t
		+ \sum_{\lvec \in L} \mathcal{B}_{j\lvec} \upd Q_{\lvec}
	\right],
\end{eqn}
or, alternatively, in the Stratonovich form
\begin{eqn}
\label{eqn:wigner-bec:fpe-bec:sde-stratonovich}
	\upd\Psi_j = \mathcal{P}_{\restbasis_j} \left[
		(\mathcal{A}_j - \mathcal{S}_j) \upd t
		+ \sum_{\lvec \in L} \mathcal{B}_{j\lvec} \upd Q_{\lvec}
	\right],
\end{eqn}
where the Stratonovich term is
\begin{eqn}
	\mathcal{S}_j
	& = \frac{1}{2} \sum_{k=1}^C \sum_{\lvec \in L}
		\mathcal{B}_{k\lvec}^*
		\frac{\fdelta}{\fdelta \Psi_k^*}
		\mathcal{B}_{j\lvec} \\
	& = \frac{1}{2} \sum_{k=1}^C \sum_{\lvec \in L}
		\delta_{\restbasis_k}(\xvec, \xvec)
		\frac{\upp^2 O_{\lvec}^*}{\upp \Psi_k^* \upp \Psi_j^*}
		\frac{\upp O_{\lvec}}{\upp \Psi_k}.
\end{eqn}
Note that the Stratonovich term is exactly equal to the correction proportional to $\delta_{\restbasis_k}$ in the loss-induced part of the drift term~\eqnref{wigner-bec:truncation:drift-term}.
This means that in the Stratonovich form the \abbrev{sde}s are actually simpler.
Physically, this corresponds to the fact that the Stratonovich form is the broad-band limit of a finite bandwidth noise equation.
Hence, one expects this to occur as the most natural equation in the Markovian master equation limit of damping.

The equations~\eqnref{wigner-bec:fpe-bec:sde} can be solved by conventional integration methods for \abbrev{sde}s (see~\appref{numerical} for details).
The required expectations of symmetrically ordered operator products can be obtained by averaging results from multiple independent integration trajectories according to \thmref{wigner:mc:moments}, since the truncated Wigner functional is a probability distribution:
\begin{eqn}
\label{eqn:wigner-bec:fpe-bec:moments}
	\left\langle
		\symprod{ \prod_{j=1}^C \Psiop_j^{r_j} (\Psiop_j^\dagger)^{s_j} }
	\right\rangle
	& = \int \fdelta^2 \bPsi\,
		\left( \prod_{j=1}^C \Psi_j^{r_j} (\Psi_j^*)^{s_j} \right) W[\bPsi] \\
	& \approx \pathavg{
		\prod_{j=1}^C \Psi_j^{r_j} (\Psi_j^*)^{s_j}
	},
\end{eqn}
where $\{r_j\}$ and $\{s_j\}$ are some sets of non-negative integers.
The second approximate equality becomes exact in the limit of the infinite number of integration trajectories.
