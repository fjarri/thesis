% =============================================================================
\section{Energy cutoff}
% =============================================================================

As was noted earlier, in order to use contact interactions, an energy cutoff has to be imposed.
We will use the formalism for multimode operators described in \charef{formalism}.

Two basis sets are used in this work.
First is the basis of plane waves:
\[
	\phi_{\nvec}(\xvec) = e^{i \kvec_{\nvec} \xvec} / \sqrt{V},
\]
whose elements are eigenfunctions of kinetic term:
\[
	\left( -\frac{\hbar^2}{2m} \nabla^2 \right) \phi_{\nvec}
	= E_{\nvec} \phi_{\nvec}
	= \frac{\hbar^2 \lvert \kvec_{\nvec} \rvert^2}{2 m} \phi_{\nvec}.
\]
Integration area for this set is the box with volume $V = \prod_{j=1}^D L_j$.
Spatial wave vectors are
\[
	\kvec_{\nvec} = \sum_{j=1}^D \frac{\pi n_j}{L_j} \vec{x}_j,
\]
where $L_j$ are dimensions of the box, and $\vec{x}_j$ are spatial unit vectors.
Field functions can be decomposed into this basis using common FFTs.

More sophisticated basis can be constructed out of harmonic oscillator modes,
which have the whole space as an integration area.
One of the ways to decompose wave function into this basis is the Gauss-Hermite quadrature
(see \appref{harmonic-transform} for technical details).

We divide mode subspaces into low- and high-energy subsets $L$ and $H$,
depending on whether $E_n$ is more or less than some cutoff energy $\ecut$,
and limit field operators to the subset $L$.
Restricted operators obey commutation relations~\eqnref{formalism:func-aux:restricted-commutators}.

\begin{figure}
\begin{center}
\subfloat[Uniform grid, $\ecut = 1000\,\hbar\omega$]{\includegraphics[width=0.5\textwidth]{%
	figures_generated/wigner/delta_uniform_1000.eps}}
\subfloat[Harmonic grid, $\ecut = 1000\,\hbar\omega$]{\includegraphics[width=0.5\textwidth]{%
	figures_generated/wigner/delta_harmonic_1000.eps}} \\
\subfloat[Uniform grid, no explicit cutoff]{\includegraphics[width=0.5\textwidth]{%
	figures_generated/wigner/delta_uniform_all.eps}}
\subfloat[Harmonic grid, no explicit cutoff]{\includegraphics[width=0.5\textwidth]{%
	figures_generated/wigner/delta_harmonic_all.eps}}
\end{center}
\caption{Absolute value of restricted delta function $\lvert \delta_P(z^\prime - z) \rvert$ with and without explicit cutoff.}
\label{fig:wigner-bec:cutoff:restricted-delta}
\end{figure}

Shape of the module of restricted delta function~\eqnref{formalism:func-calculus:restricted-delta} for two different basis sets and two different cutoffs is shown in~\figref{wigner-bec:cutoff:restricted-delta}.
The higher cutoff is, the closer its shape gets to the ``ideal'' delta function,
with nonlocality scaling as $\ecut^{-1/2}$ \todo{proof needed}.
