% =============================================================================
\section{Master equation}
% =============================================================================

The time evolution of the density matrix $\hat{\rho}$ of a multi-component \abbrev{bec} with particle losses can be written as a Markovian master equation~\cite{Jack2002}
\begin{eqn}
\label{eqn:wigner-bec:master-eqn:master-eqn}
	\frac{\upd\hat{\rho}}{\upd t} =
		- \frac{i}{\hbar} \left[ \hat{H}, \hat{\rho} \right]
		+ \sum_{\lvec \in L} \kappa_{\lvec} \int \upd \xvec
			\mathcal{L}_{\lvec} \left[ \hat{\rho} \right],
\end{eqn}
where $L$ is the set of all loss processes taken into account, and $\lvec = (l_1, l_2, \ldots, l_C)$ is a vector containing the number of particles of each component involved in the inelastic interaction causing the specific loss process (with $C$ being the number of components).
The coefficients $\kappa_{\lvec}$ are strengths of the corresponding loss processes, whose connection to experimentally measurable population losses will be discussed later in \secref{bec-noise:wigner} and is expressed by Eqn.~\eqnref{bec-noise:wigner:loss-rates-relation}.
Here we have introduced local Liouville loss terms that describe $n$-body collisional losses in the Markovian approximation,
\begin{eqn}
\label{eqn:wigner-bec:master-eqn:loss-op}
	\mathcal{L}_{\lvec} \left[ \hat{\rho} \right] =
		2\hat{O}_{\lvec} \hat{\rho} \hat{O}_{\lvec}^\dagger
		- \hat{O}_{\lvec}^\dagger \hat{O}_{\lvec} \hat{\rho}
		- \hat{\rho} \hat{O}_{\lvec}^\dagger \hat{O}_{\lvec}.
\end{eqn}
The reservoir coupling operators $\hat{O}_{\lvec}$ are products of restricted field annihilation operators
\begin{eqn}
	\hat{O}_{\lvec} \equiv \hat{O}_{\lvec} (\Psiopvec) = \prod_{j=1}^C \Psiop_j^{l_j},
\end{eqn}
describing local $n=(\sum l_c)$-body collision losses.
This definition assumes that $l_1, \ldots, l_C$ particles of internal states with quantum numbers $j=1,\ldots C$ respectively all collide simultaneously within the volume corresponding to the inverse momentum cutoff, and are removed from the Bose gas.

This minimal model of particle loss assumes that the reservoir of ``lost'' particles does not interact with the original Bose gas.
Its accuracy thus depends on the trapping mechanism since the ``lost'' particles are simply in a different quantum state (or are combined into molecules).
The model is valid if the trap is state-selective, or the collision is highly exothermic, such that the resulting particles are able to move away rapidly.
It is also possible to treat non-Markovian reservoirs within this formalism, by extending the Hamiltonian to include the detailed loss dynamics, but this is not treated in detail in this thesis.

The master equation allows us to derive an important property.

\begin{theorem}
	The time derivative of the expectation of the field operator whose evolution is described by the master equation~\eqnref{wigner-bec:master-eqn:master-eqn} with the Hamiltonian~\eqnref{wigner-bec:hamiltonian:effective-H} is
	\begin{eqn*}
		\frac{\upd}{\upd t} \langle \Psiop_j \rangle
		= \mathcal{P}_{\restbasis_j} \left[
			\left\langle
				-\frac{i}{\hbar} \sum_{k=1}^C \left(
					K_{jk} \Psiop_k
					+ U_{jk} \Psiop_k^\dagger \Psiop_k \Psiop_j
				\right)
				- \sum_{\lvec \in L} \kappa_{\lvec}
					\frac{\upp \hat{O}_{\lvec}^\dagger}{\upp \Psiop_j^\dagger} \hat{O}_{\lvec}
			\right\rangle
		\right].
	\end{eqn*}
\end{theorem}
\begin{proof}
Rewriting the expectation as a trace, and substituting the right-hand part of~\eqnref{wigner-bec:master-eqn:master-eqn}:
\begin{eqn}
	\frac{\upd}{\upd t} \langle \Psiop_j \rangle
	={} & \frac{\upd}{\upd t} \Trace{ \hat{\rho} \Psiop_j }
	= \Trace{ \frac{\upd\hat{\rho}}{\upd t} \Psiop_j } \\
	={} & \int \upd \xvec^\prime \left(
		- \frac{i}{\hbar} \sum_{m=1}^C \sum_{k=1}^C \Trace{
			\left[
				\Psiop_m^{\prime\dagger} K_{mk}^\prime \Psiop_k^\prime,
				\hat{\rho}
			\right] \Psiop_j
		} \right. \\
	& \quad - \frac{i}{2\hbar} \sum_{m=1}^C \sum_{k=1}^C U_{mk} \Trace{
			\left[
				\Psiop_m^{\prime\dagger} \Psiop_k^{\prime\dagger}
				\Psiop_m^\prime \Psiop_k^\prime,
				\hat{\rho}
			\right] \Psiop_j
		} \\
	& \quad \left. + \sum_{\lvec \in L} \kappa_{\lvec}
			\Trace{
				\mathcal{L}_{\lvec}^\prime \left[ \hat{\rho} \right]
				\Psiop_j
			}
	\right),
\end{eqn}
where $K_{mk}^\prime \equiv K_{mk}(\xvec^\prime)$, $\mathcal{L}_{\lvec}^\prime \equiv \mathcal{L} ( \Psiopvec^\prime )$.

Let us transform each term separately.
Noticing that $K_{mk}^\prime$ commutes with $\Psiop_j$, and using \lmmref{wigner:op-calculus:moment-commutators}, we can transform the first term as:
\begin{eqn}
	\Trace{
		\left[
			\Psiop_m^{\prime\dagger} K_{mk}^\prime \Psiop_k^\prime,
			\hat{\rho}
		\right] \Psiop_j
	}
	& = \Trace{
		\hat{\rho} \left(
			\Psiop_j \Psiop_m^{\prime\dagger} K_{mk}^\prime \Psiop_k^\prime
			- \Psiop_m^{\prime\dagger} K_{mk}^\prime \Psiop_k^\prime \Psiop_j
		\right)
	} \\
	& = \Trace{
		\hat{\rho} \left[
			\Psiop_j, \Psiop_m^{\prime\dagger}
		\right] K_{mk}^\prime \Psiop_k^\prime
	} \\
	& = \Trace{
		\hat{\rho} \delta_{jm} \delta_{\restbasis_j}(\xvec^\prime, \xvec) K_{mk}^\prime \Psiop_k^\prime
	}
	= \delta_{jm} \delta_{\restbasis_j}(\xvec^\prime, \xvec)
		\langle K_{mk}^\prime \Psiop_k^\prime \rangle.
\end{eqn}

The second (nonlinear) term is transformed similarly:
\begin{eqn}
	\Trace{
		\left[
			\Psiop_m^{\prime\dagger} \Psiop_k^{\prime\dagger}
			\Psiop_m^\prime \Psiop_k^\prime,
			\hat{\rho}
		\right] \Psiop_j
	}
	& = \Trace{
		\hat{\rho} \left(
			\Psiop_j \Psiop_m^{\prime\dagger} \Psiop_k^{\prime\dagger}
			\Psiop_m^\prime \Psiop_k^\prime
			- \Psiop_m^{\prime\dagger} \Psiop_k^{\prime\dagger}
			\Psiop_m^\prime \Psiop_k^\prime \Psiop_j
		\right)
	} \\
	& = \Trace{
		\hat{\rho} \left[
			\Psiop_j, \Psiop_m^{\prime\dagger} \Psiop_k^{\prime\dagger}
		\right] \Psiop_m^\prime \Psiop_k^\prime
	} \\
	& = \Trace{
		\hat{\rho} \delta_{\restbasis_j}(\xvec^\prime, \xvec) \left(
			\delta_{jm} \Psiop_k^{\prime\dagger}
			+ \delta_{jk} \Psiop_m^{\prime\dagger}
		\right) \Psiop_m^\prime \Psiop_k^\prime
	}.
\end{eqn}
Swapping $m \leftrightarrow k$ in the term corresponding to the second Kronecker delta, and summing over $m$:
\begin{eqn}
	\sum_{m=1}^C U_{mk} \Trace{
		\left[
			\Psiop_m^{\prime\dagger} \Psiop_k^{\prime\dagger}
			\Psiop_m^\prime \Psiop_k^\prime,
			\hat{\rho}
		\right] \Psiop_j
	}
	& = 2 U_{jk} \delta_{\restbasis_j}(\xvec^\prime, \xvec) \Trace{
		\hat{\rho} \Psiop_k^{\prime\dagger} \Psiop_k^\prime \Psiop_j^\prime
	} \\
	& = 2 U_{jk} \delta_{\restbasis_j}(\xvec^\prime, \xvec) \langle
		\Psiop_k^{\prime\dagger} \Psiop_k^\prime \Psiop_j^\prime
	\rangle.
\end{eqn}

Finally, for the third term we notice that $\hat{O}_{\lvec}^\prime$ commutes with $\Psiop_j$ since it is just a product of annihilation operators itself:
\begin{eqn}
	\Trace{
		\mathcal{L}_{\lvec}^\prime \left[ \hat{\rho} \right]
		\Psiop_j
	}
	& = \Trace{
		\hat{\rho} \hat{O}_{\lvec}^{\prime\dagger} \hat{O}_{\lvec}^\prime \Psiop_j
		+ \hat{\rho} \hat{O}_{\lvec}^{\prime\dagger} \Psiop_j \hat{O}_{\lvec}^\prime
		- \hat{\rho} \Psiop_j \hat{O}_{\lvec}^{\prime\dagger} \hat{O}_{\lvec}^\prime
		- \hat{\rho} \hat{O}_{\lvec}^{\prime\dagger} \hat{O}_{\lvec}^\prime \Psiop_j
	} \\
	& = \Trace{
		\hat{\rho} \left[
			\hat{O}_{\lvec}^{\prime\dagger}, \Psiop_j
		\right] \hat{O}_{\lvec}^\prime
	}.
\end{eqn}
Representing the commutator as a formal derivative of the coupling operator according to \lmmref{wigner:op-calculus:functional-commutators}:
\begin{eqn}
	\Trace{
		\mathcal{L}_{\lvec}^\prime \left[ \hat{\rho} \right]
		\Psiop_j
	}
	& = -\delta_{\restbasis_j}(\xvec^\prime, \xvec) \Trace{
		\hat{\rho} \frac{\upp \hat{O}_{\lvec}^{\prime\dagger}}{\upp \Psiop_j^{\prime\dagger}}
		\hat{O}_{\lvec}^\prime
	} \\
	& = -\delta_{\restbasis_j}(\xvec^\prime, \xvec) \left\langle
		\frac{\upp \hat{O}_{\lvec}^{\prime\dagger}}{\upp \Psiop_j^{\prime\dagger}}
		\hat{O}_{\lvec}^\prime
	\right\rangle.
\end{eqn}

Thus, the full relation is
\begin{eqn}
	\frac{d}{dt} \langle \Psiop_j \rangle
	& = \int \upd \xvec^\prime \delta_{\restbasis_j}(\xvec^\prime, \xvec) \left(
		- \frac{i}{\hbar} \sum_{k=1}^C \left(
			\langle K_{jk}^\prime \Psiop_k^\prime \rangle
			+ U_{jk} \langle
				\Psiop_k^{\prime\dagger} \Psiop_k^\prime \Psiop_j^\prime
			\rangle
		\right)
		- \sum_{\lvec} \kappa_{\lvec} \left\langle
			\frac{\upp \hat{O}_{\lvec}^{\prime\dagger}}{\upp \Psiop_j^{\prime\dagger}}
			\hat{O}_{\lvec}^\prime
		\right\rangle
	\right),
\end{eqn}
which is equivalent to the statement of the theorem after replacing the integration with the restricted delta by a projector according to~\eqnref{func-calculus:projector-via-delta}.
\end{proof}

The same method can be applied to calculating the time derivative of the per-component population.

\begin{theorem}
	For a system whose evolution is described by the master equation~\eqnref{wigner-bec:master-eqn:master-eqn} with the Hamiltonian~\eqnref{wigner-bec:hamiltonian:effective-H}, the time derivative of the population of the component $j$ is
	\begin{eqn*}
		\frac{d}{dt} \int \upd\xvec \langle \Psiop_j^\dagger \Psiop_j \rangle
		= -2 \sum_{\lvec} \kappa_{\lvec}
			\int \upd\xvec
			\left\langle
				\Psiop_j^\dagger
				\mathcal{P}_{\restbasis_j} \left[
					\frac{\upp \hat{O}_{\lvec}^\dagger}{\upp \Psiop_j^\dagger} \hat{O}_{\lvec}
				\right]
			\right\rangle.
	\end{eqn*}
\end{theorem}
\begin{proof}
Starting from the master equation, and ignoring the unitary evolution (which, obviously, will not contribute to the total population change):
\begin{eqn}
	\frac{d}{dt} \int \upd\xvec \langle \Psiop_j^\dagger \Psiop_j \rangle
	= \sum_{\lvec \in L} \kappa_{\lvec}
			\int \upd\xvec\, \upd \xvec^\prime \Trace{
				\mathcal{L}_{\lvec}^\prime \left[ \hat{\rho} \right]
				\Psiop_j^\dagger \Psiop_j
			}.
\end{eqn}
Expanding the loss operator:
\begin{eqn}
	\frac{d}{dt} \langle \Psiop_j^\dagger \Psiop_j \rangle
	={} & \sum_{\lvec \in L} \kappa_{\lvec}
			\iint \upd\xvec\, \upd\xvec^\prime \Trace{
				\hat{\rho} \left(
					2 \hat{O}_{\lvec}^{\prime\dagger} \Psiop_j^\dagger \Psiop_j \hat{O}_{\lvec}^\prime
					- \Psiop_j^\dagger \Psiop_j \hat{O}_{\lvec}^{\prime\dagger} \hat{O}_{\lvec}^\prime
					- \hat{O}_{\lvec}^{\prime\dagger} \hat{O}_{\lvec}^\prime \Psiop_j^\dagger \Psiop_j
				\right)
			} \\
	={} & \sum_{\lvec \in L} \kappa_{\lvec}
			\iint \upd\xvec\, \upd\xvec^\prime \langle
				\Psiop_j^\dagger \left[
					\hat{O}_{\lvec}^{\prime\dagger} \hat{O}_{\lvec}^\prime, \Psiop_j
				\right]
				+ \left[
					\Psiop_j^\dagger, \hat{O}_{\lvec}^{\prime\dagger} \hat{O}_{\lvec}^\prime
				\right] \Psiop_j
			\rangle \\
	={} & \sum_{\lvec \in L} \kappa_{\lvec}
			\iint \upd\xvec\, \upd\xvec^\prime 2 \Real \langle
				\Psiop_j^\dagger \left[
					\hat{O}_{\lvec}^{\prime\dagger} \hat{O}_{\lvec}^\prime, \Psiop_j
				\right]
			\rangle.
\end{eqn}
Applying \lmmref{wigner:op-calculus:functional-commutators} to transform the commutator and using~\eqnref{func-calculus:projector-via-delta} to integrate over $\xvec^\prime$:
\begin{eqn}
	\frac{d}{dt} \int \upd\xvec \langle \Psiop_j^\dagger \Psiop_j \rangle
	={} & -2\sum_{\lvec \in L} \kappa_{\lvec}
			\iint \upd\xvec\, \upd\xvec^\prime \Real \left\langle
				\Psiop_j^\dagger
				\delta_{\restbasis_j}(\xvec^\prime, \xvec)
				\frac{\upp \left(
					\hat{O}_{\lvec}^{\prime\dagger} \hat{O}_{\lvec}^\prime
					\right)}{\upp \Psiop_j^{\prime\dagger}}
			\right\rangle \\
	={} & -2\sum_{\lvec \in L} \kappa_{\lvec}
			\int \upd\xvec \left\langle
				\Psiop_j^\dagger
				\mathcal{P}_{\restbasis_j} \left[
					\frac{\upp \hat{O}_{\lvec}^\dagger}{\upp \Psiop_j^\dagger}
					\hat{O}_{\lvec}
				\right]
			\right\rangle.
\end{eqn}

\end{proof}
