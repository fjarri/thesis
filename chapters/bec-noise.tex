% =============================================================================
\chapter{Quantum noise in BEC interferometry}
\label{cha:bec-noise}
% =============================================================================

\copypaste{
Atom interferometry is an important quantum technology at the heart of many proposed future applications of ultra-cold atomic physics.
Bose-Einstein condensates (\abbrev{bec}s) or atom lasers are macroscopic quantum objects and have potential advantages as interferometric detectors and sensors, provided one can precisely extract atomic phase information.
However, unlike photons, atoms can interact strongly, causing dephasing and loss of interference fringes.
An intimate understanding of quantum many-body dynamics is the key to calculating interaction-induced dephasing in the measurement process.
This is essential for a quantitative theory of atom interferometry.
}

In this chapter we will apply the truncated Wigner method from \charef{wigner-bec} to the task of simulating the dynamics of an interferometry experiment with large atom number.
We will start from describing the mean-field approach leading to the conventional Gross-Pitaevskii equations~\cite{Pitaevskii2003} (\abbrev{gpe}s).
We then extend it to include quantum effects, such as the noise from linear and nonlinear losses.
The accurate description of nonlinear losses are especially important as \copypaste{they become dominant when atom numbers are increased to improve fringe visibility.}

\copypaste{
Importantly, we can clearly demonstrate where fringe visibility is driven by quantum fluctuations, and where it is driven by trap inhomogeneity and dynamical effects, in order to choose optimal conditions for quantum noise reduction and spin squeezing.
These calculations are a first step towards understanding mesoscopic superpositions and entanglement in ultra-cold atomic gases.
An advantage of our method compared to the variational and perturbative approaches used elsewhere~\cite{Li2009,Sakmann2009,Sinatra2011}
is that it allows us to treat a large number of independent field modes, thus including degrees of freedom that are excited due to collisional and nonlinear loss dynamics~\cite{Norrie2005,Deuar2007}.
Our theory can be readily extended to include finite-temperature initial conditions~\cite{Steel1998,Isella2006},
which will be treated elsewhere.
Nonlinear losses and finite-temperature effects can be also described within the confines of the variational approach~\cite{Li2008}.
}

\copypaste{
Quantum phase-diffusion is defined as the phase noise induced by number fluctuations which are conjugate to phase.
This is a fundamental feature of \abbrev{bec} interferometry, and can only be removed when there are no interactions.
However, there are other reasons for decoherence, which are also important.
The approach used here captures all three significant features of atom interferometry that can result in decoherence: phase-diffusion, losses, and trap inhomogeneity effects.
The results given in this chapter are applicable to simulations where the atom number per lattice point or mode is large.
}

% =============================================================================
\section{Mean-field approximation}
\label{sec:bec-noise:mean-field}
% =============================================================================

We will start from a classical model of a trapped \abbrev{bec}~--- the mean-field approximation.
While it cannot predict quantum effects, it provides the basis for comparison with the truncated Wigner method, and can be also applied to calculate the ground state of a trapped \abbrev{bec} numerically.

We will use the combined model which includes linear coupling and losses, and later in \secref{bec-noise:wigner}, dedicated to the application of the truncated Wigner method we will show how we can get almost identical equations from the first principles.


% =============================================================================
\subsection{Two-component condensate}
% =============================================================================

In the mean-field approximation the two-component \abbrev{bec} is described by wavefunctions $\Psi_j$, which are normalized such that $\int |\Psi_j|^2 \upd\xvec \equiv N_j$, where $N_j$ is the total population of the component $j$ (consequently, $|\Psi_j|^2 \equiv n_j$ is the component density).
The evolution of the condensate is described by the system of coupled Gross-Pitaevskii equations (\abbrev{cgpe}s)~\cite{Pitaevskii2003}
\begin{eqn}
\label{eqn:bec-noise:mean-field:cgpes}
	i \hbar \frac{\upd \Psi_1}{\upd t} ={} & \left(
		-\frac{\hbar^2 \nabla^2}{2 m} + V_1
		+ g_{11} \lvert \Psi_1 \rvert^2
		+ g_{12} \lvert \Psi_2 \rvert^2
		- i \hbar \Gamma_1
	\right) \Psi_1 \\
	& + \frac{\hbar \Omega}{2} \left(
		e^{i (\omega t + \alpha)} + e^{-i (\omega t + \alpha)}
	\right) \Psi_2, \\
	i \hbar \frac{\upd \Psi_2}{\upd t} ={} & \left(
		-\frac{\hbar^2 \nabla^2}{2 m} + V_2 + \hbar \omega_{hf}
		+ g_{22} \lvert \Psi_2 \rvert^2
		+ g_{12} \lvert \Psi_1 \rvert^2
		- i \hbar \Gamma_2
	\right) \Psi_2 \\
	& + \frac{\hbar \Omega}{2} \left(
		e^{i (\omega t + \alpha)} + e^{-i (\omega t + \alpha)}
	\right) \Psi_1.
\end{eqn}
Here $V_j(\xvec)$ are external potentials~\eqnref{bec-noise:system:V}, $\omega_{hf}$ is the hyperfine splitting between components, $g_{jk}$ are nonlinear interaction coefficients~\eqnref{bec-noise:system:g}, $\Gamma_j$ terms represent nonlinear losses for the component $j$, and $\Omega$ is the electromagnetic coupling strength (with $\omega$ being the frequency, and $\alpha$ the initial phase shift of the coupling).
The \abbrev{cgpe}s without loss or coupling terms were introduced by Zeng \textit{et~al}~\cite{Zeng1995}, and Ho and Shenoy~\cite{Ho1996}.
The coupling terms were first included by Ballagh \textit{et~al}~\cite{Ballagh1997}, and the loss terms by Yurovsky \textit{et~al}~\cite{Yurovsky1999}.
A detailed description of the mean-field approximation can be found in the book by Pitaevskii and Stringari~\cite{Pitaevskii2003}.

The exact expressions for the loss parameters $\Gamma_j$ depend on the loss processes in a particular experiment.
For example, the dominant three-body and two-body losses in a two-component \Rb{} \abbrev{bec} with the components ${\ket{F=1,\, m_F=-1}}$ and ${\ket{F=2,\, m_F=+1}}$ result in~\cite{Burt1997,Mertes2007}
\begin{eqn}
\label{eqn:bec-noise:mean-field:losses}
	\Gamma_1 &= \left( \gamma_{111} n_1^2 + \gamma_{12} n_2 \right) / 2, \\
	\Gamma_2 &= \left( \gamma_{12} n_1 + \gamma_{22} n_2 \right) / 2.
\end{eqn}
These equations were obtained empirically, but we will see later in this chapter how the same expressions (to leading order) appear as a result of the application of the truncated Wigner method.

The coupling frequency $\omega$ is usually slightly detuned from the hyperfine frequency in the experiment: $\omega = \omega_{hf} + \delta$, where $\delta \ll \omega_{hf}$.
It is convenient to use equations~\eqnref{bec-noise:mean-field:cgpes} in a rotating frame:
\begin{eqn}
	\Psi_1 & = \Psi_1^{(r)}, \\
	\Psi_2 & = \Psi_2^{(r)} e^{i \omega_{hf} t}.
\end{eqn}
This transformation eliminates $\omega_{hf}$ from the equations and does not change single-time observable values.
Dropping the rotating frame superscript for simplicity, we obtain the transformed equations
\begin{eqn}
	i \hbar \frac{\upd \Psi_1}{\upd t} ={} & \left(
		-\frac{\hbar^2 \nabla^2}{2 m} + V_1
		+ g_{11} \lvert \Psi_1 \rvert^2
		+ g_{12} \lvert \Psi_2 \rvert^2
		- i \hbar \Gamma_1
	\right) \Psi_1 \\
	& + \frac{\hbar \Omega}{2} \left(
		e^{i ((\omega + \omega_{hf}) t + \alpha)} + e^{-i (\delta t + \alpha)}
	\right) \Psi_2, \\
	i \hbar \frac{\upd \Psi_2}{\upd t} ={} & \left(
		-\frac{\hbar^2 \nabla^2}{2 m} + V_2
		+ g_{22} \lvert \Psi_2 \rvert^2
		+ g_{12} \lvert \Psi_1 \rvert^2
		- i \hbar \Gamma_2
	\right) \Psi_2 \\
	& + \frac{\hbar \Omega}{2} \left(
		e^{i (\delta t + \alpha)} + e^{-i ((\omega + \omega_{hf}) t + \alpha)}
	\right) \Psi_1.
\end{eqn}

Furthermore, in experiments the coupling field is typically applied for short periods of time $t_{\mathrm{pulse}}$, where $1 / \omega \ll t_{\mathrm{pulse}} \ll 1 / \delta$.
This allows us to neglect the rapidly oscillating terms proportional to $e^{i(\omega + \omega_{hf})t}$ and come to \abbrev{cgpe}s in the rotating frame:
\begin{eqn}
\label{eqn:bec-noise:mean-field:cgpes-simplified}
	i \hbar \frac{\upd \Psi_1}{\upd t} & = \left(
		-\frac{\hbar^2 \nabla^2}{2 m} + V_1
		+ g_{11} \lvert \Psi_1 \rvert^2
		+ g_{12} \lvert \Psi_2 \rvert^2
		- i \hbar \Gamma_1
	\right) \Psi_1
	+ \frac{\hbar \Omega}{2} e^{-i (\delta t + \alpha)} \Psi_2, \\
	i \hbar \frac{\upd \Psi_2}{\upd t} & = \left(
		-\frac{\hbar^2 \nabla^2}{2 m} + V_2
		+ g_{22} \lvert \Psi_2 \rvert^2
		+ g_{12} \lvert \Psi_1 \rvert^2
		- i \hbar \Gamma_2
	\right) \Psi_2 +
	\frac{\hbar \Omega}{2} e^{i (\delta t + \alpha)} \Psi_1.
\end{eqn}
When the pulse is applied twice using the same coupling field (which is the case for the Ramsey interferometry), it is the same as just setting $\Omega$ to zero after the first pulse and then restoring its value for the time of the second pulse; therefore, $\alpha$ stays the same too.
If one wants to apply pulse with the different detuning, the phase information is lost, and the value of $\alpha$ has to be regarded as random.

The application of the coupling field can be simplified when certain additional conditions are valid, namely:
\begin{itemize}
	\item $\mu / \hbar \ll \Omega$, where $\mu$ is the chemical potential of the first component;
	\item $\delta \ll \Omega$;
	\item the characteristic time of the other terms in~\eqnref{bec-noise:mean-field:cgpes} is much greater than $t_{\mathrm{pulse}}$.
\end{itemize}
This allows us to use an ``instantaneous'' pulse, multiplying the state vector by a rotation matrix:
\begin{eqn}
\label{eqn:bec-noise:mean-field:rotation-matrix}
	\begin{pmatrix}
		\Psi^\prime_1 \\ \Psi^\prime_2
	\end{pmatrix} =
	\begin{pmatrix}
		\cos \frac{\theta}{2} & -i e^{-i \phi} \sin \frac{\theta}{2} \\
		-i e^{i \phi} \sin \frac{\theta}{2} & \cos \frac{\theta}{2}
	\end{pmatrix}
	\begin{pmatrix}
		\Psi_1 \\ \Psi_2
	\end{pmatrix},
\end{eqn}
where $\theta = \Omega t_{\mathrm{pulse}}$, and $\phi = \delta t + \alpha$ is the total phase of the coupling field at the beginning of the pulse.
In particular, for the two-pulse Ramsey scheme (\figref{bec-noise:visibility:sequences},~(a)) with the time $t_R$ between pulses, $\phi_2 = \phi_1 + \delta t_R$.


% =============================================================================
\subsection{Ground state calculation}
% =============================================================================

At the beginning of the simulation, the \abbrev{bec} is assumed to be in the ground state which has the lowest possible energy.
The ground state is the solution of the stationary \abbrev{cgpe}s
\begin{eqn}
\label{eqn:bec-noise:mean-field:cgpes-stationary}
	\mu_1 \Psi_1 & = \left(
		-\frac{\hbar^2 \nabla^2}{2 m} + V_1
		+ g_{11} \lvert \Psi_1 \rvert^2
		+ g_{12} \lvert \Psi_2 \rvert^2
	\right) \Psi_1, \\
	\mu_2 \Psi_2 & = \left(
		-\frac{\hbar^2 \nabla^2}{2 m} + V_2
		+ g_{22} \lvert \Psi_2 \rvert^2
		+ g_{12} \lvert \Psi_1 \rvert^2
	\right) \Psi_2,
\end{eqn}
where $\mu_1$ and $\mu_2$ are chemical potentials of the components.

The most common method of finding the mean-field ground state is the imaginary time propagation~\cite{Chiofalo2000,Bao2004}.
The essence of the method is that the propagation of an arbitrary wavefunction using the time-dependent \abbrev{cgpe}s, but with the substitution $t \rightarrow \tau = it$, lowers its energy; therefore, after the sufficient amount of time this propagation will lead us arbitrarily close to the ground state.
The actual equations to be propagated are~\eqnref{bec-noise:mean-field:cgpes-simplified} without the loss or coupling terms:
\begin{eqn}
\label{eqn:bec-noise:mean-field:imaginary-time}
	\hbar \frac{\upd \Psi_1}{\upd \tau} & = -\left(
		-\frac{\hbar^2 \nabla^2}{2 m} + V_1
		+ g_{11} \lvert \Psi_1 \rvert^2
		+ g_{12} \lvert \Psi_2 \rvert^2
	\right) \Psi_1, \\
	\hbar \frac{\upd \Psi_2}{\upd \tau} & = -\left(
		-\frac{\hbar^2 \nabla^2}{2 m} + V_2
		+ g_{22} \lvert \Psi_2 \rvert^2
		+ g_{12} \lvert \Psi_1 \rvert^2
	\right) \Psi_2.
\end{eqn}

The rigorous proof of this method was derived by Bao and Du~\cite{Bao2004}.
The idea can be roughly illustrated by considering a one-component system with a linear Hamiltonian $\hat{H}$, whose eigenvalues are $\mu_1 < \mu_2 < ...$, where the lowest eigenvalue corresponds to ground state we want to find.
The steady solution of the time-dependent \abbrev{gpe}
\begin{eqn}
	i \hbar \frac{\upd \Psi}{\upd t} = \hat{H} \Psi
\end{eqn}
then looks like
\begin{eqn}
	\Psi(\xvec, t) = \sum_k e^{-\frac{i}{\hbar}\mu_k t} f_k(\xvec),
\end{eqn}
where $f_k$ are eigenfunctions of $\hat{H}$ corresponding to the eigenvalues $\mu_k$.
After the substitution $t \rightarrow \tau = it$ the solution becomes fading, with higher-energy terms fading faster:
\begin{eqn}
	\Psi(\xvec, \tau) = \sum_k e^{-\frac{1}{\hbar}\mu_k \tau} f_k(\xvec).
\end{eqn}

Therefore, if we take some random initial solution and propagate it long enough in imaginary time using~\eqnref{bec-noise:mean-field:imaginary-time}, the higher-energy terms will eventually die out (in comparison with the lowest-energy state) and leave us with the desired ground state.
The state obtained from the Thomas-Fermi approximated \abbrev{gpe} can be taken as the initial one since it is rather close to the desired one (and, therefore, higher-energy terms are already quite small).

Since the population will decrease exponentially after each step, and the precision of numerical calculations is limited, a renormalisation after each step will be required.
The total number of atoms in the ground state serves best in this case (because we will have to renormalise the final ground state anyway):
\begin{eqn}
	\int\limits_V \lvert \Psi(\tau, \xvec) \rvert^2 \upd V = N.
\end{eqn}

Propagation is terminated when the Gross-Pitaevskii energy of the state converges to the required precision (that is, only one eigenstate with the lowest energy is left out).
The energy of the two-component condensate~\cite{Pitaevskii2003}
\begin{eqn}
\label{eqn:bec-noise:mean-field:two-comp-energy}
	E[\Psivec] ={} & \int\limits_A \left(
		- \frac{\hbar^2 \Psi_1^* \nabla^2 \Psi_1}{2m}
		- \frac{\hbar^2 \Psi_2^* \nabla^2 \Psi_2}{2m}
	\right. \\
	& \left.
		+ (V_1 + \hbar \omega_1) n_1 + (V_2 + \hbar \omega_2) n_2
		+ \frac{g_{11}}{2} n_1^2 + \frac{g_{22}}{2} n_2^2 + g_{12} n_1 n_2
	\right) \upd\xvec
\end{eqn}
thus has to be calculated after each step and compared to the previous value, waiting for the desired precision to be reached.

\begin{figure}
\centerline{%
\includegraphics{figures_generated/mean_field/two_comp_gs_miscible.pdf}%
\includegraphics{figures_generated/mean_field/two_comp_gs_immiscible.pdf}}

\caption[Two-component ground state for miscible and immiscible regimes]{
Axial densities of two-component ground state for \textbf{(a)}~a miscible and \textbf{(b)}~an immiscible regime of \Rb{} \abbrev{bec} with $N_1 = N_2 = 40,000$ atoms in a three-dimensional harmonic trap with the frequencies $f_x = f_y = 97.6\un{Hz},\,f_z = 11.96\un{Hz}$.
Blue solid lines and red dashed lines show the axial density of the first and the second component respectively.
Intra-component scattering lengths are $a_{11} = 100.40\,r_B$, $a_{22} = 95.68\,r_B$, inter-component scattering length is \textbf{(a)}~$a_{12} = 97.0\,r_B$ and \textbf{(b)}~$a_{11} = 99.0\,r_B$, where $r_B$ is the Bohr radius.}%endcaption
\label{fig:bec-noise:mean-field:two-comp-gs}
\end{figure}

As an example, \figref{bec-noise:mean-field:two-comp-gs} shows the axial density $n_z = \int n(\xvec) \upd x \upd y$ of the two-component ground state for an equal mix of two hyperfine states of \Rb{} \abbrev{bec}.
Two panes of the figure illustrate the difference between the miscible ($a_{12}^2 < a_{11} a_{22}$) and immiscible ($a_{12}^2 > a_{11} a_{22}$) regimes.

It should be emphasized that this ground state is not a true many-body ground state, but rather a type of mean-field (single-particle) approximation, since it omits quantum correlations.


% =============================================================================
\subsection{Thomas-Fermi approximation}
% =============================================================================

As was mentioned earlier in this section, the starting state for the imaginary time calculation can be set to the Thomas-Fermi approximate state to minimize the propagation time.
Thomas-Fermi approximation consists of neglecting the kinetic term in the stationary equations~\eqnref{bec-noise:mean-field:cgpes-stationary}.
For a one-component state ($\Psi_2 \equiv 0$), the resulting equations can be easily solved analytically:
\begin{eqn}
\label{eqn:bec-noise:mean-field:tf-gs}
	| \Psi_1(\xvec) |^2 = \frac{1}{g_{11}} \max \left( \mu_1 - V_1(\xvec), 0 \right).
\end{eqn}
In the $D$-dimensional harmonic trap potential
\begin{eqn}
\label{eqn:bec-noise:mean-field:trap-potential}
	V_1(\xvec) = \frac{m}{2} \sum_{d=1}^D \omega_d^2 x_d^2,
\end{eqn}
this solution has the shape of an ellipsoid with radii $r_d = \sqrt{2\mu_1 / (m \omega_d^2)}$.

The chemical potential $\mu_1$ is fixed by the normalisation condition $\int |\Psi_1|^2 \upd \xvec = N_1$.
For a three-dimensional trap it can be shown to be
\begin{eqn}
	\mu_1^{\mathrm{(3D)}} =
		\left( \frac{15 N_1}{8 \pi} \right)^{\frac{2}{5}}
		\left( \frac{m \bar{\omega}^2}{2} \right)^{\frac{3}{5}}
		{g_{11}}^{\frac{2}{5}},
\end{eqn}
where $\bar{\omega} = \sqrt[3]{\omega_x \omega_y \omega_z}$.
For a one-dimensional trap it has the form
\begin{eqn}
	\mu_1^{\mathrm{(1D)}} =
		\left( \frac{3 g_{11} N_1}{4} \right)^{\frac{2}{3}}
		\left( \frac{m \omega_1^2}{2} \right)^{\frac{1}{3}}.
\end{eqn}

Now we can roughly estimate the conditions necessary to neglect the kinetic term from the equation.
Substituting approximate solution~\eqnref{bec-noise:mean-field:tf-gs} to~\eqnref{bec-noise:mean-field:cgpes-stationary} and comparing the kinetic term with the potential term, we get the following inequation:
\begin{eqn}
\label{eqn:bec-noise:mean-field:tf-inequation}
	\frac{\hbar^2}{2m} \left(
		\frac{m \sum_{d=1}^D \omega_d^2}{2}
		+ \frac{m^2 \sum_{d=1}^D \omega_d^4 x_d^2}
			{4 \left( \mu_1 - V_1(\xvec) \right)}
	\right) \ll
	\mu \left(\mu_1 - V_1(\xvec)\right).
\end{eqn}
Near the centre of the condensate, this inequation simplifies to
\begin{eqn}
\label{eqn:bec-noise:mean-field:tf-condition}
	\mu \gg \frac{\hbar}{2} |\bomega|,
\end{eqn}
where $\bomega \equiv (\omega_1, \ldots, \omega_D)^T$ is the vector of trap frequencies.

On the other hand, near the edges of the cloud the left-hand side of the inequation~\eqnref{bec-noise:mean-field:tf-inequation} diverges, while the right-hand side tends to zero.
This means that near the edges the Thomas-Fermi approximation fails regardless of the conditions.
Fortunately, the particle density there is low, so we can estimate the width $h$ of the ``belt'' where our first approximation of the state function is significantly incorrect.
If it happens to be small as compared to the size of the condensate, the approximation can be considered valid.

The first term at the left-hand side of the inequation~\eqnref{bec-noise:mean-field:tf-inequation} is constant and can be dropped in the limit of $V_1(\xvec) \rightarrow \mu_1$.
Then, for the sake of simplicity, we assume all but one of coordinates to be zero and the remaining one to be equal to $r_d - h_d$, where $r_d$ is the corresponding radius of the condensate.
After replacing ``$\ll$'' by ``$\approx$'' and assuming $h_d$ to be small as compared to $r_d$, we obtain the conditions for each coordinate:
\begin{eqn}
	h_d \approx \sqrt{\frac{\hbar^2}{2 \mu_1 m}},\,d \in [1, \ldots, D].
\end{eqn}
These have to be much smaller than the corresponding radii, which gives us
\begin{eqn}
	\mu_1 \gg \frac{1}{2} \hbar \max_{d \in [1, \ldots, D]} \omega_d.
\end{eqn}
This condition is less strict than the condition for the centre of the condensate.
Therefore, we have only one condition justifying the application of the Thomas-Fermi approximation is~\eqnref{bec-noise:mean-field:tf-condition}.

\begin{figure}
\centerline{%
	\includegraphics{figures_generated/mean_field/one_comp_gs_large.pdf}%
	\includegraphics{figures_generated/mean_field/one_comp_gs_small.pdf}}
\caption[Numerically calculated and Thomas-Fermi approximated ground states]{
Axial densities of numerically calculated (blue solid lines) and Thomas-Fermi approximated (red dashed lines) ground states for a one-component \Rb{} \abbrev{bec} of \textbf{(a)}~$100,000$ atoms, and \textbf{(b)}~$1,000$ atoms.}%endcaption
\label{fig:bec-noise:mean-field:tf-vs-accurate}
\end{figure}

Let us use some real-life experimental parameters and check how well the Thomas-Fermi approximation works.
For a three-dimensional trap with the frequencies $f_x = f_y = 97.6\un{Hz}$ and $f_z = 11.96\un{Hz}$ and $N_1=10^5$ \Rb{} atoms (which have a scattering length of $a_{11} = 100.4 r_B$), we have $2 \mu_1 / (\hbar |\bomega|) \approx 10.68$.
This means that the Thomas-Fermi approximation produces a solution which is close to the real one.
However, for a lower amount of atoms, say $N_1=10^3$, we obtain $2 \mu / (\hbar |\bomega|) \approx 1.69$, which is a sign that we are reaching the limit of the approximation's applicability.
\figref{bec-noise:mean-field:tf-vs-accurate} shows the axial density for both cases: for $100,000$ atoms the Thomas-Fermi approximation is very close to the numerically calculated ground state and for $1,000$ atoms it differs significantly as expected.

An extended discussion of the Thomas-Fermi approximation was given by Dalfovo \textit{et~al}~\cite{Dalfovo1999}.
It is possible to work out the Thomas-Fermi approximation for a two-component condensate, but it requires some non-trivial handling of various miscibility/immiscibility cases~\cite{Anderson2010}.
In this thesis we only single-component ground states, as these are the only ones used in the experiments we are considering.

% =============================================================================
\section{Wigner transformation}
% =============================================================================


\copypaste{
We start by assuming that the BEC has $s$-wave interactions, together with Markovian losses due to $n$-body collisions.
We employ a master equation together with the Wigner-Moyal quantum phase-space representation~\cite{Gardiner2004} and a truncation of third- and higher-order derivatives in the equations of motion.
If we regard the commonly used Gross-Pitaevskii equation as a classical, first approximation to mean-field condensate dynamics, the truncated Wigner approach is best thought of as the second term in an expansion in inverse particle number.
}

\todo{
This section should be mostly dedicated to results in \cite{Egorov2011} and \cite{Opanchuk2012} (and possibly \cite{Egorov2013}).
}


In the present Letter, we treat an ultra-cold,
interacting multi-component spinor Bose gas in $D$ effective dimensions.
The basic Hamiltonian is easily expressed using quantum fields
$\Psiop_j^{\dagger}(\xvec)$ and $\Psiop_j(\xvec)$,
where $\Psiop_j^{\dagger}(\xvec)$ creates a bosonic atom of spin $j$
at location $\xvec$, and $\Psiop_j(\xvec)$ destroys one;
the commutators are
$[\Psiop_j(\xvec),\Psiop_k^{\dagger}(\xvec^\prime)] =
\delta^{(D)}(\xvec-\xvec^\prime)\delta_{jk}.$
The resulting physics of a dilute, low-temperature Bose gas
is well-described in the $s$-wave scattering limit by an effective Hamiltonian
with contact interactions and external potentials:
\begin{equation}
    \hat{H} / \hbar = \int \upd^{D}\xvec \left\{
        \Psiop_j^{\dagger} K_{jk} \Psiop_k +
        \frac{U_{jk}}{2} \Psiop_j^{\dagger} \Psiop_k^{\dagger}
        \Psiop_k \Psiop_j
    \right\}.
\end{equation}
Here we omit the field argument $(\xvec)$ for brevity,
and use the Einstein summation convention of summing over repeated indices.
$K_{jk}$ is the single-particle Hamiltonian:
\begin{equation}
    K_{jk} = \left( -\frac{\hbar}{2m} \nabla^2 + \omega_j + V_j(\xvec) / \hbar \right) \delta_{jk} +
        \tilde{\Omega}_{jk}(t),
\end{equation}
where $m$ is the atomic mass, $V_j$ is the external trapping potential for spin $j$,
$\omega_j$ is the internal energy of spin $j$,
$\tilde{\Omega}_{jk}$ represents a time-dependent coupling
that is used to rotate one spin projection into another,
and $U_{jk}$ is the atom-atom interaction term.
Thus, $n_j = \langle \Psiop_j^{\dagger} \Psiop_j \rangle$
is the spin-$j$ atomic density.
For a dilute gas at low enough temperatures,
$U_{jk}=4\pi\hbar a_{jk} / m$, where $a_{jk}$ is the $s$-wave scattering length in three dimensions.
Here we assume a momentum cutoff $k_{c} \ll 1 / a_{jk}$,
otherwise the couplings must be renormalized~\cite{Sinatra2002}.

We proceed by using a stochastic phase-space method that allows a numerical
simulation of the quantum dynamics~\cite{Drummond1993,Steel1998,Hoffmann2008}.
Defining a Wigner function $W(\Psivec)$, where $\Psi_j$
is a c-number field corresponding to the quantum field $\hat{\Psi}_j$, this has a unitary time-evolution equation:
\begin{equation}
    \frac{\partial W}{\partial t} = \int \upd^D\xvec \left\{
        - \frac{\delta}{\delta\Psi_j} A_j
        - \frac{\delta}{\delta\Psi_j^*}A_j^*
        + \mbox{O} \left[ \frac{\delta^3}{\delta\Psi_j^3} \right]
    \right\} W.
\end{equation}
Next, higher-derivative terms of type $\mbox{O} \left[ \delta^3 / \delta\Psi_j^3 \right]$ are truncated.
This approximation neglects higher-order terms in an expansion in $1 / \sqrt{N}$,
and is therefore valid in the limit of $N \gg M$
where $N$ is the atom number and $M$ is the number of low-energy modes included~\cite{Drummond1993,Sinatra2002,Norrie2006}.
In free-space calculations it is important to maintain this mode truncation.
In the relevant limits where the technique is applicable, the equations
simply reduce to Gross-Pitaevskii equations with Gaussian fluctuations
of the initial conditions:
\begin{equation}
\label{eqn:SDE-1}
    \frac{\upd\Psi_j}{\upd t} = -i \left(
        K_{jk} \Psi_k + U_{jk} \lvert \Psi_k \rvert^2 \Psi_j
    \right).
\end{equation}
For initial conditions in interferometry it is usually sufficient to consider
a coherent state amplitude $\Psi_s^c$,
corresponding to a typical initial state with Poissonian number fluctuations,
as produced by a beam-splitter.
In this case the initial Wigner amplitude has a Gaussian random distribution, with
$\Psi_j(\xvec, t_0) = \Psi_j^c(\xvec) + \Delta \Psi_j(\xvec)$, where:
$\left\langle \Delta \Psi_j(\xvec) \Delta \Psi_k^*(\xvec^{\prime}) \right\rangle =
\delta_{jk} \delta^D(\xvec - \xvec^{\prime}) / 2.$
This initial noise is necessary because the Wigner representation generates
symmetrically ordered correlation functions, and includes vacuum fluctuations.
For greater accuracy, the initial state can be modified to account for
initial  correlations, thermal noise, or additional fluctuations.
If normal ordered correlations are measured, one has to express them
as a sum of symmetrically ordered terms.

This includes all the known nonlinear quantum noise effects of quantum dynamics,
like phase diffusion, entanglement and quantum squeezing, in the limit
of large particle number.
The initial noise terms do not occur in the semi-classical Gross-Pitaevskii
approximation, which is therefore unable to predict these effects.
Thus, while the lossless equations are identical to the Gross-Pitaevskii
equations, the inclusion of initial noise terms together with nonlinear
interactions leads to quantum phase-diffusion.
Such methods can be used for either free-space or trapped atom interferometry,
provided there is an appropriate mode truncation.

Additional quantum noise enters from the effects of damping and losses,
due to the fluctuation-dissipation theorem.
These effects are important at high densities in atomic traps.
They can be included via an additional Markovian master equation~\cite{Jack2002}
defined so that,
\begin{equation}
    \frac{\upd\hat{\rho}}{\upd t} =
        - \frac{i}{\hbar} \left[ \hat{H}, \hat{\rho} \right]
        + \sum_{n,\lvec} \kappa_{\lvec}^{(n)} \int \upd^{D}\xvec
            \mathcal{L}_{\lvec}^{(n)} \left[ \hat{\rho} \right],
\end{equation}
where $n$ is the number of interacting particles,
$\lvec = (l_1, l_2, \ldots, l_n)$ is a vector indicating the spins that are coupled,
and we have introduced local Liouville loss terms,
\begin{equation}
    \mathcal{L}_{\lvec}^{(n)} \left[ \hat{\rho} \right] =
        2\hat{O}_{\lvec}^{(n)} \hat{\rho} \hat{O}_{\lvec}^{(n)\dagger}
        - \hat{O}_{\lvec}^{(n)\dagger} \hat{O}_{\lvec}^{(n)} \hat{\rho}
        - \hat{\rho} \hat{O}_{\lvec}^{(n)\dagger} \hat{O}_{\lvec}^{(n)}.
\end{equation}
The reservoir coupling operators $\hat{O}_{\lvec}^{(n)}$ are the distinct $n$-fold products of local field annihilation operators,
$\hat{O}_{\lvec}^{(n)} = \hat{O}_{\lvec}^{(n)} (\widehat{\Psivec}) =
    \Psiop_{l_{1}} (\xvec)
    \Psiop_{l_{2}} (\xvec) \ldots
    \Psiop_{l_{n}} (\xvec),$
describing local $n$-body collision losses.

After transforming these new terms to evolution equations for the Wigner distribution, the drift term $A_j$
changes the Gross-Pitaevskii evolution to include nonlinear damping, while
the next terms in the evolution equation give rise to additional Fokker-Planck
diffusion terms associated with quantum noise from the loss reservoirs,
given by:
\begin{equation}
    \frac{\delta^{2}}{\delta\Psi_j\delta\Psi_k^{*}} \left\{
        \sum_{n,\lvec} \kappa_{\lvec}^{(n)}
            \frac{\partial O_{\lvec}^{(n)*}}{\partial\Psi_j^{*}}
            \frac{\partial O_{\lvec}^{(n)}}{\partial\Psi_k}
        \right\} W.
\end{equation}

This leads to a stochastic equation:
\begin{equation}
\label{eqn:SDE}
    \frac{\upd\Psi_j}{\upd t} =
        - i\left( K_{jk} \Psi_k + U_{jk} \lvert \Psi_k \rvert^{2} \Psi_j \right)
        - \Gamma_j
        + \sum_{n,\lvec} \beta_{\lvec,j}^{(n)} \zeta_{\lvec}^{(n)}(\xvec,t),
\end{equation}
where the nonlinear loss has the form:
\begin{equation}
    \Gamma_j = \sum_{n,\lvec}
        \kappa_{\lvec}^{(n)}
        \frac{\partial O_{\lvec}^{(n)*} (\Psivec)}{\partial\Psi_j^{*} (\xvec)}
        O_{\lvec}^{(n)}(\xvec),
\end{equation}
and $\zeta_{\lvec}^{(n)}(\xvec, t)$ is a corresponding complex,
stochastic delta-correlated Gaussian noise with
\begin{equation}
    \left\langle
        \zeta_{\lvec}^{(n)} (\xvec,t) \zeta_{\kvec}^{(m)*}(\xvec^\prime, t^\prime)
    \right\rangle =
    \delta_{\lvec \kvec} \delta^{nm} \delta^{D} \left(
        \xvec - \xvec^\prime
    \right)
    \delta \left( t - t^\prime \right).
\end{equation}
The multiplicative noise coefficient
\begin{equation}
    \beta_{\lvec,j}^{(n)} \left( \Psivec \right) =
    \sqrt{\kappa_{\lvec}^{(n)}}
    \frac{\partial O_{\lvec}^{(n)}}{\partial\Psi_j}
\end{equation}
is a fluctuation-dissipation term,
so that the Wigner variables remain equivalent to the corresponding operators.

The loss coefficients in eq.~(\ref{eqn:SDE}) can be converted to the conventional form,
which is defined using atom number losses:
\begin{equation}
    \dot{n}_j = - \gamma^{(n)}_{\lvec,j} n^{m_1}_1 n^{m_2}_2 \ldots ,
\end{equation}
where $n_j$ is the density of component $j$ and $m_j$
is the number of spin-$j$ atoms lost in the collision.
The conversion can be carried out as $\gamma^{(n)}_{\lvec,j} = 2 m_j \kappa^{(n)}_{\lvec}$.

In this work we use a basis of plane waves in the volume $V$,
and the density of component $j$ is calculated as a probabilistic average:
\begin{equation}
\label{eqn:wigner-density}
    n_j (\xvec)
        = \langle \Psi^*_j (\xvec) \Psi_j (\xvec) \rangle_{\mathrm{paths}} - \frac{M}{2V}.
\end{equation}
Here we use the fact that the approximate Wigner function is a probability distribution
equivalent to an averaged sum over different simulation paths.

\begin{figure}
    %\begin{tabular}{l l}
    %\imagetop{\hspace*{0.44in}\includegraphics[width=0.72\columnwidth]{ramsey_sequence.eps}} & \imagetop{(a)} \\
    %\imagetop{\includegraphics[width=0.85\columnwidth]{long_ramsey_visibility.eps}} & \imagetop{(b)} \\
    %\imagetop{\hspace*{0.44in}\includegraphics[width=0.72\columnwidth]{echo_sequence.eps}} & \imagetop{(c)} \\
    %\imagetop{\includegraphics[width=0.85\columnwidth]{long_rephasing_visibility.eps}} & \imagetop{(d)}
    %\end{tabular}

    \caption{
    Timeline of the experiment for Ramsey (a) and Ramsey with spin echo (c); (b) and (d) are the simulated plots of interferometric visibility.
    Classical GPE (red dashed lines) and Wigner calculations (blue solid lines) are shown.
    $N = 5.5 \times 10^4$,
    $\omega_x = \omega_y = 2 \pi \times 97.0\un{Hz}$,
    $\omega_z = 2 \pi \times 11.69\un{Hz}$,
    $a_{11} = 100.4\,a_0$, $a_{12} = 97.993\,a_0$, $a_{22} = 95.57\,a_0$~\cite{Egorov2011},
    $a_0$ is the Bohr radius.
    Nonlinear atomic losses:
    $\gamma^{(3)}_{111} = 5.4 \times 10^{-30}\un{cm^6/s}$~\cite{Mertes2007},
    $\gamma^{(2)}_{12} = 1.51 \times 10^{-14}\un{cm^3/s}$,
    $\gamma^{(2)}_{22} = 8.1 \times 10^{-14}\un{cm^3/s}$~\cite{Egorov2011}.}

    \label{fig:visibility}
\end{figure}

To illustrate the applications of this method we consider recent interferometry
experiments with a two-component BEC involving two hyperfine states
${\ket{F=1,\, m_F=-1}}$ and ${\ket{F=2,\, m_F=+1}}$ in \Rb~\cite{Egorov2011}.
A conventional Ramsey sequence (\figref{visibility},~(a)) has been used
with a BEC confined in a cigar-shaped magnetic trap with the frequencies $(97.0, 97.0, 11.69)\un{Hz}$
in a bias magnetic field of $3.23\un{G}$, so that magnetic field dephasing is largely eliminated~\cite{Hall1998}.
The first $\pi/2$ pulse prepares a non-equilibrium superposition of states ${\ket{1,-1}}$ and ${\ket{2,+1}}$
and the spatial modes of two components periodically separate and merge again~\cite{Mertes2007}.
The spatially-separated spin components evolve differently, as they have
different scattering lengths.
As a result, these collective oscillations lead to periodic dephasing and
self-rephasing of the BEC components, clearly visible in both GPE and Wigner
simulations of interference fringe visibility
$\mathcal{V}$ (\figref{visibility},~(b)).
Asymmetric losses of two states are one cause of the contrast decay.
This can be partially compensated by the application of a spin echo pulse
mid-way through the evolution (\figref{visibility},~(c)).
The GPE simulations wrongly predict (dashed lines) that visibility is largely
recovered at long evolution times using the spin echo method.
However, the addition of quantum noise (solid line) via the Wigner simulations
noticeably speeds up the visibility decay even with a spin echo pulse present.
This is in agreement with experimental observations, and shows that these
effects play a significant part in the decay of visibility, even for
large particle numbers.

The important feature of these quantum dynamical simulations
is that they are able to treat large numbers of atoms (55,000 in this case),
while correctly tracking all the quantum noise sources, and also extending the simulations to long time-scales.
Both of these features, large atom numbers and long time-scales,
are essential ingredients to accurate interferometric measurements.
The simulations give accurate predictions despite large, multi-mode dynamical motion in three dimensions
and substantial losses of most of the condensate atoms~\cite{Egorov2011}.
On longer time-scales, the experimental accuracy is limited by technical noises, and we have no data for comparisons.


