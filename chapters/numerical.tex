% =============================================================================
\chapter{Numerical methods}
\label{cha:appendix:numerical}
% =============================================================================

This appendix outlines our approach to numerical simulations performed for this thesis.
While full texts of the programs used are too long to include verbatim in the text, the main choices made will be described here.


% =============================================================================
\section{Calculations on GPU}
% =============================================================================

The phase space methods described in this thesis are inherently parallel.
Such algorithms are very suitable to run on modern \abbrev{gpu}s, as long as the data being processed fits the video memory (which was our case).
We chose nVidia's CUDA as the \abbrev{gpgpu} platform because of its maturity and the included set of libraries with effective implementations of Fast Fourier Transform (\abbrev{fft}), random number generators, and other useful algorithms.
In practice, OpenCL could be used as well: at the moment it is just as fast as CUDA on nVidia video cards, and has an additional benefit of supporting AMD cards.

Furthermore, in order to speed up prototyping we did not use the CUDA language itself (which is, essentially, a superset of C++), but Python bindings to it.
Python is a general-purpose dynamically typed language with a rich set of third-party libraries.
It is quite popular in the academic community owing to the excellent NumPy and SciPy packages~\cite{Oliphant2007}, which provide an extensive toolset for numerical calculations.
PyCUDA~\cite{Klockner2012} augments it with a convenient access to CUDA and its libraries, significantly reducing the amount of boilerplate code and making metaprogramming possible.
Of course, the dynamical facilities of Python make it slower than C++, but that negligible in our simulations, since the majority of calculations was done on \abbrev{gpu}, and the Python language overhead was masked by their asynchronous execution.

The use of \abbrev{gpgpu} allowed us to reach a hundredfold speedup of calculations on a single nVidia Tesla C2050 as compared to MatLab implementations.
This was improved even further by combining results calculated on several videocards, such as we did for the calculations in \secref{bell-ineq:ghz}, where we used seven nVidia Tesla M2090.


% =============================================================================
\section{XMDS}
% =============================================================================

For the cases where the raw speed of calculations was less important, and the custom \abbrev{gpgpu} program was not required, we used the XMDS package~\cite{Collecutt2001,Dennis2013}.
XMDS is a code generator written in Python, that, based on a brief XML description of the simulation problem, creates and compiles a C++ program that performs the integration.
The strength of XMDS lies in highly optimized integration algorithms it uses (which is improved even more by on-demand compilation), and in its transparent usage of multi-processor interface (\abbrev{mpi}), which allows it to use multiple cores of a \abbrev{cpu}, or multiple nodes in a cluster.
We used XMDS to integrate the stochastic differential equations in \secref{wigner-bec:mm}.


% =============================================================================
\section{Stochastic integration}
% =============================================================================

The problem of stochastic integration is well-studied; for some general information, one can refer to an extensive review by Drummond and Mortimer~\cite{Drummond1990} and by Werner and Drummond~\cite{Werner1997}.
The most popular approach is to use a method from the wide Runge-Kutta family, which can be applied without changes to a set of \abbrev{sde}s in Stratonovich form~\cite{Wilkie2004,Wilkie2005}.

It possible to employ variable-stepsize methods, which are strongly convergent providing that stochastic differentials on the subintervals are sampled correctly~\cite{Wilkie2005}.
Another widely-used specialized Runge-Kutta subtype is the \abbrev{rk4} interaction picture method~\cite{CaradocDavies2000}, which can greatly improve convergence properties when applied to stiff equations.
One must note though, that the reduction in the number of steps required for a given level of convergence is often neutralized by the increased number of \abbrev{fft}s that have to be performed on each step.

In our case neither adaptive stepsize, nor the transition to interaction picture did not appear necessary, and we settled on a fixed-stepsize fourth order \abbrev{rk} method optimized for low dissipation, low dispersion and low storage~\cite{Berland2006}.
In particular, the latter means that on each substep of \abbrev{rk}, the result depends only on the previous step.
This results in the algorithm requiring 6 steps in order to reach 4th order, but the low-storage property is very important in \abbrev{gpgpu} environment, where one has to keep whole arrays of intermediate results in video memory.
