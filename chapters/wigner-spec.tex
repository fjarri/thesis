% =============================================================================
\chapter{Special cases of Wigner transformation}
% =============================================================================

This section contains some theorems concerning transformations of specific operator sequences,
which will be useful when transforming the master equation.
\todo{Throughout this section all wave functions and field operators belong to restricted bases etc.
Need to write it formally.}

\begin{theorem}
\label{thm:wigner-spec:w-commutator1}
    \begin{eqn*}
        \mathcal{W} \left[ [\int d\xvec \Psiop_j^\dagger \Psiop_k, \hat{A}] \right]
        = \int d\xvec \left(
            - \frac{\delta}{\delta \Psi_j} \Psi_k
            + \frac{\delta}{\delta \Psi_k^*} \Psi_j^*
        \right) \mathcal{W}[\hat{A}].
    \end{eqn*}
\end{theorem}
\begin{proof}
\begin{eqn}
    \mathcal{W} \left[ [\int d\xvec \Psiop_j^\dagger \Psiop_k, \hat{A}] \right]
    ={} & \int d\xvec \left(
        \left(
            \Psi_j^* - \frac{1}{2} \frac{\delta}{\delta \Psi_j}
        \right)
        \left(
            \Psi_k + \frac{1}{2} \frac{\delta}{\delta \Psi_k^*}
        \right) \right. \\
    &   \left. - \left(
            \Psi_k - \frac{1}{2} \frac{\delta}{\delta \Psi_k^*}
        \right)
        \left(
            \Psi_j^* + \frac{1}{2} \frac{\delta}{\delta \Psi_j}
        \right)
    \right)
    \mathcal{W}[\hat{A}] \\
    ={} & \frac{1}{2} \int d\xvec \left(
        - \frac{\delta}{\delta \Psi_j} \Psi_k
        + \Psi_j^* \frac{\delta}{\delta \Psi_k^*}
        + \frac{\delta}{\delta \Psi_k^*} \Psi_j^*
        - \Psi_k \frac{\delta}{\delta \Psi_j}
    \right)
    \mathcal{W}[\hat{A}] \\
    ={} & (*).
\end{eqn}
Changing the order of derivatives and functions using the relation
\begin{eqn}
    \Psi_k \frac{\delta}{\delta \Psi_j} \mathcal{F}
    = \left(
        \frac{\delta}{\delta \Psi_j} \Psi_k
        - \delta_{jk} \delta_{\restbasis_j}(\xvec, \xvec)
    \right) \mathcal{F},
\end{eqn}
we get
\begin{eqn}
    (*) = \int d\xvec \left(
        - \frac{\delta}{\delta \Psi_j} \Psi_k
        + \frac{\delta}{\delta \Psi_k^*} \Psi_j^*
    \right)
    \mathcal{W}[\hat{A}].
    \qedhere
\end{eqn}
\end{proof}

Commutators with Laplacian inside require somewhat special treatment, because it acts on basis functions and, in general, cannot be dragged around like a constant.
For our purposes we only need one specific case, and, fortunately, in this case it does act like a constant.

\begin{theorem}
\label{thm:wigner-spec:w-laplacian-commutator1}
    \begin{eqn*}
        \mathcal{W} \left[
            \int d\xvec [\Psiop^\dagger(\xvec) \nabla^2 \Psiop(\xvec), \hat{A}]
        \right]
        = \int d\xvec \left(
            - \frac{\delta}{\delta \Psi} \nabla^2 \Psi
            + \frac{\delta}{\delta \Psi^*} \nabla^2 \Psi^*
        \right) \mathcal{W}[\hat{A}].
    \end{eqn*}
\end{theorem}
\begin{proof}
First, it is obvious from the definition of the Wigner transformation that
\begin{eqn}
    \mathcal{W} \left[ \int d\xvec \hat{B}(\xvec) \hat{A} \right]
    = \int d\xvec \mathcal{W} [\hat{B}(\xvec) \hat{A}]
\end{eqn}
and
\begin{eqn}
    \mathcal{W} [ \nabla^2 \hat{B}(\xvec) \hat{A} ]
    = \nabla^2 \mathcal{W} [\hat{B}(\xvec) \hat{A}].
\end{eqn}
Let us now expand the commutator and apply correspondences from \thmref{wigner:func:correspondences}:
\begin{eqn2}
    & \mathcal{W} && \left[
        \int d\xvec [\Psiop^\dagger(\xvec) \nabla^2 \Psiop(\xvec), \hat{A}]
    \right] \\
    & ={} && \mathcal{W} \left[
            \int d\xvec \Psiop^\dagger(\xvec) \nabla^2 \Psiop(\xvec) \hat{A}
        \right]
        - \mathcal{W} \left[
                \int d\xvec \hat{A} \Psiop^\dagger(\xvec) \nabla^2 \Psiop(\xvec)
        \right] \\
    & ={} && \int d\xvec \left(
            \Psi^* - \frac{1}{2} \frac{\delta}{\delta \Psi}
        \right)
        \left(
            \nabla^2 \Psi + \frac{1}{2} \nabla^2 \frac{\delta}{\delta \Psi^*}
        \right)
        \mathcal{W}[\hat{A}] \\
    & && - \int d\xvec \left(
            \nabla^2 \Psi - \frac{1}{2} \nabla^2 \frac{\delta}{\delta \Psi^*}
        \right)
        \left(
            \Psi^* + \frac{1}{2} \frac{\delta}{\delta \Psi}
        \right)
        \mathcal{W}[\hat{A}] \\
    & ={} && \int d\xvec \left(
            \Psi^* \nabla^2 \Psi
            - \frac{1}{2} \frac{\delta}{\delta \Psi} \nabla^2 \Psi
            + \frac{1}{2} \Psi^* \nabla^2 \frac{\delta}{\delta \Psi^*}
            - \frac{1}{4} \frac{\delta}{\delta \Psi} \nabla^2 \frac{\delta}{\delta \Psi^*}
        \right)
        \mathcal{W}[\hat{A}] \\
    & && - \int d\xvec \left(
            \Psi^* \nabla^2 \Psi
            - \frac{1}{2} \left( \nabla^2 \frac{\delta}{\delta \Psi^*} \right) \Psi^*
            + \frac{1}{2} \left( \nabla^2 \Psi \right) \frac{\delta}{\delta \Psi}
            - \frac{1}{4} \left(
                \nabla^2 \frac{\delta}{\delta \Psi^*}
            \right) \frac{\delta}{\delta \Psi}
        \right)
        \mathcal{W}[\hat{A}] \\
    & ={} && \frac{1}{2} \int d\xvec \left(
            - \frac{\delta}{\delta \Psi} \nabla^2 \Psi
            + \Psi^* \nabla^2 \frac{\delta}{\delta \Psi^*}
            + \left( \nabla^2 \frac{\delta}{\delta \Psi^*} \right) \Psi^*
            - \left( \nabla^2 \Psi \right) \frac{\delta}{\delta \Psi}
        \right)
        \mathcal{W}[\hat{A}] \\
    & ={} && (*)
\end{eqn2}

Using basis expansion, one can easily see that
\begin{eqn}
    \Psi^* \nabla^2 \frac{\delta}{\delta \Psi^*} \mathcal{F}[\Psi, \Psi^*]
    = \left( \nabla^2 \frac{\delta}{\delta \Psi^*} \right) \Psi^* \mathcal{F}[\Psi, \Psi^*]
    - \sum_{\nvec \in L} \phi_{\nvec}^* \nabla^2 \phi_{\nvec} \mathcal{F}[\Psi, \Psi^*]
\end{eqn}
and
\begin{eqn}
    \left( \nabla^2 \Psi \right) \frac{\delta}{\delta \Psi} \mathcal{F}[\Psi, \Psi^*]
    = \frac{\delta}{\delta \Psi} \left( \nabla^2 \Psi \right) \mathcal{F}[\Psi, \Psi^*]
    - \sum_{\nvec \in L} \phi_{\nvec}^* \nabla^2 \phi_{\nvec} \mathcal{F}[\Psi, \Psi^*].
\end{eqn}
Therefore:
\begin{eqn}
    (*)
    = \frac{1}{2} \int d\xvec \left(
        - \frac{\delta}{\delta \Psi} \nabla^2 \Psi
        + \left( \nabla^2 \frac{\delta}{\delta \Psi^*} \right) \Psi^*
        + \left( \nabla^2 \frac{\delta}{\delta \Psi^*} \right) \Psi^*
        - \frac{\delta}{\delta \Psi} \nabla^2 \Psi
    \right)
    \mathcal{W}[\hat{A}]
\end{eqn}
Now using \lmmref{func-calculus:move-laplacian} we can get the final result:
\begin{eqn}
    & = \frac{1}{2} \int d\xvec \left(
        - \frac{\delta}{\delta \Psi} \nabla^2 \Psi
        + \frac{\delta}{\delta \Psi^*} \nabla^2 \Psi^*
        + \frac{\delta}{\delta \Psi^*} \nabla^2 \Psi^*
        - \frac{\delta}{\delta \Psi} \nabla^2 \Psi
    \right)
    \mathcal{W}[\hat{A}] \\
    & = \int d\xvec \left(
        - \frac{\delta}{\delta \Psi} \nabla^2 \Psi
        + \frac{\delta}{\delta \Psi^*} \nabla^2 \Psi^*
    \right) \mathcal{W}[\hat{A}].
    \qedhere
\end{eqn}
\end{proof}

\begin{theorem}
\label{thm:wigner-spec:w-commutator2}
    \todo{Check that deltas have correct basis indices.}
    \begin{eqn*}
        & \mathcal{W} \left[
            [
                \int d\xvec \int d\xvec^\prime
                \Psiop_j^\dagger \Psiop_k^{\prime\dagger} \Psiop_j^\prime \Psiop_k,
                \hat{A}
            ]
        \right] \\
        & = \int d\xvec \int d\xvec^\prime \left(
            \frac{\delta}{\delta \Psi_j} \left(
                - \Psi_j^\prime \Psi_k \Psi_k^{\prime*}
                + \frac{1}{2} \delta_{jk} \delta_{\restbasis_j}(\xvec^\prime, \xvec^\prime) \Psi_k
                + \frac{1}{2} \delta_{\restbasis_j}(\xvec, \xvec^\prime) \Psi_j^\prime
            \right) \right . \\
        &   \left. + \frac{\delta}{\delta \Psi_j^{\prime*}} \left(
                \Psi_j^* \Psi_k \Psi_k^{\prime*}
                - \frac{1}{2} \delta_{jk} \delta_{\restbasis_j}(\xvec, \xvec) \Psi_k^{\prime*}
                - \frac{1}{2} \delta_{\restbasis_j}(\xvec, \xvec^\prime) \Psi_j^*
            \right) \right. \\
        &   \left. + \frac{\delta}{\delta \Psi_k^\prime} \left(
                - \Psi_j^\prime \Psi_j^* \Psi_k
                + \frac{1}{2} \delta_{jk} \delta_{\restbasis_j}(\xvec, \xvec) \Psi_j^\prime
                + \frac{1}{2} \delta_{\restbasis_k}(\xvec^\prime, \xvec) \Psi_k
            \right) \right .\\
        &   \left. + \frac{\delta}{\delta \Psi_k^*} \left(
                \Psi_j^\prime \Psi_j^* \Psi_k^{\prime*}
                - \frac{1}{2} \delta_{jk} \delta_{\restbasis_j}(\xvec^\prime, \xvec^\prime) \Psi_j^*
                - \frac{1}{2} \delta_{\restbasis_k}(\xvec^\prime, \xvec) \Psi_k^{\prime*}
            \right) \right. \\
        &   \left.
                + \frac{\delta}{\delta \Psi_j}
                \frac{\delta}{\delta \Psi_j^{\prime*}}
                \frac{\delta}{\delta \Psi_k^\prime}
                \frac{1}{4} \Psi_k
                - \frac{\delta}{\delta \Psi_j}
                \frac{\delta}{\delta \Psi_j^{\prime*}}
                \frac{\delta}{\delta \Psi_k^*}
                \frac{1}{4} \Psi_k^{\prime*}
            \right. \\
        &   \left.
                + \frac{\delta}{\delta \Psi_k^\prime}
                \frac{\delta}{\delta \Psi_k^*}
                \frac{\delta}{\delta \Psi_j}
                \frac{1}{4} \Psi_j^\prime
                - \frac{\delta}{\delta \Psi_k^\prime}
                \frac{\delta}{\delta \Psi_k^*}
                \frac{\delta}{\delta \Psi_j^{\prime*}}
                \frac{1}{4} \Psi_j^*
        \right) \mathcal{W}[\hat{A}].
    \end{eqn*}
\end{theorem}
\begin{proof}
Proof is the same as in case of \thmref{wigner-spec:w-commutator1}.
\end{proof}

\begin{lemma}
\label{lmm:wigner-spec:swap-differential}
    \begin{eqn*}
        \Psi^a \left( \frac{\delta}{\delta \Psi} \right)^b \mathcal{F}[\Psi, \Psi^*]
        = \sum_{j=0}^{\min(a, b)}
            \binom{b}{j} \frac{(-1)^j a!}{(a - j)!}
            \delta_{\restbasis}^j(\xvec, \xvec)
            \left( \frac{\delta}{\delta \Psi} \right)^{b - j}
            \Psi^{a - j}
            \mathcal{F}[\Psi, \Psi^*]
    \end{eqn*}
\end{lemma}
\begin{proof}
Proof by induction.
Let us assume that the statement is true for $b - 1$, and prove it for $b$
(also assuming non-trivial case of $a > 0$).
Moving a single differential to the left:
\begin{eqn}
    \Psi^a \left( \frac{\delta}{\delta \Psi} \right)^b \mathcal{F}
    = \left(
            \frac{\delta}{\delta \Psi} \Psi^a
            - a \Psi^{a - 1} \delta_{\restbasis}(\xvec, \xvec)
        \right)
        \left( \frac{\delta}{\delta \Psi} \right)^{b-1}
        \mathcal{F}
\end{eqn}
Using known relation for $b-1$:
\begin{eqn}
    ={} & \frac{\delta}{\delta \Psi} \sum_{j = 0}^{\min(a, b-1)}
            \binom{b-1}{j} \frac{(-1)^j a!}{(a-j)!} \delta_{\restbasis}^j(\xvec, \xvec)
            \left( \frac{\delta}{\delta \Psi} \right)^{b-1-j} \Psi^{a-j}
            \mathcal{F} \\
    & - a \delta_{\restbasis}(\xvec, \xvec) \sum_{j = 0}^{\min(a-1, b-1)}
            \binom{b-1}{j} \frac{(-1)^j (a-1)!}{(a-1-j)!} \delta_{\restbasis}^j(\xvec, \xvec)
            \left( \frac{\delta}{\delta \Psi} \right)^{b-1-j} \Psi^{a-1-j}
            \mathcal{F}
\end{eqn}
Merging coefficients before sums into the internal expressions:
\begin{eqn}
    ={} & \sum_{j = 0}^{\min(a, b-1)}
            \binom{b-1}{j} \frac{(-1)^j a!}{(a-j)!} \delta_{\restbasis}^j(\xvec, \xvec)
            \left( \frac{\delta}{\delta \Psi} \right)^{b-j} \Psi^{a-j}
            \mathcal{F} \\
    & + \sum_{j = 0}^{\min(a-1, b-1)}
            \binom{b-1}{j} \frac{(-1)^{j+1} a!}{(a-1-j)!} \delta_{\restbasis}^{j+1}(\xvec, \xvec)
            \left( \frac{\delta}{\delta \Psi} \right)^{b-1-j} \Psi^{a-1-j}
            \mathcal{F}
\end{eqn}
Shifting counter in the second sum:
\begin{eqn}
    ={} & \sum_{j = 0}^{\min(a, b-1)}
            \binom{b-1}{j} \frac{(-1)^j a!}{(a-j)!} \delta_{\restbasis}^j(\xvec, \xvec)
            \left( \frac{\delta}{\delta \Psi} \right)^{b-j} \Psi^{a-j}
            \mathcal{F} \\
    & + \sum_{j = 1}^{\min(a, b)}
            \binom{b-1}{j-1} \frac{(-1)^j a!}{(a-j)!} \delta_{\restbasis}^j(\xvec, \xvec)
            \left( \frac{\delta}{\delta \Psi} \right)^{b-j} \Psi^{a-j}
            \mathcal{F}
\end{eqn}
Now we can join sums, noticing that $\binom{b-1}{j} + \binom{b-1}{j-1} = \binom{b}{j}$.
There will be at most two leftover terms: first, term for $j=0$ from the first sum,
and, possibly, the term with $j=\min(a,b)$ from the second sum.
The former term appears only if $\min(a,b) > \min(a, b-1)$,
or, in other words, $a \ge b$ (which means that $\min(a, b) = b$ and $\min(a, b-1) = b-1$).
\begin{eqn}
    ={} & \binom{b-1}{0} \frac{(-1)^0 a!}{(a-0)!} \delta_{\restbasis}^0(\xvec, \xvec)
            \left( \frac{\delta}{\delta \Psi} \right)^{b-0} \Psi^{a-0}
            \mathcal{F} \\
    & + \sum_{j = 1}^{\min(a, b-1)}
            \binom{b}{j} \frac{(-1)^j a!}{(a-j)!} \delta_{\restbasis}^j(\xvec, \xvec)
            \left( \frac{\delta}{\delta \Psi} \right)^{b-j} \Psi^{a-j}
            \mathcal{F} \\
    & + H[a - b]
            \binom{b-1}{b-1} \frac{(-1)^j a!}{(a-b)!} \delta_{\restbasis}^b(\xvec, \xvec)
            \left( \frac{\delta}{\delta \Psi} \right)^{b-b} \Psi^{a-b}
            \mathcal{F},
\end{eqn}
Where $H[n]$ is the discrete Heaviside step function.
Now, since $\binom{b-1}{0} = \binom{b}{0}$ and $\binom{b-1}{b-1} = \binom{b}{b}$,
we can attach two leftover terms to the sum too, obtaining the statement of the lemma:
\begin{eqn}
    = \sum_{j = 0}^{\min(a, b)}
        \binom{b}{j} \frac{(-1)^j a!}{(a-j)!} \delta_{\restbasis}^j(\xvec, \xvec)
        \left( \frac{\delta}{\delta \Psi} \right)^{b-j} \Psi^{a-j}
        \mathcal{F}
    \qedhere
\end{eqn}
\end{proof}

\begin{lemma}[Sum rearrangement]
\label{lmm:wigner-spec:sum-rearrangement}
    \todo{Change indices to match theorem about losses?}
    For integer $l$, $u$:
    \begin{eqn*}
        \sum_{j=0}^l \sum_{k=0}^{\min(l-u,j)} f^{j-k} g(j, k)
        = \sum_{v=0}^l f^v \sum_{k=0}^{l-\max(u,v)} g(v + k, k).
    \end{eqn*}
\end{lemma}
\begin{proof}
\todo{Add picture of summation area for axes $j, k$?}
Obviously, the order $v = j - k$ of factor $f$ can vary from $0$ (say, when $j=0$ and $k=0$) to $l$ (when $j=l$ and $k=0$).
Therefore:
\begin{eqn}
    \sum_{j=0}^l \sum_{k=0}^{\min(l-u,j)} f^{j-k} g(j, k)
    = \sum_{v=0}^l f^v \sum_{k \in K(l, u, v)} g(v + k, k),
\end{eqn}
where the set $K$ is defined as
\begin{eqn}
    K(l, u, v)
    & = \{k |
        0 \le j \le l
        \wedge 0 \le k \le \min(l - u, j)
        \wedge j - k = v
    \} \\
    & = \{k |
        -v \le k \le l - v
        \wedge 0 \le k \le \min(l - u, v + k)
    \} \\
    & = \{k |
        k \le l - v
        \wedge 0 \le k \le \min(l - u, v + k)
    \}.
\end{eqn}

It is convenient to consider two cases separately $v \le u$ and $v > u$.
For the former case
\begin{eqn}
    K_{v \le u}
    = \{k |
        k \le l - v
        \wedge 0 \le k \le \min(l - u, k + v)
        \wedge v \le u
    \}.
\end{eqn}
Since $v \le u$, $k \le l - v \le l - u \le \min(l - u, k + v)$ is always true,
the first inequation is redundant:
\begin{eqn}
    = \{k |
        0 \le k \le \min(l - u, v + k)
        \wedge v \le u
    \}.
\end{eqn}
Splitting into two sets to get rid of minimum function:
\begin{eqn}
    & = \{k |
        v \le u \wedge k \ge 0
        \wedge
        (
            (k \le l - u \wedge l - u < v + k)
            \vee
            (k \le v + k \wedge l - u \ge v + k)
        )
    \} \\
    & = \{k |
        v \le u \wedge k \ge 0
        \wedge
        (
            (k \le l - u \wedge k > l - u - v)
            \vee
            (k \le l - u - v)
        )
    \} \\
    & = \{k |
        v \le u \wedge k \ge 0
        \wedge
        (k \le l - u)
    \} \\
    & = \{k | v \le u \wedge 0 \le k \le l - u \}.
\end{eqn}

For the latter case:
\begin{eqn}
    K_{v > u}
    ={} & \{k |
        k \le l - v
        \wedge 0 \le k \le \min(l - u, k + v)
        \wedge v > u
    \} \\
    ={} & \{k |
        v > u \wedge k \ge 0 \\
    &   \wedge
        (
            (k \le l - v \wedge k \le l - u \wedge l - u \le k + v)
            \vee
            (k \le l - v \wedge k \le k + v \wedge l - u > k + v)
        )
    \} \\
    ={} & \{k |
        v > u \wedge k \ge 0
        \wedge
        (
            (k \le l - v \wedge k \ge l - u - v)
            \vee
            (k \le l - v \wedge k < l - u - v)
        )
    \} \\
    ={} & \{k | v > u \wedge 0 \le k \le l - v \}.
\end{eqn}
Thus
\begin{eqn}
    K
    & = K_{v \le u} \cup K_{v > u} \\
    & = \{k | v \le u \wedge 0 \le k \le l - u \} \cup \{k | v > u \wedge 0 \le k \le l - v \} \\
    & = \{k | 0 \le k \le l - \max(u, v) \},
\end{eqn}
which gives us the statement of the lemma.
\end{proof}

\begin{theorem}
\label{thm:wigner-spec:w-losses}
    If loss operator $\hat{\mathcal{L}}_{\lvec}$ is defined as
    \begin{eqn*}
        \hat{\mathcal{L}}_{\lvec} [\hat{A}]
        = 2 \hat{O}_{\lvec} \hat{A} \hat{O}_{\lvec}^\dagger
            - \hat{O}_{\lvec}^\dagger \hat{O}_{\lvec} \hat{A}
            - \hat{A} \hat{O}_{\lvec}^\dagger \hat{O}_{\lvec},
    \end{eqn*}
    where
    \begin{eqn*}
        \hat{O}_{\lvec}
        \equiv \hat{O}_{\lvec} (\Psiopvec)
        = \prod_{c=1}^C \Psiop_c^{l_c} (\xvec),
    \end{eqn*}
    then its Wigner transformation is
    \begin{eqn*}
        \mathcal{W} \left[ \int d\xvec \hat{\mathcal{L}}_{\lvec} [\hat{A}] \right]
        = \int d\xvec
            \sum_{j_1=0}^{l_1} \sum_{k_1=0}^{l_1} \ldots
            \sum_{j_C=0}^{l_C} \sum_{k_C=0}^{l_C}
                \left(
                    \prod_{c=1}^C
                        \left( \frac{\delta}{\delta \Psi_c^*} \right)^{j_c}
                        \left( \frac{\delta}{\delta \Psi_c} \right)^{k_c}
                \right)
                L_{\lvec, \jvec, \kvec}
            \mathcal{W}[\hat{A}],
    \end{eqn*}
    where
    \begin{eqn*}
        L_{\lvec, \jvec, \kvec}
        ={} & \left( 2 - (-1)^{\sum_c j_c} - (-1)^{\sum_c k_c} \right) \\
        &   \prod_{c=1}^C \left(
                \sum_{m_c=0}^{l_c - \max(j_c, k_c)}
                Q(l_c, j_c, k_c, m_c)
                \delta_{\restbasis_c}^{m_c}(\xvec, \xvec)
                \Psi_c^{l_c - j_c - m_c}
                (\Psi_c^*)^{l_c - k_c - m_c}
            \right),
    \end{eqn*}
    and
    \begin{eqn*}
        Q(l, j, k, m)
        = \frac{(-1)^m}{2^{j + k + m}}
            \frac{(l!)^2}{m! j! k! (l - k - m)! (l - j - m)!}.
    \end{eqn*}
\end{theorem}
\begin{proof}
Let us perform the transformation for each term of loss operator.
\begin{eqn}
    W_1
    & = \mathcal{W}[\hat{O}_{\lvec} \hat{A} \hat{O}_{\lvec}^\dagger] \\
    & = \prod_{c=1}^C \left(
            \Psi_c + \frac{1}{2} \frac{\delta}{\delta \Psi_c^*}
        \right)^{l_c}
        \left(
            \Psi_c^* + \frac{1}{2} \frac{\delta}{\delta \Psi_c}
        \right)^{l_c}
        \mathcal{W}[\hat{A}] \\
    & = \prod_{c=1}^C \left(
            \sum_{j=0}^{l_c}
                \binom{l_c}{j} \left( \frac{1}{2} \right)^j
                \left( \frac{\delta}{\delta \Psi_c^*} \right)^j
                \Psi_c^{l_c - j}
            \sum_{k=0}^{l_c}
                \binom{l_c}{k} \left( \frac{1}{2} \right)^k
                \left( \frac{\delta}{\delta \Psi_c} \right)^k
                (\Psi_c^*)^{l_c - k}
        \right)
        \mathcal{W}[\hat{A}] \\
    & = \prod_{c=1}^C \left(
            \sum_{j=0}^{l_c}
            \sum_{k=0}^{l_c}
                \binom{l_c}{j} \binom{l_c}{k} \left( \frac{1}{2} \right)^{j + k}
                \left( \frac{\delta}{\delta \Psi_c^*} \right)^j
                \Psi_c^{l_c - j}
                \left( \frac{\delta}{\delta \Psi_c} \right)^k
                (\Psi_c^*)^{l_c - k}
        \right)
        \mathcal{W}[\hat{A}].
\end{eqn}
Using \lmmref{wigner-spec:swap-differential} to swap $\Psi_c$ and $\delta / \delta \Psi_c$:
\begin{eqn}
    ={} & \prod_{c=1}^C \left(
            \sum_{j=0}^{l_c}
            \sum_{k=0}^{l_c}
                \binom{l_c}{j} \binom{l_c}{k} \left( \frac{1}{2} \right)^{j + k}
                \left( \frac{\delta}{\delta \Psi_c^*} \right)^j
        \right. \\
        & \left.
                \sum_{m=0}^{\min(l_c - j, k)}
                    \binom{k}{m}
                    \frac{(-1)^m (l_c - j)!}{(l_c - j - m)!}
                    \delta_{\restbasis_c}^m(\xvec, \xvec)
                    \left( \frac{\delta}{\delta \Psi_c} \right)^{k - m}
                    \Psi_c^{l_c - j - m}
                (\Psi_c^*)^{l_c - k}
        \right)
        \mathcal{W}[\hat{A}] \\
    ={} & (*).
\end{eqn}
Using \lmmref{wigner-spec:sum-rearrangement} with
\begin{eqn}
    f & = \frac{\delta}{\delta \Psi_c}, \\
    g & = \binom{l_c}{j} \binom{l_c}{k} \left( \frac{1}{2} \right)^{j + k}
        \binom{k}{m}
        \frac{(-1)^m (l_c - j)!}{(l_c - j - m)!}
        \delta_{\restbasis_c}^m(\xvec, \xvec)
        \Psi_c^{l_c - j - m}
    (\Psi_c^*)^{l_c - k},
\end{eqn}
we obtain
\begin{eqn}
    (*) = \prod_{c=1}^C \left(
            \sum_{j=0}^{l_c}
            \sum_{k=0}^{l_c}
                \left( \frac{\delta}{\delta \Psi_c^*} \right)^j
                \left( \frac{\delta}{\delta \Psi_c} \right)^k
                \sum_{m=0}^{l_c - \max(j, k)}
                Q_1(l, j, k, m)
                \delta_{\restbasis_c}^m(\xvec, \xvec)
                \Psi_c^{l_c - j - m}
                (\Psi_c^*)^{l_c - k - m}
        \right)
        \mathcal{W}[\hat{A}],
\end{eqn}
where
\begin{eqn}
    Q_1(l, j, k, m)
    = \frac{1}{2^{j + k + m}}
        \binom{l}{j} \binom{l}{k + m} \binom{k + m}{m}
        \frac{(-1)^m (l - j)!}{(l - j - m)!}.
\end{eqn}

Similarly,
\begin{eqn}
    W_2
    & = \mathcal{W}[\hat{O}_{\lvec}^\dagger \hat{O}_{\lvec} \hat{A}] \\
    & = \prod_{c=1}^C \left(
            \sum_{j=0}^{l_c}
            \sum_{k=0}^{l_c}
                \left( \frac{\delta}{\delta \Psi_c^*} \right)^j
                \left( \frac{\delta}{\delta \Psi_c} \right)^k
                \sum_{m=0}^{l_c - \max(j, k)}
                Q_2(l, j, k, m)
                \delta_{\restbasis_c}^m(\xvec, \xvec)
                \Psi_c^{l_c - j - m}
                (\Psi_c^*)^{l_c - k - m}
        \right)
        \mathcal{W}[\hat{A}],
\end{eqn}
where
\begin{eqn}
    Q_2(l, j, k, m)
    = \frac{(-1)^k}{2^{j + k + m}}
        \binom{l}{k} \binom{l}{j + m} \binom{j + m}{m}
        \frac{(-1)^m (l - k)!}{(l - k - m)!},
\end{eqn}
and
\begin{eqn}
    W_3
    & = \mathcal{W}[\hat{A} \hat{O}_{\lvec}^\dagger \hat{O}_{\lvec}] \\
    & = \prod_{c=1}^C \left(
            \sum_{j=0}^{l_c}
            \sum_{k=0}^{l_c}
                \left( \frac{\delta}{\delta \Psi_c^*} \right)^j
                \left( \frac{\delta}{\delta \Psi_c} \right)^k
                \sum_{m=0}^{l_c - \max(j, k)}
                Q_3(l, j, k, m)
                \delta_{\restbasis_c}^m(\xvec, \xvec)
                \Psi_c^{l_c - j - m}
                (\Psi_c^*)^{l_c - k - m}
        \right)
        \mathcal{W}[\hat{A}],
\end{eqn}
where
\begin{eqn}
    Q_3(l, j, k, m)
    = (-1)^j \left( \frac{1}{2} \right)^{j + k + m}
        \binom{l}{j} \binom{l}{k + m} \binom{k + m}{m}
        \frac{(-1)^m (l - j)!}{(l - j - m)!}.
\end{eqn}

Expanding binomial coefficients, one can see that
\begin{eqn}
    Q_1 \equiv Q,\, Q_2 \equiv (-1)^k Q,\, Q_3 \equiv (-1)^j Q,
\end{eqn}
where
\begin{eqn}
    Q(l, j, k, m)
    = \frac{(-1)^m}{2^{j + k + m}}
        \frac{(l!)^2}{m! j! k! (l - k - m)! (l - j - m)!}.
\end{eqn}

The full expression for the Wigner transformation of $\hat{\mathcal{L}}_{\lvec}$ is
\begin{eqn}
    \mathcal{W}[\hat{\mathcal{L}_{\lvec}}[\hat{A}]] = 2 W_1 - W_2 - W_3,
\end{eqn}
which leads to the statement of the lemma.
\end{proof}
