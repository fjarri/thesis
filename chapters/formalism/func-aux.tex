% =============================================================================
\section{Field operators and restricted basis}
% =============================================================================

Phase-space treatment of multimode problems can be simplified by working with multimode field operators instead of single-mode operators.
It was initially introduced by Graham~\cite{Graham1970,Graham1970a}.
Examples of usage can be found in ~\cite{Steel1998,Norrie2006a}.
\todo{Move to bibliography review and extend.}

Multimode fields are described by operators $\Psiop_j^{\dagger}(\xvec)$ and $\Psiop_j(\xvec)$,
where $\Psiop_j^{\dagger}(\xvec)$ creates a bosonic atom of spin $j$ at location $\xvec$,
and $\Psiop_j(\xvec)$ destroys one;
the commutators are
\begin{equation}
\label{eqn:formalism:func-aux:commutators}
	[ \Psiop_j(\xvec), \Psiop_k^{\dagger}(\xvec^\prime) ]
	= \delta_{jk} \delta(\xvec^\prime-\xvec).
\end{equation}
Field operators can be decomposed using a single-particle basis \todo{explanation needed?}:
\[
	\Psiop_j(\xvec) = \sum\limits_{\nvec} \phi_{\nvec}(\xvec) \hat{a}_{j,\nvec},
\]
where $\phi_{\nvec}$ is some orthonormal basis,
$\nvec$ is a state vector with $D$ elements.
Single mode operators $\hat{a}_{j,\nvec}$ obey commutation relations~\eqnref{formalism:mm-aux:commutators},
the pair $j,\nvec$ serving as a mode identifier.
Orthonormality and completeness conditions for basis functions are, respectively,
\[
	\int\limits_A \phi_{\nvec}^*(\xvec) \phi_{\mvec}(\xvec) d\xvec = \delta_{\nvec\mvec},
\]
\[
	\sum\limits_{\nvec} \phi_{\nvec}^*(\xvec) \phi_{\nvec}(\xvec^\prime) = \delta(\xvec^\prime - \xvec),
\]
where the exact nature of integration area $A$ depends on the basis set.
Hereinafter we assume that the integration is always performed over $A$.

Now suppose we want to consider only modes from some subset $L$.
Corresponding projection operator can be written as
\[
	\mathcal{P} \equiv \sum\limits_{\nvec \in L} \lvert \nvec \rangle \langle \nvec \rvert,
\]
Or, in coordinate form:
\[
	\mathcal{P} [f(\xvec)]
	= \sum\limits_{\nvec \in L} \phi_{\nvec} (\xvec) \int
		d\xvec^\prime\, \phi_{\nvec}^*(\xvec^\prime) f(\xvec^\prime),
\]
where $f$ can be a common function or an operator.
Being applied to the field operator $\Psiop_j$, this operator returns the restricted field operator
\[
	\mathcal{P} [\Psiop_j]
	= \sum\limits_{\nvec \in L} \phi_{\nvec} (\xvec) \hat{a}_{j,\nvec}
	= \Psiop_{jP} (\xvec),
\]
containing only modes from subset $L$.
If $L$ is the whole mode space, then obviously $\mathcal{P} \equiv \mathds{1}$.
In order to simplify equations, we will consider all field operators in this chapter to be restricted and omit the index $P$.

Because of the restricted nature of the operator, commutation relations~\eqnref{formalism:func-aux:commutators} no longer apply.
The following ones should be used instead:
\begin{equation}
\label{eqn:formalism:func-aux:restricted-commutators}
\begin{split}
	\left[ \Psiop_j(\xvec), \Psiop_k(\xvec^\prime) \right]
	& = \left[ \Psiop_j^\dagger(\xvec), \Psiop_k^\dagger(\xvec^\prime) \right] = 0, \\
	\left[ \Psiop_j(\xvec), \Psiop_k^\dagger(\xvec^\prime) \right]
	& = \delta_{jk} \delta_P(\xvec^\prime - \xvec),
\end{split}
\end{equation}
where the restricted delta function $\delta_P$ is defined as
\begin{equation}
\label{eqn:multimode-formalism:restricted-delta}
	\delta_P(\xvec^\prime - \xvec)
	= \sum\limits_{\nvec \in L} \phi_{\nvec}^* (\xvec^\prime) \phi_{\nvec} (\xvec).
\end{equation}
Note that conjugation operator swaps variables in $\delta_P$: $\delta_P^*(\xvec^\prime - \xvec) = \delta_P(\xvec - \xvec^\prime)$.

Restricted delta function can be used to rewrite equation for projection operator $P$:
\[
	\mathcal{P} [f(\xvec)] = \int d\xvec^\prime \delta_P(\xvec^\prime - \xvec) f(\xvec^\prime).
\]
The Hermitian conjugate of $P$ is thus defined as
\[
	(\mathcal{P} [f(\xvec)])^\dagger
	= \int d\xvec^\prime \delta_P^*(\xvec^\prime - \xvec) f^\dagger(\xvec^\prime)
	= \mathcal{P}^\dagger [f^\dagger(\xvec)].
\]


Let us now find the expression for high-order commutators of restricted field operators, analogous to \lmmref{formalism:mm-aux:high-order-commutators} for single-mode operators.
It can be done using the similar recursive procedure.

\begin{lemma}
\begin{equation*}
\begin{split}
	\left[ \Psiop, ( \Psiop^{\prime\dagger} )^l \right]
	& = l \delta_P (\xvec^\prime - \xvec) ( \Psiop^{\prime\dagger} )^{l-1}, \\
	\left[ \Psiop^\dagger, ( \Psiop^\prime )^l \right]
	& = - l \delta_P^* (\xvec^\prime - \xvec) ( \Psiop^\prime )^{l-1}.
\end{split}
\end{equation*}
\end{lemma}
\begin{proof}
Given that we know the expression for $\left[ \Psiop, ( \Psiop^{\prime\dagger} )^{l-1} \right]$,
the commutator of higher order can be expanded as
\begin{equation*}
\begin{split}
	\left[ \Psiop, ( \Psiop^{\prime\dagger} )^l \right]
	& = \Psiop ( \Psiop^{\prime\dagger} )^l - ( \Psiop^{\prime\dagger} )^l \Psiop \\
	& = (
		\delta_P (\xvec^\prime - \xvec) + \Psiop^{\prime\dagger} \Psiop
	) ( \Psiop^{\prime\dagger} )^{l-1}
	- ( \Psiop^{\prime\dagger} )^l \Psiop \\
	& = \delta_P (\xvec^\prime - \xvec) ( \Psiop^{\prime\dagger} )^{l-1}
	+ \Psiop^{\prime\dagger} (
		\Psiop ( \Psiop^{\prime\dagger} )^{l-1}
		- ( \Psiop^{\prime\dagger} )^{l-1} \Psiop
	) \\
	& = \delta_P (\xvec^\prime - \xvec) ( \Psiop^{\prime\dagger} )^{l-1}
	+ \Psiop^{\prime\dagger} [
		\Psiop, ( \Psiop^{\prime\dagger} )^{l-1}
	].
\end{split}
\end{equation*}
Now we can get the commutator of any order starting from the known relation~\eqnref{formalism:func-aux:restricted-commutators}.
\end{proof}

A further generalisation of these relations is
\begin{lemma}
\label{lmm:formalism:func-aux:functional-commutators}
\begin{equation*}
\begin{split}
	\left[ \Psiop, f( \Psiop^\prime, \Psiop^{\prime\dagger} ) \right]
	& = \delta_P (\xvec^\prime - \xvec) \frac{\partial f}{\partial \Psiop^{\prime\dagger}} \\
	\left[ \Psiop^\dagger, f( \Psiop^\prime, \Psiop^{\prime\dagger} ) \right]
	& = -\delta_P^* (\xvec^\prime - \xvec) \frac{\partial f}{\partial \Psiop^\prime},
\end{split}
\end{equation*}
where $f(x, y)$ is a function that can be expanded in the power series of $x$ and $y$.
\end{lemma}
\begin{proof}
Let us prove the first relation; the procedure for the second one is the same.
Without loss of generality, we can assume that $f(\Psiop^\prime, \Psiop^{\prime\dagger})$ can be expanded in power series of normally ordered operators (otherwise we can just use commutation relations).
Thus
\begin{equation*}
\begin{split}
	\left[ \Psiop, f( \Psiop^\prime, \Psiop^{\prime\dagger} ) \right]
	& = \sum\limits_{r,s} f_{rs} [ \Psiop, (\Psiop^{\prime\dagger})^r (\Psiop^\prime)^s ] \\
	& = \sum\limits_{r,s} f_{rs} [ \Psiop, (\Psiop^{\prime\dagger})^r ] (\Psiop^\prime)^s \\
	& = \sum\limits_{r,s} f_{rs} r \delta_P(\xvec^\prime - \xvec)
		(\Psiop^{\prime\dagger})^{r-1} (\Psiop^\prime)^s \\
	& = \delta_P (\xvec^\prime - \xvec) \frac{\partial f}{\partial \Psiop^{\prime\dagger}}.
	\qedhere
\end{split}
\end{equation*}
\end{proof}

Detailed description of functional calculus is given in~\cite{Dalton2011} \todo{anywhere else?}.
Here we only provide some important results which are going to be used late on in this chapter.
