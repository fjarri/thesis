% =============================================================================
\section{Functional Wigner representation}
% =============================================================================


Phase-space treatment of multimode problems can be simplified by working with multimode field operators instead of single-mode operators~\cite{Steel1998,Norrie2006a}.

Multimode fields are described by operators $\Psiop_j^{\dagger}(\xvec)$ and $\Psiop_j(\xvec)$,
where $\Psiop_j^{\dagger}(\xvec)$ creates a bosonic atom of spin $j$ at location $\xvec$,
and $\Psiop_j(\xvec)$ destroys one;
the commutators are
\begin{equation}
\label{eqn:multimode-formalism:multimode-commutators}
	[ \Psiop_j(\xvec), \Psiop_k^{\dagger}(\xvec^\prime) ]
	= \delta_{jk} \delta(\xvec-\xvec^\prime).
\end{equation}
Field operators can be decomposed using a single-particle basis \todo{explanation needed?}:
\[
	\Psiop_j(\xvec) = \sum\limits_{\nvec} \phi_{\nvec}(\xvec) \hat{a}_{j,\nvec},
\]
where $\phi_{\nvec}$ is some orthonormal basis,
$\nvec$ is a state vector with $D$ elements.
Single mode operators $\hat{a}_{j,\nvec}$ obey commutation relations \eqnref{multimode-formalism:single-mode-commutators},
the pair $j,\nvec$ serving as a mode identifier.
Orthonormality condition for basis functions is
\[
	\int\limits_A \phi_{\nvec}^*(\xvec) \phi_{\mvec}(\xvec) d\xvec = \delta_{\nvec\mvec},
\]
where the exact nature of integration area $A$ depends on the basis set.
Hereinafter we assume that the integration is always performed over $A$.

Now suppose we want to consider only modes from some subset $L$.
Corresponding projection operator can be written as
\[
	P \equiv \sum\limits_{\nvec \in L} \lvert \nvec \rangle \langle \nvec \rvert,
\]
Or, in coordinate form:
\[
	P [f(\xvec)]
	= \sum\limits_{\nvec \in L} \phi_{\nvec} (\xvec) \int
		d\xvec^\prime\, \phi_{\nvec}^*(\xvec^\prime) f(\xvec^\prime).
\]
Being applied to the field operator $\Psiop_j$, this operator returns the restricted field operator
\[
	P [\Psiop_j]
	= \sum\limits_{\nvec \in L} \phi_{\nvec} (\xvec) \hat{a}_{j,\nvec}
	= \Psiop_{jP} (\xvec),
\]
containing only modes from subset $L$.
If $L$ is the whole mode space, then obviously $P \equiv \mathds{1}$.
In order to simplify equations, we will consider all field operators in this chapter to be restricted and omit the index $P$.

Because of the restricted nature of the operator, commutation relations~\eqnref{multimode-formalism:multimode-commutators} no longer apply.
The following ones should be used instead:
\begin{equation}
\label{eqn:multimode-formalism:restricted-commutators}
\begin{split}
	\left[ \Psiop_j(\xvec), \Psiop_k(\xvec^\prime) \right]
	& = \left[ \Psiop_j^\dagger(\xvec), \Psiop_k^\dagger(\xvec^\prime) \right] = 0, \\
	\left[ \Psiop_j(\xvec), \Psiop_k^\dagger(\xvec^\prime) \right]
	& = \delta_{jk} \delta_P(\xvec - \xvec^\prime),
\end{split}
\end{equation}
where the restricted delta function $\delta_P$ is defined as
\begin{equation}
\label{eqn:multimode-formalism:restricted-delta}
	\delta_P(\xvec - \xvec^\prime)
	= \sum\limits_{\nvec \in L} \phi_{\nvec}^* (\xvec) \phi_{\nvec} (\xvec^\prime).
\end{equation}
Note that conjugation operator swaps variables in $\delta_P$: $\delta_P^*(\xvec - \xvec^\prime) = \delta_P(\xvec^\prime - \xvec)$.

Restricted delta function can be used to rewrite equation for projection operator $P$:
\[
	P [f(\xvec)] = \int d\xvec^\prime \delta_P(\xvec^\prime - \xvec) f(\xvec^\prime).
\]
The Hermitian conjugate of $P$ is thus defined as
\[
	(P [f(\xvec)])^\dagger
	= \int d\xvec^\prime \delta_P^*(\xvec^\prime - \xvec) f^\dagger(\xvec^\prime)
	= P^\dagger [f^\dagger(\xvec)].
\]
