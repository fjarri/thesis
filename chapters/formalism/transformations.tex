% =============================================================================
\section{Specific cases of transformations}
% =============================================================================

This section contains some theorems concerning transformations of specific operator sequences,
which will be useful when transforming the master equation.

\begin{theorem}
\label{thm:formalism:transformations:w-commutator1}
\[
	\mathcal{W} \left[ [\int d\xvec \Psiop_j^\dagger \Psiop_k, \hat{A}] \right]
	= \int d\xvec \left(
		- \frac{\delta}{\delta \Psi_j} \Psi_k
		+ \frac{\delta}{\delta \Psi_k^*} \Psi_j^*
	\right) \mathcal{W}[\hat{A}].
\]
\end{theorem}
\begin{proof}
\begin{equation*}
\begin{split}
	\mathcal{W} \left[ [\int d\xvec \Psiop_j^\dagger \Psiop_k, \hat{A}] \right]
	& = \int d\xvec \left(
		\left(
			\Psi_j^* - \frac{1}{2} \frac{\delta}{\delta \Psi_j}
		\right)
		\left(
			\Psi_k + \frac{1}{2} \frac{\delta}{\delta \Psi_k^*}
		\right)
		- \left(
			\Psi_k - \frac{1}{2} \frac{\delta}{\delta \Psi_k^*}
		\right)
		\left(
			\Psi_j^* + \frac{1}{2} \frac{\delta}{\delta \Psi_j}
		\right)
	\right)
	\mathcal{W}[\hat{A}] \\
	& = \frac{1}{2} \int d\xvec \left(
		- \frac{\delta}{\delta \Psi_j} \Psi_k
		+ \Psi_j^* \frac{\delta}{\delta \Psi_k^*}
		+ \frac{\delta}{\delta \Psi_k^*} \Psi_j^*
		- \Psi_k \frac{\delta}{\delta \Psi_j}
	\right)
	\mathcal{W}[\hat{A}] \\
	& = (*).
\end{split}
\end{equation*}
Changing the order of derivatives and functions using the relation
\[
	\Psi_k \frac{\delta}{\delta \Psi_j} \mathcal{F}
	= \left(
		\frac{\delta}{\delta \Psi_j} \Psi_k
		- \delta_{jk} \delta_P(\xvec, \xvec)
	\right) \mathcal{F},
\]
we get
\[
	(*) = \int d\xvec \left(
		- \frac{\delta}{\delta \Psi_j} \Psi_k
		+ \frac{\delta}{\delta \Psi_k^*} \Psi_j^*
	\right)
	\mathcal{W}[\hat{A}].
	\qedhere
\]
\end{proof}

Commutators with Laplacian inside require somewhat special treatment,
because it acts on basis functions and, in general, cannot be dragged around like a constant.
For our purposes we only need one specific case,
and, fortunately, in this case it does act like a constant.

\begin{theorem}
\label{thm:formalism:transformations:w-laplacian-commutator1}
\[
	\mathcal{W} \left[
		\int d\xvec [\Psiop^\dagger(\xvec) \nabla^2 \Psiop(\xvec), \hat{A}]
	\right]
	= \int d\xvec \left(
		- \frac{\delta}{\delta \Psi} \nabla^2 \Psi
		+ \frac{\delta}{\delta \Psi^*} \nabla^2 \Psi^*
	\right) \mathcal{W}[\hat{A}].
\]
\end{theorem}
\begin{proof}
First, it is obvious from the definition of the Wigner transformation that
\[
	\mathcal{W} \left[ \int d\xvec \hat{B}(\xvec) \hat{A} \right]
	= \int d\xvec \mathcal{W} [\hat{B}(\xvec) \hat{A}]
\]
and
\[
	\mathcal{W} [ \nabla^2 \hat{B}(\xvec) \hat{A} ]
	= \nabla^2 \mathcal{W} [\hat{B}(\xvec) \hat{A}].
\]
Let us now expand the commutator and apply correspondences from \thmref{formalism:func-wigner:correspondences}:
\begin{equation*}
\begin{split}
	\mathcal{W} \left[
		\int d\xvec [\Psiop^\dagger(\xvec) \nabla^2 \Psiop(\xvec), \hat{A}]
	\right]
	& = \mathcal{W} \left[
			\int d\xvec \Psiop^\dagger(\xvec) \nabla^2 \Psiop(\xvec) \hat{A}
		\right]
		- \mathcal{W} \left[
				\int d\xvec \hat{A} \Psiop^\dagger(\xvec) \nabla^2 \Psiop(\xvec)
		\right] \\
	& = \int d\xvec \left(
			\Psi^* - \frac{1}{2} \frac{\delta}{\delta \Psi}
		\right)
		\left(
			\nabla^2 \Psi + \frac{1}{2} \nabla^2 \frac{\delta}{\delta \Psi^*}
		\right)
		\mathcal{W}[\hat{A}] \\
	& - \int d\xvec \left(
			\nabla^2 \Psi - \frac{1}{2} \nabla^2 \frac{\delta}{\delta \Psi^*}
		\right)
		\left(
			\Psi^* + \frac{1}{2} \frac{\delta}{\delta \Psi}
		\right)
		\mathcal{W}[\hat{A}] \\
	& = \int d\xvec \left(
			\Psi^* \nabla^2 \Psi
			- \frac{1}{2} \frac{\delta}{\delta \Psi} \nabla^2 \Psi
			+ \frac{1}{2} \Psi^* \nabla^2 \frac{\delta}{\delta \Psi^*}
			- \frac{1}{4} \frac{\delta}{\delta \Psi} \nabla^2 \frac{\delta}{\delta \Psi^*}
		\right)
		\mathcal{W}[\hat{A}] \\
	& - \int d\xvec \left(
			\Psi^* \nabla^2 \Psi
			- \frac{1}{2} \left( \nabla^2 \frac{\delta}{\delta \Psi^*} \right) \Psi^*
			+ \frac{1}{2} \left( \nabla^2 \Psi \right) \frac{\delta}{\delta \Psi}
			- \frac{1}{4} \left(
				\nabla^2 \frac{\delta}{\delta \Psi^*}
			\right) \frac{\delta}{\delta \Psi}
		\right)
		\mathcal{W}[\hat{A}] \\
	& = \frac{1}{2} \int d\xvec \left(
			- \frac{\delta}{\delta \Psi} \nabla^2 \Psi
			+ \Psi^* \nabla^2 \frac{\delta}{\delta \Psi^*}
			+ \left( \nabla^2 \frac{\delta}{\delta \Psi^*} \right) \Psi^*
			- \left( \nabla^2 \Psi \right) \frac{\delta}{\delta \Psi}
		\right)
		\mathcal{W}[\hat{A}]
	= (*)
\end{split}
\end{equation*}

Using basis expansion, one can easily see that
\[
	\Psi^* \nabla^2 \frac{\delta}{\delta \Psi^*} \mathcal{F}[\Psi, \Psi^*]
	= \left( \nabla^2 \frac{\delta}{\delta \Psi^*} \right) \Psi^* \mathcal{F}[\Psi, \Psi^*]
	- \sum_{\nvec \in L} \phi_{\nvec}^* \nabla^2 \phi_{\nvec} \mathcal{F}[\Psi, \Psi^*]
\]
and
\[
	\left( \nabla^2 \Psi \right) \frac{\delta}{\delta \Psi} \mathcal{F}[\Psi, \Psi^*]
	= \frac{\delta}{\delta \Psi} \left( \nabla^2 \Psi \right) \mathcal{F}[\Psi, \Psi^*]
	- \sum_{\nvec \in L} \phi_{\nvec}^* \nabla^2 \phi_{\nvec} \mathcal{F}[\Psi, \Psi^*].
\]
Therefore:
\begin{equation*}
\begin{split}
	(*)
	= \frac{1}{2} \int d\xvec \left(
		- \frac{\delta}{\delta \Psi} \nabla^2 \Psi
		+ \left( \nabla^2 \frac{\delta}{\delta \Psi^*} \right) \Psi^*
		+ \left( \nabla^2 \frac{\delta}{\delta \Psi^*} \right) \Psi^*
		- \frac{\delta}{\delta \Psi} \nabla^2 \Psi
	\right)
	\mathcal{W}[\hat{A}] = (*)
\end{split}
\end{equation*}
Now using \lmmref{formalism:func-calculus:move-laplacian} we can get the final result:
\begin{equation*}
\begin{split}
	(*)
	& = \frac{1}{2} \int d\xvec \left(
		- \frac{\delta}{\delta \Psi} \nabla^2 \Psi
		+ \frac{\delta}{\delta \Psi^*} \nabla^2 \Psi^*
		+ \frac{\delta}{\delta \Psi^*} \nabla^2 \Psi^*
		- \frac{\delta}{\delta \Psi} \nabla^2 \Psi
	\right)
	\mathcal{W}[\hat{A}] \\
	& = \int d\xvec \left(
		- \frac{\delta}{\delta \Psi} \nabla^2 \Psi
		+ \frac{\delta}{\delta \Psi^*} \nabla^2 \Psi^*
	\right) \mathcal{W}[\hat{A}].
	\qedhere
\end{split}
\end{equation*}
\end{proof}
