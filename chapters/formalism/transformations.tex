% =============================================================================
\section{Specific cases of transformations}
% =============================================================================

This section contains some theorems concerning transformations of specific operator sequences,
which will be useful when transforming the master equation.

\begin{theorem}
\label{thm:formalism:transformations:w-commutator1}
\[
	\mathcal{W} \left[ [\int d\xvec \Psiop_j^\dagger \Psiop_k, \hat{A}] \right]
	= \int d\xvec \left(
		- \frac{\delta}{\delta \Psi_j} \Psi_k
		+ \frac{\delta}{\delta \Psi_k^*} \Psi_j^*
	\right) \mathcal{W}[\hat{A}].
\]
\end{theorem}
\begin{proof}
\begin{equation*}
\begin{split}
	\mathcal{W} \left[ [\int d\xvec \Psiop_j^\dagger \Psiop_k, \hat{A}] \right]
	& = \int d\xvec \left(
		\left(
			\Psi_j^* - \frac{1}{2} \frac{\delta}{\delta \Psi_j}
		\right)
		\left(
			\Psi_k + \frac{1}{2} \frac{\delta}{\delta \Psi_k^*}
		\right)
		- \left(
			\Psi_k - \frac{1}{2} \frac{\delta}{\delta \Psi_k^*}
		\right)
		\left(
			\Psi_j^* + \frac{1}{2} \frac{\delta}{\delta \Psi_j}
		\right)
	\right)
	\mathcal{W}[\hat{A}] \\
	& = \frac{1}{2} \int d\xvec \left(
		- \frac{\delta}{\delta \Psi_j} \Psi_k
		+ \Psi_j^* \frac{\delta}{\delta \Psi_k^*}
		+ \frac{\delta}{\delta \Psi_k^*} \Psi_j^*
		- \Psi_k \frac{\delta}{\delta \Psi_j}
	\right)
	\mathcal{W}[\hat{A}] \\
	& = (*).
\end{split}
\end{equation*}
Changing the order of derivatives and functions using the relation
\[
	\Psi_k \frac{\delta}{\delta \Psi_j} \mathcal{F}
	= \left(
		\frac{\delta}{\delta \Psi_j} \Psi_k
		- \delta_{jk} \delta_P(\xvec, \xvec)
	\right) \mathcal{F},
\]
we get
\[
	(*) = \int d\xvec \left(
		- \frac{\delta}{\delta \Psi_j} \Psi_k
		+ \frac{\delta}{\delta \Psi_k^*} \Psi_j^*
	\right)
	\mathcal{W}[\hat{A}].
	\qedhere
\]
\end{proof}

Commutators with Laplacian inside require somewhat special treatment,
because it acts on basis functions and, in general, cannot be dragged around like a constant.
For our purposes we only need one specific case,
and, fortunately, in this case it does act like a constant.

\begin{theorem}
\label{thm:formalism:transformations:w-laplacian-commutator1}
\[
	\mathcal{W} \left[
		\int d\xvec [\Psiop^\dagger(\xvec) \nabla^2 \Psiop(\xvec), \hat{A}]
	\right]
	= \int d\xvec \left(
		- \frac{\delta}{\delta \Psi} \nabla^2 \Psi
		+ \frac{\delta}{\delta \Psi^*} \nabla^2 \Psi^*
	\right) \mathcal{W}[\hat{A}].
\]
\end{theorem}
\begin{proof}
First, it is obvious from the definition of the Wigner transformation that
\[
	\mathcal{W} \left[ \int d\xvec \hat{B}(\xvec) \hat{A} \right]
	= \int d\xvec \mathcal{W} [\hat{B}(\xvec) \hat{A}]
\]
and
\[
	\mathcal{W} [ \nabla^2 \hat{B}(\xvec) \hat{A} ]
	= \nabla^2 \mathcal{W} [\hat{B}(\xvec) \hat{A}].
\]
Let us now expand the commutator and apply correspondences from \thmref{formalism:func-wigner:correspondences}:
\begin{equation*}
\begin{split}
	\mathcal{W} \left[
		\int d\xvec [\Psiop^\dagger(\xvec) \nabla^2 \Psiop(\xvec), \hat{A}]
	\right]
	& = \mathcal{W} \left[
			\int d\xvec \Psiop^\dagger(\xvec) \nabla^2 \Psiop(\xvec) \hat{A}
		\right]
		- \mathcal{W} \left[
				\int d\xvec \hat{A} \Psiop^\dagger(\xvec) \nabla^2 \Psiop(\xvec)
		\right] \\
	& = \int d\xvec \left(
			\Psi^* - \frac{1}{2} \frac{\delta}{\delta \Psi}
		\right)
		\left(
			\nabla^2 \Psi + \frac{1}{2} \nabla^2 \frac{\delta}{\delta \Psi^*}
		\right)
		\mathcal{W}[\hat{A}] \\
	& - \int d\xvec \left(
			\nabla^2 \Psi - \frac{1}{2} \nabla^2 \frac{\delta}{\delta \Psi^*}
		\right)
		\left(
			\Psi^* + \frac{1}{2} \frac{\delta}{\delta \Psi}
		\right)
		\mathcal{W}[\hat{A}] \\
	& = \int d\xvec \left(
			\Psi^* \nabla^2 \Psi
			- \frac{1}{2} \frac{\delta}{\delta \Psi} \nabla^2 \Psi
			+ \frac{1}{2} \Psi^* \nabla^2 \frac{\delta}{\delta \Psi^*}
			- \frac{1}{4} \frac{\delta}{\delta \Psi} \nabla^2 \frac{\delta}{\delta \Psi^*}
		\right)
		\mathcal{W}[\hat{A}] \\
	& - \int d\xvec \left(
			\Psi^* \nabla^2 \Psi
			- \frac{1}{2} \left( \nabla^2 \frac{\delta}{\delta \Psi^*} \right) \Psi^*
			+ \frac{1}{2} \left( \nabla^2 \Psi \right) \frac{\delta}{\delta \Psi}
			- \frac{1}{4} \left(
				\nabla^2 \frac{\delta}{\delta \Psi^*}
			\right) \frac{\delta}{\delta \Psi}
		\right)
		\mathcal{W}[\hat{A}] \\
	& = \frac{1}{2} \int d\xvec \left(
			- \frac{\delta}{\delta \Psi} \nabla^2 \Psi
			+ \Psi^* \nabla^2 \frac{\delta}{\delta \Psi^*}
			+ \left( \nabla^2 \frac{\delta}{\delta \Psi^*} \right) \Psi^*
			- \left( \nabla^2 \Psi \right) \frac{\delta}{\delta \Psi}
		\right)
		\mathcal{W}[\hat{A}]
	= (*)
\end{split}
\end{equation*}

Using basis expansion, one can easily see that
\[
	\Psi^* \nabla^2 \frac{\delta}{\delta \Psi^*} \mathcal{F}[\Psi, \Psi^*]
	= \left( \nabla^2 \frac{\delta}{\delta \Psi^*} \right) \Psi^* \mathcal{F}[\Psi, \Psi^*]
	- \sum_{\nvec \in L} \phi_{\nvec}^* \nabla^2 \phi_{\nvec} \mathcal{F}[\Psi, \Psi^*]
\]
and
\[
	\left( \nabla^2 \Psi \right) \frac{\delta}{\delta \Psi} \mathcal{F}[\Psi, \Psi^*]
	= \frac{\delta}{\delta \Psi} \left( \nabla^2 \Psi \right) \mathcal{F}[\Psi, \Psi^*]
	- \sum_{\nvec \in L} \phi_{\nvec}^* \nabla^2 \phi_{\nvec} \mathcal{F}[\Psi, \Psi^*].
\]
Therefore:
\begin{equation*}
\begin{split}
	(*)
	= \frac{1}{2} \int d\xvec \left(
		- \frac{\delta}{\delta \Psi} \nabla^2 \Psi
		+ \left( \nabla^2 \frac{\delta}{\delta \Psi^*} \right) \Psi^*
		+ \left( \nabla^2 \frac{\delta}{\delta \Psi^*} \right) \Psi^*
		- \frac{\delta}{\delta \Psi} \nabla^2 \Psi
	\right)
	\mathcal{W}[\hat{A}] = (*)
\end{split}
\end{equation*}
Now using \lmmref{formalism:func-calculus:move-laplacian} we can get the final result:
\begin{equation*}
\begin{split}
	(*)
	& = \frac{1}{2} \int d\xvec \left(
		- \frac{\delta}{\delta \Psi} \nabla^2 \Psi
		+ \frac{\delta}{\delta \Psi^*} \nabla^2 \Psi^*
		+ \frac{\delta}{\delta \Psi^*} \nabla^2 \Psi^*
		- \frac{\delta}{\delta \Psi} \nabla^2 \Psi
	\right)
	\mathcal{W}[\hat{A}] \\
	& = \int d\xvec \left(
		- \frac{\delta}{\delta \Psi} \nabla^2 \Psi
		+ \frac{\delta}{\delta \Psi^*} \nabla^2 \Psi^*
	\right) \mathcal{W}[\hat{A}].
	\qedhere
\end{split}
\end{equation*}
\end{proof}

\begin{theorem}
\begin{equation*}
\begin{split}
	\mathcal{W} \left[
		[
			\int d\xvec d\xvec^\prime
			\Psiop_j^\dagger \Psiop_k^{\prime\dagger} \Psiop_j^\prime \Psiop_k,
			\hat{A}
		]
	\right]
	& = \int d\xvec d\xvec^\prime \left(
		\frac{\delta}{\delta \Psi_j} \left(
			- \Psi_j^\prime \Psi_k \Psi_k^{\prime*}
			+ \frac{1}{2} \delta_{jk} \delta_P(\xvec^\prime, \xvec^\prime) \Psi_k
			+ \frac{1}{2} \delta_P(\xvec, \xvec^\prime) \Psi_j^\prime
		\right) \right . \\
	&	\left. + \frac{\delta}{\delta \Psi_j^{\prime*}} \left(
			\Psi_j^* \Psi_k \Psi_k^{\prime*}
			- \frac{1}{2} \delta_{jk} \delta_P(\xvec, \xvec) \Psi_k^{\prime*}
			- \frac{1}{2} \delta_P(\xvec, \xvec^\prime) \Psi_j^*
		\right) \right. \\
	&	\left. + \frac{\delta}{\delta \Psi_k^\prime} \left(
			- \Psi_j^\prime \Psi_j^* \Psi_k
			+ \frac{1}{2} \delta_{jk} \delta_P(\xvec, \xvec) \Psi_j^\prime
			+ \frac{1}{2} \delta_P(\xvec^\prime, \xvec) \Psi_k
		\right) \right .\\
	&	\left. + \frac{\delta}{\delta \Psi_k^*} \left(
			\Psi_j^\prime \Psi_j^* \Psi_k^{\prime*}
			- \frac{1}{2} \delta_{jk} \delta_P(\xvec^\prime, \xvec^\prime) \Psi_j^*
			- \frac{1}{2} \delta_P(\xvec^\prime, \xvec) \Psi_k^{\prime*}
		\right) \right. \\
	&	\left.
			+ \frac{\delta}{\delta \Psi_j}
			\frac{\delta}{\delta \Psi_j^{\prime*}}
			\frac{\delta}{\delta \Psi_k^\prime}
			\frac{1}{4} \Psi_k
			- \frac{\delta}{\delta \Psi_j}
			\frac{\delta}{\delta \Psi_j^{\prime*}}
			\frac{\delta}{\delta \Psi_k^*}
			\frac{1}{4} \Psi_k^{\prime*}
		\right. \\
	&	\left.
			+ \frac{\delta}{\delta \Psi_k^\prime}
			\frac{\delta}{\delta \Psi_k^*}
			\frac{\delta}{\delta \Psi_j}
			\frac{1}{4} \Psi_j^\prime
			- \frac{\delta}{\delta \Psi_k^\prime}
			\frac{\delta}{\delta \Psi_k^*}
			\frac{\delta}{\delta \Psi_j^{\prime*}}
			\frac{1}{4} \Psi_j^*
	\right) \mathcal{W}[\hat{A}].
\end{split}
\end{equation*}
\end{theorem}
\begin{proof}
Proof is the same as in case of \thmref{formalism:transformations:w-commutator1}.
\end{proof}

\begin{lemma}
\[
	\Psi(\xvec)^a \left( \frac{\delta}{\delta \Psi(\xvec)} \right)^b \mathcal{F}[\Psi, \Psi^*]
	= \sum_{j=0}^{\min(a, b)}
		\binom{b}{j} \frac{(-1)^j a!}{(a - j)!}
		\delta(\xvec, \xvec)^j
		\left( \frac{\delta}{\delta \Psi(\xvec)} \right)^{b - j}
		\Psi(\xvec)^{a - j}
		\mathcal{F}[\Psi, \Psi^*]
\]
\end{lemma}
\begin{proof}
Proof by induction.
Let us assume that the statement is true for $b - 1$, and prove it for $b$
(also assuming non-trivial case of $a > 0$).
Moving a single differential to the left:
\[
	\Psi^a \left( \frac{\delta}{\delta \Psi} \right)^b \mathcal{F}
	= \left(
			\frac{\delta}{\delta \Psi} \Psi^a
			- a \Psi^{a - 1} \delta_P(\xvec, \xvec)
		\right)
		\left( \frac{\delta}{\delta \Psi} \right)^{b-1}
		\mathcal{F}
\]
Using known relation for $b-1$:
\begin{equation*}
\begin{split}
	& = \frac{\delta}{\delta \Psi} \sum_{j = 0}^{\min(a, b-1)}
			\binom{b-1}{j} \frac{(-1)^j a!}{(a-j)!} \delta_P(\xvec, \xvec)^j
			\left( \frac{\delta}{\delta \Psi} \right)^{b-1-j} \Psi^{a-j}
			\mathcal{F} \\
	& - a \delta_P(\xvec, \xvec) \sum_{j = 0}^{\min(a-1, b-1)}
			\binom{b-1}{j} \frac{(-1)^j (a-1)!}{(a-1-j)!} \delta_P(\xvec, \xvec)^j
			\left( \frac{\delta}{\delta \Psi} \right)^{b-1-j} \Psi^{a-1-j}
			\mathcal{F}
\end{split}
\end{equation*}
Merging coefficients before sums into the internal expressions:
\begin{equation*}
\begin{split}
	& = \sum_{j = 0}^{\min(a, b-1)}
			\binom{b-1}{j} \frac{(-1)^j a!}{(a-j)!} \delta_P(\xvec, \xvec)^j
			\left( \frac{\delta}{\delta \Psi} \right)^{b-j} \Psi^{a-j}
			\mathcal{F} \\
	& + \sum_{j = 0}^{\min(a-1, b-1)}
			\binom{b-1}{j} \frac{(-1)^{j+1} a!}{(a-1-j)!} \delta_P(\xvec, \xvec)^{j+1}
			\left( \frac{\delta}{\delta \Psi} \right)^{b-1-j} \Psi^{a-1-j}
			\mathcal{F}
\end{split}
\end{equation*}
Shifting counter in the second sum:
\begin{equation*}
\begin{split}
	& = \sum_{j = 0}^{\min(a, b-1)}
			\binom{b-1}{j} \frac{(-1)^j a!}{(a-j)!} \delta_P(\xvec, \xvec)^j
			\left( \frac{\delta}{\delta \Psi} \right)^{b-j} \Psi^{a-j}
			\mathcal{F} \\
	& + \sum_{j = 1}^{\min(a, b)}
			\binom{b-1}{j-1} \frac{(-1)^j a!}{(a-j)!} \delta_P(\xvec, \xvec)^j
			\left( \frac{\delta}{\delta \Psi} \right)^{b-j} \Psi^{a-j}
			\mathcal{F}
\end{split}
\end{equation*}
Now we can join sums, noticing that $\binom{b-1}{j} + \binom{b-1}{j-1} = \binom{b}{j}$.
There will be at most two leftover terms: first, term for $j=0$ from the first sum,
and, possibly, the term with $j=\min(a,b)$ from the second sum.
The former term appears only if $\min(a,b) > \min(a, b-1)$,
or, in other words, $a \ge b$ (which means that $\min(a, b) = b$ and $\min(a, b-1) = b-1$).
\begin{equation*}
\begin{split}
	& = \binom{b-1}{0} \frac{(-1)^0 a!}{(a-0)!} \delta_P(\xvec, \xvec)^0
			\left( \frac{\delta}{\delta \Psi} \right)^{b-0} \Psi^{a-0}
			\mathcal{F} \\
	& + \sum_{j = 1}^{\min(a, b-1)}
			\binom{b}{j} \frac{(-1)^j a!}{(a-j)!} \delta_P(\xvec, \xvec)^j
			\left( \frac{\delta}{\delta \Psi} \right)^{b-j} \Psi^{a-j}
			\mathcal{F} \\
	& + H[a - b]
			\binom{b-1}{b-1} \frac{(-1)^j a!}{(a-b)!} \delta_P(\xvec, \xvec)^b
			\left( \frac{\delta}{\delta \Psi} \right)^{b-b} \Psi^{a-b}
			\mathcal{F},
\end{split}
\end{equation*}
Where $H[n]$ is the discrete Heaviside step function.
Now, since $\binom{b-1}{0} = \binom{b}{0}$ and $\binom{b-1}{b-1} = \binom{b}{b}$,
we can attach two leftover terms to the sum too, obtaining the statement of the lemma:
\[
	= \sum_{j = 0}^{\min(a, b)}
		\binom{b}{j} \frac{(-1)^j a!}{(a-j)!} \delta_P(\xvec, \xvec)^j
		\left( \frac{\delta}{\delta \Psi} \right)^{b-j} \Psi^{a-j}
		\mathcal{F}
	\qedhere
\]
\end{proof}

\begin{lemma}[Sum rearrangement]
For integer $l$, $u$:
\[
	\sum_{j=0}^l \sum_{k=0}^{\min(l-u,j)} x^{j-k} Q_{jk}
	= \sum_{v=0}^l x^v \sum_{k=0}^{l-\max(u,v)} Q_{(v+k)k}.
\]
\end{lemma}
\begin{proof}
\todo{Add picture of summation area for axes $j, k$?}
Obviously, the order $v = j - k$ of factor $x$ can vary from $0$ (say, when $j=0$ and $k=0$) to $l$ (when $j=l$ and $k=0$).
Therefore:
\[
	\sum_{j=0}^l \sum_{k=0}^{\min(l-u,j)} x^{j-k} Q_{jk}
	= \sum_{v=0}^l x^v \sum_{k \in K(l, u, v)} Q_{(v+k)k},
\]
where the set $K$ is defined as
\begin{equation*}
\begin{split}
	K(l, u, v)
	& = \{k |
		0 \le j \le l
		\wedge 0 \le k \le \min(l - u, j)
		\wedge j - k = v
	\} \\
	& = \{k |
		-v \le k \le l - v
		\wedge 0 \le k \le \min(l - u, v + k)
	\} \\
	& = \{k |
		k \le l - v
		\wedge 0 \le k \le \min(l - u, v + k)
	\}.
\end{split}
\end{equation*}

It is convenient to consider two cases separately $v \le u$ and $v > u$.
For the former case
\[
	K_{v \le u}
	= \{k |
		k \le l - v
		\wedge 0 \le k \le \min(l - u, k + v)
		\wedge v \le u
	\}.
\]
Since $v \le u$, $k \le l - v \le l - u \le \min(l - u, k + v)$ is always true,
the first inequation is redundant:
\[
	= \{k |
		0 \le k \le \min(l - u, v + k)
		\wedge v \le u
	\}.
\]
Splitting into two sets to get rid of minimum function:
\begin{equation*}
\begin{split}
	& = \{k |
		v \le u \wedge k \ge 0
		\wedge
		(
			(k \le l - u \wedge l - u < v + k)
			\vee
			(k \le v + k \wedge l - u \ge v + k)
		)
	\} \\
	& = \{k |
		v \le u \wedge k \ge 0
		\wedge
		(
			(k \le l - u \wedge k > l - u - v)
			\vee
			(k \le l - u - v)
		)
	\} \\
	& = \{k |
		v \le u \wedge k \ge 0
		\wedge
		(k \le l - u)
	\} \\
	& = \{k | v \le u \wedge 0 \le k \le l - u \}.
\end{split}
\end{equation*}

For the latter case:
\begin{equation*}
\begin{split}
	K_{v > u}
	& = \{k |
		k \le l - v
		\wedge 0 \le k \le \min(l - u, k + v)
		\wedge v > u
	\} \\
	& = \{k |
		v > u \wedge k \ge 0
		\wedge
		(
			(k \le l - v \wedge k \le l - u \wedge l - u \le k + v)
			\vee
			(k \le l - v \wedge k \le k + v \wedge l - u > k + v)
		)
	\} \\
	& = \{k |
		v > u \wedge k \ge 0
		\wedge
		(
			(k \le l - v \wedge k \ge l - u - v)
			\vee
			(k \le l - v \wedge k < l - u - v)
		)
	\} \\
	& = \{k | v > u \wedge 0 \le k \le l - v \}.
\end{split}
\end{equation*}
Thus
\begin{equation*}
\begin{split}
	K
	& = K_{v \le u} \cup K_{v > u} \\
	& = \{k | v \le u \wedge 0 \le k \le l - u \} \cup \{k | v > u \wedge 0 \le k \le l - v \} \\
	& = \{k | 0 \le k \le l - \max(u, v) \},
\end{split}
\end{equation*}
which gives us the statement of the lemma.
\end{proof}

\begin{theorem}
\[
	\mathcal{W} \left[
		\int d\xvec \mathcal{L}_{\lvec} [\hat{A}]
	\right] = \todo{?} \mathcal{W}[\hat{A}],
\]
where
\[
	\mathcal{L}_{\lvec} [\hat{A}]
	= 2 \hat{O}_{\lvec} \hat{A} \hat{O}_{\lvec}^\dagger
		- \hat{O}_{\lvec}^\dagger \hat{O}_{\lvec} \hat{A}
		- \hat{A} \hat{O}_{\lvec}^\dagger \hat{O}_{\lvec},
\]
and
\[
	\hat{O}_{\lvec}
	\equiv \hat{O}_{\lvec} (\Psiopvec)
	= \Psiop_{l_{1}} (\xvec) \Psiop_{l_{2}} (\xvec) \ldots \Psiop_{l_{n}} (\xvec).
\]
\end{theorem}
\begin{proof}
\todo{Need general formula.
I know how it looks for truncated part, but is there something for all terms?}
\end{proof}