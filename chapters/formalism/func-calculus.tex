% =============================================================================
\section{Functional calculus}
% =============================================================================

Phase-space treatment of multimode problems can be simplified by working with multimode field operators instead of single-mode operators.
It was initially introduced by Graham~\cite{Graham1970,Graham1970a}.
Examples of usage can be found in ~\cite{Steel1998,Norrie2006a}.
Detailed description of functional calculus is given in~\cite{Dalton2011} \todo{anywhere else?}.
Here we only provide some important results which are going to be used later on in this chapter.
\todo{Move to bibliography review and extend.}

First we must introduce some operations on functions,
which will replace common differentials and integrals used in single and multi-mode cases and help encapsulate basis and mode populations inside wave functions and field operators.
In order to do that, we define an orthonormal basis $\phi_{\nvec}$,
where $\nvec$ is a state vector with $D$ elements.
Orthonormality and completeness conditions for basis functions are, respectively,
\[
	\int\limits_A \phi_{\nvec}^*(\xvec) \phi_{\mvec}(\xvec) d\xvec = \delta_{\nvec\mvec},
\]
\[
	\sum_{\nvec} \phi_{\nvec}^*(\xvec) \phi_{\nvec}(\xvec^\prime) = \delta(\xvec^\prime - \xvec),
\]
where the exact nature of integration area $A$ depends on the nature of the basis set
(for example, $A$ is the whole space for harmonic oscillator modes, and a box for plane waves).
Hereinafter we assume that the integration $\int d\xvec$ is always performed over $A$.

Given basis, we can define composition transformation
\[
	\mathcal{C} :: \mathbb{C}^{|L|} \rightarrow (\mathbb{R}^D \rightarrow \mathbb{C})_L
\]
\[
	\mathcal{C}(\bm{\alpha}) = \sum_{\nvec \in L} \phi_{\nvec} \alpha_{\nvec},
\]
where $L$ is some subset of the basis,
and $|L|$ is its cardinality.
Its result is a complex-valued function, which consists only of modes from $L$.
Decomposition transformation is, in turn
\[
	\mathcal{C}^{-1} :: (\mathbb{R}^D \rightarrow \mathbb{C})_L \rightarrow \mathbb{C}^{|L|}
\]
\[
	\mathcal{C}^{-1}[f]_m = \int d\xvec \phi_m^*(\xvec) f(\xvec),\,m \in L
\]
Any function can be projected to subset $L$ using the projection transformation
\[
	\mathcal{P} ::
	(\mathbb{R}^D \rightarrow \mathbb{C}) \rightarrow (\mathbb{R}^D \rightarrow \mathbb{C})_L
\]
\begin{equation}
\label{eqn:formalism:func-calculus:projector}
	\mathcal{P}[f](\xvec)
	= \sum_{\nvec \in L} \phi_{\nvec} (\xvec) \int
		d\xvec^\prime\, \phi_{\nvec}^*(\xvec^\prime) f(\xvec^\prime),
\end{equation}
If $L$ is the whole basis, then, apparently, $\mathcal{P} \equiv \mathds{1}$.

If not explicitly stated otherwise, all functions of $\xvec$ are assumed to belong to basis subset $L$ (restricted basis),
that is $\mathcal{P}[f] \equiv f$, or $f$ has type $(\mathbb{R}^D \rightarrow \mathbb{C})_L$.
Note that the result of any non-linear transformation of a function is not guaranteed to belong to $L$ and requires explicit projection to be used with other restricted functions.
This applies to the delta function which depends on coordinates.
To avoid confusion with delta function of real or complex number,
the restricted delta function is written as $\delta_P$ and defined as
\begin{equation}
\label{eqn:formalism:func-calculus:restricted-delta}
	\delta_P(\xvec^\prime - \xvec)
	= \sum_{\nvec \in L} \phi_{\nvec}^* (\xvec^\prime) \phi_{\nvec} (\xvec).
\end{equation}
Apparently, restricted delta belongs to required type $(\mathbb{R}^D \rightarrow \mathbb{C})_L$.
Note that $\delta_P$ is a Hermitian function: $\delta_P^*(\xvec^\prime - \xvec) = \delta_P(\xvec - \xvec^\prime)$.

Restricted delta function can be used to rewrite equation for $\mathcal{P}$:
\[
	\mathcal{P}[f](\xvec) = \int d\xvec^\prime \delta_P(\xvec^\prime - \xvec) f(\xvec^\prime).
\]
The conjugate of $\mathcal{P}$ is thus defined as
\[
	(\mathcal{P}[f])^*(\xvec)
	= \int d\xvec^\prime \delta_P^*(\xvec^\prime - \xvec) f^*(\xvec^\prime)
	= \mathcal{P}^* [f^*](\xvec).
\]

Let $\mathcal{F}[f] :: (\mathbb{R}^D \rightarrow \mathbb{C})_L \rightarrow (\mathbb{R}^D \rightarrow \mathbb{C})$ be some transformation
(note that the result is not guaranteed to belong to restricted basis).
Because values of types $(\mathbb{R}^D \rightarrow \mathbb{C})_L$ and $\mathbb{C}^{|L|}$ are interchangeable,
$\mathcal{F}$ can be alternatively treated as a function of a vector of complex numbers:
\[
	\mathcal{F} :: \mathbb{C}^{|L|} \rightarrow \mathbb{C}^\infty
\]
\[
	\mathcal{F}(\bm{\alpha}_f) \equiv \mathcal{C}^{-1}[\mathcal{F}[\mathcal{C}(\bm{\alpha}_f)]],
\]
where the subscript $f$ after the vector means that this is the vector which specifies function $f$.

Functional derivative is defined as
\[
	\frac{\delta}{\delta f(\xvec^\prime)} ::
	\left(
		(\mathbb{R}^D \rightarrow \mathbb{C})_L
		\rightarrow
		(\mathbb{R}^D \rightarrow \mathbb{C})
	\right)
	\rightarrow
	\left(
		(\mathbb{R}^D \rightarrow \mathbb{C})_L
		\rightarrow
		(\mathbb{R}^D \rightarrow \mathbb{R}^D \rightarrow \mathbb{C})
	\right)
\]
\begin{equation}
\label{eqn:formalism:func-aux:func-diff}
	\frac{\delta \mathcal{F}[f]}{\delta f(\xvec^\prime)}
	= \sum_{\nvec \in L} \phi_{\nvec}^* (\xvec^\prime)
		\frac{\partial \mathcal{F}(\bm{\alpha}_f)}{\partial \alpha_{f,\nvec}}.
\end{equation}
Note that the transformation being returned differs from the one which was taken:
the result of new transformation is a function depending on two variables from $\mathbb{R}^D$, not one.
The second variable comes from the function we are differentiating by.

Functional derivative definition behaves in many ways similar to common derivative.
\begin{lemma}
Functional differentiation~\eqnref{formalism:func-aux:func-diff} obeys sum, product, quotient, and chain differentiation rules.
\end{lemma}
\begin{proof}
\todo{Sum, product and quotient are more or less obvious; but should we prove chain differentiation?}
\end{proof}

\begin{lemma}
If $g(z)$ is a function that can be expanded into power series,
and functional $\mathcal{F}[f] \equiv g(f)$, then
\[
	\frac{\delta \mathcal{F}[f]}{\delta f(\xvec^\prime)} (\xvec)
	= \delta_P(\xvec^\prime - \xvec)
		\left. \frac{\partial g(z)}{\partial z} \right|_{z = f(\xvec)}
\]
\end{lemma}
\begin{proof}
We will consider $g(z) = z^k$ case first, which will straightforwardly lead to the statement of the lemma.
For $k = 1$, obviously,
\[
	\frac{\delta f}{\delta f(\xvec^\prime)} (\xvec)
	= \delta_P(\xvec^\prime - \xvec)
\]
Then for other values of $k$:
\begin{equation*}
\begin{split}
	\frac{\delta \mathcal{F}[f]}{\delta f(\xvec^\prime)} (\xvec)
	& = \frac{\delta f^k}{\delta f(\xvec^\prime)} (\xvec)
	= \sum_{\nvec \in L} \phi_{\nvec}^{\prime*}
		\frac{\partial f^k}{\partial \alpha_{\nvec}} \\
	& = \sum_{\nvec \in L} \phi_{\nvec}^{\prime*}
		\frac{\partial f^k}{\partial f}
		\frac{\partial f}{\partial \alpha_{\nvec}}
	= k f^{k-1}
		\sum_{\nvec \in L} \phi_{\nvec}^{\prime*}
		\frac{\partial f}{\partial \alpha_{\nvec}} \\
	& = k \delta_P(\xvec^\prime - \xvec) f^{k-1}(\xvec)
	= \delta_P(\xvec^\prime - \xvec)
		\left. \frac{\partial z^k}{\partial z} \right|_{z = f(\xvec)}.
	\qedhere
\end{split}
\end{equation*}
\end{proof}

\begin{lemma}
If $g(z)$ can be expanded into series of $z^n (z^*)^m$,
and functional $\mathcal{F}[f, f^*] \equiv g(f, f^*)$,
then $\delta \mathcal{F} / \delta f^\prime$ and $\delta \mathcal{F} / \delta f^{\prime*}$ can be treated as partial differentiation of the functional of two independent variables $f$ and $f^*$.
In other words:
\[
	\frac{\delta \mathcal{F}}{\delta f^\prime}
	= \delta_P(\xvec^\prime - \xvec) \left.
		\frac{\partial g(z, z^*)}{\partial z}
	\right|_{z=f(x)},
	\quad
	\frac{\delta \mathcal{F}}{\delta f^{\prime*}}
	= \delta_P^*(\xvec^\prime - \xvec) \left.
		\frac{\partial g(z, z^*)}{\partial z^*}
	\right|_{z=f^*(x)}
\]
\end{lemma}
\begin{proof}
Proof is similar to \lmmref{formalism:c-numbers:independent-vars}.
\end{proof}

Functional integration is defined as
\[
	\int \delta f ::
	(\mathbb{R}^D \rightarrow \mathbb{C})_L	\rightarrow \mathbb{C}
\]
\[
	\int \delta^2 f \mathcal{F}[f]
	= \int d^2\bm{\alpha}_f \mathcal{F}(\bm{\alpha}_f)
	= \int \ldots \int d^2\alpha_{f,1} \ldots d^2\alpha_{f,N} \mathcal{F}(\bm{\alpha}_f).
\]
If the basis contains infinite number of modes, the integral is treated as a limit $N \rightarrow \infty$ \todo{\cite{Dalton2011} has detailed explanation, do we need it here?}.

We will need delta functional:
\[
	\Delta[\Lambda]
	\equiv \prod_{\nvec \in L} \delta(\Real \lambda_{\nvec}) \delta(\Imag \lambda_{\nvec}).
\]
It has the same property as common delta function:
\[
	\int \delta^2 \Lambda \mathcal{F}[\Lambda] \Delta[\Lambda]
	= \int \ldots \int d^2\lambda_1 \ldots d^2\lambda_N \mathcal{F}(\bm{\lambda})
		\prod_{\nvec \in L} \delta(\Real \lambda_{\nvec}) \delta(\Imag \lambda_{\nvec})
	= \left. \mathcal{F}(\bm{\lambda}) \right|_{\forall \nvec\, \lambda_{\nvec} = 0}
	= \left. \mathcal{F}[\Lambda] \right|_{\Lambda \equiv 0}
\]

\begin{lemma}[Functional extension of \lmmref{formalism:c-numbers:fourier-of-moments}]
\label{lmm:formalism:func-calculus:fourier-of-moments}
If $\Psi$ and $\Lambda$ are complex-valued functions of coordinate $\xvec$,
then for any non-negative integers $r$ and $s$:
\[
	\int \delta^2\Psi\, \Psi^r (\Psi^*)^s \exp
		\int d\xvec \left( -\Lambda \Psi^* + \Lambda^* \Psi \right)
	= \pi^{2N}
		\left( -\frac{\delta}{\delta \Lambda^*} \right)^r
		\left( \frac{\delta}{\delta \Lambda} \right)^s
		\Delta[\Lambda]
\]
\end{lemma}
\begin{proof}
\begin{equation*}
\begin{split}
	& \int \delta^2\Psi\, \Psi^r (\Psi^*)^s \exp
		\int d\xvec \left( -\Lambda \Psi^* + \Lambda^* \Psi \right) \\
	& = \int \ldots \int d^2\alpha_1 \ldots d^2\alpha_N
		\left( \sum_{\nvec \in L} \phi_{\nvec} \alpha_{\nvec} \right)^r
		\left( \sum_{\nvec \in L} \phi^*_{\nvec} \alpha_{\nvec}^* \right)^s
		\prod_{\nvec \in L} \exp(-\lambda_{\nvec} \alpha_{\nvec}^* + \lambda_{\nvec}^* \alpha_{\nvec}) \\
	& = \int \ldots \int d^2\alpha_1 \ldots d^2\alpha_N
		\sum_{u_1 + \ldots + u_N = r} \binom{r}{u_1, \ldots, u_N}
			\prod_{\nvec \in L} \phi_{\nvec}^{u_{\nvec}} \alpha_{\nvec}^{u_{\nvec}} \\
	&	\sum_{v_1 + \ldots + v_N = s} \binom{s}{v_1, \ldots, v_N}
			\prod_{\nvec \in L} (\phi_{\nvec}^*)^{v_{\nvec}} (\alpha_{\nvec}^*)^{v_{\nvec}}
		\prod_{\nvec \in L} \exp(-\lambda_{\nvec} \alpha_{\nvec}^* + \lambda_{\nvec}^* \alpha_{\nvec}) \\
	& = \sum_{u_1 + \ldots + u_N = r}
		\sum_{v_1 + \ldots + v_N = s}
		\binom{r}{u_1, \ldots, u_N}
		\binom{s}{v_1, \ldots, v_N}
		\prod_{\nvec \in L}
			\phi_{\nvec}^{u_{\nvec}} (\phi_{\nvec}^*){v_{\nvec}}
			\int d^2\alpha_{\nvec}
				\alpha_{\nvec}^{u_{\nvec}}
				(\alpha_{\nvec}^*)^{v_{\nvec}}
				\exp(-\lambda_{\nvec} \alpha_{\nvec}^* + \lambda_{\nvec}^* \alpha_{\nvec}) \\
	& = \sum_{u_1 + \ldots + u_N = r}
		\sum_{v_1 + \ldots + v_N = s}
		\binom{r}{u_1, \ldots, u_N}
		\binom{s}{v_1, \ldots, v_N}
		\pi^{2N}
		\prod_{\nvec \in L}
			\phi_{\nvec}^{u_{\nvec}} (\phi_{\nvec}^*)^{v_{\nvec}}
			\left( -\frac{\partial}{\partial \lambda_{\nvec}^*} \right)^{u_{\nvec}}
			\left( \frac{\partial}{\partial \lambda_{\nvec}} \right)^{v_{\nvec}}
			\delta(\Real \lambda_{\nvec}) \delta(\Imag \lambda_{\nvec}) \\
	& = \pi^{2N}
		(-\sum_{\nvec \in L} \phi_{\nvec} \frac{\partial}{\partial \lambda_{\nvec}^*})^r
		(\sum_{\nvec \in L} \phi_{\nvec}^* \frac{\partial}{\partial \lambda_{\nvec}})^s
		\prod_{\nvec \in L} \delta(\Real \lambda_{\nvec}) \delta(\Imag \lambda_{\nvec}) \\
	& = \pi^{2N}
		\left( -\frac{\delta}{\delta \Lambda^*} \right)^r
		\left( \frac{\delta}{\delta \Lambda} \right)^s
		\Delta[\Lambda]
	\qedhere
\end{split}
\end{equation*}
\end{proof}
