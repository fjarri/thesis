% =============================================================================
\section{Relaxed operations with complex numbers}
% =============================================================================

Formally, a function of complex variable has to be holomorphic in order to be complex differentiable.
In many cases it is enough to have less strict ``physicists'\,'' complex differentiation rule.
If complex variable $z = x + iy$ and function $f(z) = u(x, y) + iv(x, y)$ then
\[
	\left( \frac{df(z)}{dz} \right)_{phys}
	= \frac{1}{2} \left(
		\frac{\partial f}{\partial x} - i \frac{\partial f}{\partial y}
	\right).
\]

\begin{lemma}
If $f(z)$ is holomorphic, then ``physicists'\,'' differentiation is equivalent to the formal one.
\end{lemma}
\begin{proof}
If $f(z)$ is holomorphic, Cauchy-Riemann equations $\partial u / \partial x = \partial v / \partial y$ and $\partial u / \partial y = -\partial v / \partial x$ are satisfied.
Thus
\begin{equation*}
\begin{split}
	\left( \frac{df(z)}{dz} \right)_{phys}
	= \frac{1}{2} \left(
		\frac{\partial f}{\partial x} - i \frac{\partial f}{\partial y}
	\right)
	= \frac{1}{2} \left(
		\frac{\partial u}{\partial x} + \frac{\partial v}{\partial y}
	\right)
	+ \frac{i}{2} \left(
		\frac{\partial v}{\partial x} - \frac{\partial u}{\partial y}
	\right)
	= \frac{\partial u}{\partial x} + i \frac{\partial v}{\partial x}.
\end{split}
\end{equation*}
The last expression being one of the forms of derivative for holomorphic functions.
\end{proof}

\begin{lemma}
(obvious) For any ``good'' (even non-holomorphic) $f(z)$, ``physicists'\,'' differentiation obeys sum, product, quotient, and chain differentiation rules.
\end{lemma}

Hereinafter we will use ``physicists'\,'' differentiation unless explicitly stated otherwise,
because some important functions we will encounter are not holomorphic.
This differentiation has all intuitively assumed properties, along with some not quite obvious ones.

\begin{lemma}
For any nonnegative integers $a$ and $b$.
\[
	\frac{d}{dz} (z^a (z^*)^b) = a z^{a-1} (z^*)^b,
	\quad
	\frac{d}{dz^*} (z^a (z^*)^b) = b z^a (z^*)^{b-1},
\]
\end{lemma}
\begin{proof}
Let us assume that the statement of the lemma is valid for some $a$ and $b$, then using chain rule
\[
	\frac{d}{dz} (z^{a+1} (z^*)^b)
	= \frac{d}{dz} (z z^a (z^*)^b)
	= z^a (z^*)^b + z \frac{d}{dz} (z^a (z^*)^b)
	= z^a (z^*)^b + a z z^{a-1} (z^*)^b
	= (a + 1) z^a (z^*)^b.
\]
The part for $d/dz^*$ can is proved in the same way.
One can easily prove (by transition to real values) that $d(z z^*)/dz = z^*$ and $d(z z^*)/dz^* = z$.
By induction, the statement is true for any natural $a$ and $b$,
and it is obviously true if $a = 0$ or $b = 0$, which proves the lemma.
\todo{This can be proved for any real $a$ and $b$, if necessary.}
\end{proof}

This is straightforwardly followed by
\begin{lemma}
\label{lmm:formalism:c-numbers:independent-vars}
If $f(z)$ can be expanded into series of $z^n (z^*)^m$, $df(z)/dz$ can be treated as partial differentiation of the function of two independent variables $z$ and $z^*$.
In other words:
\[
	\frac{d}{dz} f(z) \equiv \frac{\partial}{\partial z} f(z, z^*),
	\quad
	\frac{d}{dz^*} f(z) \equiv \frac{\partial}{\partial z^*} f(z, z^*).
\]
\end{lemma}

Now we can prove two lemmas which will help us deal with some integrals.

\begin{lemma}
\label{lmm:formalism:c-numbers:fourier-of-moments}
If $\alpha$ and $\lambda$ are complex variables and $\int d^2\alpha$ stands for the integral over the complex plane, then for any non-negative integers $r$ and $s$:
\[
	\int d^2\alpha\, \alpha^r (\alpha^*)^s \exp(-\lambda \alpha^* + \lambda^* \alpha)
	= \pi^2
		\left( -\frac{\partial}{\partial \lambda^*} \right)^r
		\left( \frac{\partial}{\partial \lambda} \right)^s
		\delta(\Real \lambda) \delta(\Imag \lambda)
\]
\end{lemma}
\begin{proof}
First, changing the variables in the integrals and using known Fourier transform relations, we can prove that for real $x$ and $v$, and non-negative integer $n$
\[
	\int\limits_{-\infty}^{\infty} dv\, v^n \exp(\pm 2 i x v)
	= \pi (\mp i / 2)^n \delta^{(n)}(x),
\]
Note that we have explicitly written integration limits here;
they are swapped when we change the variable in the first integral.

Denoting $\alpha = u + iv$ and $\lambda = x + iy$, we can expand the initial expression as
\begin{equation*}
\begin{split}
	\int d^2\alpha\, \alpha^r (\alpha^*)^s \exp(-\lambda \alpha^* + \lambda^* \alpha)
	& = \int du dv \exp(2ivx - 2iuy)
		\sum\limits_{l=0}^r \binom{r}{l} u^l (iv)^{r-l}
		\sum\limits_{m=0}^s \binom{s}{m} u^m (-iv)^{s-m} \\
	& = \sum\limits_{l=0}^r \sum\limits_{m=0}^s \binom{r}{l} \binom{s}{m}
		i^{r-l} (-i)^{s-m}
		\int du\, u^{l+m} \exp(2ivx)
		\int dv\, v^{r-l+s-m} \exp(-2iuy) \\
	& = \pi^2 \sum\limits_{l=0}^r \sum\limits_{m=0}^s \binom{r}{l} \binom{s}{m}
		i^{r-l} (-i)^{s-m}
		(-i/2)^{l+m} \delta^{(l+m)}(y)
		(i/2)^{r-l+s-m} \delta^{(r-l+s-m)}(x) \\
	& = \pi^2
		\sum\limits_{l=0}^r \binom{r}{l}
			\frac{1}{2^r}
			(-i \partial / \partial y)^l
			(-\partial / \partial x)^{r-l}
		\sum\limits_{m=0}^s \binom{s}{m}
			\frac{1}{2^s}
			(-i \partial / \partial y)^m
			(\partial / \partial x)^{s-m}
		\delta(y) \delta(x) \\
	& = \pi^2
		\left( \frac{1}{2} (-i \partial / \partial y - \partial / \partial x) \right)^r
		\left( \frac{1}{2} (-i \partial / \partial y + \partial / \partial x) \right)^s
		\delta(y) \delta(x) \\
	& = \pi^2
		\left( -\frac{\partial}{\partial \lambda^*} \right)^r
		\left( \frac{\partial}{\partial \lambda} \right)^s
		\delta(\Real \lambda) \delta(\Imag \lambda).
		\qedhere
\end{split}
\end{equation*}
\end{proof}

A notable special case of \lmmref{formalism:c-numbers:fourier-of-moments} is
\[
	\int d^2\alpha \exp(-\lambda \alpha^* + \lambda^* \alpha)
	= \pi^2 \delta(\Real \lambda) \delta(\Imag \lambda).
\]

\begin{lemma}
\label{lmm:formalism:c-numbers:zero-integrals}
For any non-negative integers $r$, $s$ and complex $\alpha$.
\begin{equation*}
\begin{split}
	\int d^2\lambda
		\frac{\partial}{\partial \lambda} \left(
			\exp(-\lambda \alpha^* + \lambda^* \alpha)
			\exp(ixy) x^r y^s
		\right)
	& = 0 \\
	\int d^2\lambda
		\frac{\partial}{\partial \lambda^*}
		\left(
			\exp(-\lambda \alpha^* + \lambda^* \alpha)
			\exp(ixy) x^r y^s
		\right)
	& = 0.
\end{split}
\end{equation*}
\end{lemma}
\begin{proof}
We will prove the first equation.
First, note that complex-valued integral of derivative is evaluated as
\begin{equation*}
\begin{split}
	\int d^2\lambda \frac{\partial}{\partial \lambda} f(\lambda, \lambda^*)
	& = \frac{1}{2} \int\limits_{-\infty}^{\infty} dx \int\limits_{-\infty}^{\infty} dy
		\left( \frac{\partial}{\partial x} - i \frac{\partial}{\partial y} \right)
		g(x, y) \\
	& = \frac{1}{2} \int\limits_{-\infty}^{\infty} dy \int\limits_{-\infty}^{\infty} dx
			\frac{\partial}{\partial x} g(x, y)
		- \frac{i}{2} \int\limits_{-\infty}^{\infty} dx \int\limits_{-\infty}^{\infty} dy
			\frac{\partial}{\partial y} g(x, y) \\
	& =	\frac{1}{2} \int\limits_{-\infty}^{\infty} dy \left(
			\left. g(x, y) \right|_{x=-\infty}^{\infty}
		\right)
		- \frac{i}{2} \int\limits_{-\infty}^{\infty} dx \left(
			\left. g(x, y) \right|_{y=-\infty}^{\infty}
		\right),
\end{split}
\end{equation*}
where we expanded $\lambda = x + iy$.
Thus
\begin{equation*}
\begin{split}
	\int d^2\lambda
		\frac{\partial}{\partial \lambda} \left(
			\exp(-\lambda \alpha^* + \lambda^* \alpha)
			\exp(ixy) x^r y^s
		\right)
	& = \frac{1}{2} \int dy \left. \left(
			\exp(2ixv - 2iyu) \exp(ixy) x^r y^s
		\right) \right|_{x = -\infty}^\infty \\
	& - \frac{i}{2} \int dx \left. \left(
			\exp(2ixv - 2iyu) \exp(ixy) x^r y^s
		\right) \right|_{y = -\infty}^\infty \\
	& = \left(
			\frac{1}{2} \exp(2ixv) x^r \int dy \exp(iy(x-2u)) y^s
		\right)_{x = -\infty}^\infty \\
	& - \left(
			\frac{i}{2} \exp(-2ixy) y^s \int dx \exp(ix(y+2v)) x^r
		\right)_{y = -\infty}^\infty \\
	& = \left(
			\frac{1}{2} \exp(2ixv) x^r 2 \pi i^s \delta^{(s)}(x-2u)
		\right)_{x = -\infty}^\infty \\
	& - \left(
			\frac{i}{2} \exp(-2ixy) y^s 2 \pi i^r \delta^{(r)}(y+2v)
		\right)_{y = -\infty}^\infty \\
	& = 0,
\end{split}
\end{equation*}
because any derivative of delta function is zero on the infinity.
\end{proof}
