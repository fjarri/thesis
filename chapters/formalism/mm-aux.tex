% =============================================================================
\section{Sets of single-mode operators}
% =============================================================================

We start from the set of single-mode operators $\hat{a}_j$, which obey bosonic commutation relations:
\begin{equation}
\label{eqn:formalism:mm-aux:commutators}
\begin{split}
	[ \hat{a}_j, \hat{a}_k ] & = [ \hat{a}_j^\dagger, \hat{a}_k^\dagger ] = 0, \\
	[ \hat{a}_j, \hat{a}_k^\dagger ] & = \delta_{jk}.
\end{split}
\end{equation}

In order to work with the moments of multimode operators we will need the equations for commutators of arbitrary single-mode operator products.

\begin{lemma}
\label{lmm:formalism:mm-aux:high-order-commutators}
\begin{equation*}
\begin{split}
	[ \hat{a}_n, \hat{a}_{m_1}^\dagger \ldots \hat{a}_{m_k}^\dagger ]
	& = \sum_{i=1}^k \delta_{n m_i}
		\prod_{j=1,j \ne i}^k \hat{a}_{m_j}^\dagger, \\
	[ \hat{a}_n^\dagger, \hat{a}_{m_1} \ldots \hat{a}_{m_k} ]
	& = - \sum_{i=1}^k \delta_{n m_i}
		\prod_{j=1,j \ne i}^k \hat{a}_{m_j}.
\end{split}
\end{equation*}
\end{lemma}
\begin{proof}
Let us find the expression for the first commutator by induction.
Providing that we know the expression for $[ \hat{a}_n, \hat{a}_{m_1}^\dagger \ldots \hat{a}_{m_{k-1}}^\dagger ]$,
commutator of order $k$ can be expanded as:
\[
	[ \hat{a}_n, \hat{a}_{m_1}^\dagger \ldots \hat{a}_{m_k}^\dagger ]
	= (1 - \delta_{n m_k})
		[ \hat{a}_n, \hat{a}_{m_1}^\dagger \ldots \hat{a}_{m_{k-1}}^\dagger ] \hat{a}_{m_k}
	+ \delta_{n m_k} (
		\hat{a}_n \hat{a}_{m_1}^\dagger \ldots \hat{a}_{m_{k-1}}^\dagger \hat{a}_n^\dagger
		- \hat{a}_{m_1}^\dagger \ldots \hat{a}_{m_{k-1}}^\dagger \hat{a}_n^\dagger \hat{a}_n
	)
	= (*).
\]
Here we have split the initial commutator into two possible outcomes, depending on whether $n = m_k$.
First term, corresponding to $n \ne m_k$, contains the known commutator of lower order.
In the second term we have substituted $\hat{a}_n$ for $\hat{a}_{m_k}$,
since the delta function outside the parentheses ensures that $n = m_k$.
Swapping $\hat{a}_n^\dagger$ and $\hat{a}_n$ in the last term and, again, recognising the known commutator:
\begin{equation*}
\begin{split}
	(*)
	& = (1 - \delta_{n m_k})
		[ \hat{a}_n, \hat{a}_{m_1}^\dagger \ldots \hat{a}_{m_{k-1}}^\dagger ] \hat{a}_{m_k}
	+ \delta_{n m_k} (
		[ \hat{a}_n, \hat{a}_{m_1}^\dagger \ldots \hat{a}_{m_{k-1}}^\dagger ] \hat{a}_n^\dagger
		+ \hat{a}_{m_1}^\dagger \ldots \hat{a}_{m_{k-1}}^\dagger
	) \\
	& = [ \hat{a}_n, \hat{a}_{m_1}^\dagger \ldots \hat{a}_{m_{k-1}}^\dagger ] \hat{a}_{m_k}
	+ \delta_{n m_k} \hat{a}_{m_1}^\dagger \ldots \hat{a}_{m_{k-1}}^\dagger.
\end{split}
\end{equation*}
Now, starting from the first-order relation $[ \hat{a}_n, \hat{a}_{m_1}^\dagger ] = \delta_{n m_1}$, we can obtain the relation for any order:
\begin{equation*}
\begin{split}
	[ \hat{a}_n, \hat{a}_{m_1}^\dagger \hat{a}_{m_2}^\dagger ]
	& = \delta_{n m_1} \hat{a}_{m_2}^\dagger + \delta_{n m_2} \hat{a}_{m_1}^\dagger, \\
	[ \hat{a}_n, \hat{a}_{m_1}^\dagger \hat{a}_{m_2}^\dagger \hat{a}_{m_3}^\dagger ]
	& = \delta_{n m_1} \hat{a}_{m_2}^\dagger \hat{a}_{m_3}^\dagger
	+ \delta_{n m_2} \hat{a}_{m_1}^\dagger \hat{a}_{m_3}^\dagger
	+ \delta_{n m_3} \hat{a}_{m_1}^\dagger \hat{a}_{m_2}^\dagger, \\
	& \ldots
\end{split}
\end{equation*}
Which gives us the statement of the lemma.
\end{proof}

Note that if $n = m_1 = \ldots = m_k$, this boils down to the well-known relation from~\cite{Louisell1990}:
\[
	[ \hat{a}, (\hat{a}^\dagger)^k ] = k (\hat{a}^\dagger)^{k-1}.
\]
