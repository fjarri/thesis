% =============================================================================
\section{Single-mode Wigner representation}
% =============================================================================

We will need the displacement operator which was first introduced by Weyl~\cite{Weyl1950}:
\begin{equation}
\label{eqn:formalism:sm-wigner:dispacement-op}
	\hat{D}(\lambda, \lambda^*) = \exp(\lambda \hat{a}^\dagger - \lambda^* \hat{a}).
\end{equation}
Using Baker-Hausdorff theorem to split non-commuting operators in the exponent,
one can find that
\begin{equation}
\label{eqn:formalism:sm-wigner:displacement-derivatives}
\begin{split}
	\frac{\partial}{\partial \lambda} \hat{D}(\lambda, \lambda^*)
	& = \hat{D}(\lambda, \lambda^*) (\hat{a}^\dagger + \frac{1}{2} \lambda^*)
	= (\hat{a}^\dagger - \frac{1}{2} \lambda^*) \hat{D}(\lambda, \lambda^*), \\
	-\frac{\partial}{\partial \lambda^*} \hat{D}(\lambda, \lambda^*)
	& = \hat{D}(\lambda, \lambda^*) (\hat{a} + \frac{1}{2} \lambda)
	= (\hat{a} - \frac{1}{2} \lambda) \hat{D}(\lambda, \lambda^*).
\end{split}
\end{equation}

In terms of displacement operator Wigner transformation $\mathcal{W}$ is defined as
\begin{equation}
\label{eqn:formalism:sm-wigner:w-transformation}
	\mathcal{W}[\hat{A}]
	= \frac{1}{\pi^2} \int d^2 \lambda \exp(-\lambda \alpha^* + \lambda^* \alpha)
		\Trace{ \hat{A} \hat{D}(\lambda, \lambda^*) }.
\end{equation}
It transforms an operator $\hat{A}$ on a Hilbert space to a function $\mathcal{W}[\hat{A}](\alpha, \alpha^*)$ on phase space.
The backward transformation (called the Weyl transformation) gives back matrix elements of the operator:
\[
	\langle \alpha \lvert \mathcal{W}^{-1}[f(\alpha, \alpha^*)] \rvert \alpha \rangle
	= \todo{find\,the\,expression},
\]
which is enough, because any operator is determined by its expectation in all coherent states~\cite{Gardiner2004}.

Thus Wigner function can be defined as a result of Wigner transformation of the density matrix: $W(\alpha, \alpha^*) = \mathcal{W}[\hat{\rho}]$.
The Wigner function always exists for any density matrix~\cite{Gardiner2004}.
The correspondence $W \leftrightarrow \hat{\rho}$ is a bijection \todo{prove it?}.

In some cases it will be convenient to use Wigner function in form~\cite{Gardiner2004}
\begin{equation}
\label{eqn:formalism:sm-wigner:w-function}
	W (\alpha, \alpha^*)
	= \frac{1}{\pi^2} \int d^2 \lambda \exp(-\lambda \alpha^* + \lambda^* \alpha)
		\chi_W (\lambda, \lambda^*),
\end{equation}
where $\chi_W (\lambda, \lambda^*)$ is the characteristic function
\[
	\chi_W (\lambda, \lambda^*)
	= \Trace{ \hat{\rho} \hat{D}(\lambda, \lambda^*) }.
\]

\begin{lemma}
\label{lmm:formalism:sm-wigner:zero-integrals}
\begin{equation*}
\begin{split}
	\int d^2\lambda
		\frac{\partial}{\partial \lambda} \left(
			\exp(-\lambda \alpha^* + \lambda^* \alpha)
			\left( \frac{\partial}{\partial \lambda} \right)^m
			\left( -\frac{\partial}{\partial \lambda^*} \right)^n
			\hat{D}(\lambda, \lambda^*)
		\right)
	& = 0 \\
	\int d^2\lambda
		\frac{\partial}{\partial \lambda^*}
		\left(
			\exp(-\lambda \alpha^* + \lambda^* \alpha)
			\left( \frac{\partial}{\partial \lambda} \right)^m
			\left( -\frac{\partial}{\partial \lambda^*} \right)^n
			\hat{D}(\lambda, \lambda^*)
		\right)
	& = 0.
\end{split}
\end{equation*}
\end{lemma}
\begin{proof}
We will prove the first equation.
Expanding $\lambda = x + iy$ and applying Baker-Hausdorff theorem:
\begin{equation*}
\begin{split}
	\hat{D}(\lambda, \lambda^*)
	= \exp(x(\hat{a}^\dagger - \hat{a}) + iy(\hat{a}^\dagger + \hat{a}))
	= \exp(ixy) \exp(x(\hat{a}^\dagger - \hat{a})) \exp(iy(\hat{a}^\dagger + \hat{a}))
\end{split}
\end{equation*}
Thus
\begin{equation*}
\begin{split}
	& \int d^2\lambda
		\frac{\partial}{\partial \lambda} \left(
			\exp(-\lambda \alpha^* + \lambda^* \alpha)
			\left( \frac{\partial}{\partial \lambda} \right)^m
			\left( -\frac{\partial}{\partial \lambda^*} \right)^n
			\hat{D}(\lambda, \lambda^*)
		\right) \\
	& = \int d^2\lambda
		\frac{\partial}{\partial \lambda} \left(
			\exp(-\lambda \alpha^* + \lambda^* \alpha)
			\frac{1}{4^{m+n}}
			\left( \frac{\partial}{\partial x} - i \frac{\partial}{\partial y} \right)^m
			\left( -\frac{\partial}{\partial x} - i \frac{\partial}{\partial y} \right)^n
			\exp(ixy)
			\sum\limits_{r=0}^{\infty} \frac{x^r (\hat{a}^\dagger - \hat{a})^r}{r!}
			\sum\limits_{s=0}^{\infty} \frac{(iy)^s (\hat{a}^\dagger + \hat{a})^s}{s!}
		\right) \\
	& = \frac{1}{4^{m+n}} \sum\limits_{r=0}^{\infty} \sum\limits_{s=0}^{\infty} \left(
		\int d^2\lambda
			\frac{\partial}{\partial \lambda} \left(
				\exp(-\lambda \alpha^* + \lambda^* \alpha)
				\left( \frac{\partial}{\partial x} - i \frac{\partial}{\partial y} \right)^m
				\left( -\frac{\partial}{\partial x} - i \frac{\partial}{\partial y} \right)^n
				\exp(ixy)
				x^r (iy)^s
			\right)
		\right)
		\frac{(\hat{a}^\dagger - \hat{a})^r}{r!}
		\frac{(\hat{a}^\dagger + \hat{a})^s}{s!} \\
	& = \sum\limits_{r=0}^{\infty} \sum\limits_{s=0}^{\infty} \left(
		\int d^2\lambda
			\frac{\partial}{\partial \lambda} \left(
				\exp(-\lambda \alpha^* + \lambda^* \alpha)
				\exp(ixy)
				f(x, y)
			\right)
		\right)
		\frac{(\hat{a}^\dagger - \hat{a})^r}{r!}
		\frac{(\hat{a}^\dagger + \hat{a})^s}{s!} \\
	& = 0,
\end{split}
\end{equation*}
where $f(x, y)$ is some finite-order polynomial of $x$ and $y$,
and we used \lmmref{formalism:c-numbers:zero-integrals} to evaluate the integral over $\lambda$.
\end{proof}

\begin{lemma}
\label{lmm:formalism:sm-wigner:moments-from-chi}
\[
	\langle \symprod{ \hat{a}^r (\hat{a}^\dagger)^s } \rangle
	= \left.
		\left( \frac{\partial}{\partial \lambda} \right)^s
		\left( -\frac{\partial}{\partial \lambda^*} \right)^r
		\chi_W (\lambda, \lambda^*)
	\right|_{\lambda=0}.
\]
\end{lemma}
\begin{proof}
The exponent in the $\chi_W$ can be expanded as
\[
	\exp (\lambda \hat{a}^\dagger - \lambda^* \hat{a})
	= \sum\limits_{r,s}
		\frac{(-\lambda^*)^r \lambda^s}{r!s!}
		\symprod{ \hat{a}^r (\hat{a}^\dagger)^s }.
\]
Thus
\[
	\chi_W(\lambda, \lambda^*)
	= \sum\limits_{r,s}
		\frac{(-\lambda^*)^r \lambda^s}{r!s!}
		\Trace{
			\hat{\rho} \symprod{ \hat{a}^r (\hat{a}^\dagger)^s }
		}
	= \sum\limits_{r,s}
		\frac{(-\lambda^*)^r \lambda^s}{r!s!}
		\langle \symprod{ \hat{a}^r (\hat{a}^\dagger)^s } \rangle
\]
Apparently, the application of $(\partial / \partial \lambda)^s$ and $(-\partial / \partial \lambda^*)^r$ will eliminate all lower order moments,
and setting $\lambda = 0$ afterwards will eliminate all higher order moments,
leaving only $\symprod{ \hat{a}^r (\hat{a}^\dagger)^s }$:
\[
	\left.
		\left( \frac{\partial}{\partial \lambda} \right)^s
		\left( -\frac{\partial}{\partial \lambda^*} \right)^r
		\chi_W (\lambda, \lambda^*)
	\right|_{\lambda=0}
	= r! s! \frac{1}{r! s!}
		\langle \symprod{ \hat{a}^r (\hat{a}^\dagger)^s } \rangle
	= \langle \symprod{ \hat{a}^r (\hat{a}^\dagger)^s } \rangle.
	\qedhere
\]
\end{proof}

Now we can get the final relation.
\begin{theorem}
\label{thm:formalism:sm-wigner:moments}
\[
	\int d^2\alpha\, \alpha^r (\alpha^*)^s W(\alpha, \alpha^*)
	= \langle \symprod{ \hat{a}^r (\hat{a}^\dagger)^s } \rangle
\]
\end{theorem}
\begin{proof}
By definition of Wigner function:
\begin{equation*}
\begin{split}
	\int d^2\alpha\, \alpha^r (\alpha^*)^s W(\alpha, \alpha^*)
	= \frac{1}{\pi^2} \Trace{ \hat{\rho}
		\int d^2\alpha\, \alpha^r (\alpha^*)^s
		\int d^2\lambda \exp(-\lambda \alpha^* + \lambda^* \alpha)
		\hat{D}(\lambda, \lambda^*)
	}
\end{split}
\end{equation*}
Integrating by parts and eliminating terms which fit \lmmref{formalism:sm-wigner:zero-integrals}:
\[
	= \frac{1}{\pi^2} \Trace{ \hat{\rho}
		\int d^2\alpha \int d^2\lambda
		\exp(-\lambda \alpha^* + \lambda^* \alpha)
		\left( \frac{\partial}{\partial \lambda} \right)^s
		\left( -\frac{\partial}{\partial \lambda^*} \right)^r
		\hat{D} (\lambda, \lambda^*)
	}
\]
Evaluating integral over $\alpha$ using \lmmref{formalism:c-numbers:fourier-of-moments}:
\begin{equation*}
\begin{split}
	& = \int d^2\lambda\,
		\delta (\Real \lambda) \delta (\Imag \lambda)
		\left( \frac{\partial}{\partial \lambda} \right)^s
		\left( -\frac{\partial}{\partial \lambda^*} \right)^r
		\Trace{
			\hat{\rho}
			\hat{D}(\lambda, \lambda^*)
		} \\
	& = \left.
		\left( \frac{\partial}{\partial \lambda} \right)^s
		\left( -\frac{\partial}{\partial \lambda^*} \right)^r
		\chi_W (\lambda, \lambda^*)
	\right|_{\lambda=0}.
\end{split}
\end{equation*}
Now, recognising the final expression as a part of \lmmref{formalism:sm-wigner:moments-from-chi},
we immideately get the statement of the theorem.
\end{proof}

\begin{theorem}[Operator correspondences]
\label{thm:formalism:sm-wigner:correspondences}
\begin{equation*}
\begin{split}
	\mathcal{W} [ \hat{a} \hat{A} ]
		& = \left( \alpha + \frac{1}{2} \frac{\partial}{\partial \alpha^*} \right) \mathcal{W}[\hat{A}],
	\quad
	\mathcal{W} [ \hat{a}^\dagger \hat{A} ]
		= \left( \alpha^* - \frac{1}{2} \frac{\partial}{\partial \alpha} \right) \mathcal{W}[\hat{A}], \\
	\mathcal{W} [ \hat{A} \hat{a} ]
		& = \left( \alpha - \frac{1}{2} \frac{\partial}{\partial \alpha^*} \right) \mathcal{W}[\hat{A}],
	\quad
	\mathcal{W} [ \hat{A} \hat{a}^\dagger ]
		= \left( \alpha^* + \frac{1}{2} \frac{\partial}{\partial \alpha} \right) \mathcal{W}[\hat{A}].
\end{split}
\end{equation*}
\end{theorem}
\begin{proof}
We will prove the first correspondence.
First, let us transform the trace using~\eqnref{formalism:sm-wigner:displacement-derivatives}:
\begin{equation*}
\begin{split}
	\Trace{ \hat{a} \hat{A} \hat{D} }
	= \Trace{ \hat{A} \hat{D} \hat{a}}
	= \Trace{ \hat{A} \left(
		-\frac{\partial}{\partial \lambda^*}
		-\frac{1}{2} \lambda
	\right) \hat{D}}
	= \left(
		-\frac{\partial}{\partial \lambda^*}
		-\frac{1}{2} \lambda
	\right) \Trace{ \hat{A} \hat{D}}
\end{split}
\end{equation*}
Now we need to somehow move this additional multiplier outside the integral in the expression for Wigner function:
\begin{equation*}
\begin{split}
	\mathcal{W} [ \hat{a} \hat{A} ]
	& = \frac{1}{\pi^2} \int d^2 \lambda \exp(-\lambda \alpha^* + \lambda^* \alpha)
		\Trace{ \hat{a} \hat{A} \hat{D}(\lambda, \lambda^*) } \\
	& = \frac{1}{\pi^2} \int d^2 \lambda \exp(-\lambda \alpha^* + \lambda^* \alpha)
		\left(
			-\frac{\partial}{\partial \lambda^*}
			-\frac{1}{2} \lambda
		\right)
		\Trace{ \hat{A} \hat{D}(\lambda, \lambda^*) } \\
	& = \frac{1}{2} \frac{\partial}{\partial \alpha^*} \mathcal{W} [\hat{A}]
	- \frac{1}{\pi^2} \int d^2 \lambda \exp(-\lambda \alpha^* + \lambda^* \alpha)
		\frac{\partial}{\partial \lambda^*}
		\Trace{ \hat{A} \hat{D}(\lambda, \lambda^*) } \\
	& = \frac{1}{2} \frac{\partial}{\partial \alpha^*} \mathcal{W} [\hat{A}]
	+ \frac{1}{\pi^2} \int d^2 \lambda \left(
		\frac{\partial}{\partial \lambda^*} \exp(-\lambda \alpha^* + \lambda^* \alpha)
	\right)
	\Trace{ \hat{A} \hat{D}(\lambda, \lambda^*) } \\
	& = \left( \alpha + \frac{1}{2} \frac{\partial}{\partial \alpha^*} \right) \mathcal{W} [\hat{A}].
\end{split}
\end{equation*}
Notice that we used~\lmmref{formalism:sm-wigner:zero-integrals} to move the partial derivative over $\lambda^*$.
\end{proof}
