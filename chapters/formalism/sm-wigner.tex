% =============================================================================
\section{Single-mode Wigner representation}
% =============================================================================

We will need the displacement operator which was first introduced by Weyl~\cite{Weyl1950}:
\[
	\hat{D}(\lambda, \lambda^*) = \exp(\lambda \hat{a}^\dagger - \lambda^* \hat{a}).
\]
Using Baker-Hausdorff theorem to split non-commuting operators in the exponent,
one can find that
\begin{equation}
\label{eqn:formalism:sm-wigner:displacement-derivatives}
\begin{split}
	\frac{\partial}{\partial \lambda} \hat{D}(\lambda, \lambda^*)
	& = \hat{D}(\lambda, \lambda^*) (\hat{a}^\dagger + \frac{1}{2} \lambda^*)
	= (\hat{a}^\dagger - \frac{1}{2} \lambda^*) \hat{D}(\lambda, \lambda^*), \\
	-\frac{\partial}{\partial \lambda^*} \hat{D}(\lambda, \lambda^*)
	& = \hat{D}(\lambda, \lambda^*) (\hat{a} + \frac{1}{2} \lambda)
	= (\hat{a} - \frac{1}{2} \lambda) \hat{D}(\lambda, \lambda^*).
\end{split}
\end{equation}

In terms of displacement operator Wigner transformation $\mathcal{W}$ is defined as
\[
	\mathcal{W}[\hat{A}]
	= \frac{1}{\pi^2} \int d^2 \lambda \exp(-\lambda \alpha^* + \lambda^* \alpha)
		\Trace{ \hat{A} \hat{D}(\lambda, \lambda^*) }.
\]
It transforms an operator $\hat{A}$ on a Hilbert space to a function $\mathcal{W}[\hat{A}](\alpha, \alpha^*)$ on phase space.
The backward transformation (called the Weyl transformation) gives back matrix elements of the operator:
\[
	\langle \alpha \lvert \mathcal{W}^{-1}[f(\alpha, \alpha^*)] \rvert \alpha \rangle
	= \todo{find the expression},
\]
which is enough, because any operator is determined by its expectation in all coherent states~\cite{Gardiner2004}.

Thus Wigner function can be defined as a result of Wigner transformation of the density matrix: $W(\alpha, \alpha^*) = \mathcal{W}[\hat{\rho}]$.
In some cases it will be convenient to use Wigner function in form~\cite{Gardiner2004}
\begin{equation}
\label{eqn:formalism:sm-wigner:w-definition}
	W (\alpha, \alpha^*)
	= \frac{1}{\pi^2} \int d^2 \lambda \exp(-\lambda \alpha^* + \lambda^* \alpha)
		\chi_W (\lambda, \lambda^*),
\end{equation}
where $\chi_W (\lambda, \lambda^*)$ is the characteristic function
\[
	\chi_W (\lambda, \lambda^*)
	= \Trace{ \hat{\rho} \hat{D}(\lambda, \lambda^*) }.
\]

\begin{lemma}
\label{lmm:formalism:sm-wigner:moments-from-chi}
\[
	\langle \symprod{ \hat{a}^r (\hat{a}^\dagger)^s } \rangle
	= \left.
		\left( \frac{\partial}{\partial \lambda} \right)^s
		\left( -\frac{\partial}{\partial \lambda^*} \right)^r
		\chi_W (\lambda, \lambda^*)
	\right|_{\lambda=0}.
\]
\end{lemma}
\begin{proof}
The exponent in the $\chi_W$ can be expanded as
\[
	\exp (\lambda \hat{a}^\dagger - \lambda^* \hat{a})
	= \sum\limits_{r,s}
		\frac{(-\lambda^*)^r \lambda^s}{r!s!}
		\symprod{ \hat{a}^r (\hat{a}^\dagger)^s }.
\]
Thus
\[
	\chi_W(\lambda, \lambda^*)
	= \sum\limits_{r,s}
		\frac{(-\lambda^*)^r \lambda^s}{r!s!}
		\Trace{
			\hat{\rho} \symprod{ \hat{a}^r (\hat{a}^\dagger)^s }
		}
	= \sum\limits_{r,s}
		\frac{(-\lambda^*)^r \lambda^s}{r!s!}
		\langle \symprod{ \hat{a}^r (\hat{a}^\dagger)^s } \rangle
\]
Apparently, the application of $(\partial / \partial \lambda)^s$ and $(-\partial / \partial \lambda^*)^r$ will eliminate all lower order moments,
and setting $\lambda = 0$ afterwards will eliminate all higher order moments,
leaving only $\symprod{ \hat{a}^r (\hat{a}^\dagger)^s }$:
\[
	\left.
		\left( \frac{\partial}{\partial \lambda} \right)^s
		\left( -\frac{\partial}{\partial \lambda^*} \right)^r
		\chi_W (\lambda, \lambda^*)
	\right|_{\lambda=0}
	= r! s! \frac{1}{r! s!}
		\langle \symprod{ \hat{a}^r (\hat{a}^\dagger)^s } \rangle
	= \langle \symprod{ \hat{a}^r (\hat{a}^\dagger)^s } \rangle.
	\qedhere
\]
\end{proof}

Now we can get the final relation.
\begin{theorem}
\label{thm:formalism:sm-wigner:moments}
\[
	\int d^2\alpha\, \alpha^r (\alpha^*)^s W(\alpha, \alpha^*)
	= \langle \symprod{ \hat{a}^r (\hat{a}^\dagger)^s } \rangle
\]
\end{theorem}
\begin{proof}
Integrating the expression to the left by parts and eliminating terms which fit \lmmref{formalism:c-numbers:zero-integrals}
(which can be done since $\chi_W$ is bounded~\cite{Gardiner2004}),
we find that
\begin{equation*}
\begin{split}
	\int d^2\alpha\, \alpha^r (\alpha^*)^s W(\alpha, \alpha^*)
	& = \frac{1}{\pi^2} \int d^2\alpha \int d^2\lambda
		\exp(-\lambda \alpha^* + \lambda^* \alpha)
		\left( \frac{\partial}{\partial \lambda} \right)^s
		\left( -\frac{\partial}{\partial \lambda^*} \right)^r
		\chi_W (\lambda, \lambda^*) \\
	& = \int d^2\lambda\,
		\delta (\Real \lambda) \delta (\Imag \lambda)
		\left( \frac{\partial}{\partial \lambda} \right)^s
		\left( -\frac{\partial}{\partial \lambda^*} \right)^r
		\chi_W (\lambda, \lambda^*) \\
	& = \left.
		\left( \frac{\partial}{\partial \lambda} \right)^s
		\left( -\frac{\partial}{\partial \lambda^*} \right)^r
		\chi_W (\lambda, \lambda^*)
	\right|_{\lambda=0},
\end{split}
\end{equation*}
where we used \lmmref{formalism:c-numbers:fourier-of-moments} to evaluate integral over $\alpha$.
Now, recognising the final expression as a part of \lmmref{formalism:sm-wigner:moments-from-chi},
we immideately get the statement of the theorem.
\end{proof}

\begin{theorem}[Operator correspondences]
\label{thm:formalism:sm-wigner:correspondences}
Considering~\eqnref{formalism:sm-wigner:w-definition} and corresponding backward transform \todo{needs explicit definition?} to be a bijection from Hilbert space to $\mathbb{R}^2$ \todo{proof needed? or just write the theorem in terms of Wigner transformation?}
(in other words, a one-to-one correspondence $\hat{\rho} \leftrightarrow W$),
the following correspondences take place:
\begin{equation*}
\begin{split}
	\hat{a} \hat{\rho}
		& \leftrightarrow
		\left( \alpha + \frac{1}{2} \frac{\partial}{\partial \alpha^*} \right) W(\alpha, \alpha^*),
	\quad
	\hat{a}^\dagger \hat{\rho}
		\leftrightarrow
		\left( \alpha^* - \frac{1}{2} \frac{\partial}{\partial \alpha} \right) W(\alpha, \alpha^*), \\
	\hat{\rho} \hat{a}
		& \leftrightarrow
		\left( \alpha - \frac{1}{2} \frac{\partial}{\partial \alpha^*} \right) W(\alpha, \alpha^*),
	\quad
	\hat{\rho} \hat{a}^\dagger
		\leftrightarrow
		\left( \alpha^* + \frac{1}{2} \frac{\partial}{\partial \alpha} \right) W(\alpha, \alpha^*).
\end{split}
\end{equation*}
\end{theorem}
\begin{proof}
We will prove the first correspondence.
First, let us transform the trace using~\eqnref{formalism:sm-wigner:displacement-derivatives}:
\begin{equation*}
\begin{split}
	\Trace{ \hat{a} \hat{\rho} \hat{D} }
	= \Trace{ \hat{\rho} \hat{D} \hat{a}}
	= \Trace{ \hat{\rho} \left(
		-\frac{\partial}{\partial \lambda^*}
		-\frac{1}{2} \lambda
	\right) \hat{D}}
	= \left(
		-\frac{\partial}{\partial \lambda^*}
		-\frac{1}{2} \lambda
	\right) \Trace{ \hat{\rho} \hat{D}}
	= \left(
		-\frac{\partial}{\partial \lambda^*}
		-\frac{1}{2} \lambda
	\right) \chi_W
\end{split}
\end{equation*}
Now we need to somehow move this additional multiplier outside the integral in the expression for Wigner function:
\begin{equation*}
\begin{split}
	\mathcal{W} [ \hat{a} \hat{\rho} ]
	& = \frac{1}{\pi^2} \int d^2 \lambda \exp(-\lambda \alpha^* + \lambda^* \alpha)
		\Trace{ \hat{a} \hat{\rho} \hat{D}(\lambda, \lambda^*) } \\
	& = \frac{1}{\pi^2} \int d^2 \lambda \exp(-\lambda \alpha^* + \lambda^* \alpha)
		\left(
			-\frac{\partial}{\partial \lambda^*}
			-\frac{1}{2} \lambda
		\right) \chi_W(\lambda, \lambda^*) \\
	& = \frac{1}{2} \frac{\partial}{\partial \alpha^*} \mathcal{W} [\hat{\rho}]
	- \frac{1}{\pi^2} \int d^2 \lambda \exp(-\lambda \alpha^* + \lambda^* \alpha)
		\frac{\partial}{\partial \lambda^*} \chi_W(\lambda, \lambda^*) \\
	& = \frac{1}{2} \frac{\partial}{\partial \alpha^*} \mathcal{W} [\hat{\rho}]
	+ \frac{1}{\pi^2} \int d^2 \lambda \left(
		\frac{\partial}{\partial \lambda^*} \exp(-\lambda \alpha^* + \lambda^* \alpha)
	\right) \chi_W(\lambda, \lambda^*) \\
	& = \left( \alpha + \frac{1}{2} \frac{\partial}{\partial \alpha^*} \right) \mathcal{W} [\hat{\rho}].
\end{split}
\end{equation*}
Notice that we used~\lmmref{formalism:sm-wigner:zero-integrals-2} to move the partial derivative over $\lambda^*$.
\end{proof}
