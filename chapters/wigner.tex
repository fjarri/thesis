\chapter{Wigner representation}
\label{cha:wigner}


% =============================================================================
\section{Hamiltonian}
% =============================================================================

In order to take into account quantum effects, we must start from the master equation.
The basic Hamiltonian is easily expressed using quantum fields $\Psiop_j^{\dagger}(\xvec)$ and $\Psiop_j(\xvec)$,
where $\Psiop_j^{\dagger}(\xvec)$ creates a bosonic atom of spin $j$ at location $\xvec$,
and $\Psiop_j(\xvec)$ destroys one;
the commutators are defined by~\eqnref{multimode-formalism:multimode-commutators}.
Second-quantized Hamiltonian for the system looks like:
\begin{equation}
\label{eqn:wigner:hamiltonian}
\begin{split}
	\hat{H} / \hbar = \int d^{D}\xvec \left\{
		\Psiop_j^{\dagger} K_{jk} \Psiop_k
		+ \frac{1}{2} \int d^D\xvec^\prime
			\Psiop_j^\dagger (\xvec) \Psiop_k^\dagger (\xvec^\prime)
			U_{jk}(\xvec - \xvec^\prime)
			\Psiop_j (\xvec^\prime) \Psiop_k (\xvec)
	\right\}.
\end{split}
\end{equation}
Here we use the Einstein summation convention of summing over repeated indices.
$U_{jk}$ is the two-body scattering potential, and $K_{jk}$ is the single-particle Hamiltonian:
\begin{equation}
	K_{jk} = \left(
			-\frac{\hbar}{2m} \nabla^2 + \omega_j + V_j(\xvec) / \hbar
		\right) \delta_{jk}
		+ \tilde{\Omega}_{jk}(t),
\end{equation}
where $V_j$ is the external trapping potential for spin $j$,
$\omega_j$ is the internal energy of spin $j$,
and $\tilde{\Omega}_{jk}$ represents a time-dependent coupling that is used to rotate one spin projection into another.
In the subspace of two coupled components $\tilde{\Omega}_{jk}$ can be defined as:
\[
	\tilde{\Omega} = \frac{\Omega}{2} \begin{pmatrix}
		0 & e^{i(\omega t + \alpha)} + e^{-i(\omega t + \alpha)} \\
		e^{i (\omega t + \alpha)} + e^{-i(\omega t + \alpha)} & 0
	\end{pmatrix},
\]
where $\omega$ and $\alpha$ are frequency and phase of the oscillator,
and $\Omega$ is Rabi frequency (cf. equation~\eqnref{mean-field:rotation-matrix}).

If we impose an energy cutoff $\ecut$ and only take into account low-energy modes,
the general scattering potential $U_{jk}(\xvec - \xvec^\prime)$ can be replaced by contact potential $U_{jk} \delta^{(D)}(\xvec - \xvec^\prime)$~\cite{Morgan2000}, giving the effective Hamiltonian
\begin{equation}
\label{eqn:wigner:hamiltonian_effective}
\begin{split}
	\hat{H} / \hbar = \int d^{D}\xvec \left\{
		\Psiop_j^{\dagger} K_{jk} \Psiop_k
		+ \frac{U_{jk}}{2} \Psiop_j^\dagger \Psiop_k^\dagger \Psiop_j \Psiop_k
	\right\}.
\end{split}
\end{equation}

For $s$-wave scattering in three dimensions the coefficient is $U_{jk} = 4 \pi \hbar a_{jk} / m$,
where $a_{jk}$ is the scattering length.
Note that in general case the coefficient must be renormalised depending on the grid~\cite{Sinatra2002},
but the change is small if $dx \gg a_{jk}$.


% =============================================================================
\section{Energy cutoff}
% =============================================================================

As was noted earlier, in order to use contact interactions, an energy cutoff has to be imposed.
We will use the formalism for multimode operators described in \appref{multimode-formalism}.

Two basis sets are used in this work.
First is the basis of plane waves:
\[
	\phi_{\nvec}(\xvec) = e^{i \kvec_{\nvec} \xvec} / \sqrt{V},
\]
whose elements are eigenfunctions of kinetic term:
\[
	\left( -\frac{\hbar^2}{2m} \nabla^2 \right) \phi_{\nvec}
	= E_{\nvec} \phi_{\nvec}
	= \frac{\hbar^2 \lvert \kvec_{\nvec} \rvert^2}{2 m} \phi_{\nvec}.
\]
Field functions can be decomposed into this basis using common FFTs.
Integration area for this set is the box with volume $V$.

More sophisticated basis can be constructed out of harmonic oscillator modes:
\[
	\phi_{\nvec}(\xvec) = \frac{1}{\sqrt[4]{\pi}}
		\prod\limits_{d=1}^{D}
			\frac{1}{\sqrt{2^{n_d} n_d! l_d}}
			H_{n_d} \left( \frac{x_d}{l_d} \right)
			\exp \left( -\frac{x_d^2}{2 l_d^2} \right),
\]
where $H_k(x)$ is the ``physicists'\,'' Hermite polynomial of order $k$,
and the oscillator length $l_d = \sqrt{\hbar / m \omega_d}$.
These are the eigenfunctions of harmonic oscillator Hamiltonian:
\[
	\left( -\frac{\hbar^2}{2m} \nabla^2 + V(\xvec) \right) \phi_{\nvec}
	= E_{\nvec} \phi_{\nvec}
	= \left(
		\sum\limits_{d=1}^D \hbar \omega_d \left(
			n_d + \frac{1}{2}
		\right)
	\right) \phi_{\nvec}.
\]
This basis set has the whole $D$-dimensional space as the integration area.
One of the ways to decompose wave function into this basis is the Gauss-Hermite quadrature
(see \appref{harmonic-transform} for technical details).

We divide mode subspaces into low- and high-energy subsets $L$ and $H$,
depending on whether $E_n$ is more or less than some cutoff energy $\ecut$,
and limit field operators to the subset $L$.
Restricted operators obey commutation relations~\eqnref{multimode-formalism:restricted-commutators}.

\begin{figure}
\begin{center}
\subfloat[Uniform grid, $\ecut = 1000\,\hbar\omega$]{\includegraphics[width=0.5\textwidth]{%
	figures_generated/wigner/delta_uniform_1000.eps}}
\subfloat[Harmonic grid, $\ecut = 1000\,\hbar\omega$]{\includegraphics[width=0.5\textwidth]{%
	figures_generated/wigner/delta_harmonic_1000.eps}} \\
\subfloat[Uniform grid, no explicit cutoff]{\includegraphics[width=0.5\textwidth]{%
	figures_generated/wigner/delta_uniform_all.eps}}
\subfloat[Harmonic grid, no explicit cutoff]{\includegraphics[width=0.5\textwidth]{%
	figures_generated/wigner/delta_harmonic_all.eps}}
\end{center}
\caption{Absolute value of restricted delta function $\lvert \delta_P(z - z^\prime) \rvert$ with and without explicit cutoff.}
\label{fig:wigner:restricted-delta}
\end{figure}

Shape of the module of restricted delta function~\eqnref{multimode-formalism:restricted-delta} for two different basis sets and two different cutoffs is shown in~\figref{wigner:restricted-delta}.
The higher cutoff is, the closer its shape gets to the ``ideal'' delta function,
with nonlocality scaling as $\ecut^{-1/2}$ \todo{proof needed}.


% =============================================================================
\section{Master equation}
% =============================================================================

Hereinafter field operators and wave functions will be assumed to be defined in restricted basis, unless explicitly stated otherwise.
The Markovian master equation for the system with the inclusion of losses can be written as~\cite{Jack2002}
\begin{equation}
	\frac{d\hat{\rho}}{dt} =
		- \frac{i}{\hbar} \left[ \hat{H}, \hat{\rho} \right]
		+ \sum\limits_{\lvec} \kappa_{\lvec} \int d\xvec
			\mathcal{L}_{\lvec} \left[ \hat{\rho} \right],
\end{equation}
where $\lvec = (l_1, l_2, \ldots, l_n)$ is a vector indicating the spins that are coupled,
$n$ being the number of interacting particles,
and we have introduced local Liouville loss terms,
\begin{equation}
	\mathcal{L}_{\lvec} \left[ \hat{\rho} \right] =
		2\hat{O}_{\lvec} \hat{\rho} \hat{O}_{\lvec}^\dagger
		- \hat{O}_{\lvec}^\dagger \hat{O}_{\lvec} \hat{\rho}
		- \hat{\rho} \hat{O}_{\lvec}^\dagger \hat{O}_{\lvec}.
\end{equation}
The reservoir coupling operators $\hat{O}_{\lvec}$ are the distinct $n$-fold products of local field annihilation operators,
$\hat{O}_{\lvec} = \hat{O}_{\lvec} (\Psiopvec) =
	\Psiop_{l_{1}} (\xvec)
	\Psiop_{l_{2}} (\xvec) \ldots
	\Psiop_{l_{n}} (\xvec),$
describing local $n$-body collision losses.

The master equation allows us to derive an important property:
\begin{equation*}
\begin{split}
	\frac{d}{dt} \langle \Psiop_j \rangle
	& = \frac{d}{dt} \Tr \left[ \hat{\rho} \Psiop_j \right]
	= \Tr \left[ \frac{d\hat{\rho}}{dt} \Psiop_j \right] \\
	& = \Tr \left[
		-\frac{i}{\hbar} \left[ \hat{H}, \hat{\rho} \right] \Psiop_j
	\right]
	+ \sum\limits_{\lvec} \kappa_{\lvec} \int d\xvec^\prime
		\Tr \left[
			\mathcal{L}_{\lvec}^\prime \left[ \hat{\rho} \right]
			\Psiop_j
		\right] \\
	& = \int d\xvec^\prime \left(
		- \frac{i}{\hbar} \Tr \left[
			\left[
				\Psiop_l^{\prime\dagger} K_{lm}^\prime \Psiop_m^\prime,
				\hat{\rho}
			\right] \Psiop_j
		\right]
		- \frac{i}{2\hbar} U_{lm} \Tr \left[
			\left[
				\Psiop_l^{\prime\dagger} \Psiop_m^{\prime\dagger}
				\Psiop_l^\prime \Psiop_m^\prime,
				\hat{\rho}
			\right] \Psiop_j
		\right]
		+ \sum\limits_{\lvec} \kappa_{\lvec}
			\Tr \left[
				\mathcal{L}_{\lvec}^\prime \left[ \hat{\rho} \right]
				\Psiop_j
			\right]
	\right),
\end{split}
\end{equation*}
where $K_{lm}^\prime \equiv K_{lm}(\Psiop(\xvec^\prime), \xvec^\prime)$,
$\mathcal{L}_{\lvec}^\prime \equiv \mathcal{L} [ O_{\lvec}^\prime ] \equiv \mathcal{L} [ O_{\lvec} ( \Psiopvec(\xvec^\prime) ) ]$,
and $\Psiop_j^\prime \equiv \Psiop_j (\xvec^\prime)$.

Let us transform each term separately.
We will make extensive use of the fact that trace is invariant under cyclic permutations to re-order the operators in terms.
In addition, the transformations are based on commutation relations~\eqnref{multimode-formalism:functional-commutators}.

Noticing that $[ K_{lm}^\prime, Psiop_j ] \equiv 0$, the first term can be transformed as:
\begin{equation*}
\begin{split}
	\Tr \left[
		\left[
			\Psiop_l^{\prime\dagger} K_{lm}^\prime \Psiop_m^\prime,
			\hat{\rho}
		\right] \Psiop_j
	\right]
	& = \Tr \left[
		\Psiop_l^{\prime\dagger} K_{lm}^\prime \Psiop_m^\prime \hat{\rho} \Psiop_j
		- \hat{\rho} \Psiop_l^{\prime\dagger} K_{lm}^\prime \Psiop_m^\prime \Psiop_j
	\right] \\
	& = \Tr \left[
		\hat{\rho} \left(
			\Psiop_j \Psiop_l^{\prime\dagger} K_{lm}^\prime \Psiop_m^\prime
			- \Psiop_l^{\prime\dagger} K_{lm}^\prime \Psiop_m^\prime \Psiop_j
		\right)
	\right] \\
	& = \Tr \left[
		\hat{\rho} \left[
			\Psiop_j \Psiop_l^{\prime\dagger}
		\right] K_{lm}^\prime \Psiop_m^\prime
	\right] \\
	& = \Tr \left[
		\hat{\rho} \delta_{jl} \delta_P(\xvec - \xvec^\prime) K_{lm}^\prime \Psiop_m^\prime
	\right]
	= \delta_P(\xvec - \xvec^\prime) \langle K_{jm}^\prime \Psiop_m^\prime \rangle
\end{split}
\end{equation*}

Second (nonlinear) term:
\begin{equation*}
\begin{split}
	U_{lm} \Tr \left[
		\left[
			\Psiop_l^{\prime\dagger} \Psiop_m^{\prime\dagger}
			\Psiop_l^\prime \Psiop_m^\prime,
			\hat{\rho}
		\right] \Psiop_j
	\right]
	& = U_{lm} \Tr \left[
		\Psiop_l^{\prime\dagger} \Psiop_m^{\prime\dagger}
		\Psiop_l^\prime \Psiop_m^\prime \hat{\rho} \Psiop_j
		- \hat{\rho} \Psiop_l^{\prime\dagger} \Psiop_m^{\prime\dagger}
		\Psiop_l^\prime \Psiop_m^\prime \Psiop_j
	\right] \\
	& = U_{lm} \Tr \left[
		\hat{\rho} \left(
			\Psiop_j \Psiop_l^{\prime\dagger} \Psiop_m^{\prime\dagger}
			\Psiop_l^\prime \Psiop_m^\prime
			- \Psiop_l^{\prime\dagger} \Psiop_m^{\prime\dagger}
			\Psiop_l^\prime \Psiop_m^\prime \Psiop_j
		\right)
	\right] \\
	& = U_{lm} \Tr \left[
		\hat{\rho} \left[
			\Psiop_j, \Psiop_l^{\prime\dagger} \Psiop_m^{\prime\dagger}
		\right] \Psiop_l^\prime \Psiop_m^\prime
	\right] \\
	& = U_{lm} \Tr \left[
		\hat{\rho} \delta_P(\xvec - \xvec^\prime) \left(
			\delta_{jl} \Psiop_m^{\prime\dagger}
			+ \delta_{jm} \Psiop_l^{\prime\dagger}
		\right) \Psiop_l^\prime \Psiop_m^\prime
	\right] \\
	& = 2 U_{jm} \delta_P(\xvec - \xvec^\prime) \Tr \left[
		\hat{\rho} \Psiop_m^{\prime\dagger} \Psiop_m^\prime \Psiop_j^\prime
	\right] \\
	& = 2 U_{jm} \delta_P(\xvec - \xvec^\prime) \langle
		\Psiop_m^{\prime\dagger} \Psiop_m^\prime \Psiop_j^\prime
	\rangle
\end{split}
\end{equation*}

The third term, coming from the losses, can be calculated as
\begin{equation*}
\begin{split}
	\Tr \left[
		\mathcal{L}_{\lvec}^\prime \left[ \hat{\rho} \right]
		\Psiop_j
	\right]
	& = \Tr \left[
		2 \hat{O}_{\lvec}^\prime \hat{\rho} \hat{O}_{\lvec}^{\prime\dagger} \Psiop_j
		- \hat{O}_{\lvec}^{\prime\dagger} \hat{O}_{\lvec}^\prime \hat{\rho} \Psiop_j
		- \hat{\rho} \hat{O}_{\lvec}^{\prime\dagger} \hat{O}_{\lvec}^\prime \Psiop_j
	\right] \\
	& = \Tr \left[
		2 \hat{\rho} \hat{O}_{\lvec}^{\prime\dagger} \Psiop_j \hat{O}_{\lvec}^\prime
		- \hat{O}_{\lvec}^{\prime\dagger} \hat{O}_{\lvec}^\prime \hat{\rho} \Psiop_j
		- \hat{\rho} \hat{O}_{\lvec}^{\prime\dagger} \hat{O}_{\lvec}^\prime \Psiop_j
	\right] \\
	& = \Tr \left[
		\hat{\rho} \hat{O}_{\lvec}^{\prime\dagger} \hat{O}_{\lvec}^\prime \Psiop_j
		+ \hat{\rho} \hat{O}_{\lvec}^{\prime\dagger} \Psiop_j \hat{O}_{\lvec}^\prime
		- \hat{\rho} \Psiop_j \hat{O}_{\lvec}^{\prime\dagger} \hat{O}_{\lvec}^\prime
		- \hat{\rho} \hat{O}_{\lvec}^{\prime\dagger} \hat{O}_{\lvec}^\prime \Psiop_j
	\right] \\
	& = \Tr \left[
		\hat{\rho} \left[
			\hat{O}_{\lvec}^{\prime\dagger}, \Psiop_j
		\right] \hat{O}_{\lvec}^\prime
	\right] \\
	& = -\delta_P(\xvec - \xvec^\prime)	\Tr \left[
		\hat{\rho} \frac{\partial \hat{O}_{\lvec}^{\prime\dagger}}{\partial \Psiop_j^{\prime\dagger}}
		\hat{O}_{\lvec}^\prime
	\right] \\
	& = -\delta_P(\xvec - \xvec^\prime) \langle
		\frac{\partial \hat{O}_{\lvec}^{\prime\dagger}}{\partial \Psiop_j^{\prime\dagger}}
		\hat{O}_{\lvec}^\prime
	\rangle.
\end{split}
\end{equation*}

Thus the full relation is
\begin{equation*}
\begin{split}
	\frac{d}{dt} \langle \Psiop_j \rangle
	& = \int d\xvec^\prime \delta_P(\xvec - \xvec^\prime) \left(
		- \frac{i}{\hbar} \langle K_{jm}^\prime \Psiop_m^\prime \rangle
		- \frac{i U_{jm}}{\hbar} \langle
			\Psiop_m^{\prime\dagger} \Psiop_m^\prime \Psiop_j^\prime
		\rangle
		- \sum\limits_{\lvec} \kappa_{\lvec} \langle
			\frac{\partial \hat{O}_{\lvec}^{\prime\dagger}}{\partial \Psiop_j^{\prime\dagger}}
			\hat{O}_{\lvec}^\prime
		\rangle
	\right) \\
	& = P \left[
		\langle
			-\frac{i}{\hbar} \left(
				K_{jm} \Psiop_m
				+ U_{jm} \Psiop_m^\dagger \Psiop_m \Psiop_j
			\right)
			- \sum\limits_{\lvec} \kappa_{\lvec}
				\frac{\partial \hat{O}_{\lvec}^\dagger}{\partial \Psiop_j^\dagger} \hat{O}_{\lvec}
		\rangle
	\right]
\end{split}
\end{equation*}

\begin{comment}
\subsection{Wigner representation}

We use a multimode Wigner function representation~\cite{Gardiner2004} of the density operator,
with operator correspondences being:
\begin{align*}
\hat{a}_{ik} \rho = \left( \alpha_{ik} + \frac{1}{2} \frac{\partial}{\partial \alpha_{ik}^*} \right) W, & &
\hat{a}_{ik}^\dagger \rho = \left( \alpha_{ik}^* - \frac{1}{2} \frac{\partial}{\partial \alpha_{ik}} \right) W, \\
\rho \hat{a}_{ik} = \left( \alpha_{ik} - \frac{1}{2} \frac{\partial}{\partial \alpha_{ik}^*} \right) W, & &
\rho \hat{a}_{ik}^\dagger = \left( \alpha_{ik}^* + \frac{1}{2} \frac{\partial}{\partial \alpha_{ik}} \right) W.
\end{align*}

Using these, we can transform the noninteracting term:
\begin{align*}
\begin{split}
\sum\limits^2_{i=1}\int\limits_V d\xvec \, \Psiop_i^\dagger(\xvec) \hat{K} \Psiop_i(\xvec) = &
\sum\limits_{i=1}^2 \sum\limits_k \hbar \omega_{k} \hat{a}_{ik}^\dagger \hat{a}_{ik} \\
\rightarrow {} & \sum\limits_{i=1}^2 \sum\limits_{k \in L} \hbar \omega_{k} \hat{a}_{ik}^\dagger \hat{a}_{ik} \\
\rightarrow {} & - \sum\limits_{i=1}^2 \sum\limits_{k \in L} \hbar \omega_k \left(
	\frac{\partial}{\partial \alpha_{ik}} \alpha_{ik} - \frac{\partial}{\partial \alpha_{ik}^*} \alpha_{ik}^*
\right) W \\
= {} & - \sum\limits_{i=1}^2 \int\limits_V d\xvec \left(
	\frac{\delta}{\delta \psi_{iP}} \hat{K} \psi_{iP} - \frac{\delta}{\delta \psi_{iP}^*} \hat{K} \psi_{iP}^*
\right)
\end{split}
\end{align*}

But for every other term it is easier to use functional correspondences, following~\cite{Norrie2006a}.
To do this, we define restricted basis wave functions:
\[
\psi_{iP} \equiv \sum\limits_{k \in L} \phi_k (\xvec) \alpha_{ik}
\]
and then the functional derivatives:
\[
\frac{\delta}{\delta \psi_{iP} (\xvec)} \equiv \sum\limits_{k \in L} \phi_k^* (\xvec) \frac{\partial}{\partial \alpha_j}.
\]
This gives us the following correspondences:
\begin{align*}
\Psiop_{iP} \rho = \left( \psi_{iP} + \frac{1}{2} \frac{\delta}{\delta \psi_{iP}^*} \right) W, & &
\Psiop_{iP}^\dagger \rho = \left( \psi_{iP}^* - \frac{1}{2} \frac{\delta}{\delta \psi_{iP}} \right) W, \\
\rho \Psiop_{iP} = \left( \psi_{iP} - \frac{1}{2} \frac{\delta}{\delta \psi_{iP}^*} \right) W, & &
\rho \Psiop_{iP}^\dagger = \left( \psi_{iP}^* + \frac{1}{2} \frac{\delta}{\delta \psi_{iP}} \right) W.
\end{align*}

Now we can transform full master equation to Wigner representation.
\textcolor{red}{[Add section about truncation and describe validity criteria.]}
Wigner truncation is performed simultaneously,
following the procedure, thoroughly described in~\cite{Norrie2006} and~\cite{Norrie2006a}:
\begin{align*}
\begin{split}
i \hbar \frac{\partial W}{\partial t} = & \int\limits_V d\xvec \left\{
	- \frac{\delta}{\delta \psi_{1P}} \left( \left(
		\hat{K} + U_{11} \lvert \psi_{1P} \rvert^2 + U_{12} \lvert \psi_{2P} \rvert^2 \right. \right. \right. \\
		& \left. \left. \left. - i \hbar \frac{\gamma_{111}}{2} \lvert \psi_{1P} \rvert^4 - i \hbar \frac{\gamma_{12}}{2} \lvert \psi_{2P} \rvert^2
	\right) \psi_{1P} +
	\frac{\hbar}{2} \left( \tilde{\Omega} e^{i \omega t} + \tilde{\Omega}^* e^{-i \omega t} \right) \psi_{2P} \right) +
	\textrm{c.c.} \right. \\
& \left. - \frac{\delta}{\delta \psi_{2P}} \left( \left(
		\hat{K} + V_{hf} + U_{22} \lvert \psi_{2P} \rvert^2 + U_{12} \lvert \psi_{1P} \rvert^2 \right. \right. \right. \\
		& \left. \left. \left. - i \hbar \frac{\gamma_{12}}{2} \lvert \psi_{1P} \rvert^2 - i \hbar \frac{\gamma_{22}}{2} \lvert \psi_{2P} \rvert^2
	\right) \psi_{2P} +
	\frac{\hbar}{2} \left( \tilde{\Omega} e^{i \omega t} + \tilde{\Omega}^* e^{-i \omega t} \right) \psi_{1P} \right) +
	\textrm{c.c.} \right. \\
& \left. + i \hbar \frac{\delta^2}{\delta \psi_{1P} \delta \psi_{1P}^*} \left(
	\frac{3 \gamma_{111}}{2} \lvert \psi_{1P} \rvert^4 + \frac{\gamma_{12}}{2} \lvert \psi_{2P} \rvert^2
\right) + \right. \\
& \left. + i \hbar \frac{\delta^2}{\delta \psi_{2P} \delta \psi_{2P}^*} \left(
	\frac{\gamma_{12}}{2} \lvert \psi_{1P} \rvert^2 + \gamma_{22} \lvert \psi_2 \rvert^2
\right) + \right. \\
& \left. + i \hbar \frac{\delta^2}{\delta \psi_{1P} \delta \psi_{2P}^*} \frac{\gamma_{12}}{2} \psi_{1P} \psi_{2P}^*
+ i \hbar \frac{\delta^2}{\delta \psi_{1P}^* \delta \psi_{2P}} \frac{\gamma_{12}}{2} \psi_{1P}^* \psi_{2P}
\right\} W
\end{split}
\end{align*}

\section{Fokker-Planck equations}

Now we can transform the whole master equation to Wigner representation, writing it down in the form of Fokker-Planck equation.
Let us first do it for the case of zero losses.
After transformation to Wigner representation, and considering $\Gamma_{12} = \Gamma_{21} $, we will get:
\begin{equation*}
\begin{split}
\frac{\partial W}{\partial t} = & \int\limits_V d\xvec \left\{ - \frac{\partial}{\partial \psi_1} i \left( - \hat{K}_1 +
\Gamma_{11} \left( 1 - \lvert \psi_1 \rvert^2 \right) +
\Gamma_{12} \left( \frac{1}{2} - \lvert \psi_2 \rvert^2 \right) \right) \psi_1 - \right. \\
& \left. - \frac{\partial}{\partial \psi_2} i \left( - \hat{K}_2 + \Gamma_{22} \left( 1 - \lvert \psi_2 \rvert^2 \right) +
\Gamma_{12} \left( \frac{1}{2} - \lvert \psi_1 \rvert^2 \right) \right) \psi_2 \right\} W +
\textrm{c.c.}
\end{split}
\end{equation*}
This leads us to stochastic differential equations (without the explicit noise term though, since there are no losses yet):
\[
\frac{\partial \psi_1}{\partial t} = i \left( - \hat{K}_1 + \Gamma_{11} \left( 1 - \lvert \psi_1 \rvert^2 \right) +
\Gamma_{12} \left( \frac{1}{2} - \lvert \psi_2 \rvert^2 \right) \right) \psi_1
\]
\[
\frac{\partial \psi_2}{\partial t} = i \left( - \hat{K}_2 + \Gamma_{22} \left( 1 - \lvert \psi_2 \rvert^2 \right) +
\Gamma_{12} \left( \frac{1}{2} - \lvert \psi_1 \rvert^2 \right) \right) \psi_2,
\]
which look almost like coupled GPEs, except for the phase shifting term.
Apparently, the following correspondences are valid:
\[
\Gamma_{11} = g_{11},\, \Gamma_{12} = g_{12},\, \Gamma_{22} = g_{22}.
\]

Now it's time to add losses to the picture using bath coupling operators~(\ref{bath_coupling_operators}).
The resulting equation in Wigner representation is:
\begin{equation}
\label{full_wigner_equation}
\begin{split}
\frac{\partial W}{\partial t} = & \int\limits_V d\xvec \left\{ - \frac{\partial}{\partial \psi_1} \left( \hat{A}_1 +
\kappa_{111} \left( \frac{9}{2} \lvert \psi_1 \rvert^2 - \frac{9}{4} - \frac{3}{2} \lvert \psi_1 \rvert^4 \right) +
\kappa_{12} \left( \frac{1}{2} - \lvert \psi_2 \rvert^2 \right) \right) \psi_1 - \right. \\
& \left. - \frac{\partial}{\partial \psi_2} \left( \hat{A}_2 + \kappa_{12} \left( \frac{1}{2} - \lvert \psi_1 \rvert^2 \right) +
\kappa_{22} \left( 1 - \lvert \psi_2 \rvert^2 \right) \right) \psi_2 +  \textrm{c.c.} + \right. \\
& \left. + \frac{\partial^2}{\partial \psi_1 \partial \psi_1^*} \left( \kappa_{111} \left(\frac{9}{2} \lvert \psi_1 \rvert^4 -
9 \lvert \psi_1 \rvert^2 + \frac{9}{4} \right) + \kappa_{12} \left( \lvert \psi_2 \rvert^2 - \frac{1}{2} \right) \right) + \right. \\
& \left. + \frac{\partial^2}{\partial \psi_2 \partial \psi_2^*} \left( \kappa_{12} \left( \lvert \psi_1 \rvert^2 - \frac{1}{2} \right) +
\kappa_{22} \left( 2 \lvert \psi_2 \rvert^2 - 1 \right) \right) + \right. \\
& \left. + \frac{\partial^2}{\partial \psi_1 \partial \psi_2^*} \kappa_{12} \psi_1 \psi_2^* +
\frac{\partial^2}{\partial \psi_1^* \partial \psi_2} \kappa_{12} \psi_1^* \psi_2
\right\} W,
\end{split}
\end{equation}
where $\hat{A}_1$ and $\hat{A}_2$ stand for previously calculated kinetic, potential and interaction terms for both components.
Comparing this equation with coupled GPEs, we can connect $\kappa$ coefficients with loss coefficients $\gamma$:
\[
\kappa_{111} = \frac{1}{3} \gamma_{111},\,
\kappa_{12} = \frac{1}{2} \gamma_{12},\,
\kappa_{22} = \frac{1}{2} \gamma_{22}.
\]

\subsection*{Discretisation}

In Wigner representation, virtual particles are added to the each mode of classical field to mimic quantum noise.
In continuous case this leads to infinities because of the infinite number of modes.
Therefore, we have to perform discretisation and use the finite grid; the process is described in detail in \cite{Sinatra2002}.
In 3D case it is the box $L_x \times L_y \times L_z$ with $N = N_x N_y N_z$ points.
The volume of the box is $V = L_x L_y L_z$ and the volume of the cell $dV$ equals $V / N$.

There are some conditions for discretised hamiltonian to be close to reality.
First, the spatial step must be smaller than healing length:
\[ dx_\nu \ll \left( 8 \pi \rho a \right)^{-\frac{1}{2}},\, \nu = x,y,z \]
where $\rho$ is the density at the centre of the condensate, and the equation is in real units.
Using Thomas-Fermi approximation and natural units, we can transform this condition to
\[ dx_\nu \ll \left( 2 \mu \right)^{-\frac{1}{2}} \]
Taking into account the size of the condensate~\eqnref{eqn:mean-field:tf-radii}, we obtain the following criteria:
\begin{equation}
\label{grid_size_criteria}
N_x \gg 2 \mu,\, N_y \gg 2 \mu,\, N_z \gg 2 \lambda \mu
\end{equation}

Secondly, the spatial step of the grid must be larger than the scattering length:
\[ dx_\nu \gg a \]
This is necessary for discretised interaction term in the hamiltonian to be correct; see \cite{Sinatra2002} for details.

Equation~(\ref{full_wigner_equation}) in discrete form will then look like:
\begin{equation*}
\begin{split}
\frac{\partial W}{\partial t} = & \sum\limits_{\xvec} \left\{
	- \frac{\partial}{\partial \psi_1} \hat{A}_1 \psi_1 -
	\frac{\partial}{\partial \psi_2} \hat{A}_2 \psi_2 -
	\frac{\partial}{\partial \psi^*_1} \hat{A}^*_1 \psi^*_1 -
	\frac{\partial}{\partial \psi^*_2} \hat{A}^*_2 \psi^*_2 +
\right. \\
& \left.
	\frac{\partial^2}{\partial \psi_1 \partial \psi_1^*} \hat{D}_{11} +
	\frac{\partial^2}{\partial \psi_2 \partial \psi_2^*} \hat{D}_{22} +
	\frac{\partial^2}{\partial \psi_1 \partial \psi_2^*} \hat{D}_{12} +
	\frac{\partial^2}{\partial \psi_1^* \partial \psi_2} \hat{D}^*_{12}
\right\} W,
\end{split}
\end{equation*}
where
\begin{align*}
\hat{A}_1 = {} & i \left( - \hat{K}_1 + g_{11} \left( \frac{1}{dV} - \lvert \psi_1 \rvert^2 \right) +
g_{12} \left( \frac{1}{2 dV} - \lvert \psi_2 \rvert^2 \right) \right) + \\
& +\frac{\gamma_{111}}{2} \left( \frac{3}{dV} \lvert \psi_1 \rvert^2 - \frac{3}{2 dV^2} - \lvert \psi_1 \rvert^4 \right) +
\frac{\gamma_{12}}{2} \left( \frac{1}{2 dV} - \lvert \psi_2 \rvert^2 \right), \\
\hat{A}_2 = {} & i \left( - \hat{K}_2 + g_{22} \left( \frac{1}{dV} - \lvert \psi_2 \rvert^2 \right) +
g_{12} \left( \frac{1}{2 dV} - \lvert \psi_1 \rvert^2 \right) \right) + \\
& + \frac{\gamma_{12}}{2} \left( \frac{1}{2 dV} - \lvert \psi_1 \rvert^2 \right) +
\frac{\gamma_{22}}{2} \left( \frac{1}{dV} - \lvert \psi_2 \rvert^2 \right), \\
\hat{D}_{11} = {} & \frac{\gamma_{111}}{2} \left(3 \lvert \psi_1 \rvert^4 -
\frac{6}{dV} \lvert \psi_1 \rvert^2 + \frac{3}{2 dV^2} \right) + \frac{\gamma_{12}}{2} \left( \lvert \psi_2 \rvert^2 - \frac{1}{2 dV} \right) \\
\hat{D}_{22} = {} & \frac{\gamma_{12}}{2} \left( \lvert \psi_1 \rvert^2 - \frac{1}{2 dV} \right) +
\frac{\gamma_{22}}{2} \left( 2 \lvert \psi_2 \rvert^2 - \frac{1}{dV} \right) \\
\hat{D}_{12} = {} & \frac{\gamma_{12}}{2} \psi_1 \psi_2^*
\end{align*}

\subsection*{Measurement}

One must not forget that $\psi$ functions in Wigner representation are not actual $\psi$-functions,
and some work is necessary in order to obtain measurable quantities from them.
Moments of field operators can be found as following:
\[
\langle \left\{ \Psiop^r ( \Psiop^\dagger )^s \right\}_{sym} \rangle =
\int \psi^r \left( \psi^* \right)^s W(\psi, \psi^*) d\psi^2,
\]
where $\{\}_{sym}$ stands for symmetrically ordered product,
which is equal to the average of all possible orderings of the product.
Since we do not know the actual Wigner function, we can use the fact that it is the probability of finding the system in state $\psi$.
Therefore, if we simulate the behavior of the system for many separate paths,
we can expect that density of the paths will be larger in areas where $W$ is large.
In other words, we can approximate the weighted integration of the moment by the averaged sum over many paths:
\begin{equation*}
\begin{split}
\int\limits_V \langle \left\{ \Psiop^r ( \Psiop^\dagger )^s \right\}_{sym} \rangle d\xvec & =
\int\limits_V \int \psi^r \left( \psi^* \right)^s W(\psi, \psi^*) d\psi^2 d\xvec \\
& \approx \sum\limits_{V} \left( \frac{1}{N_{paths}} \sum\limits_{paths} \psi^r \left( \psi^* \right)^s \right) dV \\
& = \frac{1}{N_{paths}} \sum\limits_{paths} \sum\limits_{V} \psi^r \left( \psi^* \right)^s dV
\end{split}
\end{equation*}

As an example, let us work out the formula for measuring state density.
We must take into account the discretised form of the commutator~\cite{Sinatra2002}:
\[ \left[ \Psiop(\xvec_i), \Psiop^\dagger(\xvec_j) \right] = \frac{1}{dV} \delta_{ij} \]
Using this, we can obtain the normally ordered product of operators from the symmetrical one:
\begin{equation*}
\begin{split}
n_i & = \langle \Psiop^\dagger_i \Psiop_i \rangle \\
& = \frac{1}{2} \left( \langle \Psiop^\dagger_i \Psiop_i \rangle +
\langle \Psiop_i \Psiop^\dagger_i \rangle - \frac{1}{dV} \right) \\
& = \langle \left\{ \Psiop \Psiop^\dagger \right\}_{sym} \rangle - \frac{1}{2 dV} \\
& = \frac{1}{N_{paths}} \sum\limits_{paths} \lvert \psi_i \rvert^2 - \frac{1}{2 dV}
\end{split}
\end{equation*}
Consequently, the formula for population is:
\begin{equation*}
\begin{split}
	N_i & = \int\limits_V n_i d\xvec \\
	& \approx \sum\limits_{\xvec} \left(
		\frac{1}{N_{paths}} \sum\limits_{paths} \lvert \psi_i \rvert^2 - \frac{1}{2 dV}
	\right) dV \\
	& = \frac{1}{N_{paths}} \sum\limits_{paths} \sum\limits_{\xvec} \lvert \psi_i \rvert^2 dV - \frac{N}{2}
\end{split}
\end{equation*}

Interaction term $\langle \Psiop^\dagger_1 \Psiop_2 \rangle$, which is necessary for visibility measurement,
can serve as another example. Operators $\Psiop_1$ and $\Psiop_2$ commute,
therefore their symmetrically ordered product is equal to the normally ordered product, which gives us the simple formula:
\[
\int\limits_V \langle \Psiop^\dagger_1 \Psiop_2 \rangle d\xvec =
\frac{1}{N_{paths}} \sum\limits_{paths} \sum\limits_{\xvec} \psi_1^* \psi_2 dV
\]

\subsection*{Initial state}

At zero temperatures, the initial state for Wigner representation
is the combination of classical ground state and quantum noise in all modes~\cite{Steel1998}.
The simplest way to construct it is to use the discrete position basis:
\[
\psi(\xvec) = \psi_{GP}(\xvec) + \frac{1}{2 \sqrt{dV}} \eta(\xvec),
\]
where $\eta(\xvec)$ are complex random numbers with independent normally distributed
real and imaginary parts in such a way that:
\[ \overline{\eta^*(\xvec) \eta(\xvec')} = \delta(\xvec - \xvec'), \]
and $\psi_{GP}$ is the ground state as obtained from GPE.

This method has its drawback, though: it links the cutoff for the quantum noise to the simulation grid size.
This means that if we need to use large grid (for example, in order to meet criteria~(\ref{grid_size_criteria})),
the cutoff will be too large.

\subsection*{Decomposition of the diffusion matrix}

Now there is a tricky part. Let us write down the second order derivative part of equation~(\ref{full_wigner_equation})
in matrix form, dropping out small terms according to criterion~(\ref{psi_estimation}):
\[
D = \frac{\kappa_{12} \lvert \psi_1 \rvert^2}{2}  \begin{pmatrix}
0 & 1 & 0 & \alpha \\
1 & 0 & \alpha^* & 0 \\
0 & \alpha^* & 0 & \lvert \alpha \rvert^2 + \beta \\
\alpha & 0 & \lvert \alpha \rvert^2 + \beta & 0 \end{pmatrix},
\]
where
\[
\alpha = \frac{\psi_1}{\psi_2},\, \beta = \frac{2 \kappa_{22}}{\kappa_{12}}.
\]
In order to map equation~(\ref{full_wigner_equation}) to a set of stochastic equations,
one has to decompose matrix $D$ as following (note that transpose is not conjugate):
\[ D = \frac{1}{2} B B^T \]
This is the form of so called Takagi's factorization, which is described in detail in~\citationneeded.

To simplify formulas, we will first find factorization of $D' = B' B'^T$, where $D = \left( \kappa_{12} \lvert \psi_1 \rvert^2 / 2 \right) D'$,
using the adaptation of the algorithm from~\citationneeded.
First, we need to find eigenvalues and eigenvectors of $D'D'^H$; this can be done analytically:
\[ \lambda_{1,2} = \frac{1}{2} \left( x^2 - 2 \beta - x \sqrt{x^2 - 4 \beta} \right), \]
\[ \lambda_{3,4} = \frac{1}{2} \left( x^2 - 2 \beta + x \sqrt{x^2 - 4 \beta} \right), \]
\[ v_1 = \begin{pmatrix} 0 & \frac{2 - x - \sqrt{x^2 - 4 \beta}}{2 \alpha} & 0 & 1 \end{pmatrix}^T, \]
\[ v_2 = \begin{pmatrix} \frac{2 - x - \sqrt{x^2 - 4 \beta}}{2 \alpha^*} & 0 & 1 & 0 \end{pmatrix}^T, \]
\[ v_3 = \begin{pmatrix} 0 & \frac{2 - x + \sqrt{x^2 - 4 \beta}}{2 \alpha} & 0 & 1 \end{pmatrix}^T, \]
\[ v_4 = \begin{pmatrix} \frac{2 - x + \sqrt{x^2 - 4 \beta}}{2 \alpha^*} & 0 & 1 & 0 \end{pmatrix}^T, \]
where $x = \lvert \alpha \rvert^2 + \beta + 1$.

Since eigenvalues have multiplicity greater than 1, we cannot simply construct matrix $B'$ from eigenvectors;
we have to use modified set instead:
\begin{align*}
q_1 = D' v_1^* + \sqrt{\lambda_1} v_1, & &
q_2 = D' v_2^* - \sqrt{\lambda_2} v_2, \\
q_3 = D' v_3^* + \sqrt{\lambda_3} v_3, & &
q_4 = D' v_4^* - \sqrt{\lambda_4} v_4.
\end{align*}
Now, one can prove that if $u_i$ are normalized $v_i$, and $U$ is the matrix, constructed from $u_i$ vectors,
the following equation is valid:
\[ D' = U \textrm{diag} \left( \sqrt{\lambda{1}}, -\sqrt{\lambda{2}}, \sqrt{\lambda{3}}, -\sqrt{\lambda{4}} \right) U^T \]
This means that we have found the decomposition for $D'$:
\[ B' = U \textrm{diag} \left( \sqrt[4]{\lambda{1}}, i \sqrt[4]{\lambda{2}}, \sqrt[4]{\lambda{3}}, i \sqrt[4]{\lambda{4}} \right) \]
Therefore, the decomposition of the initial matrix $D$ looks like:
\begin{equation}
\label{noise_decomposition}
B = \sqrt{\kappa_{12} \lvert \psi_1 \rvert^2} B'
\end{equation}

\subsection*{Stochastic equations with noise terms}

Now we are ready to write down the full set of stochastic equations using the master equation in form~(\ref{full_wigner_equation})
and noise matrix decomposition~(\ref{noise_decomposition}):
\[
\frac{\partial \psi_1}{\partial t} = i \left( \hat{A}_1 +
\kappa_{111} \left( \frac{9}{2} \lvert \psi_1 \rvert^2 - \frac{9}{4} - \frac{3}{2} \lvert \psi_1 \rvert^4 \right) +
\kappa_{12} \left( \frac{1}{2} - \lvert \psi_2 \rvert^2 \right) \right) \psi_1 +
\]
\[
\frac{\partial \psi_2}{\partial t} = i \left( - \hat{K}_2 + \Gamma_{22} \left( 1 - \lvert \psi_2 \rvert^2 \right) +
\Gamma_{12} \left( \frac{1}{2} - \lvert \psi_1 \rvert^2 \right) \right) \psi_2,
\]

\section*{Implementation}

The model explained above is implemented using C++ and CUDA.
Calculation is performed in three steps: first, GPE for one-component gas is solved, producing the steady state.
Then there is an optional equilibration phase (corresponding to the time before first $\frac{\pi}{2}$ pulse),
followed by the $\frac{\pi}{2}$ pulse and evolution phase.
During the evolution phase, each step propagates system state for given time step and prepares data necessary for rendering.
Preparation includes second $\frac{\pi}{2}$ pulse
(which does not spoil original wave-functions, so that they could be propagated further),
particle density calculation, projection on two orthogonal planes and writing these projections to textures for rendering.

Program uses $128\times32\times32$ lattice and calculates propagation in 1 or 4 ensembles
(for quantum noise turned off or on, correspondingly), giving the following times for different stages:

\begin{itemize}
\item Steady state calculation: 3.4 s (20 $\mu$s time step)
\item Equilibration phase (if any): 12.9 s (20 ms with 40 $\mu$s time step) per ensemble
\item Evolution: 49.2 ms per step per ensemble
\end{itemize}

When number of ensembles multiplied by number of lattice points is low,
calculation speed is affected mainly by rendering functions and decreases linearly with each additional ensemble;
with lots of ensembles (or dense lattice), the bottleneck is FFT,
and calculation speed depends on number of ensembles as $O(N\log{N})$.

\section*{Simulation results}

\begin{figure}
\begin{center}
\subfloat[Reference simulation]{\includegraphics[width=2.5in]{evolution_reference.png}}
\qquad
\subfloat[Split-step simulation]{\includegraphics[width=2.5in]{evolution_losses_detuning_40mus.png}}
\end{center}
\caption{Column density of the $\vert2\rangle$-component, 150000 atoms}
\label{evolution_vs_reference}
\end{figure}

The results produced by the program seems to be close to experiments and to reference simulations from \cite{Anderson2009}.
As an example consider figure~\ref{evolution_vs_reference},
which shows the evolution of the density of $\vert2\rangle$-state.
It contains the image from \cite{Anderson2009} and the result of split-step simulation with the same model parameters:
\[ N = 150000 \]
\[ a_{11} = 100.40 a_0, a_{12} = 97.66 a_0, a_{22} = 95.00 a_0 \]
\[ f_x = 11.962 \textrm{ Hz}, f_y = f_z = 97.62 \textrm{ Hz} \]
\[ m = 87 m_p \textrm{ (Rubidium)}\]
\[ \Delta = 1 \textrm{ Hz} \]
Although there are some discrepancies,
caused by different scales and the fact that reference image simulates condensate after expansion,
the evolution of cloud structure goes similarly on both images.

\begin{figure}
\begin{center}
\subfloat[Reference simulation and experimental data (courtesy of Russell Anderson)]{\includegraphics[width=2.5in]{visibility_reference_7e4.png}}
\qquad
\subfloat[Split-step simulation]{\includegraphics[width=2.5in]{visibility_70k.pdf}}
\end{center}
\caption{Visibility over time and revival, 70000 atoms}
\label{visibility_vs_reference}
\end{figure}

Then we may try to reproduce another experimental result, namely, visibility measurements.
Visibility can be described as:
\[ V = \max_{\phi} \frac{\lvert N_1(\phi) - N_2(\phi) \rvert}{N_1(\phi) + N_2(\phi)}, \]
where $N_1(\phi)$ and $N_2(\phi)$ are numbers of atoms in two states after $\frac{\pi}{2}$-pulse,
described by rotation matrix \ref{rotation_matrix}.
One can work out the analytical formula for visibility, which looks like:
\[ V = \frac{2 \lvert \langle \psi_1 \vert \psi_2 \rangle \rvert}{N_1 + N_2}, \]
where $N_1$ and $N_2$ are numbers of atoms in two states without any pulses (with noise term subtracted, if necessary).

As can be seen on figure~\ref{visibility_vs_reference},
split-step simulation results for 70000 atoms lie close to reference simulation (blue dotted line),
but start to differ from experimental results after approximately 400 ms.
This can be explained by the incorrect values of loss terms, or by quantum noise---as you can see,
simulation with noise seems to be more close to the experiment.

\begin{figure}
\includegraphics[width=4.5in]{visibility_150k.pdf}
\caption{Visibility over time for different quantum noise settings, 150000 atoms}
\label{visibility_noise}
\end{figure}

\begin{figure}
\includegraphics[width=4.5in]{visibility_10k.pdf}
\caption{Visibility over time, 10000 atoms}
\label{visibility_noise_10k}
\end{figure}

\begin{figure}
\includegraphics[width=4.5in]{particles_150k.pdf}
\caption{Visibility over time for different quantum noise settings, 150000 atoms}
\label{particle_number}
\end{figure}

Current model of quantum noise seems to be affected by lattice size.
As can be seen on figure~\ref{visibility_noise}, for denser lattice the results diverge noticeably from expected picture.
On the other hand, number of ensembles does not play a big role (at least, for large amounts of atoms).
For low amounts of atoms, simulation with noise gives ugly pictures like figure~\ref{visibility_noise_10k}.
Noise also contributes in particle loss over time,
see figure~\ref{particle_number} for particle number changes over time for different noise settings.

\begin{figure}
\includegraphics[width=4.5in]{axial_view.pdf}
\caption{Axial view over time}
\label{axial_view}
\end{figure}

\begin{figure}
\includegraphics[width=4.5in]{axial_pi_pulse.pdf}
\caption{Axial view over time, with $\pi$-pulse after 30ms}
\label{axial_pi_pulse}
\end{figure}

Next in line is a visualisation technique, proposed by Russell Anderson.
The idea is to project density of each state to $z$-axis
and render the difference between the density of two states along this axis in time.
The difference is divided by the sum of the densities before rendering in order to correspond to spin projection.
Fig~\ref{axial_view} shows the example of such visualisation for detuning $\Delta = -41 \textrm{ Hz}$.
The noise at the borders of the condensate is caused by the low total density.

The last exhibit is the axial view of the same system, but with additional $\pi$-pulse after 30ms,
which can be described by rotation matrix:
\[
\begin{pmatrix}
	\psi^\prime_1 \\	\psi^\prime_2
\end{pmatrix} =
\frac{1}{\sqrt{2}} \begin{pmatrix}
	1 & -i \\ -i & 1
\end{pmatrix} \cdot
\frac{1}{\sqrt{2}} \begin{pmatrix}
	1 & -i \\ -i & 1
\end{pmatrix}
\begin{pmatrix}
	\psi_1 \\ \psi_2
\end{pmatrix} =
\frac{1}{2} \begin{pmatrix}
	0 & -i \\ -i & 0
\end{pmatrix}
\begin{pmatrix}
	\psi_1 \\ \psi_2
\end{pmatrix}.
\]
The result can be seen on fig~\ref{axial_pi_pulse}.
Comparing it to the fig~\ref{axial_view}, you can see the so called 'revival' ---
the amplitude of spin projection oscillations did not decrease after 120ms.
\end{comment}
