\defaultfontfeatures{Ligatures=TeX, Scale=MatchLowercase, Mapping=tex-text, Contextuals=Swash}

\usepackage[oldstyle]{libertine}
\renewcommand{\captionfont}{\rmfamily}

%\setmathfont[StylisticSet=1]{XITS Math Mod}
%\setmathfont{Asana Math}
\setmathfont{Tex Gyre Pagella Math}
\setmathfont[range={\mathcal,\mathbfcal},StylisticSet=1]{Latin Modern Math}

\DeclareMicrotypeSet{maintext}{family={LinLibertine}}
\UseMicrotypeSet[protrusion,expansion,tracking]{maintext}

% three asterisks, used to separate sub-parts of text
\newcommand{\asterism}{{\upshape\normalfont\symbol{\string"2042}}}
% a symbol used to separate two words
\newcommand{\separator}{{\upshape\normalfont\symbol{\string"25C6}}}
% a fancy bullet (Warning: this code is specific to LinLibertine, other fonts may not have it)
\newcommand{\hbullet}{{\upshape\normalfont\symbol{\string"E001}}}

% Fleurons (which most fonts do not have)
% These symbols interact stangely with some printers, and add a dependency on another font.
% Commenting for now and using what LinLibertine has.
%\newfontface{\fleurons}{JUnicode}
%\newcommand{\vfleuron}{{\upshape\fleurons\symbol{\string"2766}}} % vertical
%\newcommand{\hfleuron}{{\upshape\fleurons\symbol{\string"2767}}} % horizontal
