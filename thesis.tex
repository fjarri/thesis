\documentclass[a4paper,11pt]{report}
%\usepackage{fullpage}

% A bit of magic to make amsfonts work with unicode-math
\usepackage[no-math]{fontspec}
\ExplSyntaxOn
\int_new:N \l_mathcode_minus_int
\int_new:N \l_mathcode_equal_int
\exp_args:Nx \AtBeginDocument {
  \exp_not:n {
    \int_set:Nn \l_mathcode_minus_int { \XeTeXmathcodenum `\- }
    \int_set:Nn \l_mathcode_equal_int { \XeTeXmathcodenum `\= }
  }
  \mathcode \int_eval:n { `\- } = \number \mathcode `\- \scan_stop:
  \mathcode \int_eval:n { `\= } = \number \mathcode `\= \scan_stop:
}
\usepackage{amsmath}
\AtBeginDocument {
  \XeTeXmathcodenum `\- = \l_mathcode_minus_int
  \XeTeXmathcodenum `\= = \l_mathcode_equal_int
}
\ExplSyntaxOff
\usepackage[italic,noendash]{mathastext}

\usepackage{amsfonts}
\usepackage{amssymb}
\usepackage{amsthm}

\usepackage{unicode-math}
\usepackage{dsfont}

\defaultfontfeatures{Scale=MatchLowercase,Mapping=tex-text}
\setmainfont[Ligatures=TeX,Numbers={Lining,Proportional}]{Baskerville}
\setmathfont{XITS Math Mod}
%\setmainfont[Ligatures=TeX,Numbers={Lining,Proportional}]{Palatino}
%\setmathfont{Asana Math}
\setsansfont[Ligatures=TeX,Numbers={Lining,Proportional}]{Helvetica}
\setmonofont{Menlo}
\usepackage{microtype}

\usepackage{polyglossia}
\setmainlanguage[variant=british]{english}

%\usepackage[xetex]{graphicx}
\usepackage[margin=5pt]{subfig}
\usepackage{amsthm}
\usepackage{verbatim}
\usepackage{psfrag}
\usepackage[nottoc]{tocbibind}
\usepackage{environ} % gives \NewEnviron macro
\usepackage{chngcntr} % provides \counterwithin to add appendix letter to theorems, lemmas etc

\usepackage{xcolor}
\definecolor{citation}{HTML}{A60000}
\definecolor{link}{HTML}{3515B0}
\usepackage[unicode,pdfencoding=auto,backref=page,breaklinks,pdftitle={PhD thesis: Two-component BEC dynamics simulations}, pdfauthor={Bogdan Opanchuk},colorlinks,urlcolor=blue,citecolor=citation,linkcolor=link]{hyperref}
%\renewcommand{\backref}[1]{[pp #1]}
\renewcommand*{\backref}[1]{}% for backref < 1.33 necessary
\renewcommand*{\backrefalt}[4]{%
\ifcase #1 % No citations.%
\or
[p #2]%
\else
[pp #2]%
\fi }

% Page layout
% 1.5 interval
\renewcommand{\baselinestretch}{1.5}
\textwidth=15cm
\evensidemargin=-0.28cm
\oddsidemargin=1.13cm
\pretolerance=2000

\newtheorem{theorem}{Theorem}
\newtheorem{lemma}{Lemma}
\newtheorem{definition}{Definition}

\newcommand{\todo}[1]{\textcolor{red}{[#1]}}
\newcommand{\citationneeded}{\textcolor{red}{[citation needed]}}

\newcommand{\balpha}{\mathbf{\alpha}}
\newcommand{\bdelta}{\mathbf{\delta}}
\newcommand{\blambda}{\mathbf{\lambda}}
\newcommand{\bfeta}{\mathbf{\eta}}
\newcommand{\bxi}{\mathbf{\xi}}
\newcommand{\bPsi}{\mathbf{\Psi}}
\newcommand{\bLambda}{\mathbf{\Lambda}}
\newcommand{\bPhi}{\mathbf{\Phi}}
\newcommand{\bXi}{\mathbf{\Xi}}

\newcommand{\avec}{\mathbf{a}}
\newcommand{\evec}{\mathbf{e}}
\newcommand{\jvec}{\mathbf{j}}
\newcommand{\kvec}{\mathbf{k}}
\newcommand{\lvec}{\mathbf{l}}
\newcommand{\mvec}{\mathbf{m}}
\newcommand{\nvec}{\mathbf{n}}
\newcommand{\pvec}{\mathbf{p}}
\newcommand{\svec}{\mathbf{s}}
\newcommand{\xvec}{\mathbf{x}}
\newcommand{\zvec}{\mathbf{z}}
\newcommand{\Zvec}{\mathbf{Z}}

\newcommand{\bpartial}{\mathbf{\partial}}

\newcommand{\ecut}{\epsilon_{\mathrm{cut}}}
\newcommand{\Tr}{\operatorname{Tr}}
\newcommand{\Trace}[1]{\Tr \left\{ #1 \right\}}

\newcommand{\symprod}[1]{\left\{ #1 \right\}_{\mathrm{sym}}}
\newcommand{\pathavg}[1]{\langle #1 \rangle_{\mathrm{paths}}}
\newcommand{\Real}{\operatorname{Re}}
\newcommand{\Imag}{\operatorname{Im}}

\newcommand{\Psivec}{\mathbf{\Psi}}
\newcommand{\Psiop}{\hat{\Psi}}
\newcommand{\Psiopvec}{\hat{\mathbf{\Psi}}}

% {C, H} - i.e. either complex number or operator from Hilbert space
% making it a macro, because I'm not sure what the letter should be
\newcommand{\BasicType}{\todo{remove that}}
\newcommand{\fullbasis}{\mathbb{B}}
\newcommand{\restbasis}{\mathbb{M}}

\newcommand{\eqnref}[1]{(\ref{eqn:#1})}
\newcommand{\figref}[1]{Fig.~\ref{fig:#1}}
\newcommand{\charef}[1]{Chapter~\ref{cha:#1}}
\newcommand{\appref}[1]{Appendix~\ref{cha:appendix:#1}}
\newcommand{\thmref}[1]{Theorem~\ref{thm:#1}}
\newcommand{\lmmref}[1]{Lemma~\ref{lmm:#1}}
\newcommand{\defref}[1]{Definition~\ref{def:#1}}

\NewEnviron{eqn}{\begin{align}\begin{split}\BODY\end{split}\end{align}}
\NewEnviron{eqns}{\begin{align} \BODY \end{align}}
\NewEnviron{eqn*}{\begin{equation*}\begin{split} \BODY \end{split}\end{equation*}}

% Split equation with three columns
% - use && as the second separator
% - write equal signs as ={} to get correct spacing
\NewEnviron{eqn2}{\begin{equation}\begin{alignedat}{2} \BODY \end{alignedat}\end{equation}}
\NewEnviron{eqn2*}{\begin{equation*}\begin{alignedat}{2} \BODY \end{alignedat}\end{equation*}}

\title{Two-component BEC dynamics simulations}
\author{Bogdan Opanchuk}

\begin{document}

\maketitle

\tableofcontents

% =============================================================================
\chapter{Introduction}
% =============================================================================

The problem of calculating the dynamics of quantum systems has been around since the dawn of quantum mechanics itself.
In most cases the exact simulation of such systems is intractable or at least extremely slow due to the exponential growth of the system's Hilbert space with particle number.
The continuing increase of the available computational power has made it possible to apply the direct diagonalisation method for relatively large systems with as many as 100 particles~\cite{Sakmann2009} \todo{actually not a good reference, it's another method?}.
But larger systems, such as Bose-Einstein condensates (\abbrev{bec}s), still remain unreachable for exact simulation approaches.

This lead Feynman to postulate as early as 1982~\cite{Feynman1982} that quantum computers are the most perspective way to simulate quantum systems effectively.
But even today quantum computers of any useable size are not readily available, and prognoses about the speed of their development seem rather grim.
At the moment of writing this thesis, the largest universal quantum computer operates with 6 qubits~\cite{Lanyon2011}.
If one restricts himself to a particular algorithm, somewhat larger numbers are available (8 qubits for quantum factorisation, for instance~\cite{Xu2012}).
Among the problems current quantum computers are facing today are decoherence, circuit errors, and the exponentially growing number of measurements (and, consequently, experiments) one has to perform in order to get the result of a computation.
Therefore, even despite a number of recent developments in the field, it will require a major breakthrough for quantum computers to overcome classical ones in the near future.

That is why, in order to handle existing quantum dynamics problems, approximations of varying accuracy have been developed in parallel with the quantum computing research.
This thesis is dedicated to one of such approaches --- quasiprobabilities.


% =============================================================================
\section{Rationale}
% =============================================================================

The essence of the quasiprobability methods is representing the system's density matrix in the form of a probability distribution, or at least a probability distribution-like function.
This function can then be propagated in time (directly or by means of a Monte-Karlo approach) and used to obtain required observables.
First quasiprobability representations, the Wigner function~\cite{Wigner1932,Dirac1945,Moyal1947} and Husimi Q-function~\cite{Husimi1940} were introduced as early as the first half of the 20th century.
They were followed by the Glauber-Sudarshan P-representation~\cite{Sudarshan1963,Glauber1963b,Glauber1963} and its improved version by Drummond, Gardiner and Walls, the positive-P representation~\cite{Drummond1980,Drummond1981}.
These representations circumvent Feynman's claim (based on the Bell's theorem~\cite{Bell1964}) about the impossibility of simulating quantum systems probabilistically~\cite{Feynman1982}.
They use the complex phase space, have the domain larger than the values of observables predicted by quantum mechanics, and only give the correct values of those observables on average~\cite{Opanchuk2013-bell-sim}.

This thesis has arisen from the task of simulating non-classical effects in \abbrev{bec} experiments, and is focused primarily on this area.
Since different representation may perform better or worse depending on the system in question, we had to pick one that was best suited for our problem.

Q-function is usually difficult to propagate in time, although it can be extremely efficient for the sampling of static states (it will make a brief appearance in \charef{bell-ineq}).
Positive-P representation is characterised by sampling error quickly growing in time.
This can be in some cases handled by exploiting its non-uniqueness and tailoring the exact form of the function for the task, resulting in the gauge-P representation~\cite{Deuar2002}.
Alternatively, one may project the distribution on the required part of Hilbert space, thus preventing it from venturing into ``useless'' states, which will cancel out during measurement, yet still affect the total error \todo{cite projection paper if there is any}.

Wigner distribution is, in general, not positive, which presents problems when simulating systems with a small number of particles.
Fortunately, assuming that the number of particles is large enough (our case), one can make certain approximations (see \secref{wigner-bec:truncation} for details) which will truncate the Wigner function and make it strictly positive.
This turns it into a probability distribution, thus providing a way to reduce the initial master equation to a set of stochastic differential equations, for which an extensive set of numerical integration methods exists.
Of course, it is not the only method for this type of the system; a very perspective group of two-mode variational methods has been used by different research groups~\cite{Li2008,Li2009,Sinatra2011},
but it has some difficulties handling nonlinear losses in \abbrev{bec}s, and the approximation starts to break down in the presence of many populated modes.

This combination of features made the truncated Wigner representation the best choice for the task of simulating the dynamics of bosonic quantum fields, including optical fields~\cite{Drummond1993,Drummond1993a,Corney2006,Corney2008} and \abbrev{bec}s.
For \abbrev{bec} systems, the representation has been successfully used to describe fragmentation~\cite{Isella2005,Isella2006,Gross2011}, dissipative atom transport~\cite{Ruostekoski2005}, dynamically unstable lattice dynamics~\cite{Shrestha2009}, dark solitons~\cite{Martin2010,Martin2010a}, turbulence~\cite{Norrie2005,Norrie2006}, quantum noise and decoherence~\cite{Steel1998,Norrie2006a,Egorov2011}, squeezing~\cite{Opanchuk2012} and entanglement~\cite{Opanchuk2012a}.
\copypaste{The truncated Wigner method is particularly useful in low-dimensional and trap environments, where it has successfully predicted quantum squeezing and phase-diffusion effects, in good agreement with dynamical experiments in photonic quantum soliton propagation~\cite{Carter1987,Corney2008}.}

The comparison of the results of the truncated Wigner method with analytical predictions has generally shown an excellent agreement~\cite{Corney2006,Deuar2007}.
Other quasiprobabilitiy representations, such as the positive-P are known to work better near the threshold of applicability of the truncation condition~\cite{Deuar2007}.
\copypaste{There are a number of studies of applicability that compare the truncated Wigner method with the exact positive-P method~\cite{Chaturvedi2002,Dechoum2004} or, where feasible, Bloch-basis approaches.
The typical result found is that the truncated Wigner method gives correct results out to a
characteristic break time.
At this stage, the accumulated errors can lead to large discrepancies in quantum correlations.
The method is weakest when dealing with nonlinear quantum tunneling~\cite{Drummond1989,Kinsler1991a}, which depends on both long time dynamics and quantum correlations.
Within its domain of applicability the technique is remarkably accurate and stable.
The overall picture of how this method is related to other techniques for quantum dynamics has been recently reviewed~\cite{He2012}.}

Although initially Wigner representation was formulated for a single-mode system, the definition and associated methods were later extended to operate on field operators and wave functions,
which facilitates the phase-space treatment of multimode problems.
First such description was produced by Graham~\cite{Graham1970,Graham1970a}, followed by its usage in various other works~\cite{Steel1998,Gardiner2003,Isella2006,Norrie2006,Norrie2006a,Blakie2008,Martin2010} without a formal introduction of corresponding definitions and accompanying theorems.
A more detailed description was given by Polkovnikov~\cite{Polkovnikov2010} in his review paper of phase-space methods.

Direct numerical integration of the partial differential equation resulting from the application of the Wigner transformation is, in general, very cumbersome, and one has to truncate third-order derivative terms~\cite{Drummond1993,Steel1998,Sinatra2002} and apply projection to remove modes with low population.
This adds to the complexity of the formal description of the method.
Moreover, nonlinear inelastic interactions, which were not approached methodically before, became important.
\copypaste{Accordingly, much of the mathematical derivation of these techniques is not readily available.}
This thesis intends to provide a rigorous formal description of the functional truncated Wigner method for simulating the dynamics of multimode \abbrev{bec}s, along with the examples of its application to existing experiments.
The core of the theory described in this thesis has been published in~\cite{Opanchuk2013}.


% =============================================================================
\section{Thesis structure}
% =============================================================================

This thesis is laid out in the mathematical tradition, with the formalism preceeding its application.
Supplementary information, methods, and parts of the formalism that are not directly connected to quantum mechanics can be found in the Appendices.

First three chapters contain the foundation for the functional Wigner transformation.
\charef{mm-wigner} introduces the Wigner transformation.
In this chapter we also extend it to work with sets of single-mode operators.
This chapter contains proofs of known properties of the transformation, and presents the single-mode transformation in a way facilitating further extension into the functional domain.
\charef{wigner} introduces the functional calculus, restricted basis formalism and their application to bosonic field operators.
We then intriduce the functional Wigner transformation and prove several main theorems that govern the transformation of field operators and the measurement of their moments.
Finally, \charef{wigner-spec} uses these theorems to derive general identities which can be used to transform parts of the initial master equation describing a \abbrev{bec}.
\copypaste{We focus especially on the problem of nonlinear damping.
This is a dominant relaxation mechanism in BEC systems, and is often ignored or (incorrectly) approximated using linear loss terms.}

\charef{wigner-bec} applies the formalism from the previous chapters to transform a master equation describing a \abbrev{bec} to a set of stochastic differential equations (\abbrev{sde}s).
\copypaste{We successively reduce the problem in its initial form, the master equation for bosonic field operators, to a system of stochastic differential equations, which have significantly lower computational complexity.
While there is a price for making the truncation approximation, we emphasize that this is a systematic expansion in a small parameter, $1/N$, where $N$ is the particle number. Such expansions are also relevant to stochastic diagram techniques~\cite{Chaturvedi1999}, which can be
used to formally calculate order-by-order behaviour in such equations.}

\copypaste{We derive the correct Fokker-Planck drift and noise terms for general multicomponent damping using the $1/N$ expansion, which transforms to an expansion in the inverse particle density for quantum fields.
Even in the single-component case, the drift term has both a leading (classical) term and a quantum noise correction to the damping. This is needed to predict the loss behaviour correctly,
and is important in high-accuracy simulations. Such corrections --- both in the drift and noise --- are relevant to topics like EPR correlations, entanglement and quantum squeezing in the presence of nonlinear reservoirs, a topic of increasing importance in areas ranging from quantum optics and BEC physics to nanomechanical oscillators~\cite{Chaturvedi1977,Reid1986a,Rabl2004}.}

\copypaste{We derive the resulting stochastic differential equations from the functional Fokker-Planck equations, and show when the corresponding truncation approximations are applicable.
The final equations can be treated using standard computational techniques for solving ordinary
and partial stochastic differential equations~\cite{Drummond1990,Werner1997,Wilkie2005}.
There are code generator packages and public domain websites with code available for this purpose~\cite{Collecutt2001,Dennis2013}.}

Next three chapters describe several applications of the truncated Wigner formalism.
\charef{bec-noise} is dedicated to the theoretical description of the quantum interferometry experiments performed in Swinburne University.
It shows how truncated Wigner can predict the visibility dynamics (including its decay) in the experiment, along with the growth of the phase noise.
\charef{bec-squeezing} illustrates how truncated Wigner can be used to calculate squeezing in the interferometry experiments with complex dynamics.
\charef{epr-two-well} shows how truncated Wigner was applied to predict the entanglement in the two-well system with nonlinear interactions.

\charef{bell-ineq} stands a bit aside, since it mostly makes use of the different quasiprobability method, Husimi Q-function (in its \abbrev{su(2)} variation).
The Q-function is used to sample the ``Shr\"{o}dinger cat'' state and to show that the probabilistical methods can indeed violate the Bell inequality.

Finally, \charef{conclusion} summarizes the thesis and discusses some possible directions of the development in the field of quasiprobability representations.

The thesis includes several Appendices which deal with auxiliary topics.
\appref{c-numbers} briefly describes the relaxed complex (Wirtinger) differentiation and associated integration, which are commonly used in the field of quasiprobabilities.
\appref{func-calculus} applies these differentiation rules to define the similar formalism for functionals.
\appref{fpe-sde} contains several theorem which deal with the equivalence correspondence between Fokker-Planck equation and a set of stochastic differential equations, for complex variables and functional form.
\appref{numerical} outlines numerical methods used in the thesis to obtain simulation results.


% theory
% =============================================================================
\section{Field operator calculus}
% =============================================================================

Field operators can simplify both the analysis of a master equation, and its transformation to a \abbrev{fpe} using phase-space methods, which is the main topic of this thesis.
In this section we are going to outline the functional operator calculus.
It is quite similar to the calculus of functional operators, described in \appref{func-calculus}).
The similartiy and interconnection of field operators and functional operators will become even more evident during the description of the functional Wigner transformation in \charef{wigner}.

Multimode field of a $C$-component \abbrev{bec} in $D$ effective dimensions is described by field operators $\Psiop_j^{\dagger}(\xvec)$ and $\Psiop_j(\xvec)$, where $\Psiop_j^{\dagger}$ creates a bosonic atom of spin $j$, $j = 1 \ldots C$ at a location defined by a $D$-component coordinate vector $\xvec$, and $\Psiop_j$ destroys one.
We will use the same scheme as with functions of coordinates in \appref{func-calculus}, abbreviating $\Psiop_j \equiv \Psiop_j(\xvec)$ and $\Psiop_j^\prime \equiv \Psiop_j(\xvec^\prime)$.

The commutators are
\begin{eqn}
\label{eqn:wigner:op-calculus:commutators}
    [ \Psiopf_j, \Psiopf_k^{\prime} ]
    & = [ \Psiopf_j^\dagger, \Psiopf_k^{\prime\dagger} ]
    = 0, \\
    [ \Psiopf_j, \Psiopf_k^{\prime\dagger} ]
    & = \delta_{jk} \delta(\xvec^\prime-\xvec).
\end{eqn}
Field operators have type $\Psiop_j \in (\mathbb{R}^D \rightarrow \mathbb{H}_j) \equiv \mathbb{FH}_j$, where per-component Hilbert spaces $\mathbb{H}_j$ consitute the system Hilbert space $\mathbb{H} = \bigotimes_{j=1}^C \mathbb{H}_j$.
Field operators can be decomposed using a single-particle orthonormal basis (see~\eqnref{func-calculus:basis}):
\begin{eqn}
\label{eqn:wigner:op-calculus:field}
    \Psiopf_j = \sum_{\nvec \in \fullbasis_j} \phi_{j,\nvec} \hat{a}_{j,\nvec}.
\end{eqn}
Note that each component can have its own basis.
Single mode operators $\hat{a}_{j,\nvec}$ obey commutation relations~\eqnref{mm-wigner:mm:commutators}, with the pair $j,\nvec$ serving as a mode identifier.

In practice, one cannot use an infinitely sized basis in numerical calculations; some subset of modes is always chosen.
To take this into account we will restrict ourselves to a subset of modes from full basis for each component: $\restbasis_j \subset \fullbasis_j$.
New restricted field operators are
\begin{eqn}
\label{eqn:wigner:op-calculus:restricted-field}
    \Psiop_j = \sum_{\nvec \in \restbasis_j} \phi_{j,\nvec} \hat{a}_{j,\nvec}.
\end{eqn}
They map coordinates to a restricted Hilbert subspaces: $\Psiop_j \in (\mathbb{R} \rightarrow \mathbb{H}_{\restbasis_j}) = \mathbb{FH}_{\restbasis_j}$.
Because of the restricted nature of these operators, commutation relations~\eqnref{wigner:op-calculus:commutators} no longer apply.
The following ones should be used instead:
\begin{eqn}
\label{eqn:wigner:op-calculus:restricted-commutators}
    \left[ \Psiop_j, \Psiop_k^\prime \right]
    & = \left[ \Psiop_j^\dagger, \Psiop_k^{\prime\dagger} \right] = 0, \\
    \left[ \Psiop_j, \Psiop_k^{\prime\dagger} \right]
    & = \delta_{jk} \delta_{\restbasis_j}(\xvec^\prime, \xvec).
\end{eqn}

Let us now find the expression for high-order commutators of restricted field operators, analogous to the one for single-mode operators which can be found in the book by Louisell~\cite{Louisell1990}.

\begin{lemma}
\label{lmm:wigner:op-calculus:moment-commutators}
    For $\Psiop \in \mathbb{FH}_{\restbasis}$
    \begin{eqn*}
        \left[ \Psiop, ( \Psiop^{\prime\dagger} )^l \right]
        & = l \delta_{\restbasis} (\xvec^\prime, \xvec) ( \Psiop^{\prime\dagger} )^{l-1}, \\
        \left[ \Psiop^\dagger, ( \Psiop^\prime )^l \right]
        & = - l \delta_{\restbasis}^* (\xvec^\prime, \xvec) ( \Psiop^\prime )^{l-1}.
    \end{eqn*}
\end{lemma}
\begin{proof}
Proved by induction.
Given that we know the expression for $\left[ \Psiop, ( \Psiop^{\prime\dagger} )^{l-1} \right]$,
the commutator of higher order can be expanded as
\begin{eqn}
    \left[ \Psiop, ( \Psiop^{\prime\dagger} )^l \right]
    & = \Psiop ( \Psiop^{\prime\dagger} )^l - ( \Psiop^{\prime\dagger} )^l \Psiop \\
    & = (
        \delta_{\restbasis} (\xvec^\prime, \xvec) + \Psiop^{\prime\dagger} \Psiop
    ) ( \Psiop^{\prime\dagger} )^{l-1}
    - ( \Psiop^{\prime\dagger} )^l \Psiop \\
    & = \delta_{\restbasis} (\xvec^\prime, \xvec) ( \Psiop^{\prime\dagger} )^{l-1}
    + \Psiop^{\prime\dagger} (
        \Psiop ( \Psiop^{\prime\dagger} )^{l-1}
        - ( \Psiop^{\prime\dagger} )^{l-1} \Psiop
    ) \\
    & = \delta_{\restbasis} (\xvec^\prime, \xvec) ( \Psiop^{\prime\dagger} )^{l-1}
    + \Psiop^{\prime\dagger} [
        \Psiop, ( \Psiop^{\prime\dagger} )^{l-1}
    ].
\end{eqn}
Now we can get the commutator of any order starting from the known relation~\eqnref{wigner:op-calculus:restricted-commutators}.
\end{proof}

A further generalisation of these relations is
\begin{lemma}
\label{lmm:wigner:op-calculus:functional-commutators}
    For $\Psiop \in \mathbb{FH}_{\restbasis}$
    \begin{eqn*}
        \left[ \Psiop, f( \Psiop^\prime, \Psiop^{\prime\dagger} ) \right]
        & = \delta_{\restbasis} (\xvec^\prime, \xvec) \frac{\partial f}{\partial \Psiop^{\prime\dagger}}, \\
        \left[ \Psiop^\dagger, f( \Psiop^\prime, \Psiop^{\prime\dagger} ) \right]
        & = -\delta_{\restbasis}^* (\xvec^\prime, \xvec) \frac{\partial f}{\partial \Psiop^\prime},
    \end{eqn*}
    where $f(x, y)$ is a function that can be expanded in the power series of $x$ and $y$.
\end{lemma}
\begin{proof}
Let us prove the first relation; the procedure for the second one is the same.
Without loss of generality, we can assume that $f(\Psiop^\prime, \Psiop^{\prime\dagger})$ can be expanded in power series of normally ordered operators (otherwise we can just use commutation relations).
Thus
\begin{eqn}
    \left[ \Psiop, f( \Psiop^\prime, \Psiop^{\prime\dagger} ) \right]
    & = \sum_{r,s} f_{rs} [ \Psiop, (\Psiop^{\prime\dagger})^r (\Psiop^\prime)^s ] \\
    & = \sum_{r,s} f_{rs} [ \Psiop, (\Psiop^{\prime\dagger})^r ] (\Psiop^\prime)^s \\
    & = \sum_{r,s} f_{rs} r \delta_P(\xvec^\prime, \xvec)
        (\Psiop^{\prime\dagger})^{r-1} (\Psiop^\prime)^s \\
    & = \delta_P (\xvec^\prime, \xvec) \frac{\partial f}{\partial \Psiop^{\prime\dagger}}.
    \qedhere
\end{eqn}
\end{proof}

% =============================================================================
\chapter{Functional Wigner transformation}
% =============================================================================

% =============================================================================
\section{Single-mode Wigner representation}
% =============================================================================

We will need the displacement operator which was first introduced by Weyl~\cite{Weyl1950}.

\begin{definition}
	\begin{eqn*}
	\label{eqn:wigner:sm:displacement-op}
		\hat{D}(\lambda, \lambda^*) = \exp(\lambda \hat{a}^\dagger - \lambda^* \hat{a}),
	\end{eqn*}
	where $\hat{a}^\dagger$ and $\hat{a}$ are bosonic creation and annihilation operators, and $\lambda$ is a complex variable.
\end{definition}
Using Baker-Hausdorff theorem to split non-commuting operators in the exponent,
one can find that
\begin{eqn}
\label{eqn:wigner:sm:displacement-derivatives}
	\frac{\partial}{\partial \lambda} \hat{D}(\lambda, \lambda^*)
	& = \hat{D}(\lambda, \lambda^*) (\hat{a}^\dagger + \frac{1}{2} \lambda^*)
	= (\hat{a}^\dagger - \frac{1}{2} \lambda^*) \hat{D}(\lambda, \lambda^*), \\
	-\frac{\partial}{\partial \lambda^*} \hat{D}(\lambda, \lambda^*)
	& = \hat{D}(\lambda, \lambda^*) (\hat{a} + \frac{1}{2} \lambda)
	= (\hat{a} - \frac{1}{2} \lambda) \hat{D}(\lambda, \lambda^*).
\end{eqn}

Using the displacement operator we can define Wigner transformation.

\begin{definition}
\label{def:wigner:sm:w-transformation}
	Wigner transformation converts an operator $\hat{A}$ on a Hilbert space to a complex-valued function $\mathcal{W}[\hat{A}](\alpha, \alpha^*)$ on phase space.
	\begin{eqn*}
		\mathcal{W}\left[\hat{A}\right]
		= \frac{1}{\pi^2} \int d^2 \lambda \exp(-\lambda \alpha^* + \lambda^* \alpha)
			\Trace{ \hat{A} \hat{D}(\lambda, \lambda^*) }.
	\end{eqn*}
	The backward transformation (called the Weyl transformation) gives back the operator:
	\begin{eqn*}
		\mathcal{W}^{-1}[f]
		= \frac{1}{\pi} \int d^2 \xi \hat{D}^{\dagger}(\xi, \xi^*)
			\int d^2 \eta \exp(-\eta \xi^* + \eta^* \xi) f(\eta, \eta^*).
	\end{eqn*}
\end{definition}

It is easy to demonstrate that Wigner and Weyl transformations define a bijection $\mathbb{H} \leftrightarrow (\mathbb{C} \rightarrow \mathbb{C})$ \todo{at least, for Hilbert-Schmidt operators and square-integrable functions~\cite{Cahill1969}}.
Let us assume $\hat{A} \equiv \mathcal{W}^{-1}[f]$.
Then, using the fact that $\Trace{\hat{D}^{\dagger}(\xi, \xi^*) \hat{D}(\lambda, \lambda^*)} = \pi \delta(\Real \xi - \Real \lambda) \delta(\Imag \xi - \Imag \lambda)$~\cite{Cahill1969} and applying \lmmref{c-numbers:fourier-of-moments}:
\begin{eqn}
	\mathcal{W}[\hat{A}]
	={} & \frac{1}{\pi^3} \int d^2 \lambda \exp(-\lambda \alpha^* + \lambda^* \alpha) \\
	&	\times \Trace{
			\int d^2 \xi \hat{D}^{\dagger}(\xi, \xi^*)
				\int d^2 \eta \exp(-\eta \xi^* + \eta^* \xi) f(\eta, \eta^*)
			\hat{D}(\lambda, \lambda^*)
		} \\
	={} & \frac{1}{\pi^2} \int d^2 \lambda \int d^2 \eta
	 	\exp(-\lambda \alpha^* + \lambda^* \alpha)
		\exp(-\eta \lambda^* + \eta^* \lambda) f(\eta, \eta^*) \\
	={} & \frac{1}{\pi^2} \int d^2 \eta \int d^2 \lambda
	 	\exp(-\lambda (\alpha^* - \eta^*) + \lambda^* (\alpha - \eta)) f(\eta, \eta^*) \\
	={} & \int d^2 \eta \delta(\Real \alpha - \Real \eta) \delta(\Imag \alpha - \Imag \eta) f(\eta, \eta^*) \\
	={} & f(\alpha, \alpha^*).
\end{eqn}
The proof for the other direction looks the same.

\begin{theorem}
\label{thm:wigner:sm:w-real}
	If an operator $\hat{A}$ is Hermitian, its Wigner transformation $W[\hat{A}]$ is a real function.
	\todo{It is a sufficient condition, but is it necessary?}
\end{theorem}
\begin{proof}
Let us calculate the conjugation of $W[\hat{A}]$:
\begin{eqn}
	(W[\hat{A}])^*
	& = \frac{1}{\pi^2} \int d^2 \lambda \exp(-\lambda^* \alpha + \lambda \alpha^*)
		(\Trace{ \hat{A} \exp(\lambda \hat{a}^\dagger - \lambda^* \hat{a}) })^*.
\end{eqn}
Changing variables as $\lambda \rightarrow -\lambda$ and using the fact that $(\Tr{\hat{B}})^* = \Tr{\hat{B}^\dagger}$:
\begin{eqn}
	& = \frac{1}{\pi^2} \int d^2 \lambda \exp(\lambda^* \alpha - \lambda \alpha^*)
		\Trace{ \hat{A}^\dagger \exp(-\lambda^* \hat{a} + \lambda \hat{a}^\dagger) } \\
	& = W[\hat{A}^\dagger].
\end{eqn}
Therefore, if $\hat{A} = \hat{A}^\dagger$, the conjugate of $W[\hat{A}]$ is equal to itself and therefore is real.
\end{proof}

\begin{definition}
\label{def:wigner:sm:w-function}
	Wigner function is a Wigner transformation of the density matrix:
	\begin{eqn*}
		W(\alpha, \alpha^*) \equiv \mathcal{W}[\hat{\rho}].
	\end{eqn*}
	The Wigner function always exists for any density matrix~\cite{Gardiner2004}.
	Since the density matrix is Hermitian, according to \thmref{wigner:sm:w-real} $W$ is a real function.
\end{definition}

In some cases it will be convenient to use Wigner function in form~\cite{Gardiner2004}
\begin{eqn}
	W (\alpha, \alpha^*)
	= \frac{1}{\pi^2} \int d^2 \lambda \exp(-\lambda \alpha^* + \lambda^* \alpha)
		\chi_W (\lambda, \lambda^*),
\end{eqn}
where $\chi_W (\lambda, \lambda^*)$ is the characteristic function
\begin{eqn}
	\chi_W (\lambda, \lambda^*)	= \Trace{ \hat{\rho} \hat{D}(\lambda, \lambda^*) }.
\end{eqn}

\begin{theorem}[Operator correspondences]
\label{thm:wigner:sm:correspondences}
	For any Hilbert-Schmidt operator $\hat{A}$ \todo{needs definition?}
	\begin{eqn*}
		\mathcal{W} [ \hat{a} \hat{A} ]
			& = \left( \alpha + \frac{1}{2} \frac{\partial}{\partial \alpha^*} \right) \mathcal{W}[\hat{A}],
		\quad
		\mathcal{W} [ \hat{a}^\dagger \hat{A} ]
			= \left( \alpha^* - \frac{1}{2} \frac{\partial}{\partial \alpha} \right) \mathcal{W}[\hat{A}], \\
		\mathcal{W} [ \hat{A} \hat{a} ]
			& = \left( \alpha - \frac{1}{2} \frac{\partial}{\partial \alpha^*} \right) \mathcal{W}[\hat{A}],
		\quad
		\mathcal{W} [ \hat{A} \hat{a}^\dagger ]
			= \left( \alpha^* + \frac{1}{2} \frac{\partial}{\partial \alpha} \right) \mathcal{W}[\hat{A}].
	\end{eqn*}
\end{theorem}
\begin{proof}
We will prove the first correspondence.
First, let us transform the trace using~\eqnref{wigner:sm:displacement-derivatives}:
\begin{eqn}
	\Trace{ \hat{a} \hat{A} \hat{D} }
	= \Trace{ \hat{A} \hat{D} \hat{a}}
	= \Trace{ \hat{A} \left(
		-\frac{\partial}{\partial \lambda^*}
		-\frac{1}{2} \lambda
	\right) \hat{D}}
	= \left(
		-\frac{\partial}{\partial \lambda^*}
		-\frac{1}{2} \lambda
	\right) \Trace{ \hat{A} \hat{D}}
\end{eqn}
Using this:
\begin{eqn}
	\mathcal{W} [ \hat{a} \hat{A} ]
	& = \frac{1}{\pi^2} \int d^2 \lambda \exp(-\lambda \alpha^* + \lambda^* \alpha)
		\Trace{ \hat{a} \hat{A} \hat{D}(\lambda, \lambda^*) } \\
	& = \frac{1}{\pi^2} \int d^2 \lambda \exp(-\lambda \alpha^* + \lambda^* \alpha)
		\left(
			-\frac{\partial}{\partial \lambda^*}
			-\frac{1}{2} \lambda
		\right)
		\Trace{ \hat{A} \hat{D}(\lambda, \lambda^*) } \\
	& = \frac{1}{2} \frac{\partial}{\partial \alpha^*} \mathcal{W} [\hat{A}]
	- \frac{1}{\pi^2} \int d^2 \lambda \exp(-\lambda \alpha^* + \lambda^* \alpha)
		\frac{\partial}{\partial \lambda^*}
		\Trace{ \hat{A} \hat{D}(\lambda, \lambda^*) }.
\end{eqn}
The second term is almost the definition of the Wigner function, except for the partial derivative over $\lambda^*$.
Moving it using integration by parts and \lmmref{c-numbers:zero-integrals} ($\hat{A}$ is Hilbert-Schmidt, which means that $\Trace{\hat{A} \hat{D}}$ is square-integrable~\cite{Cahill1969}):
\begin{eqn}
	= \frac{1}{2} \frac{\partial}{\partial \alpha^*} \mathcal{W} [\hat{A}]
	+ \frac{1}{\pi^2} \int d^2 \lambda \left(
		\frac{\partial}{\partial \lambda^*} \exp(-\lambda \alpha^* + \lambda^* \alpha)
	\right)
	\Trace{ \hat{A} \hat{D}(\lambda, \lambda^*) } \\
	= \left( \alpha + \frac{1}{2} \frac{\partial}{\partial \alpha^*} \right) \mathcal{W} [\hat{A}].
	\qedhere
\end{eqn}
\end{proof}

\begin{lemma}
\label{lmm:wigner:sm:moments-from-chi}
	\begin{eqn*}
		\langle \symprod{ \hat{a}^r (\hat{a}^\dagger)^s } \rangle
		= \left.
			\left( \frac{\partial}{\partial \lambda} \right)^s
			\left( -\frac{\partial}{\partial \lambda^*} \right)^r
			\chi_W (\lambda, \lambda^*)
		\right|_{\lambda=0}.
	\end{eqn*}
\end{lemma}
\begin{proof}
The exponent in the expression for $\chi_W$ can be expanded as
\begin{eqn}
	\exp (\lambda \hat{a}^\dagger - \lambda^* \hat{a})
	= \sum_{r,s}
		\frac{(-\lambda^*)^r \lambda^s}{r!s!}
		\symprod{ \hat{a}^r (\hat{a}^\dagger)^s }.
\end{eqn}
Thus
\begin{eqn}
	\chi_W(\lambda, \lambda^*)
	& = \sum_{r,s}
		\frac{(-\lambda^*)^r \lambda^s}{r!s!}
		\Trace{
			\hat{\rho} \symprod{ \hat{a}^r (\hat{a}^\dagger)^s }
		} \\
	& = \sum_{r,s}
		\frac{(-\lambda^*)^r \lambda^s}{r!s!}
		\langle \symprod{ \hat{a}^r (\hat{a}^\dagger)^s } \rangle
\end{eqn}
Apparently, the application of $(\partial / \partial \lambda)^s$ and $(-\partial / \partial \lambda^*)^r$ will eliminate all lower order moments,
and setting $\lambda = 0$ afterwards will eliminate all higher order moments,
leaving only $\symprod{ \hat{a}^r (\hat{a}^\dagger)^s }$:
\begin{eqn}
	\left.
		\left( \frac{\partial}{\partial \lambda} \right)^s
		\left( -\frac{\partial}{\partial \lambda^*} \right)^r
		\chi_W (\lambda, \lambda^*)
	\right|_{\lambda=0}
	= r! s! \frac{1}{r! s!}
		\langle \symprod{ \hat{a}^r (\hat{a}^\dagger)^s } \rangle
	= \langle \symprod{ \hat{a}^r (\hat{a}^\dagger)^s } \rangle.
	\qedhere
\end{eqn}
\end{proof}

Now we can get the final relation.
\begin{theorem}
\label{thm:wigner:sm:moments}
	\begin{eqn*}
		\int d^2\alpha\, \alpha^r (\alpha^*)^s W(\alpha, \alpha^*)
		= \langle \symprod{ \hat{a}^r (\hat{a}^\dagger)^s } \rangle
	\end{eqn*}
\end{theorem}
\begin{proof}
By definition of the Wigner function:
\begin{eqn}
	\int d^2\alpha\, \alpha^r (\alpha^*)^s W(\alpha, \alpha^*)
	= \frac{1}{\pi^2}
		\int d^2\alpha\, \alpha^r (\alpha^*)^s
		\int d^2\lambda \exp(-\lambda \alpha^* + \lambda^* \alpha)
		\chi_W (\lambda, \lambda^*).
\end{eqn}
Integrating over $\alpha$ using \lmmref{c-numbers:fourier-of-moments}:
\begin{eqn}
	= \int d^2\lambda
		\left(
			\left( \frac{\partial}{\partial \lambda} \right)^s
			\left( -\frac{\partial}{\partial \lambda^*} \right)^r
			\delta(\Real \lambda) \delta(\Imag \lambda)
		\right)
		\chi_W (\lambda, \lambda^*).
\end{eqn}
Integrating by parts and eliminating terms which fit \lmmref{c-numbers:zero-delta-integrals}:
\begin{eqn}
	& = \int d^2\lambda
		\delta(\Real \lambda) \delta(\Imag \lambda)
		\left( \frac{\partial}{\partial \lambda} \right)^s
		\left( -\frac{\partial}{\partial \lambda^*} \right)^r
		\chi_W (\lambda, \lambda^*) \\
	& = \left.
		\left( \frac{\partial}{\partial \lambda} \right)^s
		\left( -\frac{\partial}{\partial \lambda^*} \right)^r
		\chi_W (\lambda, \lambda^*)
	\right|_{\lambda=0}.
\end{eqn}
Now, recognising the final expression as a part of \lmmref{wigner:sm:moments-from-chi}, we immideately get the statement of the theorem.
\end{proof}

% =============================================================================
\section{Sets of single-mode operators}
% =============================================================================

We start from the set of single-mode operators $\hat{a}_j$, which obey bosonic commutation relations:
\begin{eqn}
\label{eqn:wigner:mm-aux:commutators}
	[ \hat{a}_j, \hat{a}_k ] & = [ \hat{a}_j^\dagger, \hat{a}_k^\dagger ] = 0, \\
	[ \hat{a}_j, \hat{a}_k^\dagger ] & = \delta_{jk}.
\end{eqn}

In order to work with the moments of multimode operators we will need the equations for commutators of arbitrary single-mode operator products.

\begin{lemma}
\label{lmm:wigner:mm-aux:high-order-commutators}
	\begin{eqn*}
		[ \hat{a}_n, \hat{a}_{m_1}^\dagger \ldots \hat{a}_{m_k}^\dagger ]
		& = \sum_{i=1}^k \delta_{n m_i}
			\prod_{j=1,j \ne i}^k \hat{a}_{m_j}^\dagger, \\
		[ \hat{a}_n^\dagger, \hat{a}_{m_1} \ldots \hat{a}_{m_k} ]
		& = - \sum_{i=1}^k \delta_{n m_i}
			\prod_{j=1,j \ne i}^k \hat{a}_{m_j}.
	\end{eqn*}
\end{lemma}
\begin{proof}
Let us find the expression for the first commutator by induction.
Providing that we know the expression for $[ \hat{a}_n, \hat{a}_{m_1}^\dagger \ldots \hat{a}_{m_{k-1}}^\dagger ]$,
commutator of order $k$ can be expanded as:
\begin{eqn}
	[ \hat{a}_n, \hat{a}_{m_1}^\dagger \ldots \hat{a}_{m_k}^\dagger ]
	={} & (1 - \delta_{n m_k})
		[ \hat{a}_n, \hat{a}_{m_1}^\dagger \ldots \hat{a}_{m_{k-1}}^\dagger ] \hat{a}_{m_k} \\
	& + \delta_{n m_k} (
		\hat{a}_n \hat{a}_{m_1}^\dagger \ldots \hat{a}_{m_{k-1}}^\dagger \hat{a}_n^\dagger
		- \hat{a}_{m_1}^\dagger \ldots \hat{a}_{m_{k-1}}^\dagger \hat{a}_n^\dagger \hat{a}_n
	).
\end{eqn}
Here we have split the initial commutator into two possible outcomes, depending on whether $n = m_k$.
First term, corresponding to $n \ne m_k$, contains the known commutator of lower order.
In the second term we have substituted $\hat{a}_n$ for $\hat{a}_{m_k}$,
since the delta function outside the parentheses ensures that $n = m_k$.
Swapping $\hat{a}_n^\dagger$ and $\hat{a}_n$ in the last term and, again, recognising the known commutator:
\begin{eqn}
	& = (1 - \delta_{n m_k})
		[ \hat{a}_n, \hat{a}_{m_1}^\dagger \ldots \hat{a}_{m_{k-1}}^\dagger ] \hat{a}_{m_k}
	+ \delta_{n m_k} (
		[ \hat{a}_n, \hat{a}_{m_1}^\dagger \ldots \hat{a}_{m_{k-1}}^\dagger ] \hat{a}_n^\dagger
		+ \hat{a}_{m_1}^\dagger \ldots \hat{a}_{m_{k-1}}^\dagger
	) \\
	& = [ \hat{a}_n, \hat{a}_{m_1}^\dagger \ldots \hat{a}_{m_{k-1}}^\dagger ] \hat{a}_{m_k}
	+ \delta_{n m_k} \hat{a}_{m_1}^\dagger \ldots \hat{a}_{m_{k-1}}^\dagger.
\end{eqn}
Now, starting from the first-order relation $[ \hat{a}_n, \hat{a}_{m_1}^\dagger ] = \delta_{n m_1}$, we can obtain the relation for any order:
\begin{eqn}
	[ \hat{a}_n, \hat{a}_{m_1}^\dagger \hat{a}_{m_2}^\dagger ]
	& = \delta_{n m_1} \hat{a}_{m_2}^\dagger + \delta_{n m_2} \hat{a}_{m_1}^\dagger, \\
	[ \hat{a}_n, \hat{a}_{m_1}^\dagger \hat{a}_{m_2}^\dagger \hat{a}_{m_3}^\dagger ]
	& = \delta_{n m_1} \hat{a}_{m_2}^\dagger \hat{a}_{m_3}^\dagger
	+ \delta_{n m_2} \hat{a}_{m_1}^\dagger \hat{a}_{m_3}^\dagger
	+ \delta_{n m_3} \hat{a}_{m_1}^\dagger \hat{a}_{m_2}^\dagger, \\
	& \ldots
\end{eqn}
Which gives us the statement of the lemma.
\end{proof}

Note that if $n = m_1 = \ldots = m_k$, this boils down to the well-known relation from~\cite{Louisell1990}:
\begin{eqn}
	[ \hat{a}, (\hat{a}^\dagger)^k ] = k (\hat{a}^\dagger)^{k-1}.
\end{eqn}

% =============================================================================
\section{Multimode Wigner representation}
% =============================================================================


Single-mode \defref{mm-wigner:sm:w-transformation} of Wigner transformation can be extended to the case of many modes.
Mode operators $\hat{a}_\nvec$, $\nvec \in \restbasis$, obey bosonic commutation relations:
\begin{eqn}
\label{eqn:mm-wigner:mm:commutators}
	[ \hat{a}_{\mvec}, \hat{a}_{\nvec} ]
	& = [ \hat{a}_{\mvec}^\dagger, \hat{a}_{\nvec}^\dagger ] = 0, \\
	[ \hat{a}_{\mvec}, \hat{a}_{\nvec}^\dagger ] & = \delta_{\mvec,\nvec}.
\end{eqn}

\begin{definition}
\label{def:mm-wigner:mm:w-transformation}
	Let $\blambda$ and $\balpha$ be vectors of $\lambda_{\nvec}$ and $\alpha_{\nvec}$ values respectively.
	Multimode Wigner transformation is
	\begin{eqn*}
		\mathcal{W}[\hat{A}]
		= \frac{1}{\pi^{2|\restbasis|}}
			\int \upd^2 \blambda
			\left(
				\prod_{\nvec \in \restbasis} \exp(-\lambda_{\nvec} \alpha_{\nvec}^* + \lambda_{\nvec}^* \alpha_{\nvec})
			\right)
			\Trace{
				\hat{A}
				\prod_{\nvec \in \restbasis} \hat{D}_{\nvec} (\lambda_{\nvec})
			},
	\end{eqn*}
	where $\hat{D}_{\nvec}(\lambda_{\nvec}) = \exp(\lambda_{\nvec} \hat{a}_{\nvec}^\dagger - \lambda_{\nvec}^* \hat{a}_{\nvec})$, $\int \upd^2 \blambda \equiv \int \prod_{\nvec \in \restbasis} \upd \lambda_{\nvec}$ , and $|\restbasis|$ stands for the cardinality of $\restbasis$.
	Multi-component Weyl transformation is, correspondingly,
	\begin{eqn*}
		\mathcal{W}^{-1}[f]
		= \frac{1}{\pi^{|\restbasis|}} \int \upd^2 \bxi
			\left( \prod_{\nvec \in \restbasis} \hat{D}_{\nvec}^{\dagger}(\xi_{\nvec}) \right)
			\int \upd^2 \bfeta
				\left( \prod_{\nvec \in \restbasis}
					\exp(-\eta_{\nvec} \xi_{\nvec}^* + \eta_{\nvec}^* \xi_{\nvec})
				\right) f(\bfeta).
	\end{eqn*}
\end{definition}

It can be proved, same as in \thmref{mm-wigner:sm:w-real} for the single-mode case, that $W[\hat{A}]$ is real if $\hat{A}$ is Hermitian.
The equality $\mathcal{W}[\mathcal{W}^-1[f]] \equiv f$ is proved analogously to the single-mode case as well.

Corresponding definitions of multimode characteristic function and Wigner function are therefore
\begin{eqn}
	\chi_W (\blambda)
	= \Trace{
		\hat{\rho}
		\prod_{\nvec \in \restbasis} \hat{D}_{\nvec} (\lambda_{\nvec})
	},
\end{eqn}
and
\begin{eqn}
	W (\balpha)
	\equiv \mathcal{W}[\hat{\rho}]
	= \frac{1}{\pi^{2|\restbasis|}} \int \upd^2 \blambda
		\left(
			\prod_{\nvec \in \restbasis}
			\exp(-\lambda_{\nvec} \alpha_{\nvec}^* + \lambda_{\nvec}^* \alpha_{\nvec})
		\right)
		\chi_W (\blambda).
\end{eqn}

Following the single-mode scheme, we can formulate two theorems that govern the transformation of a master equation and subsequent calculation of observables.

\begin{theorem}[Multimode extension of \thmref{mm-wigner:sm:correspondences}]
\label{thm:mm-wigner:mm:correspondences}
	For any Hilbert-Schmidt operator $\hat{A}$
	\begin{eqn*}
		\mathcal{W} [ \hat{a}_{\nvec} \hat{A} ]
			& = \left( \alpha_{\nvec} + \frac{1}{2} \frac{\cwd}{\cwd \alpha_{\nvec}^*} \right)
				\mathcal{W}[\hat{A}],
		\quad
		\mathcal{W} [ \hat{a}_{\nvec}^\dagger \hat{A} ]
			= \left( \alpha_{\nvec}^* - \frac{1}{2} \frac{\cwd}{\cwd \alpha_{\nvec}} \right)
				\mathcal{W}[\hat{A}], \\
		\mathcal{W} [ \hat{A} \hat{a}_{\nvec} ]
			& = \left( \alpha_{\nvec} - \frac{1}{2} \frac{\cwd}{\cwd \alpha_{\nvec}^*} \right)
				\mathcal{W}[\hat{A}],
		\quad
		\mathcal{W} [ \hat{A} \hat{a}_{\nvec}^\dagger ]
			= \left( \alpha_{\nvec}^* + \frac{1}{2} \frac{\cwd}{\cwd \alpha_{\nvec}} \right)
				\mathcal{W}[\hat{A}].
	\end{eqn*}
\end{theorem}
\begin{proof}
The procedure is the same as in \thmref{mm-wigner:sm:correspondences}.
\end{proof}

\begin{lemma}[Multimode extension of \lmmref{mm-wigner:sm:moments-from-chi}]
\label{lmm:mm-wigner:mm:moments-from-chi}
	For a system with the density matrix $\rho$ and the corresponding characteristic function $\chi_W$:
	\begin{eqn*}
		\langle \symprod{ \prod_{\nvec \in \restbasis}
			\hat{a}_{\nvec}^{r_{\nvec}} (\hat{a}_{\nvec}^\dagger)^{s_{\nvec}} } \rangle
		= \left.
			\left(
				\prod_{\nvec \in \restbasis}
				\left( \frac{\cwd}{\cwd \lambda_{\nvec}} \right)^{s_{\nvec}}
				\left( -\frac{\cwd}{\cwd \lambda_{\nvec}^*} \right)^{r_{\nvec}}
			\right)
			\chi_W (\blambda)
		\right|_{\blambda=0}.
	\end{eqn*}
\end{lemma}
\begin{proof}
Mode operators with different indices commute, so
\begin{eqn}
	\chi_W (\blambda)
	& = \Trace{
		\hat{\rho}
		\prod_{\nvec \in \restbasis}
			\exp( \lambda_{\nvec} \hat{a}_{\nvec}^\dagger - \lambda_{\nvec}^* \hat{a}_{\nvec})
	} \\
	& = \Trace{
		\hat{\rho}
		\prod_{\nvec \in \restbasis}
			\sum_{j_{\nvec} \in \restbasis, k_{\nvec} \in \restbasis}
			\frac{(-\lambda_{\nvec}^*)^{j_{\nvec}} \lambda_{\nvec}^{k_{\nvec}}}{j_{\nvec}! k_{\nvec}!}
			\symprod{ \hat{a}_{\nvec}^{j_{\nvec}} (\hat{a}_{\nvec}^\dagger)^{k_{\nvec}}}
	} \\
	& = \Trace{
		\hat{\rho}
		\left( \prod_{\nvec \in \restbasis} \sum_{j_{\nvec}, k_{\nvec}} \right)
		\left(
			\prod_{\nvec \in \restbasis}
			\frac{(-\lambda_{\nvec}^*)^{j_{\nvec}} \lambda_{\nvec}^{k_{\nvec}}}{j_{\nvec}! k_{\nvec}!}
			\symprod{ \hat{a}_{\nvec}^{j_{\nvec}} (\hat{a}_{\nvec}^\dagger)^{k_{\nvec}}}
		\right)
	},
\end{eqn}
where we have used $\left( \prod_{\nvec \in \restbasis} \sum_{j_{\nvec}, k_{\nvec}} \right)$ in a sense of a sequence of summations $\ldots \sum_{j_{\nvec}, k_{\nvec}} \ldots$ for all values of $\nvec \in \restbasis$.
Separating operator and scalar parts:
\begin{eqn}
	& = \Trace{
		\left( \prod_{\nvec \in \restbasis} \sum_{j_{\nvec}, k_{\nvec}} \right)
		\left(
			\prod_{\nvec \in \restbasis}
			\frac{(-\lambda_{\nvec}^*)^{j_{\nvec}} \lambda_{\nvec}^{k_{\nvec}}}{j_{\nvec}! k_{\nvec}!}
		\right)
		\hat{\rho}
		\symprod{ \prod_{\nvec \in \restbasis} \hat{a}_{\nvec}^{j_{\nvec}} (\hat{a}_{\nvec}^\dagger)^{k_{\nvec}}}
	} \\
	& = \left( \prod_{\nvec \in \restbasis} \sum_{j_{\nvec}, k_{\nvec}} \right)
		\left(
			\prod_{\nvec \in \restbasis}
			\frac{(-\lambda_{\nvec}^*)^{j_{\nvec}} \lambda_{\nvec}^{k_{\nvec}}}{j_{\nvec}! k_{\nvec}!}
		\right)
		\langle
			\symprod{ \prod_{\nvec \in \restbasis} \hat{a}_{\nvec}^{j_{\nvec}} (\hat{a}_{\nvec}^\dagger)^{k_{\nvec}}}
		\rangle.
\end{eqn}
Same as in \lmmref{mm-wigner:sm:moments-from-chi}, differentiating the resulting expression and setting $\blambda = 0$ will leave only term with required $j_{\nvec} = r_{\nvec}$ and $k_{\nvec} = s_{\nvec}$.
\end{proof}

Moments of multimode Wigner function correspond to the averages of symmetrically ordered products in the same way as for the single-mode case.

\begin{theorem}[Multimode extension of \thmref{mm-wigner:sm:moments}]
\label{thm:mm-wigner:mm:moments}
	For a system with the density matrix $\rho$ and the corresponding Wigner function $W(\balpha)$, and any non-negative integer $r$, $s$:
	\begin{eqn*}
		\langle \symprod{
			\prod_{\nvec \in \restbasis}
			\hat{a}_{\nvec}^{r_{\nvec}} (\hat{a}_{\nvec}^\dagger)^{s_{\nvec}}
		} \rangle
		= \int \upd^2 \balpha
			\left(
				\prod_{\nvec \in \restbasis} \alpha_{\nvec}^{r_{\nvec}} (\alpha_{\nvec}^*)^{s_{\nvec}}
			\right) W(\balpha).
	\end{eqn*}
\end{theorem}
\begin{proof}
The proof is carried out similarly to \thmref{mm-wigner:sm:moments}: integrals over $\alpha_{\nvec}$ are eliminated one by one using \lmmref{c-numbers:fourier-of-moments}, resulting integrals over $\lambda_{\nvec}$ are dealt with using integration by parts and \lmmref{c-numbers:zero-delta-integrals}, until we get the right part of \lmmref{mm-wigner:mm:moments-from-chi}.
\end{proof}

% =============================================================================
\section{Functional Wigner representation}
% =============================================================================

Now we have all the tools we need to work with the functional Wigner representation.
First, we will define functional analogue of the displacement operator~\eqnref{mm-wigner:sm:displacement-op}:
\begin{definition}
\label{def:wigner:func:displacement-op}
Functional displacement operator $\hat{D} \in \mathbb{F}_{\restbasis} \rightarrow \mathbb{H}_{\restbasis}$ is
\begin{eqn}
	\hat{D}[\Lambda] = \exp \int \upd\xvec \left(
		\Lambda \Psiop^\dagger - \Lambda^* \Psiop
	\right).
\end{eqn}
\end{definition}

It can be shown that the displacement operator has properties similar to~\eqnref{mm-wigner:sm:displacement-derivatives}.

\begin{lemma}
\label{lmm:wigner:func:displacement-derivatives}
	\begin{eqn*}
		\frac{\fdelta}{\fdelta \Lambda^\prime} \hat{D}[\Lambda]
		& = \hat{D}[\Lambda] (\Psiop^{\prime\dagger} + \frac{1}{2} \Lambda^{\prime*})
		= (\Psiop^{\prime\dagger} - \frac{1}{2} \Lambda^{\prime*}) \hat{D}[\Lambda], \\
		-\frac{\fdelta}{\fdelta \Lambda^{\prime*}} \hat{D}[\Lambda]
		& = \hat{D}[\Lambda] (\Psiop^\prime + \frac{1}{2} \Lambda^\prime)
		= (\Psiop^\prime - \frac{1}{2} \Lambda^\prime) \hat{D}[\Lambda].
	\end{eqn*}
\end{lemma}
\begin{proof}
We will prove the second part of the first equation.
Using Baker-Hausdorff theorem:
\begin{eqn}
	\hat{D}[\Lambda]
	& = \exp \left( \int \upd\xvec \Lambda \Psiop^\dagger \right)
		\exp \left( -\int \upd\xvec \Lambda^* \Psiop \right)
		\exp \frac{1}{2} \left[
			\int \upd\xvec^\prime \Lambda^\prime \Psiop^{\prime\dagger},
			\int \upd\xvec \Lambda^* \Psiop
		\right] \\
	& = \exp \left( \int \upd\xvec \Lambda \Psiop^\dagger \right)
		\exp \left( -\int \upd\xvec \Lambda^* \Psiop \right)
		\exp \left(
			-\frac{1}{2} \iint \upd\xvec \upd\xvec^\prime
			\Lambda^\prime \Lambda^* \delta_{\restbasis}(\xvec^\prime, \xvec)
		\right) \\
	& = \exp \left( \int \upd\xvec \Lambda \Psiop^\dagger \right)
		\exp \left( -\int \upd\xvec \Lambda^* \Psiop \right)
		\exp \left(
			-\frac{1}{2} \int \upd\xvec \Lambda \Lambda^*
		\right).
\end{eqn}
Note that, since $\Lambda \in \mathbb{F}_{\restbasis}$, it projects to itself, and so does $\Psiop^\dagger$.
Thus
\begin{eqn}
	\frac{\fdelta}{\fdelta \Lambda^\prime} \hat{D}[\Lambda]
	& = \left(
		\int \upd\xvec \Psiop^\dagger \delta_{\restbasis}(\xvec^\prime, \xvec)
		- \frac{1}{2} \int \upd\xvec \Lambda^* \delta_{\restbasis}(\xvec^\prime, \xvec)
	\right) \hat{D}[\Lambda] \\
	& = (\Psiop^{\prime\dagger} - \frac{1}{2} \Lambda^{\prime *}) \hat{D}[\Lambda].
	\qedhere
\end{eqn}
\end{proof}

The functional Wigner transformation is the same Fourier transform of the trace as in \defref{mm-wigner:sm:w-transformation}, but expressed in functional terms.

\begin{definition}
\label{def:wigner:func:w-transformation}
	Functional Wigner transformation $\mathcal{W} \in \mathbb{FH}_{\restbasis} \rightarrow \mathbb{F}_{\restbasis}$ is defined as
	\begin{eqn*}
		\mathcal{W}[\hat{A}]
		= \frac{1}{\pi^{2|\restbasis|}} \int \fdelta^2 \Lambda\,
			D[\Lambda, \Psi]
			\Trace{ \hat{A} \hat{D}[\Lambda] }.
	\end{eqn*}
	It transforms an operator $\hat{A}$ on a restricted subset of a Hilbert space to a functional $(\mathcal{W}[\hat{A}])[\Psi]$.
	The corresponding functional Weyl transformation is:
	\begin{eqn*}
		\mathcal{W}^{-1}[F]
		= \frac{1}{\pi^{|\restbasis|}} \int \fdelta^2 \Xi\, \hat{D}^{\dagger}[\Xi]
			\int \fdelta^2 \Phi\, D[\Phi, \Xi] F[\Phi].
	\end{eqn*}
\end{definition}

It can be proved, same as in \thmref{mm-wigner:sm:w-real} for the single-mode case, that $\mathcal{W}[\hat{A}]$ is real if $\hat{A}$ is Hermitian, and $\mathcal{W}[\mathcal{W}^{-1}[F]] \equiv F$ (by mode expansion).
Following the single-mode and multi-mode cases we define the characteristic functional $\chi_W [\Lambda] \in \mathbb{F}_{\restbasis} \rightarrow \mathbb{C}$
\begin{eqn}
	\chi_W [\Lambda] = \Trace{ \hat{\rho} \hat{D}[\Lambda] },
\end{eqn}
and the Wigner functional $W \in \mathbb{F}_{\restbasis} \rightarrow \mathbb{R}$
\begin{eqn}
	W [\Psi]
	\equiv \mathcal{W}[\hat{\rho}]
	= \frac{1}{\pi^{2|\restbasis|}} \int \fdelta^2 \Lambda\,
		D[\Lambda, \Psi]
		\chi_W [\Lambda].
\end{eqn}

Correspondence relations and moment extraction theorems have the same form as in the single-mode case.

\begin{theorem}[Functional extension of \thmref{mm-wigner:sm:correspondences}]
\label{thm:wigner:func:correspondences}
	For any Hilbert-Schmidt operator $\hat{A}$
	\begin{eqn*}
		\mathcal{W} [ \Psiop \hat{A} ]
			& = \left( \Psi + \frac{1}{2} \frac{\fdelta}{\fdelta \Psi^*} \right) \mathcal{W}[\hat{A}],
		\quad
		\mathcal{W} [ \Psiop^\dagger \hat{A} ]
			= \left( \Psi^* - \frac{1}{2} \frac{\fdelta}{\fdelta \Psi} \right) \mathcal{W}[\hat{A}], \\
		\mathcal{W} [ \hat{A} \Psiop ]
			& = \left( \Psi - \frac{1}{2} \frac{\fdelta}{\fdelta \Psi^*} \right) \mathcal{W}[\hat{A}],
		\quad
		\mathcal{W} [ \hat{A} \Psiop^\dagger ]
			= \left( \Psi^* + \frac{1}{2} \frac{\fdelta}{\fdelta \Psi} \right) \mathcal{W}[\hat{A}].
	\end{eqn*}
\end{theorem}
\begin{proof}
We will prove the first correspondence.
First, let us transform the trace using \lmmref{wigner:func:displacement-derivatives}:
\begin{eqn}
	\Trace{ \Psiop \hat{A} \hat{D} }
	& = \Trace{ \hat{A} \hat{D} \Psiop}
	= \Trace{ \hat{A} \left(
		-\frac{\fdelta}{\fdelta \Lambda^*}
		-\frac{1}{2} \Lambda
	\right) \hat{D}} \\
	& = \left(
		-\frac{\fdelta}{\fdelta \Lambda^*}
		-\frac{1}{2} \Lambda
	\right) \Trace{ \hat{A} \hat{D}}.
\end{eqn}
Moving additional multiplier outside the integral:
\begin{eqn}
	\mathcal{W} [ \hat{\Psi} \hat{A} ]
	& = \frac{1}{\pi^{2|\restbasis|}} \int \fdelta^2 \Lambda
		\left( \exp \int \upd\xvec \left( -\Lambda \Psi^* + \Lambda^* \Psi \right) \right)
		\Trace{ \Psiop \hat{A} \hat{D}[\Lambda] } \\
	& = \frac{1}{\pi^{2|\restbasis|}} \int \fdelta^2 \Lambda
		\left( \exp \int \upd\xvec \left( -\Lambda \Psi^* + \Lambda^* \Psi \right) \right)
		\left(
			-\frac{\fdelta}{\fdelta \Lambda^*}
			-\frac{1}{2} \Lambda
		\right)
		\Trace{ \hat{A} \hat{D}[\Lambda] } \\
	& = \frac{1}{2} \frac{\fdelta}{\fdelta \Psi^*} \mathcal{W} [\hat{A}]
	- \frac{1}{\pi^{2|\restbasis|}} \int \fdelta^2 \Lambda
		\left( \exp \int \upd\xvec \left( -\Lambda \Psi^* + \Lambda^* \Psi \right) \right)
		\frac{\fdelta}{\fdelta \Lambda^*}
		\Trace{ \hat{A} \hat{D}[\Lambda] }.
\end{eqn}
Using \lmmref{func-calculus:zero-integrals} to move the partial derivative over $\Lambda^*$ ($\hat{D}[\Lambda]$ equals to a product of single-mode displacement operators, which makes the operator in the trace Hilbert-Schmidt, and the trace itself bounded):
\begin{eqn}
	& = \frac{1}{2} \frac{\fdelta}{\fdelta \Psi^*} \mathcal{W} [\hat{A}]
	+ \frac{1}{\pi^{2|\restbasis|}} \int \fdelta^2 \Lambda \left(
		\frac{\fdelta}{\fdelta \Lambda^*}
		\exp \int \upd\xvec \left( -\Lambda \Psi^* + \Lambda^* \Psi \right)
	\right)
	\Trace{ \hat{A} \hat{D}[\Lambda] } \\
	& = \left( \Psi + \frac{1}{2} \frac{\fdelta}{\fdelta \Psi^*} \right) \mathcal{W} [\hat{A}].
	\qedhere
\end{eqn}
\end{proof}

\begin{lemma}[Functional extension of \lmmref{mm-wigner:sm:moments-from-chi}]
\label{lmm:wigner:func:moments-from-chi}
	For a system with the density matrix $\hat{\rho}$ and the corresponding characteristic function $\chi_W$:
	\begin{eqn*}
		\langle \symprod{ (\Psiop^\prime)^r (\Psiop^{\prime\dagger})^s } \rangle
		= \left.
			\left( \frac{\fdelta}{\fdelta \Lambda^\prime} \right)^s
			\left( -\frac{\fdelta}{\fdelta \Lambda^{\prime*}} \right)^r
			\chi_W [\Lambda]
		\right|_{\Lambda \equiv 0}.
	\end{eqn*}
\end{lemma}
\begin{proof}
The proof follows the same general scheme as in the single-mode case.
The exponent in the $\chi_W$ can be expanded as
\begin{eqn}
	\exp (\Lambda \Psiop^\dagger - \Lambda^* \Psiop)
	= \sum_{r,s}
		\frac{
			\symprod{
				\left( \int \upd\xvec \Lambda \Psiop^\dagger \right)^r
				\left( -\int \upd\xvec \Lambda^* \Psiop \right)^s
			}
		}
		{r!s!}.
\end{eqn}
We can swap a functional derivative with both integration and multiplication by an independent function, so:
\begin{eqn}
	\frac{\fdelta}{\fdelta \Lambda^\prime} \left( \int \upd\xvec \Lambda \Psiop^\dagger \right)^r
	& = r \int \upd\xvec \frac{\fdelta \Lambda}{\fdelta \Lambda^\prime} \Psiop^\dagger
		\left( \int \upd\xvec \Lambda \Psiop^\dagger \right)^{r-1} \\
	& = r \int \upd\xvec\, \delta_{\restbasis}(\xvec^\prime, \xvec) \Psiop^\dagger
		\left( \int \upd\xvec\, \Lambda \Psiop^\dagger \right)^{r-1} \\
	& = r \Psiop^{\prime\dagger} \left( \int \upd\xvec\, \Lambda \Psiop^\dagger \right)^{r-1},
\end{eqn}
and multiple application of the differential gives us
\begin{eqn}
	\left( \frac{\fdelta}{\fdelta \Lambda^\prime} \right)^r
	\left( \int \upd\xvec\, \Lambda \Psiop^\dagger \right)^r
	= r! ( \Psiop^{\prime\dagger} )^r.
\end{eqn}
Similarly for the other differential:
\begin{eqn}
	\left( -\frac{\fdelta}{\fdelta \Lambda^{\prime*}} \right)^s
	\left( -\int \upd\xvec\, \Lambda \Psiop^\dagger \right)^s
	= s! ( \Psiop^{\prime\dagger} )^s.
\end{eqn}

Thus, same as in the single-mode case, the differentiation will eliminate all lower order terms in the expansion, and all higher order terms will be eliminated by setting $\Lambda \equiv 0$, leaving only one operator product with the required order:
\begin{eqn}
	\left.
		\left( \frac{\fdelta}{\fdelta \Lambda^\prime} \right)^s
		\left( -\frac{\fdelta}{\fdelta \Lambda^{\prime*}} \right)^r
		\chi_W [\Lambda]
	\right|_{\Lambda \equiv 0}
	& = r! s! \frac{1}{r! s!}
		\langle \symprod{ (\Psiop^\prime)^r (\Psiop^{\prime\dagger})^s } \rangle \\
	& = \langle \symprod{ (\Psiop^\prime)^r (\Psiop^{\prime\dagger})^s } \rangle.
	\qedhere
\end{eqn}
\end{proof}

\begin{theorem}[Functional extension of \thmref{mm-wigner:sm:moments}]
\label{thm:wigner:func:moments}
	For a system with the density matrix $\hat{\rho}$ and the corresponding Wigner functional $W[\Psi]$, and any non-negative integer $r$, $s$:
	\begin{eqn*}
		\langle \symprod{ \Psiop^r (\Psiop^\dagger)^s } \rangle
		= \int \fdelta^2\Psi\, \Psi^r (\Psi^*)^s W[\Psi].
	\end{eqn*}
\end{theorem}
\begin{proof}
By definition of the Wigner functional:
\begin{eqn}
	\int & \fdelta^2\Psi\, \Psi^r (\Psi^*)^s W[\Psi] \\
	={} & \frac{1}{\pi^{2|\restbasis|}} \Trace{ \hat{\rho}
		\int \fdelta^2\Psi\, \Psi^r (\Psi^*)^s
		\int \fdelta^2\Lambda \exp(-\Lambda \Psi^* + \Lambda^* \Psi)
		\hat{D}[\Lambda]
	}
\end{eqn}
Evaluating integral over $\Psi$ using \lmmref{func-calculus:fourier-of-moments}:
\begin{eqn}
	= \int \fdelta^2\Lambda
		\left(
			\left( \frac{\fdelta}{\fdelta \Lambda} \right)^s
			\left( -\frac{\fdelta}{\fdelta \Lambda^*} \right)^r
			\Delta_{\restbasis}[\Lambda]
		\right)
		\Trace{
			\hat{\rho}
			\hat{D}[\Lambda]
		}.
\end{eqn}
Integrating by parts and eliminating terms which fit \lmmref{func-calculus:zero-delta-integrals}:
\begin{eqn}
	& = \int \fdelta^2\Lambda\,
		\Delta_{\restbasis}[\Lambda]
		\left( \frac{\fdelta}{\fdelta \Lambda} \right)^s
		\left( -\frac{\fdelta}{\fdelta \Lambda^*} \right)^r
		\Trace{
			\hat{\rho}
			\hat{D}[\Lambda]
		} \\
	& = \left.
		\left( \frac{\fdelta}{\fdelta \Lambda} \right)^s
		\left( -\frac{\fdelta}{\fdelta \Lambda^*} \right)^r
		\chi_W [\Lambda]
	\right|_{\Lambda \equiv 0}.
\end{eqn}
Now, recognising the final expression as a part of \lmmref{wigner:func:moments-from-chi},
we immideately get the statement of the theorem.
\end{proof}

% =============================================================================
\section{Multi-component functional transformation}
% =============================================================================

Many experiments with \abbrev{bec}s involve several interacting components, so it is useful to extend the definition of the functional Wigner transformation from the previous section, along with \thmref{wigner:func:correspondences} and \thmref{wigner:func:moments}, to the case of several components.

First, definitions of the displacement operator and the Wigner functional are extended to the case multiple components.

\begin{definition}
	The functional displacement operator $\hat{D}_j \in \mathbb{F}_{\restbasis_j} \rightarrow \mathbb{H}_{\restbasis_j}$ for the component $j$ is defined as
	\begin{eqn*}
		\hat{D}_j[\Lambda, \Lambda^*] = \exp \int d\xvec \left(
			\Lambda \Psiop_j^\dagger - \Lambda^* \Psiop_j
		\right),
	\end{eqn*}
	where $\Psiop_j^\dagger$ and $\Psiop_j$ are creation and annihilation field operators for this component.
\end{definition}

\begin{definition}
\label{def:wigner:mc:w-transformation}
	The multi-component functional Wigner transformation $\mathcal{W}$ is defined as
	\begin{eqn*}
		& \mathcal{W} \in \left( \mathbb{R}^D \rightarrow \prod_{j=1}^C \mathbb{H}_{\restbasis_j} \right)
			\rightarrow \prod_{j=1}^C \mathbb{F}_{\restbasis_j}
			\rightarrow \mathbb{C} \\
		& \mathcal{W}[\hat{A}]
		= \frac{1}{\pi^{2 \sum|\restbasis_j|}} \int \fdelta^2 \bLambda
			\left( \prod_{j=1}^C D[\Lambda_j, \Psi_j] \right)
			\Trace{ \hat{A} \prod_{j=1}^C \hat{D}_j[\Lambda_j] },
	\end{eqn*}
	where $\Lambda_j \in \mathbb{F}_{\restbasis_j}$, and $\int \fdelta^2 \bLambda \equiv \int \fdelta^2 \Lambda_1 \ldots \int \fdelta^2 \Lambda_C$.
	This transforms an operator $\hat{A}$ on a restricted subset of the Hilbert space to a functional $(\mathcal{W}[\hat{A}])[\bPsi]$.
	The corresponding Weyl transformation is
	\begin{eqn*}
		\mathcal{W}^{-1}[F]
		= \frac{1}{\pi^{\sum |\restbasis_j|}} \int \fdelta^2 \bXi
			\left( \prod_{j=1}^C \hat{D}_j^{\dagger}[\Xi_j] \right)
			\int \fdelta^2 \bPhi
				\left( \prod_{j=1}^C D[\Phi_j, \Xi_j] \right)
				F[\bPhi].
	\end{eqn*}
\end{definition}

\begin{definition}
\label{def:wigner:mc:w-functional}
	The Wigner functional $W \in \prod_{j=1}^C \mathbb{F}_{\restbasis_j} \rightarrow \mathbb{R}$ is defined as
	\begin{eqn*}
		W [\bPsi]
		\equiv \mathcal{W}[\hat{\rho}]
		= \frac{1}{\pi^{2 \sum|\restbasis_j|}} \int \fdelta^2 \bLambda
			\left( \prod_{j=1}^C D[\Lambda_j, \Psi_j] \right)
			\chi_W [\bLambda],
	\end{eqn*}
	where $\chi_W [\bLambda] \in \prod_{j=1}^C \mathbb{F}_{\restbasis_j} \rightarrow \mathbb{C}$ is the characteristic functional
	\begin{eqn*}
		\chi_W [\bLambda]
		= \Trace{ \hat{\rho} \prod_{j=1}^C \hat{D}_j[\Lambda_j] }.
	\end{eqn*}
\end{definition}

The two central theorems for the single-component functional Wigner transformation can be straightforwardly reformulated for multiple components.

\begin{theorem}[multi-component extension of \thmref{wigner:func:correspondences}]
\label{thm:wigner:mc:correspondences}
	For any Hilbert-Schmidt operator $\hat{A}$,
	\begin{eqn*}
		\mathcal{W} [ \Psiop_j \hat{A} ]
			& = \left( \Psi_j + \frac{1}{2} \frac{\fdelta}{\fdelta \Psi_j^*} \right) \mathcal{W}[\hat{A}],
		\quad
		\mathcal{W} [ \Psiop_j^\dagger \hat{A} ]
			= \left( \Psi_j^* - \frac{1}{2} \frac{\fdelta}{\fdelta \Psi_j} \right) \mathcal{W}[\hat{A}], \\
		\mathcal{W} [ \hat{A} \Psiop_j ]
			& = \left( \Psi_j - \frac{1}{2} \frac{\fdelta}{\fdelta \Psi_j^*} \right) \mathcal{W}[\hat{A}],
		\quad
		\mathcal{W} [ \hat{A} \Psiop_j^\dagger ]
			= \left( \Psi_j^* + \frac{1}{2} \frac{\fdelta}{\fdelta \Psi_j} \right) \mathcal{W}[\hat{A}].
	\end{eqn*}
\end{theorem}
\begin{proof}
Proved in the same way as \thmref{wigner:func:correspondences}.
\end{proof}

\begin{theorem}[multi-component extension of \thmref{wigner:func:moments}]
\label{thm:wigner:mc:moments}
	For a system with a density matrix $\hat{\rho}$ and a corresponding Wigner functional $W[\bPsi]$, given any non-negative integers $r$, $s$:
	\begin{eqn*}
		\left\langle
			\symprod{ \prod_{j=1}^C \Psiop_j^{r_j} (\Psiop_j^\dagger)^{s_j} }
		\right\rangle
		= \int \fdelta^2 \bPsi\,
			\left( \prod_{j=1}^C \Psi_j^{r_j} (\Psi_j^*)^{s_j} \right) W[\bPsi].
	\end{eqn*}
\end{theorem}
\begin{proof}
Proved in the same way as \thmref{wigner:func:moments}, processing each component successively.
\end{proof}


% =============================================================================
\chapter{Transformation of the master equation}
\label{cha:wigner-spec}
% =============================================================================

Although the functional correspondences from \thmref{wigner:mc:correspondences} are relatively straightforward, there is a certain amount of work required to apply them to actual master equations arising in real world problems.
This chapter contains several theorems describing transformations of specific operator sequences,
encountered in master equations describing \abbrev{bec} evolution.


% =============================================================================
\section{Unitary evolution}
% =============================================================================

The simplest non-trivial element of the Hamiltonian one is the density operator $\Psiop_j^\dagger \Psiop_k$, which describes, for instance, a potential or a coupling field.

\begin{theorem}
\label{thm:wigner-spec:w-commutator1}
    For a Hilbert-Schmidt operator $\hat{A}$ with the corresponding Wigner functional $\mathcal{W}[\hat{A}] \equiv (\mathcal{W}[\hat{A}])[\bPsi]$, 
    \begin{eqn*}
        \mathcal{W} \left[ [\int \upd\xvec \Psiop_j^\dagger \Psiop_k, \hat{A}] \right]
        = \int \upd\xvec \left(
            - \frac{\fdelta}{\fdelta \Psi_j} \Psi_k
            + \frac{\fdelta}{\fdelta \Psi_k^*} \Psi_j^*
        \right) \mathcal{W}[\hat{A}].
    \end{eqn*}
\end{theorem}
\begin{proof}
Expanding the commutator and applying \thmref{wigner:mc:correspondences}:
\begin{eqn}
    \mathcal{W} \left[ [\int \upd\xvec \Psiop_j^\dagger \Psiop_k, \hat{A}] \right]
    ={} & \int \upd\xvec \left(
        \left(
            \Psi_j^* - \frac{1}{2} \frac{\fdelta}{\fdelta \Psi_j}
        \right)
        \left(
            \Psi_k + \frac{1}{2} \frac{\fdelta}{\fdelta \Psi_k^*}
        \right) \right. \\
    &   \left. - \left(
            \Psi_k - \frac{1}{2} \frac{\fdelta}{\fdelta \Psi_k^*}
        \right)
        \left(
            \Psi_j^* + \frac{1}{2} \frac{\fdelta}{\fdelta \Psi_j}
        \right)
    \right)
    \mathcal{W}[\hat{A}] \\
    ={} & \frac{1}{2} \int \upd\xvec \left(
        - \frac{\fdelta}{\fdelta \Psi_j} \Psi_k
        + \Psi_j^* \frac{\fdelta}{\fdelta \Psi_k^*}
        + \frac{\fdelta}{\fdelta \Psi_k^*} \Psi_j^*
        - \Psi_k \frac{\fdelta}{\fdelta \Psi_j}
    \right)
    \mathcal{W}[\hat{A}].
\end{eqn}
Changing the order of derivatives and functions using the relation
\begin{eqn}
    \Psi_k \frac{\fdelta}{\fdelta \Psi_j} \mathcal{F}
    = \left(
        \frac{\fdelta}{\fdelta \Psi_j} \Psi_k
        - \delta_{jk} \delta_{\restbasis_j}(\xvec, \xvec)
    \right) \mathcal{F},
\end{eqn}
we get
\begin{eqn}
    \mathcal{W} \left[ [\int \upd\xvec \Psiop_j^\dagger \Psiop_k, \hat{A}] \right]
    = \int \upd\xvec \left(
        - \frac{\fdelta}{\fdelta \Psi_j} \Psi_k
        + \frac{\fdelta}{\fdelta \Psi_k^*} \Psi_j^*
    \right)
    \mathcal{W}[\hat{A}],
\end{eqn}
which is the statement of the theorem.
\end{proof}

Commutators with the Laplacian inside require somewhat special treatment, because it acts on basis functions and, in general, cannot be dragged around like a constant.
For our purposes we only need one specific case, and, fortunately, in this case it does act like a constant.
The target expression describes kinetic energy in a Hamiltonian.

\begin{theorem}
\label{thm:wigner-spec:w-laplacian-commutator1}
    For a Hilbert-Schmidt operator $\hat{A}$ with the corresponding Wigner functional $\mathcal{W}[\hat{A}] \equiv (\mathcal{W}[\hat{A}])[\bPsi]$,
    \begin{eqn*}
        \mathcal{W} \left[
            \int \upd\xvec [\Psiop^\dagger(\xvec) \nabla^2 \Psiop(\xvec), \hat{A}]
        \right]
        = \int \upd\xvec \left(
            - \frac{\fdelta}{\fdelta \Psi} \nabla^2 \Psi
            + \frac{\fdelta}{\fdelta \Psi^*} \nabla^2 \Psi^*
        \right) \mathcal{W}[\hat{A}].
    \end{eqn*}
\end{theorem}
\begin{proof}
First, it is obvious from the definition of the Wigner transformation that an integral or a derivative acting on coordinates can be moved in and out of the transformation:
\begin{eqn}
    \mathcal{W} \left[ \int \upd\xvec \hat{B}(\xvec) \hat{A} \right]
    & = \int \upd\xvec \mathcal{W} [\hat{B}(\xvec) \hat{A}], \\
    \mathcal{W} [ \nabla^2 \hat{B}(\xvec) \hat{A} ]
    & = \nabla^2 \mathcal{W} [\hat{B}(\xvec) \hat{A}].
\end{eqn}
Let us now expand the commutator and apply correspondences from \thmref{wigner:func:correspondences}:
\begin{eqn2}
    & \mathcal{W} && \left[
        \int \upd\xvec [\Psiop^\dagger(\xvec) \nabla^2 \Psiop(\xvec), \hat{A}]
    \right] \\
    & ={} && \int \upd\xvec \left(
            \Psi^* - \frac{1}{2} \frac{\fdelta}{\fdelta \Psi}
        \right)
        \left(
            \nabla^2 \Psi + \frac{1}{2} \nabla^2 \frac{\fdelta}{\fdelta \Psi^*}
        \right)
        \mathcal{W}[\hat{A}] \\
    & && - \int \upd\xvec \left(
            \nabla^2 \Psi - \frac{1}{2} \nabla^2 \frac{\fdelta}{\fdelta \Psi^*}
        \right)
        \left(
            \Psi^* + \frac{1}{2} \frac{\fdelta}{\fdelta \Psi}
        \right)
        \mathcal{W}[\hat{A}] \\
    & ={} && \frac{1}{2} \int \upd\xvec \left(
            - \frac{\fdelta}{\fdelta \Psi} \nabla^2 \Psi
            + \Psi^* \nabla^2 \frac{\fdelta}{\fdelta \Psi^*}
            + \left( \nabla^2 \frac{\fdelta}{\fdelta \Psi^*} \right) \Psi^*
            - \left( \nabla^2 \Psi \right) \frac{\fdelta}{\fdelta \Psi}
        \right)
        \mathcal{W}[\hat{A}].
\end{eqn2}

Using the basis expansion, one can easily check that
\begin{eqn}
    \Psi^* \nabla^2 \frac{\fdelta}{\fdelta \Psi^*} \mathcal{F}[\Psi]
    = \left( \nabla^2 \frac{\fdelta}{\fdelta \Psi^*} \right) \Psi^* \mathcal{F}[\Psi]
    - \sum_{\nvec \in \restbasis} \phi_{\nvec}^* \nabla^2 \phi_{\nvec} \mathcal{F}[\Psi],
\end{eqn}
and
\begin{eqn}
    \left( \nabla^2 \Psi \right) \frac{\fdelta}{\fdelta \Psi} \mathcal{F}[\Psi]
    = \frac{\fdelta}{\fdelta \Psi} \left( \nabla^2 \Psi \right) \mathcal{F}[\Psi]
    - \sum_{\nvec \in \restbasis} \phi_{\nvec}^* \nabla^2 \phi_{\nvec} \mathcal{F}[\Psi].
\end{eqn}
Therefore:
\begin{eqn}
    & \mathcal{W} \left[
        \int \upd\xvec [\Psiop^\dagger(\xvec) \nabla^2 \Psiop(\xvec), \hat{A}]
    \right] \\
    & = \frac{1}{2} \int \upd\xvec \left(
        - \frac{\fdelta}{\fdelta \Psi} \nabla^2 \Psi
        + \left( \nabla^2 \frac{\fdelta}{\fdelta \Psi^*} \right) \Psi^*
        + \left( \nabla^2 \frac{\fdelta}{\fdelta \Psi^*} \right) \Psi^*
        - \frac{\fdelta}{\fdelta \Psi} \nabla^2 \Psi
    \right)
    \mathcal{W}[\hat{A}].
\end{eqn}
Now using \lmmref{func-calculus:move-laplacian} we can get the final result:
\begin{eqn}
    & = \frac{1}{2} \int \upd\xvec \left(
        - \frac{\fdelta}{\fdelta \Psi} \nabla^2 \Psi
        + \frac{\fdelta}{\fdelta \Psi^*} \nabla^2 \Psi^*
        + \frac{\fdelta}{\fdelta \Psi^*} \nabla^2 \Psi^*
        - \frac{\fdelta}{\fdelta \Psi} \nabla^2 \Psi
    \right)
    \mathcal{W}[\hat{A}] \\
    & = \int \upd\xvec \left(
        - \frac{\fdelta}{\fdelta \Psi} \nabla^2 \Psi
        + \frac{\fdelta}{\fdelta \Psi^*} \nabla^2 \Psi^*
    \right) \mathcal{W}[\hat{A}].
    \qedhere
\end{eqn}
\end{proof}

The last usual term in \abbrev{bec} Hamiltonians is a fourth order nonlinear interaction term.
We will provide the transformation for its general non-local form, which can be later simplified to the local one by intergration with the delta function.

\begin{theorem}
\label{thm:wigner-spec:w-commutator2}
    For a Hilbert-Schmidt operator $\hat{A}$ with the corresponding Wigner functional $\mathcal{W}[\hat{A}] \equiv (\mathcal{W}[\hat{A}])[\bPsi]$,
    \begin{eqn*}
        & \mathcal{W} \left[
            [
                \int \upd\xvec \int \upd\xvec^\prime
                \Psiop_j^\dagger \Psiop_k^{\prime\dagger} \Psiop_j \Psiop_k^\prime,
                \hat{A}
            ]
        \right] \\
        & = \int \upd\xvec \int \upd\xvec^\prime \left(
            -\frac{\fdelta}{\fdelta \Psi_j} \mathcal{Q}_{jk} \right.
            + \frac{\fdelta}{\fdelta \Psi_j^*} \mathcal{Q}_{jk}^*
            - \frac{\fdelta}{\fdelta \Psi_k} \mathcal{Q}_{kj}
            + \frac{\fdelta}{\fdelta \Psi_k^*} \mathcal{Q}_{kj}^* \\
        &   \left. \quad + \frac{\fdelta}{\fdelta \Psi_j^*}
            \frac{\fdelta}{\fdelta \Psi_j}
            \left(
                \frac{\fdelta}{\fdelta \Psi_k^\prime}
                \frac{\Psi_k^\prime}{4}
                - \frac{\fdelta}{\fdelta \Psi_k^{\prime *}}
                \frac{\Psi_k^{\prime *}}{4}
            \right)
            + \frac{\fdelta}{\fdelta \Psi_k^*}
            \frac{\fdelta}{\fdelta \Psi_k}
            \left(
                \frac{\fdelta}{\fdelta \Psi_j^\prime}
                \frac{\Psi_j^\prime}{4}
                - \frac{\fdelta}{\fdelta \Psi_j^{\prime *}}
                \frac{\Psi_j^{\prime *}}{4}
            \right)
        \right) \mathcal{W}[\hat{A}],
    \end{eqn*}
    where we denoted
    \begin{eqn*}
        \mathcal{Q}_{jk}[\bPsi](\xvec, \xvec^\prime)
        = \Psi_j | \Psi_k^\prime |^2
            -\frac{1}{2} \Psi_j \delta_{\restbasis_k}(\xvec^\prime, \xvec^\prime)
            -\frac{\delta_{j k} }{2} \Psi_j^\prime \delta_{\restbasis_k}(\xvec^\prime, \xvec).
    \end{eqn*}
\end{theorem}
\begin{proof}
Proved by straightforward application of \thmref{wigner:mc:correspondences} and simplification of the resulting expression, similarly to \thmref{wigner-spec:w-commutator1}.
\end{proof}


% =============================================================================
\section{Losses}
% =============================================================================

In this section we will state and proof the theorem which describes the transformation of the nonlinear operator
\begin{eqn}
\label{eqn:wigner-spec:loss-operator}
    \hat{\mathcal{L}}_{\lvec} [\hat{A}]
    = 2 \hat{O}_{\lvec} \hat{A} \hat{O}_{\lvec}^\dagger
        - \hat{O}_{\lvec}^\dagger \hat{O}_{\lvec} \hat{A}
        - \hat{A} \hat{O}_{\lvec}^\dagger \hat{O}_{\lvec},
\end{eqn}
where $\lvec = (l_1,\,\ldots,\,l_C)^T$ is the identifier of a loss process, with $l_j$ specifying the number of particles of component $j$ lost in the event, and
\begin{eqn}
    \hat{O}_{\lvec}
    \equiv \hat{O}_{\lvec} (\Psiopvec)
    = \prod_{j=1}^C \Psiop_j^{l_j} (\xvec).
\end{eqn}
The operator of this form appears when one describes losses in a \abbrev{bec}.

Before approaching the main theorem we will need two auxiliary lemmas.
After the application of \thmref{wigner:mc:correspondences} to the operator we will have terms with mixed order of $\Psi_j$ and $\fdelta / \fdelta \Psi_j$, while we need all derivatives to be grouped in the beginning in order to apply \thmref{fpe-sde:corr:fpe-sde-func} to the resulting differential equation.
The first lemma will provide a way to do that.

\begin{lemma}
\label{lmm:wigner-spec:swap-differential}
    For a function $\Psi \in \mathbb{F}_{\restbasis}$ and a functional operator $\mathcal{F} \in \mathbb{F}_{\restbasis} \rightarrow \mathbb{F}$, and non-negative integer $a$, $b$
    \begin{eqn*}
        \Psi^a \left( \frac{\fdelta}{\fdelta \Psi} \right)^b \mathcal{F}[\Psi]
        = \sum_{j=0}^{\min(a, b)}
            \binom{b}{j} \frac{(-1)^j a!}{(a - j)!}
            \delta_{\restbasis}^j(\xvec, \xvec)
            \left( \frac{\fdelta}{\fdelta \Psi} \right)^{b - j}
            \Psi^{a - j}
            \mathcal{F}[\Psi].
    \end{eqn*}
\end{lemma}
\begin{proof}
Proof by induction.
Let us assume that the statement is true for $b - 1$, and prove it for $b$
(also assuming non-trivial case of $a > 0$).
Moving a single differential to the left:
\begin{eqn}
    \Psi^a \left( \frac{\fdelta}{\fdelta \Psi} \right)^b \mathcal{F}
    = \left(
            \frac{\fdelta}{\fdelta \Psi} \Psi^a
            - a \Psi^{a - 1} \delta_{\restbasis}(\xvec, \xvec)
        \right)
        \left( \frac{\fdelta}{\fdelta \Psi} \right)^{b-1}
        \mathcal{F}.
\end{eqn}
Using the known relation for $b-1$:
\begin{eqn}
    ={} & \frac{\fdelta}{\fdelta \Psi} \sum_{j = 0}^{\min(a, b-1)}
            \binom{b-1}{j} \frac{(-1)^j a!}{(a-j)!} \delta_{\restbasis}^j(\xvec, \xvec)
            \left( \frac{\fdelta}{\fdelta \Psi} \right)^{b-1-j} \Psi^{a-j}
            \mathcal{F} \\
    & - a \delta_{\restbasis}(\xvec, \xvec) \sum_{j = 0}^{\min(a-1, b-1)}
            \binom{b-1}{j} \frac{(-1)^j (a-1)!}{(a-1-j)!} \delta_{\restbasis}^j(\xvec, \xvec)
            \left( \frac{\fdelta}{\fdelta \Psi} \right)^{b-1-j} \Psi^{a-1-j}
            \mathcal{F}.
\end{eqn}
Merging coefficients in front of the sums into internal expressions:
\begin{eqn}
    ={} & \sum_{j = 0}^{\min(a, b-1)}
            \binom{b-1}{j} \frac{(-1)^j a!}{(a-j)!} \delta_{\restbasis}^j(\xvec, \xvec)
            \left( \frac{\fdelta}{\fdelta \Psi} \right)^{b-j} \Psi^{a-j}
            \mathcal{F} \\
    & + \sum_{j = 0}^{\min(a-1, b-1)}
            \binom{b-1}{j} \frac{(-1)^{j+1} a!}{(a-1-j)!} \delta_{\restbasis}^{j+1}(\xvec, \xvec)
            \left( \frac{\fdelta}{\fdelta \Psi} \right)^{b-1-j} \Psi^{a-1-j}
            \mathcal{F}.
\end{eqn}
Shifting counter in the second sum:
\begin{eqn}
    ={} & \sum_{j = 0}^{\min(a, b-1)}
            \binom{b-1}{j} \frac{(-1)^j a!}{(a-j)!} \delta_{\restbasis}^j(\xvec, \xvec)
            \left( \frac{\fdelta}{\fdelta \Psi} \right)^{b-j} \Psi^{a-j}
            \mathcal{F} \\
    & + \sum_{j = 1}^{\min(a, b)}
            \binom{b-1}{j-1} \frac{(-1)^j a!}{(a-j)!} \delta_{\restbasis}^j(\xvec, \xvec)
            \left( \frac{\fdelta}{\fdelta \Psi} \right)^{b-j} \Psi^{a-j}
            \mathcal{F}.
\end{eqn}
Now we can join sums, noticing that $\binom{b-1}{j} + \binom{b-1}{j-1} = \binom{b}{j}$.
There will be at most two leftover terms: first, term for $j=0$ from the first sum,
and, possibly, the term with $j=\min(a,b)$ from the second sum.
The former term appears only if $\min(a,b) > \min(a, b-1)$,
or, in other words, $a \ge b$ (which means that $\min(a, b) = b$ and $\min(a, b-1) = b-1$).
\begin{eqn}
    ={} & \binom{b-1}{0} \frac{(-1)^0 a!}{(a-0)!} \delta_{\restbasis}^0(\xvec, \xvec)
            \left( \frac{\fdelta}{\fdelta \Psi} \right)^{b-0} \Psi^{a-0}
            \mathcal{F} \\
    & + \sum_{j = 1}^{\min(a, b-1)}
            \binom{b}{j} \frac{(-1)^j a!}{(a-j)!} \delta_{\restbasis}^j(\xvec, \xvec)
            \left( \frac{\fdelta}{\fdelta \Psi} \right)^{b-j} \Psi^{a-j}
            \mathcal{F} \\
    & + H[a - b]
            \binom{b-1}{b-1} \frac{(-1)^j a!}{(a-b)!} \delta_{\restbasis}^b(\xvec, \xvec)
            \left( \frac{\fdelta}{\fdelta \Psi} \right)^{b-b} \Psi^{a-b}
            \mathcal{F},
\end{eqn}
Where $H[n]$ is the discrete Heaviside step function.
Now, since $\binom{b-1}{0} \equiv \binom{b}{0}$ and $\binom{b-1}{b-1} \equiv \binom{b}{b}$,
we can attach two leftover terms to the sum as well:
\begin{eqn}
    = \sum_{j = 0}^{\min(a, b)}
        \binom{b}{j} \frac{(-1)^j a!}{(a-j)!} \delta_{\restbasis}^j(\xvec, \xvec)
        \left( \frac{\fdelta}{\fdelta \Psi} \right)^{b-j} \Psi^{a-j}
        \mathcal{F},
\end{eqn}
obtaining the statement of the lemma.
\end{proof}

\begin{lemma}[Sum rearrangement]
\label{lmm:wigner-spec:sum-rearrangement}
    For non-negative integer $a$, $b$:
    \begin{eqn*}
        \sum_{k=0}^a \sum_{m=0}^{\min(a-b,k)} f_{k-m} g_{k, m}
        = \sum_{v=0}^a f_v \sum_{m=0}^{a-\max(b,v)} g_{v + m, m}.
    \end{eqn*}
\end{lemma}
\begin{proof}
Obviously, the index $v = k - m$ of the factor $f$ can vary from $0$ (when $m=k$) to $a$ (when $k=a$ and $m=0$).
Therefore
\begin{eqn}
    \sum_{k=0}^a \sum_{m=0}^{\min(a-b,k)} f_{k-m} g_{k, m}
    = \sum_{v=0}^a f_v \sum_{m \in K(a, b, v)} g_{v + m, m},
\end{eqn}
where the set $K$ is defined as
\begin{eqn}
    K(a, b, v)
    & = \{m |
        0 \le k \le a
        \wedge 0 \le m \le \min(a - b, k)
        \wedge k - m = v
    \} \\
    & = \{m |
        -v \le m \le a - v
        \wedge 0 \le m \le \min(a - b, v + m)
    \} \\
    & = \{m |
        m \le a - v
        \wedge 0 \le m \le \min(a - b, v + m)
    \}.
\end{eqn}

It is convenient to consider the cases of $v \le b$ and $v > b$ separately.
For the former case
\begin{eqn}
    K_{v \le b}
    = \{m |
        m \le a - v
        \wedge 0 \le m \le \min(a - b, m + v)
        \wedge v \le b
    \}.
\end{eqn}
Since $v \le b$, $m \le a - v \le a - b \le \min(a - b, m + v)$ is always true, and the first inequation is redundant:
\begin{eqn}
    = \{m |
        0 \le m \le \min(a - b, v + m)
        \wedge v \le b
    \}.
\end{eqn}
Splitting into two sets to get rid of minimum function:
\begin{eqn}
    ={} & \{m |
        v \le b \wedge m \ge 0
        \\
    &   \wedge (
            (m \le a - b \wedge a - b < v + m)
            \vee
            (m \le v + m \wedge a - b \ge v + m)
        )
    \} \\
    ={} & \{m |
        v \le b \wedge m \ge 0
        \wedge
        (
            (m \le a - b \wedge m > a - b - v)
            \vee
            (m \le a - b - v)
        )
    \} \\
    ={} & \{m |
        v \le b \wedge m \ge 0
        \wedge
        (m \le a - b)
    \} \\
    ={} & \{m | v \le b \wedge 0 \le m \le a - b \}.
\end{eqn}
For the latter case $v < U$ we have
\begin{eqn}
    K_{v > b}
    ={} & \{m |
        m \le a - v
        \wedge 0 \le m \le \min(a - b, m + v)
        \wedge v > b
    \} \\
    ={} & \{m |
        v > b \wedge m \ge 0 \\
    &   \wedge (
            (m \le a - v \wedge m \ge a - b - v)
            \vee
            (m \le a - v \wedge m < a - b - v)
        )
    \} \\
    ={} & \{m | v > b \wedge 0 \le m \le a - v \}.
\end{eqn}

The final result is a union of these two cases:
\begin{eqn}
    K
    & = K_{v \le b} \cup K_{v > b} \\
    & = \{m | v \le b \wedge 0 \le m \le a - b \} \cup \{m | v > b \wedge 0 \le m \le a - v \} \\
    & = \{m | 0 \le m \le a - \max(b, v) \},
\end{eqn}
which gives us the statement of the lemma.
\end{proof}

With the help of these two lemmas we can perform the Wigner transformation of the nonlinear loss operator.

\begin{theorem}
\label{thm:wigner-spec:w-losses}
    The Wigner transformation of the loss operator $\hat{\mathcal{L}}_{\lvec}$ of form~\eqnref{wigner-spec:loss-operator} is
    \begin{eqn*}
        \mathcal{W} \left[ \hat{\mathcal{L}}_{\lvec} [\hat{A}] \right]
        =
            \sum_{j_1=0}^{l_1} \sum_{k_1=0}^{l_1} \ldots
            \sum_{j_C=0}^{l_C} \sum_{k_C=0}^{l_C}
                \left(
                    \prod_{c=1}^C
                        \left( \frac{\fdelta}{\fdelta \Psi_c^*} \right)^{j_c}
                        \left( \frac{\fdelta}{\fdelta \Psi_c} \right)^{k_c}
                \right)
                Z_{\lvec, \jvec, \kvec}
            \mathcal{W}[\hat{A}],
    \end{eqn*}
    where
    \begin{eqn*}
        Z_{\lvec, \jvec, \kvec}
        ={} & \left( 2 - (-1)^{\sum_c j_c} - (-1)^{\sum_c k_c} \right) \\
        &   \times \prod_{c=1}^C \left(
                \frac{1}{2^{j_c + k_c}}
                \binom{l_c}{j_c} \binom{l_c}{k_c}
                \exp \left(
                    -\frac{\delta_{\restbasis_c}(\xvec, \xvec)}{2}
                    \frac{\upp^2}{\upp \Psi_c \upp \Psi_c^*}
                \right)
                \Psi_c^{l_c-j_c} (\Psi_c^*)^{l_c-k_c}
            \right).
    \end{eqn*}
\end{theorem}
\begin{proof}
Let us perform the transformation for each term of the loss operator.
\begin{eqn}
    W_1
    & \equiv \mathcal{W}[\hat{O}_{\lvec} \hat{A} \hat{O}_{\lvec}^\dagger] \\
    & = \prod_{c=1}^C \left(
            \Psi_c + \frac{1}{2} \frac{\fdelta}{\fdelta \Psi_c^*}
        \right)^{l_c}
        \left(
            \Psi_c^* + \frac{1}{2} \frac{\fdelta}{\fdelta \Psi_c}
        \right)^{l_c}
        \mathcal{W}[\hat{A}] \\
    & = \prod_{c=1}^C \left(
            \sum_{j=0}^{l_c}
                \binom{l_c}{j} \left( \frac{1}{2} \right)^j
                \left( \frac{\fdelta}{\fdelta \Psi_c^*} \right)^j
                \Psi_c^{l_c - j}
            \sum_{k=0}^{l_c}
                \binom{l_c}{k} \left( \frac{1}{2} \right)^k
                \left( \frac{\fdelta}{\fdelta \Psi_c} \right)^k
                (\Psi_c^*)^{l_c - k}
        \right)
        \mathcal{W}[\hat{A}] \\
    & = \prod_{c=1}^C \left(
            \sum_{j=0}^{l_c}
            \sum_{k=0}^{l_c}
                \binom{l_c}{j} \binom{l_c}{k} \left( \frac{1}{2} \right)^{j + k}
                \left( \frac{\fdelta}{\fdelta \Psi_c^*} \right)^j
                \Psi_c^{l_c - j}
                \left( \frac{\fdelta}{\fdelta \Psi_c} \right)^k
                (\Psi_c^*)^{l_c - k}
        \right)
        \mathcal{W}[\hat{A}].
\end{eqn}
Using \lmmref{wigner-spec:swap-differential} to swap $\Psi_c$ and $\fdelta / \fdelta \Psi_c$:
\begin{eqn}
    ={} & \prod_{c=1}^C \left(
            \sum_{j=0}^{l_c}
            \sum_{k=0}^{l_c}
                \binom{l_c}{j} \binom{l_c}{k} \left( \frac{1}{2} \right)^{j + k}
                \left( \frac{\fdelta}{\fdelta \Psi_c^*} \right)^j
        \right. \\
        & \times \left.
                \sum_{m=0}^{\min(l_c - j, k)}
                    \binom{k}{m}
                    \frac{(-1)^m (l_c - j)!}{(l_c - j - m)!}
                    \delta_{\restbasis_c}^m(\xvec, \xvec)
                    \left( \frac{\fdelta}{\fdelta \Psi_c} \right)^{k - m}
                    \Psi_c^{l_c - j - m}
                (\Psi_c^*)^{l_c - k}
        \right)
        \mathcal{W}[\hat{A}].
\end{eqn}
Using \lmmref{wigner-spec:sum-rearrangement} with $a=l_c$ and $b=j$, and
\begin{eqn}
    f_{k-m} & = \left( \frac{\fdelta}{\fdelta \Psi_c} \right)^{k-m}, \\
    g_{k,m} & = \binom{l_c}{j} \binom{l_c}{k} \left( \frac{1}{2} \right)^{j + k}
        \binom{k}{m}
        \frac{(-1)^m (l_c - j)!}{(l_c - j - m)!}
        \delta_{\restbasis_c}^m(\xvec, \xvec)
        \Psi_c^{l_c - j - m}
    (\Psi_c^*)^{l_c - k},
\end{eqn}
we obtain
\begin{eqn}
    W_1 ={} & \prod_{c=1}^C \left(
            \sum_{j=0}^{l_c}
            \sum_{k=0}^{l_c}
                \left( \frac{\fdelta}{\fdelta \Psi_c^*} \right)^j
                \left( \frac{\fdelta}{\fdelta \Psi_c} \right)^k
            \right. \\
            & \left. \times \sum_{m=0}^{l_c - \max(j, k)}
                Q(l_c, j, k, m)
                \delta_{\restbasis_c}^m(\xvec, \xvec)
                \Psi_c^{l_c - j - m}
                (\Psi_c^*)^{l_c - k - m}
        \right)
        \mathcal{W}[\hat{A}],
\end{eqn}
where
\begin{eqn}
    Q(l, j, k, m)
    & = \frac{1}{2^{j + k + m}}
        \binom{l}{j} \binom{l}{k + m} \binom{k + m}{m}
        \frac{(-1)^m (l - j)!}{(l - j - m)!} \\
    & = \frac{(-1)^m m!}{2^{j + k + m}}
        \binom{l}{j} \binom{l}{k} \binom{l-k}{m} \binom{l-j}{m}.
\end{eqn}
Similarly,
\begin{eqn}
    W_2
    \equiv {} & \mathcal{W}[\hat{O}_{\lvec}^\dagger \hat{O}_{\lvec} \hat{A}] \\
    ={} & \prod_{c=1}^C \left(
            \sum_{j=0}^{l_c}
            \sum_{k=0}^{l_c}
                \left( \frac{\fdelta}{\fdelta \Psi_c^*} \right)^j
                \left( \frac{\fdelta}{\fdelta \Psi_c} \right)^k
            \right. \\
            & \left. \times \sum_{m=0}^{l_c - \max(j, k)}
                (-1)^k Q(l_c, j, k, m)
                \delta_{\restbasis_c}^m(\xvec, \xvec)
                \Psi_c^{l_c - j - m}
                (\Psi_c^*)^{l_c - k - m}
        \right)
        \mathcal{W}[\hat{A}],
\end{eqn}
and
\begin{eqn}
    W_3
    \equiv {} & \mathcal{W}[\hat{A} \hat{O}_{\lvec}^\dagger \hat{O}_{\lvec}] \\
    ={} & \prod_{c=1}^C \left(
            \sum_{j=0}^{l_c}
            \sum_{k=0}^{l_c}
                \left( \frac{\fdelta}{\fdelta \Psi_c^*} \right)^j
                \left( \frac{\fdelta}{\fdelta \Psi_c} \right)^k
            \right. \\
            & \left. \times \sum_{m=0}^{l_c - \max(j, k)}
                (-1)^j Q(l_c, j, k, m)
                \delta_{\restbasis_c}^m(\xvec, \xvec)
                \Psi_c^{l_c - j - m}
                (\Psi_c^*)^{l_c - k - m}
        \right)
        \mathcal{W}[\hat{A}].
\end{eqn}
The full expression for the Wigner transformation of $\hat{\mathcal{L}}_{\lvec}$ is therefore
\begin{eqn}
    \mathcal{W}[\hat{\mathcal{L}_{\lvec}}[\hat{A}]] = 2 W_1 - W_2 - W_3,
\end{eqn}

The expressions for $W_1$, $W_2$ and $W_3$ can be simplified further using the formal differentiation notation
\begin{eqn}
    \frac{\upp \left( \Psi^{l_c-j} (\Psi^*)^{l_c-k} \right)}{\upp^m \Psi \upp^m \Psi^*}
    \equiv{} & H[l_c-j-m] H[l_c-k-m] \\
    & \times
        \binom{l_c-j}{m} \binom{l_c-k}{m} (m!)^2
        \Psi^{l_c-j-m} (\Psi^*)^{l_c-k-m},
\end{eqn}
where $H[n]$ is the discrete Heaviside function, which equals $1$ for $n \ge 0$, and $0$ otherwise.
With this in mind, the internal summation over $m$ can be rewritten as
\begin{eqn}
    & \sum_{m=0}^{l_c - \max(j, k)}
        Q(l_c, j, k, m)
        \delta_{\restbasis_c}^{m}(\xvec, \xvec)
        \Psi_c^{l_c - j - m}
        (\Psi_c^*)^{l_c - k - m} \\
    & = \sum_{m=0}^{l_c}
        \frac{(-1)^{m}}{2^{j + k + m} m!}
        \binom{l_c}{j} \binom{l_c}{k}
        \delta_{\restbasis_c}^{m}(\xvec, \xvec)
        \frac{\upp \left( \Psi_c^{l_c-j} (\Psi_c^*)^{l_c-k} \right)}{\upp^{m} \Psi_c \upp^{m} \Psi_c^*}.
\end{eqn}
Furthermore, we can extend the summation over $m$ from $l_c$ to $\infty$, since all the new terms will be equal to zero.
This turns the expression into the power series for the exponential:
\begin{eqn}
    = \frac{1}{2^{j + k}}
        \binom{l_c}{j} \binom{l_c}{k}
        \exp \left(
            -\frac{\delta_{\restbasis_c}(\xvec, \xvec)}{2}
            \frac{\upp^2}{\upp \Psi_c \upp \Psi_c^*}
        \right)
        \Psi_c^{l_c-j} (\Psi_c^*)^{l_c-k}.
\end{eqn}
Substituting this into $W_1$, $W_2$ and $W_3$, and swapping the product and summations over $j_c$ and $k_c$, we get the statement of the lemma.
\end{proof}

% =============================================================================
\chapter{Wigner representation of BEC}
\label{cha:wigner-bec}
% =============================================================================

In this chapter we will apply the functional operator calculus and the functional Wigner transformation from \charef{wigner} along with the transformation theorems from \charef{wigner-spec} to the master equation of the \abbrev{bec} dynamics.
We will discuss the applicability of the truncation approximation, which is required to get rid of the possible negativity of the Wigner functional.
Finally, we will then employ the correspondences from \appref{fpe-sde} to derive the stochastic equations which can be solved numerically.


% =============================================================================
\section{Hamiltonian}
% =============================================================================

In order to take into account quantum effects, we must start from the master equation.
The basic Hamiltonian is easily expressed using quantum fields $\Psiop_j^{\dagger}(\xvec)$ and $\Psiop_j(\xvec)$, where $\xvec$ is a $D$-dimensional coordinate vector, $\Psiop_j^{\dagger}(\xvec)$ creates a bosonic atom of spin $j$ at location $\xvec$, and $\Psiop_j(\xvec)$ destroys one; the commutators are defined by~\eqnref{wigner:op-calculus:commutators}.
Second-quantized Hamiltonian for the system looks like:
\begin{eqn}
\label{eqn:wigner-bec:hamiltonian:H}
	\hat{H} / \hbar = \int d\xvec \left\{
		\Psiop_j^{\dagger} K_{jk} \Psiop_k
		+ \frac{1}{2} \int d\xvec^\prime
			\Psiop_j^\dagger \Psiop_k^{\prime\dagger}
			U_{jk}(\xvec - \xvec^\prime)
			\Psiop_j^\prime \Psiop_k
	\right\}.
\end{eqn}
Here we use the Einstein summation convention of summing over repeated indices.
$U_{jk}$ is the two-body scattering potential, and $K_{jk}$ is the single-particle Hamiltonian:
\begin{eqn}
	K_{jk} = \left(
			-\frac{\hbar}{2m} \nabla^2 + \omega_j + V_j(\xvec) / \hbar
		\right) \delta_{jk}
		+ \tilde{\Omega}_{jk}(t),
\end{eqn}
where $V_j$ is the external trapping potential for spin $j$,
$\omega_j$ is the internal energy of spin $j$,
and $\tilde{\Omega}_{jk}$ represents a time-dependent coupling that is used to rotate one spin projection into another.
In the subspace of two coupled components $\tilde{\Omega}_{jk}$ can be defined as:
\begin{eqn}
	\tilde{\Omega} = \frac{\Omega}{2} \begin{pmatrix}
		0 & e^{i(\omega t + \alpha)} + e^{-i(\omega t + \alpha)} \\
		e^{i (\omega t + \alpha)} + e^{-i(\omega t + \alpha)} & 0
	\end{pmatrix},
\end{eqn}
where $\omega$ and $\alpha$ are frequency and phase of the oscillator,
and $\Omega$ is Rabi frequency (cf. equation~\eqnref{mean-field:rotation-matrix}).

If we impose an energy cutoff $\ecut$ and only take into account low-energy modes,
the general scattering potential $U_{jk}(\xvec - \xvec^\prime)$ can be replaced by contact potential $U_{jk} \delta(\xvec - \xvec^\prime)$~\cite{Morgan2000}, giving the effective Hamiltonian
\begin{eqn}
\label{eqn:wigner-bec:hamiltonian:effective-H}
	\hat{H} / \hbar = \int d\xvec \left\{
		\Psiop_j^{\dagger} K_{jk} \Psiop_k
		+ \frac{U_{jk}}{2} \Psiop_j^\dagger \Psiop_k^\dagger \Psiop_j \Psiop_k
	\right\}.
\end{eqn}

For $s$-wave scattering in three dimensions the coefficient is $U_{jk} = 4 \pi \hbar a_{jk} / m$,
where $a_{jk}$ is the scattering length.
Note that in general case the coefficient must be renormalised depending on the grid~\cite{Sinatra2002},
but the change is small if $dx \gg a_{jk}$.

% =============================================================================
\section{Energy cutoff}
% =============================================================================

As has been noted earlier, in order to use contact interactions, an energy cutoff has to be imposed.
We use two different bases in numerical simulations, plane waves and harmonic oscillator modes (see \appref{bases} for details).
Both have analytical expressions for modes and corresponding energies, which makes the selection of modes straightforward.

Besides being a requirement for using the contact interaction in the Hamiltonian, the energy cutoff has other important functions.
First, it allows one to check for the convergence of integration with respect to decreasing cell size of the spatial grid.
It effectively separates the propagation in momentum space (which remains constant) from the propagation of the nonlinear parts of the equation in coordinate space (which, hopefully, becomes more precise when the cell size is decreased).
Alternatively, the energy cutoff can work in a different direction, lowering the amount of modes under consideration while keeping the spatial grid constant.
This helps to satisfy the Wigner truncation condition (see \secref{wigner-bec:truncation} for details).

% =============================================================================
\section{Master equation}
% =============================================================================

Hereinafter field operators and wave functions will be assumed to be defined in restricted basis, unless explicitly stated otherwise.
The Markovian master equation for the system with the inclusion of losses can be written as~\cite{Jack2002}
\begin{eqn}
\label{eqn:wigner-bec:master-eqn:master-eqn}
	\frac{d\hat{\rho}}{dt} =
		- \frac{i}{\hbar} \left[ \hat{H}, \hat{\rho} \right]
		+ \sum_{\lvec} \kappa_{\lvec} \int d\xvec
			\mathcal{L}_{\lvec} \left[ \hat{\rho} \right],
\end{eqn}
where $\lvec = (l_1, l_2, \ldots, l_C)$ is a vector containing the number of particles of the corresponding components participating in the interaction, $C$ being the number of components, and we have introduced local Liouville loss terms,
\begin{eqn}
	\mathcal{L}_{\lvec} \left[ \hat{\rho} \right] =
		2\hat{O}_{\lvec} \hat{\rho} \hat{O}_{\lvec}^\dagger
		- \hat{O}_{\lvec}^\dagger \hat{O}_{\lvec} \hat{\rho}
		- \hat{\rho} \hat{O}_{\lvec}^\dagger \hat{O}_{\lvec}.
\end{eqn}
The reservoir coupling operators $\hat{O}_{\lvec}$ are the distinct $n$-fold products of local field annihilation operators, $\hat{O}_{\lvec} = \hat{O}_{\lvec} (\Psiopvec) = \prod_{c=1}^C \Psiop_c^{l_c}$, describing local $(\sum l_c)$-body collision losses.

The master equation allows us to derive an important property.

\begin{theorem}
	\begin{eqn*}
		\frac{d}{dt} \langle \Psiop_j \rangle
		= \mathcal{P}_{\restbasis_j} \left[
			\langle
				-\frac{i}{\hbar} \left(
					K_{jm} \Psiop_m
					+ U_{jm} \Psiop_m^\dagger \Psiop_m \Psiop_j
				\right)
				- \sum_{\lvec} \kappa_{\lvec}
					\frac{\partial \hat{O}_{\lvec}^\dagger}{\partial \Psiop_j^\dagger} \hat{O}_{\lvec}
			\rangle
		\right]
	\end{eqn*}
\end{theorem}
\begin{proof}
\begin{eqn}
	\frac{d}{dt} \langle \Psiop_j \rangle
	={} & \frac{d}{dt} \Trace{ \hat{\rho} \Psiop_j }
	= \Trace{ \frac{d\hat{\rho}}{dt} \Psiop_j } \\
	={} & \Trace{ -\frac{i}{\hbar} \left[ \hat{H}, \hat{\rho} \right] \Psiop_j }
	+ \sum_{\lvec} \kappa_{\lvec} \int d\xvec^\prime
		\Trace{
			\mathcal{L}_{\lvec}^\prime \left[ \hat{\rho} \right]
			\Psiop_j
		} \\
	={} & \int d\xvec^\prime \left(
		- \frac{i}{\hbar} \Trace{
			\left[
				\Psiop_l^{\prime\dagger} K_{lm}^\prime \Psiop_m^\prime,
				\hat{\rho}
			\right] \Psiop_j
		}
		- \frac{i}{2\hbar} U_{lm} \Trace{
			\left[
				\Psiop_l^{\prime\dagger} \Psiop_m^{\prime\dagger}
				\Psiop_l^\prime \Psiop_m^\prime,
				\hat{\rho}
			\right] \Psiop_j
		} \right. \\
	& \left. + \sum_{\lvec} \kappa_{\lvec}
			\Trace{
				\mathcal{L}_{\lvec}^\prime \left[ \hat{\rho} \right]
				\Psiop_j
			}
	\right),
\end{eqn}
where $K_{lm}^\prime \equiv K_{lm}(\xvec^\prime)$, $\mathcal{L}_{\lvec}^\prime \equiv \mathcal{L} [ O_{\lvec}^\prime ] \equiv \mathcal{L} [ O_{\lvec} ( \Psiopvec^\prime ) ]$.

Let us transform each term separately.
We will make extensive use of the fact that trace is invariant under cyclic permutations to re-order the operators in terms.
In addition, the transformations are based on commutation relations proved in \lmmref{formalism:func-operators:functional-commutators}.

Noticing that $[ K_{lm}^\prime, \Psiop_j ] \equiv 0$, the first term can be transformed as:
\begin{eqn}
	\Trace{
		\left[
			\Psiop_l^{\prime\dagger} K_{lm}^\prime \Psiop_m^\prime,
			\hat{\rho}
		\right] \Psiop_j
	}
	& = \Trace{
		\Psiop_l^{\prime\dagger} K_{lm}^\prime \Psiop_m^\prime \hat{\rho} \Psiop_j
		- \hat{\rho} \Psiop_l^{\prime\dagger} K_{lm}^\prime \Psiop_m^\prime \Psiop_j
	} \\
	& = \Trace{
		\hat{\rho} \left(
			\Psiop_j \Psiop_l^{\prime\dagger} K_{lm}^\prime \Psiop_m^\prime
			- \Psiop_l^{\prime\dagger} K_{lm}^\prime \Psiop_m^\prime \Psiop_j
		\right)
	} \\
	& = \Trace{
		\hat{\rho} \left[
			\Psiop_j \Psiop_l^{\prime\dagger}
		\right] K_{lm}^\prime \Psiop_m^\prime
	} \\
	& = \Trace{
		\hat{\rho} \delta_{jl} \delta_{\restbasis_j}(\xvec^\prime - \xvec) K_{lm}^\prime \Psiop_m^\prime
	}
	= \delta_{\restbasis_j}(\xvec^\prime - \xvec) \langle K_{jm}^\prime \Psiop_m^\prime \rangle
\end{eqn}

Second (nonlinear) term:
\begin{eqn}
	U_{lm} \Trace{
		\left[
			\Psiop_l^{\prime\dagger} \Psiop_m^{\prime\dagger}
			\Psiop_l^\prime \Psiop_m^\prime,
			\hat{\rho}
		\right] \Psiop_j
	}
	& = U_{lm} \Trace{
		\Psiop_l^{\prime\dagger} \Psiop_m^{\prime\dagger}
		\Psiop_l^\prime \Psiop_m^\prime \hat{\rho} \Psiop_j
		- \hat{\rho} \Psiop_l^{\prime\dagger} \Psiop_m^{\prime\dagger}
		\Psiop_l^\prime \Psiop_m^\prime \Psiop_j
	} \\
	& = U_{lm} \Trace{
		\hat{\rho} \left(
			\Psiop_j \Psiop_l^{\prime\dagger} \Psiop_m^{\prime\dagger}
			\Psiop_l^\prime \Psiop_m^\prime
			- \Psiop_l^{\prime\dagger} \Psiop_m^{\prime\dagger}
			\Psiop_l^\prime \Psiop_m^\prime \Psiop_j
		\right)
	} \\
	& = U_{lm} \Trace{
		\hat{\rho} \left[
			\Psiop_j, \Psiop_l^{\prime\dagger} \Psiop_m^{\prime\dagger}
		\right] \Psiop_l^\prime \Psiop_m^\prime
	} \\
	& = U_{lm} \Trace{
		\hat{\rho} \delta_{\restbasis_j}(\xvec^\prime - \xvec) \left(
			\delta_{jl} \Psiop_m^{\prime\dagger}
			+ \delta_{jm} \Psiop_l^{\prime\dagger}
		\right) \Psiop_l^\prime \Psiop_m^\prime
	} \\
	& = 2 U_{jm} \delta_{\restbasis_j}(\xvec^\prime - \xvec) \Trace{
		\hat{\rho} \Psiop_m^{\prime\dagger} \Psiop_m^\prime \Psiop_j^\prime
	} \\
	& = 2 U_{jm} \delta_{\restbasis_j}(\xvec^\prime - \xvec) \langle
		\Psiop_m^{\prime\dagger} \Psiop_m^\prime \Psiop_j^\prime
	\rangle
\end{eqn}

The third term, coming from the losses, can be calculated as
\begin{eqn}
	\Trace{
		\mathcal{L}_{\lvec}^\prime \left[ \hat{\rho} \right]
		\Psiop_j
	}
	& = \Trace{
		2 \hat{O}_{\lvec}^\prime \hat{\rho} \hat{O}_{\lvec}^{\prime\dagger} \Psiop_j
		- \hat{O}_{\lvec}^{\prime\dagger} \hat{O}_{\lvec}^\prime \hat{\rho} \Psiop_j
		- \hat{\rho} \hat{O}_{\lvec}^{\prime\dagger} \hat{O}_{\lvec}^\prime \Psiop_j
	} \\
	& = \Trace{
		2 \hat{\rho} \hat{O}_{\lvec}^{\prime\dagger} \Psiop_j \hat{O}_{\lvec}^\prime
		- \hat{O}_{\lvec}^{\prime\dagger} \hat{O}_{\lvec}^\prime \hat{\rho} \Psiop_j
		- \hat{\rho} \hat{O}_{\lvec}^{\prime\dagger} \hat{O}_{\lvec}^\prime \Psiop_j
	} \\
	& = \Trace{
		\hat{\rho} \hat{O}_{\lvec}^{\prime\dagger} \hat{O}_{\lvec}^\prime \Psiop_j
		+ \hat{\rho} \hat{O}_{\lvec}^{\prime\dagger} \Psiop_j \hat{O}_{\lvec}^\prime
		- \hat{\rho} \Psiop_j \hat{O}_{\lvec}^{\prime\dagger} \hat{O}_{\lvec}^\prime
		- \hat{\rho} \hat{O}_{\lvec}^{\prime\dagger} \hat{O}_{\lvec}^\prime \Psiop_j
	} \\
	& = \Trace{
		\hat{\rho} \left[
			\hat{O}_{\lvec}^{\prime\dagger}, \Psiop_j
		\right] \hat{O}_{\lvec}^\prime
	} \\
	& = -\delta_{\restbasis_j}(\xvec^\prime - \xvec) \Trace{
		\hat{\rho} \frac{\partial \hat{O}_{\lvec}^{\prime\dagger}}{\partial \Psiop_j^{\prime\dagger}}
		\hat{O}_{\lvec}^\prime
	} \\
	& = -\delta_{\restbasis_j}(\xvec^\prime - \xvec) \langle
		\frac{\partial \hat{O}_{\lvec}^{\prime\dagger}}{\partial \Psiop_j^{\prime\dagger}}
		\hat{O}_{\lvec}^\prime
	\rangle.
\end{eqn}

Thus the full relation is
\begin{eqn}
	\frac{d}{dt} \langle \Psiop_j \rangle
	& = \int d\xvec^\prime \delta_{\restbasis_j}(\xvec^\prime - \xvec) \left(
		- \frac{i}{\hbar} \langle K_{jm}^\prime \Psiop_m^\prime \rangle
		- \frac{i U_{jm}}{\hbar} \langle
			\Psiop_m^{\prime\dagger} \Psiop_m^\prime \Psiop_j^\prime
		\rangle
		- \sum_{\lvec} \kappa_{\lvec} \langle
			\frac{\partial \hat{O}_{\lvec}^{\prime\dagger}}{\partial \Psiop_j^{\prime\dagger}}
			\hat{O}_{\lvec}^\prime
		\rangle
	\right),
\end{eqn}
which is equivalent to the statement of the theorem.
\end{proof}

% =============================================================================
\section{Wigner truncation}
% =============================================================================

In order to solve operator equation~\eqnref{wigner-bec:master-eqn:master-eqn} numerically,
we will transform it to ordinary differential equation using Wigner transformation~\eqnref{formalism:func-wigner:w-transformation}.

Namely, the term with $K_j$ is transformed using \thmref{formalism:transformations:w-commutator1} and \thmref{formalism:transformations:w-laplacian-commutator1}
(since $K_j$ is basically a sum of Laplacian operator and functions of $\xvec$):
\[
	\mathcal{W} \left[ [ \int d\xvec \Psiop_j^\dagger K_{jk} \Psiop_k, \hat{\rho} ] \right]
	= \int d\xvec \left(
			- \frac{\delta}{\delta \Psi_j} K_{jk} \Psi_k
			+ \frac{\delta}{\delta \Psi_k^*} K_{jk} \Psi_j^*
		\right)
		W,
\]
where Wigner function $W = \mathcal{W}[\hat{\rho}]$.
Nonlinear term is transformed with \thmref{formalism:transformations:w-commutator2}
(minding the locality of interaction and integrating over $\xvec$):
\begin{equation*}
\begin{split}
	\mathcal{W} \left[
		[
			\int d\xvec \frac{U_{jk}}{2}
				\Psiop_j^\dagger K_{jk} \Psiop_k^\dagger \Psiop_j \Psiop_k,
			\hat{\rho}
		]
	\right]
	& = \int d\xvec \frac{U_{jk}}{2} \left(
		\frac{\delta}{\delta \Psi_j} \left(
			- \Psi_j \Psi_k \Psi_k^*
			+ \frac{\delta_P(\xvec, \xvec)}{2} ( \delta_{jk} \Psi_k + \Psi_j )
		\right) \right. \\
	&	\left. + \frac{\delta}{\delta \Psi_j^*} \left(
			\Psi_j^* \Psi_k \Psi_k^*
			- \frac{\delta_P(\xvec, \xvec)}{2} ( \delta_{jk} \Psi_k^* + \Psi_j^* )
		\right) \right. \\
	&	\left. + \frac{\delta}{\delta \Psi_k} \left(
			- \Psi_j \Psi_j^* \Psi_k
			+ \frac{\delta_P(\xvec, \xvec)}{2} ( \delta_{jk} \Psi_j + \Psi_k )
		\right) \right. \\
	&	\left. + \frac{\delta}{\delta \Psi_k^*} \left(
			\Psi_j \Psi_j^* \Psi_k^*
			- \frac{\delta_P(\xvec, \xvec)}{2} ( \delta_{jk} \Psi_j^* + \Psi_k^* )
		\right) \right. \\
	&	\left.
			+ \frac{\delta}{\delta \Psi_j}
			\frac{\delta}{\delta \Psi_j^*}
			\frac{\delta}{\delta \Psi_k}
			\frac{1}{4} \Psi_k
			- \frac{\delta}{\delta \Psi_j}
			\frac{\delta}{\delta \Psi_j^*}
			\frac{\delta}{\delta \Psi_k^*}
			\frac{1}{4} \Psi_k^*
		\right. \\
	&	\left.
			+ \frac{\delta}{\delta \Psi_k}
			\frac{\delta}{\delta \Psi_k^*}
			\frac{\delta}{\delta \Psi_j}
			\frac{1}{4} \Psi_j
			- \frac{\delta}{\delta \Psi_k}
			\frac{\delta}{\delta \Psi_k^*}
			\frac{\delta}{\delta \Psi_j^*}
			\frac{1}{4} \Psi_j^*
		\right) W.
\end{split}
\end{equation*}
Assuming $U_{kj} = U_{jk}$, it simplifies to
\begin{equation*}
\begin{split}
	\mathcal{W} \left[
		[
			\int d\xvec \frac{U_{jk}}{2}
				\Psiop_j^\dagger K_{jk} \Psiop_k^\dagger \Psiop_j \Psiop_k,
			\hat{\rho}
		]
	\right]
	& = \int d\xvec U_{jk} \left(
		\frac{\delta}{\delta \Psi_j} \left(
			- \Psi_j \Psi_k \Psi_k^*
			+ \frac{\delta_P(\xvec, \xvec)}{2} ( \delta_{jk} \Psi_k + \Psi_j )
		\right) \right. \\
	&	\left. + \frac{\delta}{\delta \Psi_j^*} \left(
			\Psi_j^* \Psi_k \Psi_k^*
			- \frac{\delta_P(\xvec, \xvec)}{2} ( \delta_{jk} \Psi_k^* + \Psi_j^* )
		\right) \right. \\
	&	\left.
			+ \frac{\delta}{\delta \Psi_j}
			\frac{\delta}{\delta \Psi_j^*}
			\frac{\delta}{\delta \Psi_k}
			\frac{1}{4} \Psi_k
			- \frac{\delta}{\delta \Psi_j}
			\frac{\delta}{\delta \Psi_j^*}
			\frac{\delta}{\delta \Psi_k^*}
			\frac{1}{4} \Psi_k^*
		\right) W.
\end{split}
\end{equation*}

Loss operator is transformed with \thmref{formalism:transformations:w-losses}.
\todo{Not writing the resulting expression here, because with the absence of truncation it is too long,
and is practically the same as in theorem statement.}

We can only solve the resulting differential equation if it does not have functional derivatives of order more than 2 \todo{citation needed}.
\todo{Need to formally derive truncation condition.
For now I am assuming that we can drop third- and higher-order derivatives and consider $\delta_P(\xvec, \xvec) \ll | \Psi_j |^2$ for any $j$.}

\begin{lemma}
Assuming the conditions for Wigner truncation are satisfied,
the result of Wigner transformation of the nonlinear term can be written as
\[
	\mathcal{W} \left[
		[
			\frac{U_{jk}}{2}
				\Psiop_j^\dagger K_{jk} \Psiop_k^\dagger \Psiop_j \Psiop_k,
			\hat{\rho}
		]
	\right]
	= U_{jk} \left(
		\frac{\delta}{\delta \Psi_j^*} \Psi_j^* \Psi_k \Psi_k^*
		- \frac{\delta}{\delta \Psi_j} \Psi_j \Psi_k \Psi_k^*
	\right) W.
\]
\end{lemma}

\begin{theorem}
Assuming the conditions for Wigner truncation are satisfied,
i.e. we can drop differentials higher than the second order and consider $\delta_P(\xvec, \xvec) \ll | \Psi_j |^2$,
the result of Wigner transformation of the loss term can be written as
\begin{equation*}
\begin{split}
	\mathcal{W}[\mathcal{L}_{\lvec}[\hat{\rho}]]
	= \sum_{n=1}^C
			\frac{\delta}{\delta \Psi_n^*} \frac{\partial O_{\lvec}}{\partial \Psi_n} O_{\lvec}^*
	+ \sum_{n=1}^C
		\frac{\delta}{\delta \Psi_n} \frac{\partial O_{\lvec}^*}{\partial \Psi_n^*} O_{\lvec}
	+ \sum_{n=1}^C \sum_{p=1}^C
		\frac{\delta^2}{\delta \Psi_n^* \delta \Psi_p}
		\frac{\partial O_{\lvec}}{\partial \Psi_n}
		\frac{\partial O_{\lvec}^*}{\partial \Psi_p^*}.
\end{split}
\end{equation*}
where $O_{\lvec} \equiv O_{\lvec}(\Psivec) = \prod_{c=1}^C \Psi_c^{l_c}$.
\end{theorem}
\begin{proof}
The proof is basically a simplification of the result of \thmref{formalism:transformations:w-losses} under certain conditions.
First, we are neglecting all occurrences of $\delta_P$, which means setting $m_c = 0$ for every $c$.
Second, we are dropping all terms with high order differentials,
which can be expressed as limiting $\sum j_c + \sum k_c \le 2$.
The only combinations of $j_c$ and $k_c$ for which $Z(\jvec, \kvec)$ is not zero are thus
$\{ j_c = \delta_{cn}, k_c = 0, n \in [1, C] \}$,
$\{ j_c = 0, k_c = \delta_{cn}, n \in [1, C] \}$ and
$\{ j_c = \delta_{cn}, k_c = \delta_{cp}, n \in [1, C], p \in [1, C] \}$.
These combinations produce terms with $\frac{\delta}{\delta \Psi_n^*}$,
$\frac{\delta}{\delta \Psi_n}$ and
$\frac{\delta^2}{\delta \Psi_p \delta \Psi_n^*}$ respectively:
\begin{equation*}
\begin{split}
	\mathcal{W}[\mathcal{L}_{\lvec}[\hat{\rho}]]
	& = \sum_{n=1}^C \frac{\delta}{\delta \Psi_n^*}
		2 H[l_n - 1] Q_n(1, 0, 0) \Psi_n^{l_n - 1} (\Psi_n^*)^{l_n}
		\prod_{c=1, c \ne n}^C Q_c(0, 0, 0) \Psi_c^{l_c} (\Psi_c^*)^{l_c} \\
	& + \sum_{n=1}^C \frac{\delta}{\delta \Psi_n}
		2 H[l_n - 1] Q_n(0, 1, 0) \Psi_n^{l_n} (\Psi_n^*)^{l_n - 1}
		\prod_{c=1, c \ne n}^C Q_c(0, 0, 0) \Psi_c^{l_c} (\Psi_c^*)^{l_c} \\
	& + \sum_{n=1}^C \sum_{p=1, p \ne n}^C \frac{\delta^2}{\delta \Psi_n^* \delta \Psi_p}
		4 H[l_n - 1] Q_n(1, 0, 0) \Psi_n^{l_n - 1} (\Psi_n^*)^{l_n}
		H[l_p - 1] Q_p(0, 1, 0) \Psi_p^{l_p} (\Psi_p^*)^{l_p - 1} \\
	&	\prod_{c=1, c \ne n, c \ne p}^C Q_c(0, 0, 0) \Psi_c^{l_c} (\Psi_c^*)^{l_c} \\
	& + \sum_{n=1}^C \frac{\delta^2}{\delta \Psi_n^* \delta \Psi_n}
		4 H[l_n - 1] Q_n(1, 1, 0) \Psi_n^{l_n - 1} (\Psi_n^*)^{l_n - 1}
		\prod_{c=1, c \ne n}^C Q_c(0, 0, 0) \Psi_c^{l_c} (\Psi_c^*)^{l_c},
\end{split}
\end{equation*}
where $H[n]$ is the discrete Heavyside function.

One can easily find that $Q_n(1, 0, 0) = Q_n(0, 1, 0) = l_n / 2$, $Q_n(0, 0, 0) = 1$ and $Q_n(1, 1, 0) =
l_n^2 / 4$.
Therefore:
\begin{equation*}
\begin{split}
	& = \sum_{n=1}^C \frac{\delta}{\delta \Psi_n^*}
		\left( H[l_n - 1] l_n \Psi_n^{l_n - 1} \prod_{c=1, c \ne n}^C \Psi_c^{l_c} \right)
		O_{\lvec}^* \\
	& + \sum_{n=1}^C \frac{\delta}{\delta \Psi_n}
		\left( H[l_n - 1] l_n (\Psi_n^*)^{l_n - 1} \prod_{c=1, c \ne n}^C (\Psi_c^*)^{l_c} \right)
		O_{\lvec} \\
	& + \sum_{n=1}^C \sum_{p=1, p \ne n}^C \frac{\delta^2}{\delta \Psi_n^* \delta \Psi_p}
		\left( H[l_n - 1] l_n \Psi_n^{l_n - 1} \prod_{c=1, c \ne n}^C \Psi_c^{l_c} \right)
		\left( H[l_p - 1] l_p (\Psi_p^*)^{l_p - 1} \prod_{c=1, c \ne p}^C (\Psi_c^*)^{l_c} \right) \\
	& + \sum_{n=1}^C \frac{\delta^2}{\delta \Psi_n^* \delta \Psi_n}
		\left( H[l_n - 1] l_n \Psi_n^{l_n - 1} \prod_{c=1, c \ne n}^C \Psi_c^{l_c} \right)
		\left( H[l_n - 1] l_n (\Psi_n^*)^{l_n - 1} \prod_{c=1, c \ne n}^C (\Psi_c^*)^{l_c} \right) \\
	& = \sum_{n=1}^C \frac{\delta}{\delta \Psi_n^*}
		\frac{\partial O_{\lvec}}{\partial \Psi_n} O_{\lvec}^*
	+ \sum_{n=1}^C \frac{\delta}{\delta \Psi_n}
		\frac{\partial O_{\lvec}^*}{\partial \Psi_n^*} O_{\lvec}
	+ \sum_{n=1}^C \sum_{p=1, p \ne n}^C \frac{\delta^2}{\delta \Psi_n^* \delta \Psi_p}
		\frac{\partial O_{\lvec}^*}{\partial \Psi_p^*}
		\frac{\partial O_{\lvec}}{\partial \Psi_n}
	+ \sum_{n=1}^C \frac{\delta^2}{\delta \Psi_n^* \delta \Psi_n}
		\frac{\partial O_{\lvec}^*}{\partial \Psi_n^*}
		\frac{\partial O_{\lvec}}{\partial \Psi_n} \\
	& = \sum_{n=1}^C \frac{\delta}{\delta \Psi_n^*}
		\frac{\partial O_{\lvec}}{\partial \Psi_n} O_{\lvec}^*
	+ \sum_{n=1}^C \frac{\delta}{\delta \Psi_n}
		\frac{\partial O_{\lvec}^*}{\partial \Psi_n^*} O_{\lvec}
	+ \sum_{n=1}^C \sum_{p=1}^C \frac{\delta^2}{\delta \Psi_n^* \delta \Psi_p}
		\frac{\partial O_{\lvec}}{\partial \Psi_n}
		\frac{\partial O_{\lvec}^*}{\partial \Psi_p^*}
	\qedhere
\end{split}
\end{equation*}
\end{proof}

Resulting Wigner-transformed master equation is
\begin{equation}
\label{eqn:wigner-bec:truncation:fpe}
\begin{split}
	\frac{dW}{dt}
	& = \int d\xvec \left(
		- \sum_{j=1}^C \frac{\delta}{\delta \Psi_j} \left(
			-\frac{i}{\hbar} \left(
				\sum_{k=1}^C K_{jk} \Psi_k
				+ \sum_{k=1}^C U_{jk} \Psi_j \Psi_k \Psi_k^*
			\right)
			- \sum_{\lvec} \kappa_{\lvec} \frac{\partial O_{\lvec}^*}{\partial \Psi_j^*} O_{\lvec}
		\right)
	\right. \\
	& \left.
		+ \sum_{j=1}^C \frac{\delta}{\delta \Psi_j^*} \left(
			-\frac{i}{\hbar} \left(
				\sum_{k=1}^C K_{jk} \Psi_k^*
				+ \sum_{k=1}^C U_{jk} \Psi_j^* \Psi_k \Psi_k^*
			\right)
			+ \sum_{\lvec} \kappa_{\lvec} \frac{\partial O_{\lvec}}{\partial \Psi_j} O_{\lvec}^*
		\right)
	\right. \\
	& \left. + \sum_{j=1}^C \sum_{k=1}^C \frac{\delta^2}{\delta \Psi_j^* \delta \Psi_k}
		\sum_{\lvec} \kappa_{\lvec}
		\frac{\partial O_{\lvec}}{\partial \Psi_j}
		\frac{\partial O_{\lvec}^*}{\partial \Psi_k^*}
	\right) W
\end{split}
\end{equation}


% =============================================================================
\section{Fokker-Planck equation for the BEC}
% =============================================================================

The general approach to numerical solution of the Fokker-Planck equation~\eqnref{wigner-bec:truncation:fpe} is to transform it to the equivalent set of stochastic differential equations (SDEs) for $\Psi_j$.
Since the transformation is defined for real-valued variables only \todo{citation needed}, we have to modify the equation.

First, noticing that $K_{jk}$, $U_{jk}$ and $\kappa_{\lvec}$ are real-valued (which is important for the further transformations), we can rewrite equation~\eqnref{wigner-bec:truncation:fpe} as
\begin{eqn}
	\frac{dW}{dt}
	= \int d\xvec \left(
		- \sum_{j=1}^C \frac{\delta}{\delta \Psi_j} \mathcal{A}_j
		- \sum_{j=1}^C \frac{\delta}{\delta \Psi_j^*} \mathcal{A}_j^*
		+ \sum_{j=1}^C \sum_{k=1}^C \frac{\delta^2}{\delta \Psi_j^* \delta \Psi_k} D_{jk}
	\right) W,
\end{eqn}
where
\begin{eqn}
	\mathcal{A}_j = -\frac{i}{\hbar} \left(
			\sum_{k=1}^C K_{jk} \Psi_k
			+ \sum_{k=1}^C U_{jk} \Psi_j \Psi_k \Psi_k^*
		\right)
		- \sum_{\lvec} \kappa_{\lvec} \frac{\partial O_{\lvec}^*}{\partial \Psi_j^*} O_{\lvec},
\end{eqn}
and
\begin{eqn}
	D_{jk} = \sum_{\lvec} \kappa_{\lvec}
		\frac{\partial O_{\lvec}}{\partial \Psi_j}
		\frac{\partial O_{\lvec}^*}{\partial \Psi_k^*}.
\end{eqn}

This allows us to apply \thmref{wigner-bec:fpe:fpe-sde-func} with
\begin{eqn}
	\mathcal{B}_{\lvec}^{(j)}
	= \sqrt{\kappa_{\lvec}} \frac{\partial O_{\lvec}^*}{\partial \Psi_j^*}.
\end{eqn}
to get the equivalent set of SDEs in It\^{o} form
\begin{eqn}
	d\Psi_j = \mathcal{P}_{\restbasis_j} \left[
		\mathcal{A}^{(j)} dt + \sum_{\lvec} \mathcal{B}_{\lvec}^{(j)} dQ_{\lvec}
	\right].
\end{eqn}

% =============================================================================
\section{Initial states}
% =============================================================================

\subsection{Coherent state}

The simplest case of initial state is a coherent state.

\begin{theorem}
	The Wigner distribution for a multimode coherent state with with expectation values
	$\alpha_{\nvec}(0) = \alpha_{\nvec}^{(0)}$, $\nvec \in \restbasis$ is
	\begin{eqn*}
		W_{\mathrm{coh}} (\balpha^{(0)})
		= \left( \frac{2}{\pi} \right)^{|\restbasis|} \prod_{\nvec \in \restbasis}
			\exp(-2 |\alpha_{\nvec} - \alpha_{\nvec}^{(0)}|^2).
	\end{eqn*}
\end{theorem}
\begin{proof}
The density matrix of the state is
\begin{eqn}
	\hat{\rho}
	= \vert \alpha_{\nvec}^{(0)},\, \nvec \in \restbasis \rangle
		\langle \alpha_{\nvec}^{(0)},\, \nvec \in \restbasis \vert
	= \left( \prod_{\nvec \in \restbasis} \vert \alpha_{\nvec}^{(0)} \rangle \right)
		\left( \prod_{\nvec \in \restbasis} \langle \alpha_{\nvec}^{(0)} \vert \right).
\end{eqn}
Then the characteristic function is
\begin{eqn}
	\chi_W (\balpha^{(0)})
	& = \Trace{
		\left( \prod_{\nvec \in \restbasis} \vert \alpha_{\nvec}^{(0)} \rangle \right)
		\left( \prod_{\nvec \in \restbasis} \langle \alpha_{\nvec}^{(0)} \vert \right)
		\left( \prod_{\nvec \in \restbasis} \hat{D}_{\nvec} (\lambda_{\nvec}, \lambda_{\nvec}^*) \right)
	} \\
	& = \Trace{
		\left( \prod_{\nvec \in \restbasis} \langle \alpha_{\nvec}^{(0)} \vert \right)
		\left( \prod_{\nvec \in \restbasis} \hat{D}_{\nvec} (\lambda_{\nvec}, \lambda_{\nvec}^*) \right)
		\left( \prod_{\nvec \in \restbasis} \vert \alpha_{\nvec}^{(0)} \rangle \right)
	} \\
	& = \prod_{\nvec \in \restbasis}
		\langle \alpha_{\nvec}^{(0)} \vert
		\hat{D}_{\nvec} (\lambda_{\nvec}, \lambda_{\nvec}^*)
		\vert \alpha_{\nvec}^{(0)} \rangle
\end{eqn}
Displacement operators obey multiplication law \todo{proof needed? taken from Cahill and Glauber}
\begin{eqn}
	\hat{D}(\lambda, \lambda^*) \hat{D}(\alpha, \alpha^*)
	= \hat{D}(\lambda + \alpha, \lambda^* + \alpha^*)
		\exp(\frac{1}{2} (\lambda \alpha^* - \lambda^* \alpha)),
\end{eqn}
and the scalar product of two coherent state is calculated as \todo{again, taken from Cahill and Glauber}
\begin{eqn}
	\langle \beta \vert \alpha \rangle
	= \exp(-\frac{1}{2} |\alpha|^2 - \frac{1}{2} |\beta|^2 + \beta^* \alpha).
\end{eqn}
Therefore
\begin{eqn}
	\hat{D}(\lambda, \lambda^*) \vert \alpha \rangle
	& = \hat{D}(\lambda, \lambda^*) \hat{D}(\alpha, \alpha^*) \vert 0 \rangle
	= \hat{D}(\lambda + \alpha, \lambda^* + \alpha^*)
		\exp(\frac{1}{2} (\lambda \alpha^* - \lambda^* \alpha))
		\vert 0 \rangle \\
	& = \exp(\frac{1}{2} (\lambda \alpha^* - \lambda^* \alpha))
		\vert \lambda + \alpha \rangle,
\end{eqn}
and the characteristic function can be simplified as:
\begin{equation*}
\begin{split}
	\chi_W (\balpha^{(0)})
	& = \prod_{\nvec \in \restbasis}
		\exp(\frac{1}{2} (\lambda_{\nvec} (\alpha_{\nvec}^{(0)})^*
			- \lambda_{\nvec}^* \alpha_{\nvec}^{(0)}))
		\langle \alpha_{\nvec}^{(0)} \vert \lambda_{\nvec} + \alpha_{\nvec}^{(0)} \rangle \\
	& = \prod_{\nvec \in \restbasis}
		\exp(
			- \lambda_{\nvec}^* \alpha_{\nvec}^{(0)}
			+ \lambda_{\nvec} (\alpha_{\nvec}^{(0)})^*
			- \frac{1}{2} |\lambda|^2
		).
\end{split}
\end{equation*}

Finally, Wigner function is
\begin{equation*}
\begin{split}
	W_c (\balpha^{(0)})
	& = \frac{1}{\pi^{2|\restbasis|}} \int d^2\blambda
		\left( \prod_{\nvec \in \restbasis} \exp(
			- \lambda_{\nvec} \alpha_{\nvec}^*
			+ \lambda_{\nvec}^* \alpha_{\nvec}
		) \right)
		\left( \prod_{\nvec \in \restbasis} \exp(
			- \lambda_{\nvec}^* \alpha_{\nvec}^{(0)}
			+ \lambda_{\nvec} (\alpha_{\nvec}^{(0)})^*
			- \frac{1}{2} |\lambda|^2
		) \right) \\
	& = \frac{1}{\pi^{2|\restbasis|}} \prod_{\nvec \in \restbasis} \left(
		\int d^2\lambda_{\nvec}
			\exp(
				- \lambda_{\nvec} (\alpha_{\nvec}^* - (\alpha_{\nvec}^{(0)})^*)
				+ \lambda_{\nvec}^* (\alpha_{\nvec} - \alpha_{\nvec}^{(0)})
				- \frac{1}{2} |\lambda|^2
			)
	\right) \\
	& = \frac{1}{\pi^{2|\restbasis|}} \prod_{\nvec \in \restbasis}
		2 \pi \exp(-2 |\alpha_{\nvec} - \alpha_{\nvec}^{(0)}|^2) \\
	& = \left( \frac{2}{\pi} \right)^{|\restbasis|} \prod_{\nvec \in \restbasis}
		\exp(-2 |\alpha_{\nvec} - \alpha_{\nvec}^{(0)}|^2).
	\qedhere
\end{split}
\end{equation*}
\end{proof}

The resulting Wigner distribution is a product of independent complex-valued Gaussian distributions for each mode,
with the expectation value equal to the expectation value of the mode,
and the variance equal to $\frac{1}{2}$.
Therefore the initial state can be sampled as
\begin{eqn}
	\alpha_{\nvec} = \alpha_{\nvec}^{(0)} + \frac{1}{\sqrt{2}} \eta_{\nvec},
\end{eqn}
where $\eta_{\nvec}$ are normally distributed complex random numbers with zero mean,
$\langle \eta_{\mvec} \eta_{\nvec} \rangle = 0$ and
$\langle \eta_{\mvec} \eta_{\nvec}^* \rangle = \delta_{\mvec,\nvec}$
(in other words, with components distributed independently with variance $\frac{1}{2}$).
This looks like adding half a ``vacuum particle'' to each mode.
In functional form this can be written as
\begin{eqn}
	\Psi_j(\xvec, 0)
	= \Psi_j^{(0)}(\xvec, 0)
		+ \sum_{\nvec \in \restbasis} \frac{\eta_{j,\nvec}}{\sqrt{2}} \phi_{\nvec}(\xvec),
\end{eqn}
where $\Psi_j^{(0)}(\xvec, 0)$ is the ``classical'' ground state of the system.

% =============================================================================
\section{Usage examples}
% =============================================================================

The general theorems in the sections above may seem complicated, so in this section we will see into two examples of how they can be applied to real systems.

% =============================================================================
\subsection{Single-component example}
% =============================================================================

First, we will first consider a simple case with a single component \abbrev{bec}, with 3-body loss and no unitary evolution (the same as described by Norrie~\textit{et~al}~\cite{Norrie2006a}).
For this system we have $K \equiv 0$, $U \equiv 0$ and $\hat{O} = \Psiop^3$ (and, consequently, $O = \Psi^3$), and we also denote $\gamma = 6\kappa$.
The \abbrev{fpe} for this system is therefore
\begin{eqn}
    \frac{\upd W}{\upd t}
    ={} & -\frac{\fdelta}{\fdelta\Psi} \left(
            - \frac{\gamma}{2} |\Psi|^4 \Psi
            + \frac{3\gamma}{2} |\Psi|^2 \Psi \delta_{\restbasis}(\xvec, \xvec)
            - \frac{3\gamma}{4} \Psi \delta_{\restbasis}^2(\xvec, \xvec)
        \right) \\
    & - \frac{\fdelta}{\fdelta \Psi^*} \left(
            - \frac{\gamma}{2} |\Psi|^4 \Psi^*
            + \frac{3\gamma}{2} |\Psi|^2 \Psi^* \delta_{\restbasis}(\xvec, \xvec)
            - \frac{3\gamma}{4} \Psi^* \delta_{\restbasis}^2(\xvec, \xvec)
        \right) \\
    & + \frac{\fdelta^2}{\fdelta\Psi^* \fdelta\Psi} \left(
            \frac{3\gamma}{2} |\Psi|^4
            - 3\gamma |\Psi|^2 \delta_{\restbasis}(\xvec, \xvec)
            + \frac{3\gamma}{4} \delta_{\restbasis}^2(\xvec, \xvec)
        \right) \\
    & + \frac{\fdelta^3}{\fdelta\Psi^* \fdelta\Psi^2} \left(
            \frac{3\gamma}{8} |\Psi|^2 \Psi
            - \frac{3\gamma}{8} \Psi \delta_{\restbasis}(\xvec, \xvec)
        \right) \\
    & + \frac{\fdelta^3}{\fdelta(\Psi^*)^2 \fdelta\Psi} \left(
            \frac{3\gamma}{8} |\Psi|^2 \Psi^*
            - \frac{3\gamma}{8} \Psi^* \delta_{\restbasis}(\xvec, \xvec)
        \right) \\
    & + \frac{\fdelta}{\fdelta\Psi^3} \left(
            \frac{\gamma}{24} \Psi^3
        \right)
        + \frac{\fdelta^3}{\fdelta(\Psi^*)^3} \left(
            \frac{\gamma}{24} (\Psi^*)^3
        \right)
        + \mathcal{O}\left[ \frac{1}{N^4} \right].
\end{eqn}
After the truncation, the resulting stochastic equation describing
the system is
\begin{eqn}
    \upd\Psi
    & = \mathcal{P}_{\restbasis} \left[
            - \frac{\gamma}{6} \left(
                \frac{\upp O^*}{\upp \Psi^*} O
                - \frac{1}{2} \delta_{\restbasis}(\xvec, \xvec) \frac{\upp^2 O^*}{\upp(\Psi^*)^2}
                    \frac{\upp O}{\upp \Psi}
            \right) \upd t
            + \sqrt{\frac{\gamma}{6}} \frac{\upp O^*}{\upp \Psi^*} \upd Q(\xvec,t)
        \right] \\
    & = \mathcal{P}_{\restbasis} \left[
            - \left(
                \frac{\gamma}{2} |\Psi|^4 \Psi
                - \frac{3\gamma}{2} \delta_{\restbasis}(\xvec, \xvec) |\Psi|^2 \Psi
            \right) \upd t
            + \sqrt{\frac{3\gamma}{2}}(\Psi^*)^2 \upd Q(\xvec,t)
        \right].
\end{eqn}
The equation coincides with the one given by Norrie \textit{et al}, except for the additional correction to the drift term, which is of order $1/N$ and therefore cannot be omitted.

If we calculate the rate population change over time using \thmref{wigner-bec:fpe-bec:population-change}, we obtain
\begin{eqn}
    \frac{\upd N}{\upd t}
    = -\gamma \int \upd\xvec \left(
        \pathavg{|\Psi|^6}
        - \frac{9}{2} \delta_{\restbasis}(\xvec, \xvec) \pathavg{|\Psi|^4}
    \right).
\end{eqn}
This can be transformed further to more conventional form.
Using the equivalence~\eqnref{wigner-bec:fpe-bec:moments} and the ordering transformation formula~\eqnref{wigner-bec:fpe-bec:ordering-transformation} for field operators, we get
\begin{eqn}
    \pathavg{|\Psi|^4}
    & = g^{(2)} n^2
        + 2 \delta_{\restbasis}(\xvec, \xvec) n
        + \frac{1}{2} \delta_{\restbasis}^2(\xvec, \xvec), \\
    \pathavg{|\Psi|^6}
    & = g^{(3)} n^3
        + \frac{9}{2} \delta_{\restbasis}(\xvec, \xvec) g^{(2)} n^2
        + \frac{9}{2} \delta_{\restbasis}^2(\xvec, \xvec) n
        + \frac{3}{4} \delta_{\restbasis}^3(\xvec, \xvec).
\end{eqn}
Here $n = \langle \Psiop^\dagger \Psiop \rangle$ is the particle density, and $g^{(k)}$ are correlation factors.
Substituting above expressions into the equation for the population change rate:
\begin{eqn}
    \frac{\upd N}{\upd t}
    = - \gamma \int \upd\xvec \left(
        g^{(3)} n^3
        - \frac{9}{2} \delta_{\restbasis}^2(\xvec, \xvec) n
        - \frac{3}{2} \delta_{\restbasis}^3(\xvec, \xvec)
    \right).
\end{eqn}
We see that the second highest term in the expression is canceled, which agrees with the expansion being correct up to the order $1/N$.
If the quantum correction term to the drift is omitted, one finds that a physically incorrect quadratic nonlinear term proportional to $n^2$ is obtained, which is inconsistent with an exact short-time solution to the master equation~\cite{Norrie2006a}.


% =============================================================================
\subsection{Two-component example}
% =============================================================================

As a more involved example, let us consider a two component $^{87}$Rb \abbrev{bec} from recent experiments~\cite{Egorov2011,Opanchuk2012}.
In this case we have both unitary evolution (including nonlinear interaction)~\eqnref{wigner-bec:hamiltonian:effective-H}, and three sources of losses: three-body recombination $\hat{O}_{111}=\Psiop_{1}^3$, two-body interspecies loss $\hat{O}_{12}=\Psiop_{1}\Psiop_{2}$ and two-body intraspecies loss $\hat{O}_{22}=\Psiop_{2}^2$.
This gives us \abbrev{sde}s~\eqnref{wigner-bec:fpe-bec:sde} with drift terms
\begin{eqn}
    \mathcal{A}_1
    ={} & - \frac{i}{\hbar} \left(
            \sum_{k=1}^2 K_{1k} \Psi_k
            + \Psi_1 \sum_{k=1}^2 U_{1k} \left(
                |\Psi_k|^2 - \frac{\delta_{1k} + 1}{2} \delta_{\restbasis_k}(\xvec, \xvec)
            \right)
        \right) \\
    & - 3\kappa_{111} \left( |\Psi_1|^2
        - 3 \delta_{\restbasis_1}(\xvec, \xvec) \right) |\Psi_1|^2 \Psi_1
        - \kappa_{12} \left( |\Psi_{2}|^2
        - \frac{\delta_{\restbasis_2}(\xvec, \xvec)}{2} \right) \Psi_1, \\
    \mathcal{A}_2
    ={} & - \frac{i}{\hbar} \left(
            \sum_{k=1}^2 K_{2k} \Psi_k
            + \Psi_2 \sum_{k=1}^2 U_{2k} \left(
                |\Psi_{k}|^2 - \frac{\delta_{2k} + 1}{2} \delta_{\restbasis_k}(\xvec, \xvec)
            \right)
        \right) \\
    & - \kappa_{12} \left(|\Psi_1|^2 - \frac{\delta_{\restbasis_1}(\xvec, \xvec)}{2} \right) \Psi_2
    - 2\kappa_{22} \left(|\Psi_2|^2 - \delta_{\restbasis_2}(\xvec, \xvec) \right)\Psi_2,
\end{eqn}
and noise terms
\begin{eqn}
    \mathcal{B}_{1,111} = 3 \sqrt{\kappa_{111}} \left( \Psi_1^* \right)^2,\quad
    \mathcal{B}_{1,12} = \sqrt{\kappa_{12}} \Psi_2^*,\quad
    \mathcal{B}_{1,22} = 0,
\end{eqn}
\begin{eqn}
    \mathcal{B}_{2,111} = 0,\quad
    \mathcal{B}_{2,12} = \sqrt{\kappa_{12}} \Psi_1^*,\quad
    \mathcal{B}_{2,22} = 2\sqrt{\kappa_{22}} \Psi_2^*.
\end{eqn}

This type of stochastic equation is needed to treat coherent \abbrev{bec} interferometry in the presence of nonlinear loss terms caused by two and three body collisions.

% =============================================================================
\section{Comparison with an exact method}
% =============================================================================

It is usually problematic to perform a comparison with an exact method for the functional truncated Wigner, since it is applied to the systems with many modes and many particles, for which no exact methods exist.
It is possible though to do this for the multimode truncated Wigner with only a few modes.
Such systems are well within the area of applicability for first-principle exact methods such as number state expansions.

In the paper~\cite{Opanchuk2012a} we considered a two-well four-mode \abbrev{bec} system with the Hamiltonian
\begin{eqn}
    \hat{H}
    = \frac{1}{2} \sum_{i,j=1}^2 \tilde{g}_{ij} \left(
        \hat{a}_i^\dagger \hat{a}_j^\dagger \hat{a}_j \hat{a}_i
        + \hat{b}_i^\dagger \hat{b}_j^\dagger \hat{b}_j \hat{b}_i
        \right)
\end{eqn}
and the initial coherent state
\begin{eqn}
\label{eqn:wigner-bec:mm:initial-cond}
    \Psi(0)
    =
        \ket{\sqrt{N_a / 2}}_{a_1}
        \ket{\sqrt{N_a / 2}}_{a_2}
        \ket{\sqrt{N_b / 2}}_{b_1}
        \ket{\sqrt{N_b / 2}}_{b_2},
\end{eqn}
where $N_a$ and $N_b$ stand for the initial total number of atoms in the first and the second well respectively.
The evolution of the system is governed by the master equation
\begin{eqn}
\label{eqn:wigner-bec:mm:master-eqn}
    \frac{\upd \hat{\rho}}{\upd \tau}
    = -i \left[ \hat{H}, \hat{\rho} \right].
\end{eqn}
After the transformation to the \abbrev{fpe} form with \thmref{mm-wigner:mm:correspondences}, truncation, and further transformation with \thmref{fpe-sde:corr:mc-fpe-sde}, this results in the equivalent system of \abbrev{sde}s
\begin{eqn}
    \upd \begin{pmatrix}
        \alpha_1 \\ \beta_1 \\ \alpha_2 \\ \beta_2
    \end{pmatrix}
    = -i \begin{pmatrix}
        \alpha_1 (\tilde{g}_{11} |\alpha_1|^2 + \tilde{g}_{12} |\alpha_2|^2) \\
        \beta_1 (\tilde{g}_{11} |\beta_1|^2 + \tilde{g}_{12} |\beta_2|^2) \\
        \alpha_2 (\tilde{g}_{22} |\alpha_2|^2 + \tilde{g}_{12} |\alpha_1|^2) \\
        \beta_2 (\tilde{g}_{22} |\beta_2|^2 + \tilde{g}_{12} |\beta_1|^2)
    \end{pmatrix} \upd \tau.
\end{eqn}
These equations can be solved numerically, which in our case was done with XMDS~\cite{Collecutt2001,Dennis2013}, and the operator moments given by \thmref{mm-wigner:mm:moments} can be used to obtain required observables.

On the other hand, the master equation~\eqnref{wigner-bec:mm:master-eqn} can be solved exactly using a number state expansion~\cite{Opanchuk2012a}.
In the Heisenberg representation the equations of motion for the first well (with the absence of coupling of any kind, the evolution in the wells is independent) are
\begin{eqn}
    \frac{\upd \hat{a}_i}{\upd t}
    = i \left[ \hat{H}, \hat{a}_i \right]
    = -i \sum_{j=1}^2 g_{ij} \hat{N}_j \hat{a}_i,
\end{eqn}
and their solution is
\begin{eqn}
\label{eqn:wigner-bec:mm:exact-a}
    \hat{a}_i(\tau)
    = \exp \left( -i \sum_{j=1}^2 g_{ij} \hat{N_j} \tau \right).
\end{eqn}
The initial conditions~\eqnref{wigner-bec:mm:initial-cond} can be, in turn, expanded as
\begin{eqn}
    \Psi_a(0)
    =
        \sum_{m=0}^{\infty} C_m^{(1)} \ket{m}_{a_1}
        \sum_{n=0}^{\infty} C_n^{(2)} \ket{n}_{a_2},
\end{eqn}
and used to calculate any required moments $f(\hat{a}_1^\dagger, \hat{a}_1, \hat{a}_2^\dagger, \hat{a}_2)$ at time $\tau$ as
\begin{eqn}
\label{eqn:wigner-bec:mm:exact-f}
    \langle f(\hat{a}_1^\dagger, \hat{a}_1, \hat{a}_2^\dagger, \hat{a}_2) \rangle
    ={} &
        \sum_{k,l,m,n=0}^{\infty} C_k^{(1)} C_l^{(2)} C_m^{(1)} C_n^{(2)} \\
    & \quad \times \bra{k}_{a_1} \bra{l}_{a_2}
        f(\hat{a}_1^\dagger(\tau), \hat{a}_1(\tau), \hat{a}_2^\dagger(\tau), \hat{a}_2(\tau))
        \ket{m}_{a_1} \ket{n}_{a_2}.
\end{eqn}

For the comparison we will take several quantities that were calculated in~\cite{Opanchuk2012a} for the system in question.
We will not go into detail about their physical meaning here, as they is not the primary focus of this thesis.
First we introduce phase-rotated Schwinger spin operator measurements for each well as
\begin{eqn}
    \hat{J}_A^X
    & = \frac{1}{2} \left(
            \hat{a}_{2}^{\dagger} \hat{a}_{1} e^{i\Delta\theta}
            +\hat{a}_{1}^{\dagger} \hat{a}_{2} e^{-i\Delta\theta}
        \right),\\
    \hat{J}_A^Y & = \frac{1}{2i} \left(
            \hat{a}_{2}^{\dagger} \hat{a}_{1} e^{i\Delta\theta}
            - \hat{a}_{1}^{\dagger} \hat{a}_{2} e^{-i\Delta\theta}
        \right),\\
    \hat{J}_A^Z & = \frac{1}{2} \left(
        \hat{a}_{2}^{\dagger} \hat{a}_{2}
        - \hat{a}_{1}^{\dagger} \hat{a}_{1}\right),
\end{eqn}
where $\Delta\theta = \pi / 2 - \arg \langle \hat{a}_2^\dagger \hat{a}_1 \rangle$ is the phase difference between the two modes in a well.

We consider all possible orthogonal pairs of spin operators $\hat{J}^\theta$, $\hat{J}^{\theta+\pi/2}$ in the plane orthogonal to $\hat{J}_Y^A$, where $\hat{J}^\theta$ is defined as
\begin{eqn}
    \hat{J}^\theta
    =   \hat{J}^{Z} \cos \theta
        + \hat{J}^{X} \sin \theta.
\end{eqn}
These pairs obey the Heisenberg uncertainty relation
\begin{eqn}
    \Delta\hat{J}^{\theta} \Delta\hat{J}^{\theta+\pi/2}
    \geq
    |\langle\hat{J}^{Y}\rangle|/2.
\end{eqn}
Even with this limit on the pair, the variance of one of the spins in the pair can be reduced below the Heisenberg limit:
\begin{eqn}
    \Delta^{2}\hat{J}^{\theta} < |\langle\hat{J}^{Y}\rangle|/2,
\end{eqn}
which is referred to as the ``spin squeezed state''.

It can be shown~\cite{Opanchuk2012a} that the optimal squeezing angle is
\begin{eqn}
    \theta
    = \frac{1}{2} \arctan \left(
        \frac{2\langle\hat{J}_{A}^{Z},\hat{J}_{A}^{X}\rangle}{%
            \Delta^{2}\hat{J}_{A}^{Z} - \Delta^{2}\hat{J}_{A}^{X}}
    \right),
\end{eqn}
and the degree of squeezing can be quantified as
\begin{eqn}
\label{eqn:wigner-bec:mm:squeezing}
    S^{\theta,\theta+\pi/2}
    = \frac{\Delta^{2}\hat{J}_{A}^{\theta,\theta+\pi/2}}{%
        \vert\langle\hat{J}_{A}^{Y}\rangle\vert/2}.
\end{eqn}
Here we denoted the pair correlation
\begin{eqn}
    \langle\hat{J}_{A}^{Z},\hat{J}_{A}^{X}\rangle
    = \frac{1}{2} \left(
            \langle\hat{J}_{A}^{Z}\hat{J}_{A}^{X}\rangle
            + \langle\hat{J}_{A}^{X}\hat{J}_{A}^{Z}\rangle
            - 2\langle\hat{J}_{A}^{Z}\rangle\langle\hat{J}_{A}^{X}\rangle
        \right).
\end{eqn}

\begin{figure}
    \centerline{%
    \includegraphics{figures_generated/exact/squeezing_nocc_100.pdf}%
    \includegraphics{figures_generated/exact/squeezing_cc_100.pdf}}

    \caption{
    Comparison of the Wigner simulated time-dependent local squeezing (blue bands; the width corresponds to the estimated sampling error) against the exact results (black lines).
    Trajectories used: $2,000$.
    The plots correspond to \textbf{(a)} the case of no inter-component interaction ($\tilde{g}_{12} = 0$, $\tilde{g}_{11} = \tilde{g}_{22} = 1 / N_A$), and \textbf{(b)} strong inter-component interaction ($\tilde{g}_{ij} = a_{ij} / (a_{11} N_A)$, where $a_{11} = 100.4$, $a_{12} = 80.8$ and $a_{22} = 95.5$).}%endcaption

    \label{fig:wigner-bec:mm:squeezing-comparison}
\end{figure}

\begin{figure}
    \centerline{%
    \includegraphics{figures_generated/exact/squeezing_nocc_err.pdf}%
    \includegraphics{figures_generated/exact/squeezing_cc_err.pdf}}

    \caption{
    Difference of the Wigner simulated local squeezing and the exact results (solid lines) as compared to the sampling errors in Wigner simulations (dashed lines).
    Plotted are the results for $2,000$ trajectories (blue lines) and $20,000$ trajectories (red lines).
    The plots correspond to \textbf{(a)} the case of no inter-component interaction, and \textbf{(b)} strong inter-component interaction (the interaction coefficients are the same as in~\figref{wigner-bec:mm:squeezing-comparison}).}%endcaption

    \label{fig:wigner-bec:mm:squeezing-error-comparison}
\end{figure}

The expectations above can be expressed in terms of creation and annihilation operators, and calculated either in Wigner representation using \thmref{mm-wigner:mm:moments}, or with the number state expansion using~\eqnref{wigner-bec:mm:exact-a} and~\eqnref{wigner-bec:mm:exact-f} (courtesy of Q.~Y.~He).
The results are plotted in~\figref{wigner-bec:mm:squeezing-comparison}.
The Wigner method shows good agreement with the exact results in the time range of interest; the growing sampling error can be reduced to a desirable extent by using more simulation trajectories.
Nevertheless, the Wigner results show a systematic error (\figref{wigner-bec:mm:squeezing-error-comparison}), which is inside the sampling error range for the low number of trajectories, but preserves its magnitude as the number of trajectories is increased (and the sampling error decreased).
This is a sign of the failing truncation approximation.
The large oscillations of the systematic error at the start of the simulation can be explained by accumulating uncertainties in the denominator of~\eqnref{wigner-bec:mm:squeezing}.

Note that while in this particular example the exact method gives more accurate results at comparable or lower computational cost, it becomes unapplicable as soon as one wants to include tunneling and nonlinear losses into the model, while the Wigner method continues to perform with the same effectiveness~\cite{Opanchuk2012a}.



% applications
% =============================================================================
\chapter{Quantum noise in BEC interferometry}
\label{cha:bec-noise}
% =============================================================================

\copypaste{
Atom interferometry is an important quantum technology at the heart of many proposed future applications of ultra-cold atomic physics.
Bose-Einstein condensates (\abbrev{bec}s) or atom lasers are macroscopic quantum objects and have potential advantages as interferometric detectors and sensors, provided one can precisely extract atomic phase information.
However, unlike photons, atoms can interact strongly, causing dephasing and loss of interference fringes.
An intimate understanding of quantum many-body dynamics is the key to calculating interaction-induced dephasing in the measurement process.
This is essential for a quantitative theory of atom interferometry.
}

In this chapter we will apply the truncated Wigner method from \charef{wigner-bec} to the task of simulating the dynamics of an interferometry experiment with large atom number.
We will start from describing the mean-field approach leading to the conventional Gross-Pitaevskii equations~\cite{Pitaevskii2003} (\abbrev{gpe}s).
We then extend it to include quantum effects, such as the noise from linear and nonlinear losses.
The accurate description of nonlinear losses are especially important as \copypaste{they become dominant when atom numbers are increased to improve fringe visibility.}

\copypaste{
Importantly, we can clearly demonstrate where fringe visibility is driven by quantum fluctuations, and where it is driven by trap inhomogeneity and dynamical effects, in order to choose optimal conditions for quantum noise reduction and spin squeezing.
These calculations are a first step towards understanding mesoscopic superpositions and entanglement in ultra-cold atomic gases.
An advantage of our method compared to the variational and perturbative approaches used elsewhere~\cite{Li2009,Sakmann2009,Sinatra2011}
is that it allows us to treat a large number of independent field modes, thus including degrees of freedom that are excited due to collisional and nonlinear loss dynamics~\cite{Norrie2005,Deuar2007}.
Our theory can be readily extended to include finite-temperature initial conditions~\cite{Steel1998,Isella2006},
which will be treated elsewhere.
Nonlinear losses and finite-temperature effects can be also described within the confines of the variational approach~\cite{Li2008}.
}

\copypaste{
Quantum phase-diffusion is defined as the phase noise induced by number fluctuations which are conjugate to phase.
This is a fundamental feature of \abbrev{bec} interferometry, and can only be removed when there are no interactions.
However, there are other reasons for decoherence, which are also important.
The approach used here captures all three significant features of atom interferometry that can result in decoherence: phase-diffusion, losses, and trap inhomogeneity effects.
The results given in this chapter are applicable to simulations where the atom number per lattice point or mode is large.
}

% =============================================================================
\section{Mean-field approximation}
\label{sec:bec-noise:mean-field}
% =============================================================================

We will start from a classical model of a trapped \abbrev{bec}~--- the mean-field approximation.
While it cannot predict quantum effects, it provides the basis for comparison with the truncated Wigner method, and can be also applied to calculate the ground state of a trapped \abbrev{bec} numerically.

We will use the combined model which includes linear coupling and losses, and later in \secref{bec-noise:wigner}, dedicated to the application of the truncated Wigner method we will show how we can get almost identical equations from the first principles.


% =============================================================================
\subsection{Two-component condensate}
% =============================================================================

In the mean-field approximation the two-component \abbrev{bec} is described by wavefunctions $\Psi_j$, which are normalized such that $\int |\Psi_j|^2 \upd\xvec \equiv N_j$, where $N_j$ is the total population of the component $j$ (consequently, $|\Psi_j|^2 \equiv n_j$ is the component density).
The evolution of the condensate is described by the system of coupled Gross-Pitaevskii equations (\abbrev{cgpe}s)~\cite{Pitaevskii2003}
\begin{eqn}
\label{eqn:bec-noise:mean-field:cgpes}
	i \hbar \frac{\upd \Psi_1}{\upd t} ={} & \left(
		-\frac{\hbar^2 \nabla^2}{2 m} + V_1
		+ g_{11} \lvert \Psi_1 \rvert^2
		+ g_{12} \lvert \Psi_2 \rvert^2
		- i \hbar \Gamma_1
	\right) \Psi_1 \\
	& + \frac{\hbar \Omega}{2} \left(
		e^{i (\omega t + \alpha)} + e^{-i (\omega t + \alpha)}
	\right) \Psi_2, \\
	i \hbar \frac{\upd \Psi_2}{\upd t} ={} & \left(
		-\frac{\hbar^2 \nabla^2}{2 m} + V_2 + \hbar \omega_{hf}
		+ g_{22} \lvert \Psi_2 \rvert^2
		+ g_{12} \lvert \Psi_1 \rvert^2
		- i \hbar \Gamma_2
	\right) \Psi_2 \\
	& + \frac{\hbar \Omega}{2} \left(
		e^{i (\omega t + \alpha)} + e^{-i (\omega t + \alpha)}
	\right) \Psi_1.
\end{eqn}
Here $V_j(\xvec)$ are external potentials~\eqnref{bec-noise:system:V}, $\omega_{hf}$ is the hyperfine splitting between components, $g_{jk}$ are nonlinear interaction coefficients~\eqnref{bec-noise:system:g}, $\Gamma_j$ terms represent nonlinear losses for the component $j$, and $\Omega$ is the electromagnetic coupling strength (with $\omega$ being the frequency, and $\alpha$ the initial phase shift of the coupling).
The \abbrev{cgpe}s without loss or coupling terms were introduced by Zeng \textit{et~al}~\cite{Zeng1995}, and Ho and Shenoy~\cite{Ho1996}.
The coupling terms were first included by Ballagh \textit{et~al}~\cite{Ballagh1997}, and the loss terms by Yurovsky \textit{et~al}~\cite{Yurovsky1999}.
A detailed description of the mean-field approximation can be found in the book by Pitaevskii and Stringari~\cite{Pitaevskii2003}.

The exact expressions for the loss parameters $\Gamma_j$ depend on the loss processes in a particular experiment.
For example, the dominant three-body and two-body losses in a two-component \Rb{} \abbrev{bec} with the components ${\ket{F=1,\, m_F=-1}}$ and ${\ket{F=2,\, m_F=+1}}$ result in~\cite{Burt1997,Mertes2007}
\begin{eqn}
\label{eqn:bec-noise:mean-field:losses}
	\Gamma_1 &= \left( \gamma_{111} n_1^2 + \gamma_{12} n_2 \right) / 2, \\
	\Gamma_2 &= \left( \gamma_{12} n_1 + \gamma_{22} n_2 \right) / 2.
\end{eqn}
These equations were obtained empirically, but we will see later in this chapter how the same expressions (to leading order) appear as a result of the application of the truncated Wigner method.

The coupling frequency $\omega$ is usually slightly detuned from the hyperfine frequency in the experiment: $\omega = \omega_{hf} + \delta$, where $\delta \ll \omega_{hf}$.
It is convenient to use equations~\eqnref{bec-noise:mean-field:cgpes} in a rotating frame:
\begin{eqn}
	\Psi_1 & = \Psi_1^{(r)}, \\
	\Psi_2 & = \Psi_2^{(r)} e^{i \omega_{hf} t}.
\end{eqn}
This transformation eliminates $\omega_{hf}$ from the equations and does not change single-time observable values.
Dropping the rotating frame superscript for simplicity, we obtain the transformed equations
\begin{eqn}
	i \hbar \frac{\upd \Psi_1}{\upd t} ={} & \left(
		-\frac{\hbar^2 \nabla^2}{2 m} + V_1
		+ g_{11} \lvert \Psi_1 \rvert^2
		+ g_{12} \lvert \Psi_2 \rvert^2
		- i \hbar \Gamma_1
	\right) \Psi_1 \\
	& + \frac{\hbar \Omega}{2} \left(
		e^{i ((\omega + \omega_{hf}) t + \alpha)} + e^{-i (\delta t + \alpha)}
	\right) \Psi_2, \\
	i \hbar \frac{\upd \Psi_2}{\upd t} ={} & \left(
		-\frac{\hbar^2 \nabla^2}{2 m} + V_2
		+ g_{22} \lvert \Psi_2 \rvert^2
		+ g_{12} \lvert \Psi_1 \rvert^2
		- i \hbar \Gamma_2
	\right) \Psi_2 \\
	& + \frac{\hbar \Omega}{2} \left(
		e^{i (\delta t + \alpha)} + e^{-i ((\omega + \omega_{hf}) t + \alpha)}
	\right) \Psi_1.
\end{eqn}

Furthermore, in experiments the coupling field is typically applied for short periods of time $t_{\mathrm{pulse}}$, where $1 / \omega \ll t_{\mathrm{pulse}} \ll 1 / \delta$.
This allows us to neglect the rapidly oscillating terms proportional to $e^{i(\omega + \omega_{hf})t}$ and come to \abbrev{cgpe}s in the rotating frame:
\begin{eqn}
\label{eqn:bec-noise:mean-field:cgpes-simplified}
	i \hbar \frac{\upd \Psi_1}{\upd t} & = \left(
		-\frac{\hbar^2 \nabla^2}{2 m} + V_1
		+ g_{11} \lvert \Psi_1 \rvert^2
		+ g_{12} \lvert \Psi_2 \rvert^2
		- i \hbar \Gamma_1
	\right) \Psi_1
	+ \frac{\hbar \Omega}{2} e^{-i (\delta t + \alpha)} \Psi_2, \\
	i \hbar \frac{\upd \Psi_2}{\upd t} & = \left(
		-\frac{\hbar^2 \nabla^2}{2 m} + V_2
		+ g_{22} \lvert \Psi_2 \rvert^2
		+ g_{12} \lvert \Psi_1 \rvert^2
		- i \hbar \Gamma_2
	\right) \Psi_2 +
	\frac{\hbar \Omega}{2} e^{i (\delta t + \alpha)} \Psi_1.
\end{eqn}
When the pulse is applied twice using the same coupling field (which is the case for the Ramsey interferometry), it is the same as just setting $\Omega$ to zero after the first pulse and then restoring its value for the time of the second pulse; therefore, $\alpha$ stays the same too.
If one wants to apply pulse with the different detuning, the phase information is lost, and the value of $\alpha$ has to be regarded as random.

The application of the coupling field can be simplified when certain additional conditions are valid, namely:
\begin{itemize}
	\item $\mu / \hbar \ll \Omega$, where $\mu$ is the chemical potential of the first component;
	\item $\delta \ll \Omega$;
	\item the characteristic time of the other terms in~\eqnref{bec-noise:mean-field:cgpes} is much greater than $t_{\mathrm{pulse}}$.
\end{itemize}
This allows us to use an ``instantaneous'' pulse, multiplying the state vector by a rotation matrix:
\begin{eqn}
\label{eqn:bec-noise:mean-field:rotation-matrix}
	\begin{pmatrix}
		\Psi^\prime_1 \\ \Psi^\prime_2
	\end{pmatrix} =
	\begin{pmatrix}
		\cos \frac{\theta}{2} & -i e^{-i \phi} \sin \frac{\theta}{2} \\
		-i e^{i \phi} \sin \frac{\theta}{2} & \cos \frac{\theta}{2}
	\end{pmatrix}
	\begin{pmatrix}
		\Psi_1 \\ \Psi_2
	\end{pmatrix},
\end{eqn}
where $\theta = \Omega t_{\mathrm{pulse}}$, and $\phi = \delta t + \alpha$ is the total phase of the coupling field at the beginning of the pulse.
In particular, for the two-pulse Ramsey scheme (\figref{bec-noise:visibility:sequences},~(a)) with the time $t_R$ between pulses, $\phi_2 = \phi_1 + \delta t_R$.


% =============================================================================
\subsection{Ground state calculation}
% =============================================================================

At the beginning of the simulation, the \abbrev{bec} is assumed to be in the ground state which has the lowest possible energy.
The ground state is the solution of the stationary \abbrev{cgpe}s
\begin{eqn}
\label{eqn:bec-noise:mean-field:cgpes-stationary}
	\mu_1 \Psi_1 & = \left(
		-\frac{\hbar^2 \nabla^2}{2 m} + V_1
		+ g_{11} \lvert \Psi_1 \rvert^2
		+ g_{12} \lvert \Psi_2 \rvert^2
	\right) \Psi_1, \\
	\mu_2 \Psi_2 & = \left(
		-\frac{\hbar^2 \nabla^2}{2 m} + V_2
		+ g_{22} \lvert \Psi_2 \rvert^2
		+ g_{12} \lvert \Psi_1 \rvert^2
	\right) \Psi_2,
\end{eqn}
where $\mu_1$ and $\mu_2$ are chemical potentials of the components.

The most common method of finding the mean-field ground state is the imaginary time propagation~\cite{Chiofalo2000,Bao2004}.
The essence of the method is that the propagation of an arbitrary wavefunction using the time-dependent \abbrev{cgpe}s, but with the substitution $t \rightarrow \tau = it$, lowers its energy; therefore, after the sufficient amount of time this propagation will lead us arbitrarily close to the ground state.
The actual equations to be propagated are~\eqnref{bec-noise:mean-field:cgpes-simplified} without the loss or coupling terms:
\begin{eqn}
\label{eqn:bec-noise:mean-field:imaginary-time}
	\hbar \frac{\upd \Psi_1}{\upd \tau} & = -\left(
		-\frac{\hbar^2 \nabla^2}{2 m} + V_1
		+ g_{11} \lvert \Psi_1 \rvert^2
		+ g_{12} \lvert \Psi_2 \rvert^2
	\right) \Psi_1, \\
	\hbar \frac{\upd \Psi_2}{\upd \tau} & = -\left(
		-\frac{\hbar^2 \nabla^2}{2 m} + V_2
		+ g_{22} \lvert \Psi_2 \rvert^2
		+ g_{12} \lvert \Psi_1 \rvert^2
	\right) \Psi_2.
\end{eqn}

The rigorous proof of this method was derived by Bao and Du~\cite{Bao2004}.
The idea can be roughly illustrated by considering a one-component system with a linear Hamiltonian $\hat{H}$, whose eigenvalues are $\mu_1 < \mu_2 < ...$, where the lowest eigenvalue corresponds to ground state we want to find.
The steady solution of the time-dependent \abbrev{gpe}
\begin{eqn}
	i \hbar \frac{\upd \Psi}{\upd t} = \hat{H} \Psi
\end{eqn}
then looks like
\begin{eqn}
	\Psi(\xvec, t) = \sum_k e^{-\frac{i}{\hbar}\mu_k t} f_k(\xvec),
\end{eqn}
where $f_k$ are eigenfunctions of $\hat{H}$ corresponding to the eigenvalues $\mu_k$.
After the substitution $t \rightarrow \tau = it$ the solution becomes fading, with higher-energy terms fading faster:
\begin{eqn}
	\Psi(\xvec, \tau) = \sum_k e^{-\frac{1}{\hbar}\mu_k \tau} f_k(\xvec).
\end{eqn}

Therefore, if we take some random initial solution and propagate it long enough in imaginary time using~\eqnref{bec-noise:mean-field:imaginary-time}, the higher-energy terms will eventually die out (in comparison with the lowest-energy state) and leave us with the desired ground state.
The state obtained from the Thomas-Fermi approximated \abbrev{gpe} can be taken as the initial one since it is rather close to the desired one (and, therefore, higher-energy terms are already quite small).

Since the population will decrease exponentially after each step, and the precision of numerical calculations is limited, a renormalisation after each step will be required.
The total number of atoms in the ground state serves best in this case (because we will have to renormalise the final ground state anyway):
\begin{eqn}
	\int\limits_V \lvert \Psi(\tau, \xvec) \rvert^2 \upd V = N.
\end{eqn}

Propagation is terminated when the Gross-Pitaevskii energy of the state converges to the required precision (that is, only one eigenstate with the lowest energy is left out).
The energy of the two-component condensate~\cite{Pitaevskii2003}
\begin{eqn}
\label{eqn:bec-noise:mean-field:two-comp-energy}
	E[\Psivec] ={} & \int\limits_A \left(
		- \frac{\hbar^2 \Psi_1^* \nabla^2 \Psi_1}{2m}
		- \frac{\hbar^2 \Psi_2^* \nabla^2 \Psi_2}{2m}
	\right. \\
	& \left.
		+ (V_1 + \hbar \omega_1) n_1 + (V_2 + \hbar \omega_2) n_2
		+ \frac{g_{11}}{2} n_1^2 + \frac{g_{22}}{2} n_2^2 + g_{12} n_1 n_2
	\right) \upd\xvec
\end{eqn}
thus has to be calculated after each step and compared to the previous value, waiting for the desired precision to be reached.

\begin{figure}
\centerline{%
\includegraphics{figures_generated/mean_field/two_comp_gs_miscible.pdf}%
\includegraphics{figures_generated/mean_field/two_comp_gs_immiscible.pdf}}

\caption[Two-component ground state for miscible and immiscible regimes]{
Axial densities of two-component ground state for \textbf{(a)}~a miscible and \textbf{(b)}~an immiscible regime of \Rb{} \abbrev{bec} with $N_1 = N_2 = 40,000$ atoms in a three-dimensional harmonic trap with the frequencies $f_x = f_y = 97.6\un{Hz},\,f_z = 11.96\un{Hz}$.
Blue solid lines and red dashed lines show the axial density of the first and the second component respectively.
Intra-component scattering lengths are $a_{11} = 100.40\,r_B$, $a_{22} = 95.68\,r_B$, inter-component scattering length is \textbf{(a)}~$a_{12} = 97.0\,r_B$ and \textbf{(b)}~$a_{11} = 99.0\,r_B$, where $r_B$ is the Bohr radius.}%endcaption
\label{fig:bec-noise:mean-field:two-comp-gs}
\end{figure}

As an example, \figref{bec-noise:mean-field:two-comp-gs} shows the axial density $n_z = \int n(\xvec) \upd x \upd y$ of the two-component ground state for an equal mix of two hyperfine states of \Rb{} \abbrev{bec}.
Two panes of the figure illustrate the difference between the miscible ($a_{12}^2 < a_{11} a_{22}$) and immiscible ($a_{12}^2 > a_{11} a_{22}$) regimes.

It should be emphasized that this ground state is not a true many-body ground state, but rather a type of mean-field (single-particle) approximation, since it omits quantum correlations.


% =============================================================================
\subsection{Thomas-Fermi approximation}
% =============================================================================

As was mentioned earlier in this section, the starting state for the imaginary time calculation can be set to the Thomas-Fermi approximate state to minimize the propagation time.
Thomas-Fermi approximation consists of neglecting the kinetic term in the stationary equations~\eqnref{bec-noise:mean-field:cgpes-stationary}.
For a one-component state ($\Psi_2 \equiv 0$), the resulting equations can be easily solved analytically:
\begin{eqn}
\label{eqn:bec-noise:mean-field:tf-gs}
	| \Psi_1(\xvec) |^2 = \frac{1}{g_{11}} \max \left( \mu_1 - V_1(\xvec), 0 \right).
\end{eqn}
In the $D$-dimensional harmonic trap potential
\begin{eqn}
\label{eqn:bec-noise:mean-field:trap-potential}
	V_1(\xvec) = \frac{m}{2} \sum_{d=1}^D \omega_d^2 x_d^2,
\end{eqn}
this solution has the shape of an ellipsoid with radii $r_d = \sqrt{2\mu_1 / (m \omega_d^2)}$.

The chemical potential $\mu_1$ is fixed by the normalisation condition $\int |\Psi_1|^2 \upd \xvec = N_1$.
For a three-dimensional trap it can be shown to be
\begin{eqn}
	\mu_1^{\mathrm{(3D)}} =
		\left( \frac{15 N_1}{8 \pi} \right)^{\frac{2}{5}}
		\left( \frac{m \bar{\omega}^2}{2} \right)^{\frac{3}{5}}
		{g_{11}}^{\frac{2}{5}},
\end{eqn}
where $\bar{\omega} = \sqrt[3]{\omega_x \omega_y \omega_z}$.
For a one-dimensional trap it has the form
\begin{eqn}
	\mu_1^{\mathrm{(1D)}} =
		\left( \frac{3 g_{11} N_1}{4} \right)^{\frac{2}{3}}
		\left( \frac{m \omega_1^2}{2} \right)^{\frac{1}{3}}.
\end{eqn}

Now we can roughly estimate the conditions necessary to neglect the kinetic term from the equation.
Substituting approximate solution~\eqnref{bec-noise:mean-field:tf-gs} to~\eqnref{bec-noise:mean-field:cgpes-stationary} and comparing the kinetic term with the potential term, we get the following inequation:
\begin{eqn}
\label{eqn:bec-noise:mean-field:tf-inequation}
	\frac{\hbar^2}{2m} \left(
		\frac{m \sum_{d=1}^D \omega_d^2}{2}
		+ \frac{m^2 \sum_{d=1}^D \omega_d^4 x_d^2}
			{4 \left( \mu_1 - V_1(\xvec) \right)}
	\right) \ll
	\mu \left(\mu_1 - V_1(\xvec)\right).
\end{eqn}
Near the centre of the condensate, this inequation simplifies to
\begin{eqn}
\label{eqn:bec-noise:mean-field:tf-condition}
	\mu \gg \frac{\hbar}{2} |\bomega|,
\end{eqn}
where $\bomega \equiv (\omega_1, \ldots, \omega_D)^T$ is the vector of trap frequencies.

On the other hand, near the edges of the cloud the left-hand side of the inequation~\eqnref{bec-noise:mean-field:tf-inequation} diverges, while the right-hand side tends to zero.
This means that near the edges the Thomas-Fermi approximation fails regardless of the conditions.
Fortunately, the particle density there is low, so we can estimate the width $h$ of the ``belt'' where our first approximation of the state function is significantly incorrect.
If it happens to be small as compared to the size of the condensate, the approximation can be considered valid.

The first term at the left-hand side of the inequation~\eqnref{bec-noise:mean-field:tf-inequation} is constant and can be dropped in the limit of $V_1(\xvec) \rightarrow \mu_1$.
Then, for the sake of simplicity, we assume all but one of coordinates to be zero and the remaining one to be equal to $r_d - h_d$, where $r_d$ is the corresponding radius of the condensate.
After replacing ``$\ll$'' by ``$\approx$'' and assuming $h_d$ to be small as compared to $r_d$, we obtain the conditions for each coordinate:
\begin{eqn}
	h_d \approx \sqrt{\frac{\hbar^2}{2 \mu_1 m}},\,d \in [1, \ldots, D].
\end{eqn}
These have to be much smaller than the corresponding radii, which gives us
\begin{eqn}
	\mu_1 \gg \frac{1}{2} \hbar \max_{d \in [1, \ldots, D]} \omega_d.
\end{eqn}
This condition is less strict than the condition for the centre of the condensate.
Therefore, we have only one condition justifying the application of the Thomas-Fermi approximation is~\eqnref{bec-noise:mean-field:tf-condition}.

\begin{figure}
\centerline{%
	\includegraphics{figures_generated/mean_field/one_comp_gs_large.pdf}%
	\includegraphics{figures_generated/mean_field/one_comp_gs_small.pdf}}
\caption[Numerically calculated and Thomas-Fermi approximated ground states]{
Axial densities of numerically calculated (blue solid lines) and Thomas-Fermi approximated (red dashed lines) ground states for a one-component \Rb{} \abbrev{bec} of \textbf{(a)}~$100,000$ atoms, and \textbf{(b)}~$1,000$ atoms.}%endcaption
\label{fig:bec-noise:mean-field:tf-vs-accurate}
\end{figure}

Let us use some real-life experimental parameters and check how well the Thomas-Fermi approximation works.
For a three-dimensional trap with the frequencies $f_x = f_y = 97.6\un{Hz}$ and $f_z = 11.96\un{Hz}$ and $N_1=10^5$ \Rb{} atoms (which have a scattering length of $a_{11} = 100.4 r_B$), we have $2 \mu_1 / (\hbar |\bomega|) \approx 10.68$.
This means that the Thomas-Fermi approximation produces a solution which is close to the real one.
However, for a lower amount of atoms, say $N_1=10^3$, we obtain $2 \mu / (\hbar |\bomega|) \approx 1.69$, which is a sign that we are reaching the limit of the approximation's applicability.
\figref{bec-noise:mean-field:tf-vs-accurate} shows the axial density for both cases: for $100,000$ atoms the Thomas-Fermi approximation is very close to the numerically calculated ground state and for $1,000$ atoms it differs significantly as expected.

An extended discussion of the Thomas-Fermi approximation was given by Dalfovo \textit{et~al}~\cite{Dalfovo1999}.
It is possible to work out the Thomas-Fermi approximation for a two-component condensate, but it requires some non-trivial handling of various miscibility/immiscibility cases~\cite{Anderson2010}.
In this thesis we only single-component ground states, as these are the only ones used in the experiments we are considering.

% =============================================================================
\section{Wigner transformation}
% =============================================================================


\copypaste{
We start by assuming that the BEC has $s$-wave interactions, together with Markovian losses due to $n$-body collisions.
We employ a master equation together with the Wigner-Moyal quantum phase-space representation~\cite{Gardiner2004} and a truncation of third- and higher-order derivatives in the equations of motion.
If we regard the commonly used Gross-Pitaevskii equation as a classical, first approximation to mean-field condensate dynamics, the truncated Wigner approach is best thought of as the second term in an expansion in inverse particle number.
}

\todo{
This section should be mostly dedicated to results in \cite{Egorov2011} and \cite{Opanchuk2012} (and possibly \cite{Egorov2013}).
}


In the present Letter, we treat an ultra-cold,
interacting multi-component spinor Bose gas in $D$ effective dimensions.
The basic Hamiltonian is easily expressed using quantum fields
$\Psiop_j^{\dagger}(\xvec)$ and $\Psiop_j(\xvec)$,
where $\Psiop_j^{\dagger}(\xvec)$ creates a bosonic atom of spin $j$
at location $\xvec$, and $\Psiop_j(\xvec)$ destroys one;
the commutators are
$[\Psiop_j(\xvec),\Psiop_k^{\dagger}(\xvec^\prime)] =
\delta^{(D)}(\xvec-\xvec^\prime)\delta_{jk}.$
The resulting physics of a dilute, low-temperature Bose gas
is well-described in the $s$-wave scattering limit by an effective Hamiltonian
with contact interactions and external potentials:
\begin{equation}
    \hat{H} / \hbar = \int \upd^{D}\xvec \left\{
        \Psiop_j^{\dagger} K_{jk} \Psiop_k +
        \frac{U_{jk}}{2} \Psiop_j^{\dagger} \Psiop_k^{\dagger}
        \Psiop_k \Psiop_j
    \right\}.
\end{equation}
Here we omit the field argument $(\xvec)$ for brevity,
and use the Einstein summation convention of summing over repeated indices.
$K_{jk}$ is the single-particle Hamiltonian:
\begin{equation}
    K_{jk} = \left( -\frac{\hbar}{2m} \nabla^2 + \omega_j + V_j(\xvec) / \hbar \right) \delta_{jk} +
        \tilde{\Omega}_{jk}(t),
\end{equation}
where $m$ is the atomic mass, $V_j$ is the external trapping potential for spin $j$,
$\omega_j$ is the internal energy of spin $j$,
$\tilde{\Omega}_{jk}$ represents a time-dependent coupling
that is used to rotate one spin projection into another,
and $U_{jk}$ is the atom-atom interaction term.
Thus, $n_j = \langle \Psiop_j^{\dagger} \Psiop_j \rangle$
is the spin-$j$ atomic density.
For a dilute gas at low enough temperatures,
$U_{jk}=4\pi\hbar a_{jk} / m$, where $a_{jk}$ is the $s$-wave scattering length in three dimensions.
Here we assume a momentum cutoff $k_{c} \ll 1 / a_{jk}$,
otherwise the couplings must be renormalized~\cite{Sinatra2002}.

We proceed by using a stochastic phase-space method that allows a numerical
simulation of the quantum dynamics~\cite{Drummond1993,Steel1998,Hoffmann2008}.
Defining a Wigner function $W(\Psivec)$, where $\Psi_j$
is a c-number field corresponding to the quantum field $\hat{\Psi}_j$, this has a unitary time-evolution equation:
\begin{equation}
    \frac{\partial W}{\partial t} = \int \upd^D\xvec \left\{
        - \frac{\delta}{\delta\Psi_j} A_j
        - \frac{\delta}{\delta\Psi_j^*}A_j^*
        + \mbox{O} \left[ \frac{\delta^3}{\delta\Psi_j^3} \right]
    \right\} W.
\end{equation}
Next, higher-derivative terms of type $\mbox{O} \left[ \delta^3 / \delta\Psi_j^3 \right]$ are truncated.
This approximation neglects higher-order terms in an expansion in $1 / \sqrt{N}$,
and is therefore valid in the limit of $N \gg M$
where $N$ is the atom number and $M$ is the number of low-energy modes included~\cite{Drummond1993,Sinatra2002,Norrie2006}.
In free-space calculations it is important to maintain this mode truncation.
In the relevant limits where the technique is applicable, the equations
simply reduce to Gross-Pitaevskii equations with Gaussian fluctuations
of the initial conditions:
\begin{equation}
\label{eqn:SDE-1}
    \frac{\upd\Psi_j}{\upd t} = -i \left(
        K_{jk} \Psi_k + U_{jk} \lvert \Psi_k \rvert^2 \Psi_j
    \right).
\end{equation}
For initial conditions in interferometry it is usually sufficient to consider
a coherent state amplitude $\Psi_s^c$,
corresponding to a typical initial state with Poissonian number fluctuations,
as produced by a beam-splitter.
In this case the initial Wigner amplitude has a Gaussian random distribution, with
$\Psi_j(\xvec, t_0) = \Psi_j^c(\xvec) + \Delta \Psi_j(\xvec)$, where:
$\left\langle \Delta \Psi_j(\xvec) \Delta \Psi_k^*(\xvec^{\prime}) \right\rangle =
\delta_{jk} \delta^D(\xvec - \xvec^{\prime}) / 2.$
This initial noise is necessary because the Wigner representation generates
symmetrically ordered correlation functions, and includes vacuum fluctuations.
For greater accuracy, the initial state can be modified to account for
initial  correlations, thermal noise, or additional fluctuations.
If normal ordered correlations are measured, one has to express them
as a sum of symmetrically ordered terms.

This includes all the known nonlinear quantum noise effects of quantum dynamics,
like phase diffusion, entanglement and quantum squeezing, in the limit
of large particle number.
The initial noise terms do not occur in the semi-classical Gross-Pitaevskii
approximation, which is therefore unable to predict these effects.
Thus, while the lossless equations are identical to the Gross-Pitaevskii
equations, the inclusion of initial noise terms together with nonlinear
interactions leads to quantum phase-diffusion.
Such methods can be used for either free-space or trapped atom interferometry,
provided there is an appropriate mode truncation.

Additional quantum noise enters from the effects of damping and losses,
due to the fluctuation-dissipation theorem.
These effects are important at high densities in atomic traps.
They can be included via an additional Markovian master equation~\cite{Jack2002}
defined so that,
\begin{equation}
    \frac{\upd\hat{\rho}}{\upd t} =
        - \frac{i}{\hbar} \left[ \hat{H}, \hat{\rho} \right]
        + \sum_{n,\lvec} \kappa_{\lvec}^{(n)} \int \upd^{D}\xvec
            \mathcal{L}_{\lvec}^{(n)} \left[ \hat{\rho} \right],
\end{equation}
where $n$ is the number of interacting particles,
$\lvec = (l_1, l_2, \ldots, l_n)$ is a vector indicating the spins that are coupled,
and we have introduced local Liouville loss terms,
\begin{equation}
    \mathcal{L}_{\lvec}^{(n)} \left[ \hat{\rho} \right] =
        2\hat{O}_{\lvec}^{(n)} \hat{\rho} \hat{O}_{\lvec}^{(n)\dagger}
        - \hat{O}_{\lvec}^{(n)\dagger} \hat{O}_{\lvec}^{(n)} \hat{\rho}
        - \hat{\rho} \hat{O}_{\lvec}^{(n)\dagger} \hat{O}_{\lvec}^{(n)}.
\end{equation}
The reservoir coupling operators $\hat{O}_{\lvec}^{(n)}$ are the distinct $n$-fold products of local field annihilation operators,
$\hat{O}_{\lvec}^{(n)} = \hat{O}_{\lvec}^{(n)} (\widehat{\Psivec}) =
    \Psiop_{l_{1}} (\xvec)
    \Psiop_{l_{2}} (\xvec) \ldots
    \Psiop_{l_{n}} (\xvec),$
describing local $n$-body collision losses.

After transforming these new terms to evolution equations for the Wigner distribution, the drift term $A_j$
changes the Gross-Pitaevskii evolution to include nonlinear damping, while
the next terms in the evolution equation give rise to additional Fokker-Planck
diffusion terms associated with quantum noise from the loss reservoirs,
given by:
\begin{equation}
    \frac{\delta^{2}}{\delta\Psi_j\delta\Psi_k^{*}} \left\{
        \sum_{n,\lvec} \kappa_{\lvec}^{(n)}
            \frac{\partial O_{\lvec}^{(n)*}}{\partial\Psi_j^{*}}
            \frac{\partial O_{\lvec}^{(n)}}{\partial\Psi_k}
        \right\} W.
\end{equation}

This leads to a stochastic equation:
\begin{equation}
\label{eqn:SDE}
    \frac{\upd\Psi_j}{\upd t} =
        - i\left( K_{jk} \Psi_k + U_{jk} \lvert \Psi_k \rvert^{2} \Psi_j \right)
        - \Gamma_j
        + \sum_{n,\lvec} \beta_{\lvec,j}^{(n)} \zeta_{\lvec}^{(n)}(\xvec,t),
\end{equation}
where the nonlinear loss has the form:
\begin{equation}
    \Gamma_j = \sum_{n,\lvec}
        \kappa_{\lvec}^{(n)}
        \frac{\partial O_{\lvec}^{(n)*} (\Psivec)}{\partial\Psi_j^{*} (\xvec)}
        O_{\lvec}^{(n)}(\xvec),
\end{equation}
and $\zeta_{\lvec}^{(n)}(\xvec, t)$ is a corresponding complex,
stochastic delta-correlated Gaussian noise with
\begin{equation}
    \left\langle
        \zeta_{\lvec}^{(n)} (\xvec,t) \zeta_{\kvec}^{(m)*}(\xvec^\prime, t^\prime)
    \right\rangle =
    \delta_{\lvec \kvec} \delta^{nm} \delta^{D} \left(
        \xvec - \xvec^\prime
    \right)
    \delta \left( t - t^\prime \right).
\end{equation}
The multiplicative noise coefficient
\begin{equation}
    \beta_{\lvec,j}^{(n)} \left( \Psivec \right) =
    \sqrt{\kappa_{\lvec}^{(n)}}
    \frac{\partial O_{\lvec}^{(n)}}{\partial\Psi_j}
\end{equation}
is a fluctuation-dissipation term,
so that the Wigner variables remain equivalent to the corresponding operators.

The loss coefficients in eq.~(\ref{eqn:SDE}) can be converted to the conventional form,
which is defined using atom number losses:
\begin{equation}
    \dot{n}_j = - \gamma^{(n)}_{\lvec,j} n^{m_1}_1 n^{m_2}_2 \ldots ,
\end{equation}
where $n_j$ is the density of component $j$ and $m_j$
is the number of spin-$j$ atoms lost in the collision.
The conversion can be carried out as $\gamma^{(n)}_{\lvec,j} = 2 m_j \kappa^{(n)}_{\lvec}$.

In this work we use a basis of plane waves in the volume $V$,
and the density of component $j$ is calculated as a probabilistic average:
\begin{equation}
\label{eqn:wigner-density}
    n_j (\xvec)
        = \langle \Psi^*_j (\xvec) \Psi_j (\xvec) \rangle_{\mathrm{paths}} - \frac{M}{2V}.
\end{equation}
Here we use the fact that the approximate Wigner function is a probability distribution
equivalent to an averaged sum over different simulation paths.

\begin{figure}
    %\begin{tabular}{l l}
    %\imagetop{\hspace*{0.44in}\includegraphics[width=0.72\columnwidth]{ramsey_sequence.eps}} & \imagetop{(a)} \\
    %\imagetop{\includegraphics[width=0.85\columnwidth]{long_ramsey_visibility.eps}} & \imagetop{(b)} \\
    %\imagetop{\hspace*{0.44in}\includegraphics[width=0.72\columnwidth]{echo_sequence.eps}} & \imagetop{(c)} \\
    %\imagetop{\includegraphics[width=0.85\columnwidth]{long_rephasing_visibility.eps}} & \imagetop{(d)}
    %\end{tabular}

    \caption{
    Timeline of the experiment for Ramsey (a) and Ramsey with spin echo (c); (b) and (d) are the simulated plots of interferometric visibility.
    Classical GPE (red dashed lines) and Wigner calculations (blue solid lines) are shown.
    $N = 5.5 \times 10^4$,
    $\omega_x = \omega_y = 2 \pi \times 97.0\un{Hz}$,
    $\omega_z = 2 \pi \times 11.69\un{Hz}$,
    $a_{11} = 100.4\,a_0$, $a_{12} = 97.993\,a_0$, $a_{22} = 95.57\,a_0$~\cite{Egorov2011},
    $a_0$ is the Bohr radius.
    Nonlinear atomic losses:
    $\gamma^{(3)}_{111} = 5.4 \times 10^{-30}\un{cm^6/s}$~\cite{Mertes2007},
    $\gamma^{(2)}_{12} = 1.51 \times 10^{-14}\un{cm^3/s}$,
    $\gamma^{(2)}_{22} = 8.1 \times 10^{-14}\un{cm^3/s}$~\cite{Egorov2011}.}

    \label{fig:visibility}
\end{figure}

To illustrate the applications of this method we consider recent interferometry
experiments with a two-component BEC involving two hyperfine states
${\ket{F=1,\, m_F=-1}}$ and ${\ket{F=2,\, m_F=+1}}$ in \Rb~\cite{Egorov2011}.
A conventional Ramsey sequence (\figref{visibility},~(a)) has been used
with a BEC confined in a cigar-shaped magnetic trap with the frequencies $(97.0, 97.0, 11.69)\un{Hz}$
in a bias magnetic field of $3.23\un{G}$, so that magnetic field dephasing is largely eliminated~\cite{Hall1998}.
The first $\pi/2$ pulse prepares a non-equilibrium superposition of states ${\ket{1,-1}}$ and ${\ket{2,+1}}$
and the spatial modes of two components periodically separate and merge again~\cite{Mertes2007}.
The spatially-separated spin components evolve differently, as they have
different scattering lengths.
As a result, these collective oscillations lead to periodic dephasing and
self-rephasing of the BEC components, clearly visible in both GPE and Wigner
simulations of interference fringe visibility
$\mathcal{V}$ (\figref{visibility},~(b)).
Asymmetric losses of two states are one cause of the contrast decay.
This can be partially compensated by the application of a spin echo pulse
mid-way through the evolution (\figref{visibility},~(c)).
The GPE simulations wrongly predict (dashed lines) that visibility is largely
recovered at long evolution times using the spin echo method.
However, the addition of quantum noise (solid line) via the Wigner simulations
noticeably speeds up the visibility decay even with a spin echo pulse present.
This is in agreement with experimental observations, and shows that these
effects play a significant part in the decay of visibility, even for
large particle numbers.

The important feature of these quantum dynamical simulations
is that they are able to treat large numbers of atoms (55,000 in this case),
while correctly tracking all the quantum noise sources, and also extending the simulations to long time-scales.
Both of these features, large atom numbers and long time-scales,
are essential ingredients to accurate interferometric measurements.
The simulations give accurate predictions despite large, multi-mode dynamical motion in three dimensions
and substantial losses of most of the condensate atoms~\cite{Egorov2011}.
On longer time-scales, the experimental accuracy is limited by technical noises, and we have no data for comparisons.



% =============================================================================
\chapter{Squeezing in 3D BEC interferometry}
\label{cha:bec-squeezing}
% =============================================================================

This chapter contains squeezing results from \cite{Opanchuk2012}.

% =============================================================================
\section{Moments of field operator in Wigner representation}
% =============================================================================


This chapter shows how to calculate different kinds of observables from wavefunctions in Wigner representation.
From the definition of Wigner function~\cite{Gardiner2004}:
\[
	\langle \symprod{ \hat{a}^r ( \hat{a}^\dagger)^s } \rangle
	= \int \alpha^r (\alpha^*)^s W (\alpha, \alpha^*) d^2\alpha ,
\]
where $\{\}_{\mathrm{sym}}$ stands for symmetrically ordered operator product.
It can be shown that similar relation applies for the multimode field operator:
\[
	\langle \symprod{ \Psiop^r ( \Psiop^\dagger)^s } \rangle
	= \int \Psi^r (\Psi^*)^s W (\Psi, \Psi^*) \delta^2\Psi.
\]
This equation can be further generalised for multi-component field.
In simulations, Wigner function $W$ can be treated as the probability distribution, allowing to replace the integral by average over simulation paths:
\[
	\int \Psi^r (\Psi^*)^s W (\Psi, \Psi^*) \delta^2\Psi
	= \pathavg{ \Psi^r (\Psi^*)^s }
	= \frac{1}{N_{\mathrm{paths}}} \sum\limits_{j=1}^{N_{\mathrm{paths}}}
		\Psi^{(j)r} (\Psi^{(j)*})^s,
\]
where superscript $(j)$ denotes the value taken from $j$-th simulation path.


% =============================================================================
\subsection{Number of atoms}
% =============================================================================

First example is the calculation of atom density:
\begin{equation*}
\begin{split}
		\langle \hat{n} (\xvec) \rangle
		& = \langle \Psiop^\dagger (\xvec) \Psiop (\xvec) \rangle \\
		& = \langle
				\symprod{ \Psiop^\dagger \Psiop }
			\rangle - \frac{1}{2} \delta_P (\xvec, \xvec) \\
		& = \pathavg{ \Psi^* (\xvec) \Psi (\xvec) }
			- \frac{1}{2} \delta_P (\xvec, \xvec) \\
		& = \pathavg{ n (\xvec) }
			- \frac{1}{2} \delta_P (\xvec, \xvec).
\end{split}
\end{equation*}
Defining population operator $\hat{N}$ as
\[
	\hat{N} = \int \hat{n} (\xvec) d\xvec,
\]
we can get the average of total population:
\[
		\langle \hat{N} \rangle
		= \int \langle \hat{n}(\xvec) \rangle d\xvec
		= \int \pathavg{ n(\xvec) } d\xvec - \frac{M}{2}
		= \pathavg{ \int n(\xvec) d\xvec } d\xvec - \frac{M}{2}
		= \pathavg{N} - \frac{M}{2},
\]
where $\delta_P (\xvec, \xvec)$ is a restricted delta function from \defref{func-calculus:restricted-delta}.
Its integral over space equals to the number of modes $M$ in the restricted basis.

Variance of total number $N$ is expressed in a slightly more complicated way.
\[
	(\Delta N)^2
		= \langle \hat{N}^2 \rangle - \langle \hat{N} \rangle^2
\]
Average of $\hat{N}^2$ requires some work.
Denoting $\Psiop(\xvec) \equiv \Psiop$ and $\Psiop(\xvec^\prime) \equiv \Psiop^\prime$ for simplicity:
\[
	\hat{N}^2
		= \int \Psiop^\dagger \Psiop d\xvec
			\int \Psiop^\dagger \Psiop d\xvec
		= \int
			\Psiop^\dagger \Psiop
			\Psiop^{\prime\dagger} \Psiop^\prime
			d\xvec d\xvec^\prime
\]
\begin{equation*}
\begin{split}
	\langle
		\Psiop^\dagger \Psiop \Psiop^{\prime\dagger} \Psiop^\prime
	\rangle
	& = \langle
		\symprod{ \Psiop^{\prime\dagger} \Psiop^\prime \Psiop^\dagger \Psiop}
		- \frac{\delta_P(\xvec^\prime,\xvec^\prime)}{2} \symprod{\Psiop^\dagger \Psiop}
		- \frac{\delta_P(\xvec,\xvec)}{2} \symprod{\Psiop^{\prime\dagger} \Psiop^\prime} \\
	& - \frac{\delta_P(\xvec,\xvec^\prime)}{2} \symprod{\Psiop^{\prime\dagger} \Psiop}
		+ \frac{\delta_P(\xvec^\prime,\xvec)}{2} \symprod{\Psiop^\dagger \Psiop^\prime}
		+ \frac{\delta_P(\xvec,\xvec) \delta_P(\xvec^\prime,\xvec^\prime)}{2}
	\rangle.
\end{split}
\end{equation*}
Therefore the average of $\hat{N}^2$ is:
\begin{eqn*}
	\langle \hat{N}^2 \rangle & = \int
		\langle
			\Psiop^\dagger \Psiop \Psiop^{\prime\dagger} \Psiop^\prime
		\rangle
	d\xvec d\xvec^\prime \\
	& = \int \pathavgleft
		\Psi^* \Psi \Psi^{\prime *} \Psi^\prime
		- \frac{\delta_P(\xvec^\prime,\xvec^\prime)}{2} \Psi^* \Psi
		- \frac{\delta_P(\xvec,\xvec) }{2} \Psi^{\prime *} \Psi^\prime \right. \\
	&	\left. - \frac{\delta_P(\xvec,\xvec^\prime)}{2} \Psi^{\prime *} \Psi
		+ \frac{\delta_P(\xvec^\prime,\xvec)}{2} \Psi^* \Psi^\prime
		+ \frac{\delta_P(\xvec,\xvec) \delta_P(\xvec^\prime,\xvec^\prime)}{2}
	\pathavgright d\xvec d\xvec^\prime \\
	& = \pathavgleft
		\int \Psi^* \Psi d\xvec \int \Psi^* \Psi d\xvec
		- \frac{M}{2} \int \Psi^* \Psi d\xvec
		- \frac{M}{2} \int \Psi^* \Psi d\xvec \right. \\
	&	\left. - \int \frac{\delta_P(\xvec,\xvec^\prime)}{2} \Psi^{\prime *} \Psi d\xvec d\xvec^\prime
		+ \int \frac{\delta_P(\xvec^\prime,\xvec)}{2} \Psi^* \Psi^\prime d\xvec d\xvec^\prime
		+ \frac{M^2}{2}
	\pathavgright \\
	& = \pathavg{N^2 - M N + \frac{M^2}{2}}.
\end{eqn*}
Here we used the correspondence between the average of the symmetric product of field operators and the average of wavefunctions over simulation paths.
Then we can split variables in double integrals, allowing us to group terms and simplify the whole equation.
Substituting this into equation for $(\Delta N)^2$:
\begin{equation}
\label{eqn:moments-calculation:delta-N}
	(\Delta N)^2
		= \langle \hat{N}^2 \rangle - \langle \hat{N} \rangle^2
		= \pathavg{N^2 - M N + \frac{M^2}{2}} - (\pathavg{N} - \frac{M}{2})^2
		= \pathavg{N^2} - \pathavg{N}^2 + \frac{M^2}{4}
\end{equation}


% =============================================================================
\subsection{Spin vector}
% =============================================================================

Another example is the spin vector, whose averages and variances are required for squeezing calculation~\cite{Li2009}.
Spin operators are defined as following:
\begin{equation}
\label{eqn:moments-calculation:spin-operators}
\begin{split}
	\hat{S}_x & = \frac{1}{2} \int \left(
		\Psiop^\dagger_2 \Psiop_1 + \Psiop^\dagger_1 \Psiop_2
	\right) d\xvec, \\
	\hat{S}_y & = \frac{i}{2} \int \left(
		\Psiop^\dagger_2 \Psiop_1 - \Psiop^\dagger_1 \Psiop_2
	\right) d\xvec, \\
	\hat{S}_z & = \frac{1}{2} \int \left(
		\Psiop^\dagger_1 \Psiop_1 - \Psiop^\dagger_2 \Psiop_2
	\right) d\xvec.
\end{split}
\end{equation}
Averages of spin operators can be calculated straightforwardly (using the fact that interspecies commutators $[\Psiop_1, \Psiop_2] = [\Psiop^\dagger_1, \Psiop_2] = 0$):
\begin{equation*}
\begin{split}
	\langle \hat{S}_x \rangle
		& = \pathavg{\Real \int \Psi^*_1 \Psi_2 d\xvec }
		= \pathavg{\Real I}
		= \pathavg{S_x}, \\
	\langle \hat{S}_y \rangle
		& = \pathavg{\Imag \int \Psi^*_1 \Psi_2 d\xvec }
		= \pathavg{\Imag I}
		= \pathavg{S_y}, \\
	\langle \hat{S}_z \rangle
		& = \frac{1}{2} \pathavg{\int \Psi^*_1 \Psi_1 d\xvec - \int \Psi^*_2 \Psi_2 d\xvec}
		= \frac{1}{2} (\pathavg{N_1 - N_2})
		= \pathavg{S_z},
\end{split}
\end{equation*}
where we introduced auxiliary per-path interaction values $I^{j}$ and per-path spin component values, whose definitions are an intuitive consequence of equations~\eqnref{moments-calculation:spin-operators}:
\begin{equation*}
\begin{split}
	S^{(j)}_x & = \frac{1}{2} \int \left(
		\Psi^{(j)*} \Psi^{(j)}_1 + \Psi^{(j)*}_1 \Psi^{(j)}_2
	\right) d\xvec, \\
	S^{(j)}_y & = \frac{i}{2} \int \left(
		\Psi^{(j)*}_2 \Psi^{(j)}_1 - \Psi^{(j)*}_1 \Psi^{(j)}_2
	\right) d\xvec, \\
	S^{(j)}_z & = \frac{1}{2} \int \left(
		\Psi^{(j)*}_1 \Psi^{(j)}_1 - \Psi^{(j)*}_2 \Psi^{(j)}_2
	\right) d\xvec,
\end{split}
\end{equation*}
where $j$ stands for the number of the simulation path.

Second-order moments of spin operators can be obtained similarly to second-order moment of population operator, by transforming normally ordered field operator products to symmetrically ordered ones, substituting them for path averages of wavefunction moments and grouping terms.
\begin{equation*}
\begin{split}
	\langle \hat{S}^2_x \rangle
	& = \frac{1}{4} \langle \int \left(
		\Psiop^\dagger_2 \Psiop_1 + \Psiop^\dagger_1 \Psiop_2
	\right)
	\left(
		\Psiop^{\prime\dagger}_2 \Psiop^\prime_1 + \Psiop^{\prime\dagger}_1 \Psiop^\prime_2
	\right) d\xvec d\xvec^\prime \rangle \\
	& = \frac{1}{4} \langle \int \left(
		\symprod{ \Psiop^\dagger_2 \Psiop_1 \Psiop^{\prime\dagger}_2 \Psiop^\prime_1 }
		+ \symprod{ \Psiop^\dagger_1 \Psiop_2 \Psiop^{\prime\dagger}_2 \Psiop^\prime_1 }
		+ \symprod{ \Psiop^\dagger_1 \Psiop_2 \Psiop^{\prime\dagger}_1 \Psiop^\prime_2 }
		+ \symprod{ \Psiop^\dagger_2 \Psiop_1 \Psiop^{\prime\dagger}_1 \Psiop^\prime_2 }
	\right. \\
	& \left.
		+ \frac{\delta_P(\xvec,\xvec^\prime)}{2} \left(
			- \symprod{ \Psiop_2 \Psiop^{\prime\dagger}_2 }
			- \symprod{ \Psiop_1 \Psiop^{\prime\dagger}_1 }
			+ \symprod{ \Psiop^\dagger_2 \Psiop^\prime_1 }
			+ \symprod{ \Psiop^\dagger_1 \Psiop^\prime_2 }
		\right)
	\right) d\xvec d\xvec^\prime \rangle \\
	& = \frac{1}{4} \pathavgleft
		\int \Psi^*_2 \Psi_1 d\xvec \int \Psi^*_2 \Psi_1 d\xvec
		+ \int \Psi^*_1 \Psi_2 d\xvec \int \Psi^*_2 \Psi_1 d\xvec \right. \\
	&	\left. + \int \Psi^*_1 \Psi_2 d\xvec \int \Psi^*_1 \Psi_2 d\xvec
		+ \int \Psi^*_2 \Psi_1 d\xvec \int \Psi^*_1 \Psi_2 d\xvec \pathavgright \\
	& = \frac{1}{4} \pathavg{ (I^*)^2 + I I^* + I^2 + I^* I } \\
	& = \pathavg{ (\Real I)^2 } = \pathavg{ S^2_x }
\end{split}
\end{equation*}

\begin{equation*}
\begin{split}
	\langle \hat{S}^2_y \rangle
	& = - \frac{1}{4} \langle \int \left(
		\Psiop^\dagger_2 \Psiop_1 - \Psiop^\dagger_1 \Psiop_2
	\right)
	\left(
		\Psiop^{\prime\dagger}_2 \Psiop^\prime_1 - \Psiop^{\prime\dagger}_1 \Psiop^\prime_2
	\right) d\xvec d\xvec^\prime \rangle \\
	& = - \frac{1}{4} \langle \int \left(
		\symprod{ \Psiop^\dagger_2 \Psiop_1 \Psiop^{\prime\dagger}_2 \Psiop^\prime_1 }
		- \symprod{ \Psiop^\dagger_1 \Psiop_2 \Psiop^{\prime\dagger}_2 \Psiop^\prime_1 }
		+ \symprod{ \Psiop^\dagger_1 \Psiop_2 \Psiop^{\prime\dagger}_1 \Psiop^\prime_2 }
		- \symprod{ \Psiop^\dagger_2 \Psiop_1 \Psiop^{\prime\dagger}_1 \Psiop^\prime_2 }
	\right. \\
	& \left.
		+ \frac{\delta_P(\xvec,\xvec^\prime)}{2} \left(
			\symprod{ \Psiop_2 \Psiop^{\prime\dagger}_2 }
			+ \symprod{ \Psiop_1 \Psiop^{\prime\dagger}_1 }
			- \symprod{ \Psiop^\dagger_2 \Psiop^\prime_2 }
			- \symprod{ \Psiop^\dagger_1 \Psiop^\prime_1 }
		\right)
	\right) d\xvec d\xvec^\prime \rangle \\
	& = - \frac{1}{4} \pathavgleft
		\int \Psi^*_2 \Psi_1 d\xvec \int \Psi^*_2 \Psi_1 d\xvec
		- \int \Psi^*_1 \Psi_2 d\xvec \int \Psi^*_2 \Psi_1 d\xvec \right. \\
	&	\left. + \int \Psi^*_1 \Psi_2 d\xvec \int \Psi^*_1 \Psi_2 d\xvec
		- \int \Psi^*_2 \Psi_1 d\xvec \int \Psi^*_1 \Psi_2 d\xvec \pathavgright \\
	& = - \frac{1}{4} \pathavg{ (I^*)^2 - I I^* + I^2 - I^* I } \\
	& = \pathavg{ ( \Imag I )^2 } = \pathavg{ S^2_y }
\end{split}
\end{equation*}

\begin{equation*}
\begin{split}
	\langle \hat{S}^2_z \rangle
	& = \frac{1}{4} \langle \int \left(
		\Psiop^\dagger_1 \Psiop_1 - \Psiop^\dagger_2 \Psiop_2
	\right)
	\left(
		\Psiop^{\prime\dagger}_1 \Psiop^\prime_1 - \Psiop^{\prime\dagger}_2 \Psiop^\prime_2
	\right) d\xvec d\xvec^\prime \rangle \\
	& = \frac{1}{4} \langle \int \left(
		\symprod{ \Psiop^\dagger_1 \Psiop_1 \Psiop^{\prime\dagger}_1 \Psiop^\prime_1 }
		- \symprod{ \Psiop^\dagger_1 \Psiop_1 \Psiop^{\prime\dagger}_2 \Psiop^\prime_2 }
		- \symprod{ \Psiop^\dagger_2 \Psiop_2 \Psiop^{\prime\dagger}_1 \Psiop^\prime_1 }
		+ \symprod{ \Psiop^\dagger_2 \Psiop_2 \Psiop^{\prime\dagger}_2 \Psiop^\prime_2 }
	\right. \\
	& \left.
		+ \frac{\delta_P(\xvec,\xvec^\prime)}{2} \left(
			\symprod{ \Psiop^\dagger_1 \Psiop^\prime_1 }
			- \symprod{ \Psiop_2 \Psiop^{\prime\dagger}_2 }
			+ \symprod{ \Psiop^\dagger_2 \Psiop^\prime_2 }
			- \symprod{ \Psiop_1 \Psiop^{\prime\dagger}_1 }
		\right)
	\right) d\xvec d\xvec^\prime \rangle \\
	& = \frac{1}{4} \pathavgleft
		\int \Psi^*_1 \Psi_1 d\xvec \int \Psi^*_1 \Psi_1 d\xvec
		- \int \Psi^*_1 \Psi_1 d\xvec \int \Psi^*_2 \Psi_2 d\xvec \right. \\
	&	\left. - \int \Psi^*_2 \Psi_2 d\xvec \int \Psi^*_1 \Psi_1 d\xvec
		+ \int \Psi^*_2 \Psi_2 d\xvec \int \Psi^*_2 \Psi_2 d\xvec \pathavgright \\
	& = \frac{1}{4} \pathavg{ N^2_1 - N_1 N_2 - N_2 N_1 + N^2_2 } \\
	& = \frac{1}{4} \pathavg{ (N_1 - N_2)^2 } = \pathavg{ S^2_z }
\end{split}
\end{equation*}

\begin{equation*}
\begin{split}
	\langle \hat{S}_x \hat{S}_y + \hat{S}_y \hat{S}_x \rangle
	& = \frac{i}{4} \langle \int \left(
		\left(
			\Psiop^\dagger_2 \Psiop_1 + \Psiop^\dagger_1 \Psiop_2
		\right)
		\left(
			\Psiop^{\prime\dagger}_2 \Psiop^\prime_1 - \Psiop^{\prime\dagger}_1 \Psiop^\prime_2
		\right)
		+ \left(
			\Psiop^\dagger_2 \Psiop_1 - \Psiop^\dagger_1 \Psiop_2
		\right)
		\left(
			\Psiop^{\prime\dagger}_2 \Psiop^\prime_1 + \Psiop^{\prime\dagger}_1 \Psiop^\prime_2
		\right)
	\right) d\xvec d\xvec^\prime \rangle \\
	& = \frac{i}{2} \langle \int \left(
		\symprod{ \Psiop^\dagger_2 \Psiop_1 \Psiop^{\prime\dagger}_2 \Psiop^\prime_1 }
		- \symprod{ \Psiop^\dagger_1 \Psiop_2 \Psiop^{\prime\dagger}_1 \Psiop^\prime_2 }
	\right) d\xvec d\xvec^\prime \rangle \\
	& = \frac{i}{2} \pathavg{
		\int \Psi^*_2 \Psi_1 d\xvec \int \Psi^*_2 \Psi_1 d\xvec
		- \int \Psi^*_1 \Psi_2 d\xvec \int \Psi^*_1 \Psi_2 d\xvec } \\
	& = \frac{i}{2} \pathavg{ (I^*)^2 - I^2 } \\
	& = 2 \pathavg{ \Real I \, \Imag I } = 2 \pathavg{ S_x S_y }
\end{split}
\end{equation*}

\begin{equation*}
\begin{split}
	\langle \hat{S}_x \hat{S}_z + \hat{S}_z \hat{S}_x \rangle
	& = \frac{1}{4} \langle \int \left(
		\left(
			\Psiop^\dagger_2 \Psiop_1 + \Psiop^\dagger_1 \Psiop_2
		\right)
		\left(
			\Psiop^{\prime\dagger}_1 \Psiop^\prime_1 - \Psiop^{\prime\dagger}_2 \Psiop^\prime_2
		\right)
		+ \left(
			\Psiop^\dagger_1 \Psiop_1 - \Psiop^\dagger_2 \Psiop_2
		\right)
		\left(
			\Psiop^{\prime\dagger}_2 \Psiop^\prime_1 + \Psiop^{\prime\dagger}_1 \Psiop^\prime_2
		\right)
	\right) d\xvec d\xvec^\prime \rangle \\
	& = \frac{1}{4} \langle \int \left(
		\symprod{ \Psiop^\dagger_1 \Psiop_1 \Psiop^{\prime\dagger}_2 \Psiop^\prime_1 }
		- \symprod{ \Psiop^\dagger_2 \Psiop_2 \Psiop^{\prime\dagger}_1 \Psiop^\prime_2 }
		+ \symprod{ \Psiop^\dagger_2 \Psiop_1 \Psiop^{\prime\dagger}_1 \Psiop^\prime_1 }
		+ \symprod{ \Psiop^\dagger_1 \Psiop_2 \Psiop^{\prime\dagger}_1 \Psiop^\prime_1 }
	\right. \\
	& \left.
		- \symprod{ \Psiop^\dagger_2 \Psiop_1 \Psiop^{\prime\dagger}_2 \Psiop^\prime_2 }
		+ \symprod{ \Psiop^\dagger_1 \Psiop_1 \Psiop^{\prime\dagger}_1 \Psiop^\prime_2 }
		- \symprod{ \Psiop^\dagger_2 \Psiop_2 \Psiop^{\prime\dagger}_2 \Psiop^\prime_1 }
		- \symprod{ \Psiop^\dagger_1 \Psiop_2 \Psiop^{\prime\dagger}_2 \Psiop^\prime_2 }
	\right) d\xvec d\xvec^\prime \rangle \\
	& = \frac{1}{4} \pathavgleft
		\int \Psi^*_1 \Psi_1 d\xvec \int \Psi^*_2 \Psi_1 d\xvec
		- \int \Psi^*_2 \Psi_2 d\xvec \int \Psi^*_1 \Psi_2 d\xvec
		+ \int \Psi^*_2 \Psi_1 d\xvec \int \Psi^*_1 \Psi_1 d\xvec
		+ \int \Psi^*_1 \Psi_2 d\xvec \int \Psi^*_1 \Psi_1 d\xvec \right. \\
	&	\left. - \int \Psi^*_2 \Psi_1 d\xvec \int \Psi^*_2 \Psi_2 d\xvec
		+ \int \Psi^*_1 \Psi_1 d\xvec \int \Psi^*_1 \Psi_2 d\xvec
		- \int \Psi^*_2 \Psi_2 d\xvec \int \Psi^*_2 \Psi_1 d\xvec
		- \int \Psi^*_1 \Psi_2 d\xvec \int \Psi^*_2 \Psi_2 d\xvec
	\pathavgright \\
	& = \frac{1}{4} \pathavg{
		N_1 I^*
		- N_2 I
		+ I^* N_1
		+ I N_1
		- I^* N_2
		+ N_1 I
		- N_2 I^*
		- I N_2
	} \\
	& = \pathavg{ (N_1 - N_2) \Real I } = 2 \pathavg{ S_x S_z }
\end{split}
\end{equation*}

\begin{equation*}
\begin{split}
	\langle \hat{S}_y \hat{S}_z + \hat{S}_z \hat{S}_y \rangle
	& = \frac{i}{4} \langle \int \left(
		\left(
			\Psiop^\dagger_2 \Psiop_1 - \Psiop^\dagger_1 \Psiop_2
		\right)
		\left(
			\Psiop^{\prime\dagger}_1 \Psiop^\prime_1 - \Psiop^{\prime\dagger}_2 \Psiop^\prime_2
		\right)
		+ \left(
			\Psiop^\dagger_1 \Psiop_1 - \Psiop^\dagger_2 \Psiop_2
		\right)
		\left(
			\Psiop^{\prime\dagger}_2 \Psiop^\prime_1 - \Psiop^{\prime\dagger}_1 \Psiop^\prime_2
		\right)
	\right) d\xvec d\xvec^\prime \rangle \\
	& = \frac{i}{4} \langle \int \left(
		\symprod{ \Psiop^\dagger_1 \Psiop_1 \Psiop^{\prime\dagger}_2 \Psiop^\prime_1 }
		+ \symprod{ \Psiop^\dagger_2 \Psiop_2 \Psiop^{\prime\dagger}_1 \Psiop^\prime_2 }
		+ \symprod{ \Psiop^\dagger_2 \Psiop_1 \Psiop^{\prime\dagger}_1 \Psiop^\prime_1 }
		- \symprod{ \Psiop^\dagger_1 \Psiop_2 \Psiop^{\prime\dagger}_1 \Psiop^\prime_1 }
	\right. \\
	& \left.
		- \symprod{ \Psiop^\dagger_2 \Psiop_1 \Psiop^{\prime\dagger}_2 \Psiop^\prime_2 }
		- \symprod{ \Psiop^\dagger_1 \Psiop_1 \Psiop^{\prime\dagger}_1 \Psiop^\prime_2 }
		- \symprod{ \Psiop^\dagger_2 \Psiop_2 \Psiop^{\prime\dagger}_2 \Psiop^\prime_1 }
		+ \symprod{ \Psiop^\dagger_1 \Psiop_2 \Psiop^{\prime\dagger}_2 \Psiop^\prime_2 }
	\right) d\xvec d\xvec^\prime \rangle \\
	& = \frac{i}{4} \pathavgleft
		\int \Psi^*_1 \Psi_1 d\xvec \int \Psi^*_2 \Psi_1 d\xvec
		+ \int \Psi^*_2 \Psi_2 d\xvec \int \Psi^*_1 \Psi_2 d\xvec
		+ \int \Psi^*_2 \Psi_1 d\xvec \int \Psi^*_1 \Psi_1 d\xvec
		- \int \Psi^*_1 \Psi_2 d\xvec \int \Psi^*_1 \Psi_1 d\xvec \right. \\
	&	\left. - \int \Psi^*_2 \Psi_1 d\xvec \int \Psi^*_2 \Psi_2 d\xvec
		- \int \Psi^*_1 \Psi_1 d\xvec \int \Psi^*_1 \Psi_2 d\xvec
		- \int \Psi^*_2 \Psi_2 d\xvec \int \Psi^*_2 \Psi_1 d\xvec
		+ \int \Psi^*_1 \Psi_2 d\xvec \int \Psi^*_2 \Psi_2 d\xvec
	\pathavgright \\
	& = \frac{i}{4} \pathavg{
		N_1 I^*
		+ N_2 I
		+ I^* N_1
		- I N_1
		- I^* N_2
		- N_1 I
		- N_2 I^*
		+ I N_2
	} \\
	& = \pathavg{ (N_1 - N_2) \Imag I } = 2 \pathavg{ S_y S_z }
\end{split}
\end{equation*}

As it turns out, unlike the equation~\eqnref{moments-calculation:delta-N}, formulas for second-order moments for spin operators do not contain any additional terms depending on $M$.
Now we can calculate all spin correlations from~\cite{Li2009}:
\begin{equation*}
\begin{split}
	\Delta S^2_i
		& = \langle \hat{S}^2_i \rangle - \langle \hat{S}_i \rangle^2
		= \pathavg{ S^2_i } - \pathavg{ S_i }^2, \\
	\Delta_{ij}
		& = \langle \hat{S}_i \hat{S}_j + \hat{S}_j \hat{S}_i \rangle
		- 2 \langle \hat{S}_i \rangle \langle \hat{S}_j \rangle
		= 2 ( \pathavg{ S_i S_j } - \pathavg{ S_i } \pathavg { S_j } )
\end{split}
\end{equation*}
In other words, we proved that in the simulator application we can first calculate spin components $S^{(j)}_i$ for each simulation path, and then use common average and variance functions on resulting arrays to obtain required correlations.


% =============================================================================
\chapter{EPR entanglement in the 2-well system}
% =============================================================================

% =============================================================================
\chapter{Probabilistic representation of Bell inequalities violation}
% =============================================================================

This paper is not published yet.


% =============================================================================
\chapter{Conclusion}
\label{cha:conclusion}
% =============================================================================

In this thesis we formally introduced the functional Wigner transformation.
We proved the required theorems from Wirtinger and functional calculi, and used them to derive the essential properties of the Wigner transformation: sequential transformation of operator products, and the correspondence between operator expectations and moments of the Wigner functional.

We then applied this framework to the exact operator equation governing the dynamics of a \abbrev{bec} and showed how to transform it into an equivalent partial differential equation.
An approximation (Wigner truncation) was introduced, which allowed us to simplify this equation further, and turn it into a set of \abbrev{sde}s for trajectories in phase space.
All of this was done while keeping the functional nature of the equations intact and retaining the inherent mode cutoff of field operators and wavefunctions, which is unavoidable in numerical simulations.

Thus, it allowed us to link the coefficients and terms in the master equation directly with those of \abbrev{sde}s, resulting in a simple method of numerical simulation of \abbrev{bec} dynamics.
The method includes the effect of nonlinear losses in a natural way, is capable of producing any high-order correlations without changes to the initial state or the propagation, and is highly parallelisable.
The latter is a huge advantage in the modern world of multi-core and multi-node computations and the recent advent of general calculations on \abbrev{gpu}s.
In particular, the use of modern \abbrev{gpu}s allowed us to simulate the dynamics of hundreds of thousands of atoms with thousands of modes with a high degree of accuracy on a desktop in the order of hours.

As a simple test of the truncation accuracy, we applied the multimode form of the Wigner representation to a two-mode system, for which an exact quantum mechanical description is possible.
We have shown that the Wigner method gives correct results in the characteristic time frame for large squeezing, with systematic errors much lower than is testable in an experiment.

We then used the truncated Wigner method to model some recent \abbrev{bec} interferometry experiments, both local and reported by other experimental teams.
The method showed reasonable agreement with the experimental results for such important observables as interferometric contrast (a second-order correlation), phase noise and degree of spin squeezing (fourth-order correlations).
The method allows one to account for experimental imperfections (measurement and apparatus noises) naturally, which greatly increases the accuracy of predictions.
There are clearly some currently unknown noise sources in these experiments.

In the last chapter we considered a more fundamental topic of the ability of phase space methods to simulate quantum mechanical systems that violate Bell inequalities, and their differences from \abbrev{lhv} theories.
We showed that, due to weaker restrictions on the range of the functions of phase-space variables that correspond to physical observables, such methods are indeed capable of demonstrating such essential quantum mechanical properties as the violation of Bell inequalities.
We were able to calculate the correlations involving every particle in the system in highly entangled \abbrev{ghzm} states for up to $60$ particles.
This result is far beyond the capabilities of current experimental techniques.

\centerline{\asterism}

In conclusion, we have shown that the truncated Wigner method is a convenient and fast tool that can be used to plan future experiments, and to get better insight into existing ones.
There is, of course, still room for further improvement.

First, we have only used coherent initial states and zero temperature for our simulations.
While it is a reasonable first approximation, for some experiments this may not be acceptable.
The obvious next step here is to include finite-temperature initial conditions using Bogoliubov modes~\cite{Steel1998,Sinatra2002,Ruostekoski2005,Isella2006,Blakie2008}.
During the simulation, the validity of the truncation needs to be estimated more accurately than by using the condition~\eqnref{wigner-bec:truncation:delta-condition}.
This condition turns out to be, in fact, more of a guideline as there are examples of good agreement with the exact methods even when it is not satisfied.
More accurate test can be performed by calculating the quantum correction~\cite{Polkovnikov2010}.

For phase space methods in general, despite demonstrating excellent results in sampling static quantum states, these can struggle in simulating the quantum dynamics of these systems.
In the extreme quantum limit, the positive-P method is known to display exponentially growing sampling error with time because of the redundant dimensions it uses.
In some cases, this can be neutralized by using a gauge-P representation~\cite{Deuar2002,Deuar2005a}, or adding a projection.

Finally, it is possible to make the Wigner representation positive in the same way it was done for the P representation~\cite{Plimak2001}.
The applicability of this ``positive-Wigner'' representation to various simulation problem is yet to be investigated.


\appendix
\counterwithin{lemma}{chapter}
\counterwithin{theorem}{chapter}
\counterwithin{definition}{chapter}

% =============================================================================
\chapter{Wirtinger calculus}
\label{cha:appendix:c-numbers}
% =============================================================================

Formally, a function of complex variable has to be holomorphic in order to be complex differentiable.
In many cases, however, it is enough to have less strict ``physicists'\,'' complex differentiation rules, which only require the function's real and complex part to be differentiable, without imposing additional constraints.
Such rules were developed by Wirtinger~\cite{Wirtinger1927}; further extension to vectors and matrices was performed by Hj{\o}rungnes and Gesbert~\cite{Hjorungnes2007}.
A very good review and a thorough description of their application was made by Kreutz-Delgado~\cite{Kreutz-Delgado2009}.
This section will outline Wirtinger differentiation rules and provide some lemmas based on them, which, in turn, are going to be used in the further Appendices, and in the main body of the thesis.


% =============================================================================
\section{Differentiation}
% =============================================================================

We will start from the definition of the differentiation:

\begin{definition}
\label{def:c-numbers:wirtinger}
	For a complex variable $z \equiv x + iy$, and a function $f(z) \equiv u(x, y) + iv(x, y)$
	\begin{eqn*}
		\frac{\cwd f(z)}{\cwd z}
		\equiv \frac{1}{2} \left(
			\frac{\upd f}{\upd x} - i \frac{\upd f}{\upd y}
		\right).
	\end{eqn*}
\end{definition}

One can easily check that if $f(z)$ is holomorphic, this definition is equivalent to the standard complex differentiation.
Wirtinger differentiation is quite intuitive in the sense that it obeys all the basic rules associated with a real-valued differentiation:

\begin{theorem}
\label{thm:c-numbers:diff-properties}
	For any $f(z)$ with differentiable real and complex parts, Wirtinger differentiation obeys sum, product, quotient, and chain differentiation rules.
	The latter one is applied as if the function $f(z)$ had two independent arguments $z$ and $z^*$:
	\begin{eqn*}
		\frac{\cwd f(g(z))}{\cwd z}
		= \frac{\cwd f}{\cwd g} \frac{\cwd g}{\cwd z}
			+ \frac{\cwd f}{\cwd g^*} \frac{\cwd g^*}{\cwd z}.
	\end{eqn*}
\end{theorem}

Some important functions we will encounter in this thesis are not holomorphic.
Therefore, hereinafter we will use Wirtinger differentiation unless explicitly stated otherwise,
along with the notation $\cwd f/\cwd z$.
Consequently, by ``differentiability'' we will mean the property used in the above theorem, namely the existence of partial derivatives of $\Real f$ and $\Imag f$ over real and complex axes.

It is convenient to connect symbolic rules for Wirtinger differentiation with the rules for common real-valued differentiation.

\begin{theorem}
\label{thm:c-numbers:independent-vars}
	If a function $f(z)$ can be expanded into the series of $z^n (z^*)^m$, then $\cwd f/\cwd z$ and $\cwd f/\cwd z^*$ can be calculated as partial derivatives of the function $f$ expressed in terms of $z$ and $z^*$, over $z$ and $z^*$ respectively:
	\begin{eqn*}
		\frac{\cwd f(z)}{\cwd z} = \frac{\upd f(z, z^*)}{\upd z},
		\quad
		\frac{\cwd f(z)}{\cwd z^*} = \frac{\upd f(z, z^*)}{\upd z^*}.
	\end{eqn*}
\end{theorem}
\begin{proof}
We will prove the first identity.
Without loss of generality, we can consider $f(z) = z^r (z^*)^s$.
First, one can easily prove (by the transition to real values) that $\cwd (z z^*)/\cwd z = z^*$ and $\cwd (z z^*)/\cwd z^* = z$.
Let us assume that the identity is correct for some $r$ and $s$; then, using the product rule,
\begin{eqn}
	\frac{\cwd}{\cwd z} (z^{r+1} (z^*)^s)
	& = \frac{\cwd}{\cwd z} (z z^r (z^*)^s)
		= z^r (z^*)^s + z \frac{\cwd}{\cwd z} (z^r (z^*)^s) \\
	& = z^r (z^*)^s + r z z^{r-1} (z^*)^s
		= (r + 1) z^r (z^*)^s.
\end{eqn}
By induction, the statement is correct for any natural $r$ and $s$, and it is obviously true if $r = 0$ or $s = 0$, which proves the theorem.
\end{proof}

The chain differentiation rule in \thmref{c-numbers:diff-properties} and the above theorem give rise to the common notation used in conjunction with Wirtinger differentiation.
In order to emphasize the ``independent'' behavior of $z$ and $z^*$, function arguments are sometimes written as $f(z, z^*)$, even though technically they are not independent.
We will not use such notation in this thesis because it will create a lot of clutter later on.


% =============================================================================
\section{Integration}
% =============================================================================

Wirtinger differentiation can be paired with the somewhat more common integration over the complex plane:

\begin{definition}
\label{def:c-numbers:integration}
	For a complex $z = x + iy$
	\begin{eqn*}
		\int \upd^2 z \equiv \int_{-\infty}^{\infty} \int_{-\infty}^{\infty} \upd x\, \upd y.
	\end{eqn*}
\end{definition}

This definition allows us to prove an analogue of one of the properties of the Fourier transform expressed in terms of Wirtinger differentiation:

\begin{lemma}
\label{lmm:c-numbers:fourier-of-moments}
	For a complex $\beta$ and any non-negative integers $r$ and $s$
	\begin{eqn*}
		\int \upd^2\alpha\, \alpha^r (\alpha^*)^s \exp(-\beta \alpha^* + \beta^* \alpha)
		= \pi^2
			\left( -\frac{\cwd}{\cwd \beta^*} \right)^r
			\left( \frac{\cwd}{\cwd \beta} \right)^s
			\delta(\Real \beta) \delta(\Imag \beta).
	\end{eqn*}
\end{lemma}
\begin{proof}
First, by changing a variable in the integral and using known Fourier transform relations, we can prove that for real $x$ and $v$, and non-negative integer $n$
\begin{eqn}
\label{eqn:c-numbers:fourier-real}
	\int\limits_{-\infty}^{\infty} \upd v\, v^n \exp(\pm 2 i x v)
	= \pi (\mp i / 2)^n \delta^{(n)}(x).
\end{eqn}
Note that we have explicitly written integration limits here; they are swapped when we change the variable in the first integral.

Denoting $\alpha = u + iv$ and $\beta = x + iy$, we can expand the initial expression as
\begin{eqn2}
	& \int && \upd^2\alpha\, \alpha^r (\alpha^*)^s \exp(-\beta \alpha^* + \beta^* \alpha) \\
	& ={} && \int \upd u\, \upd v\, \exp(2ivx - 2iuy) \\
	& && \times \sum_{m=0}^r \binom{r}{m} u^m (iv)^{r-m}
		\sum_{n=0}^s \binom{s}{n} u^n (-iv)^{s-n} \\
	& ={} && \sum_{m=0}^r \sum_{n=0}^s \binom{r}{m} \binom{s}{n}
		i^{r-m} (-i)^{s-n} \\
	& && \times \int \upd u\, u^{m+n} \exp(2ivx)
		\int \upd v\, v^{r-m+s-n} \exp(-2iuy).
\end{eqn2}
Applying~\eqnref{c-numbers:fourier-real} and grouping differentials:
\begin{eqn}
	={} & \pi^2 \sum_{m=0}^r \sum_{n=0}^s \binom{r}{m} \binom{s}{n} \\
	& \times i^{r-m} (-i)^{s-n}
		(-i/2)^{m+n} \delta^{(m+n)}(y)
		(i/2)^{r-m+s-n} \delta^{(r-m+s-n)}(x) \\
	={} & \pi^2
		\sum_{m=0}^r \binom{r}{m}
			\frac{1}{2^r}
			(-i \upd / \upd y)^m
			(-\upd / \upd x)^{r-m} \\
	& \times \sum_{n=0}^s \binom{s}{n}
			\frac{1}{2^s}
			(-i \upd / \upd y)^n
			(\upd / \upd x)^{s-n}
		\delta(y) \delta(x).
\end{eqn}
Collapsing sums and recognising \defref{c-numbers:wirtinger}:
\begin{eqn}
	& = \pi^2
		\left( \frac{1}{2} (-i \upd / \upd y - \upd / \upd x) \right)^r
		\left( \frac{1}{2} (-i \upd / \upd y + \upd / \upd x) \right)^s
		\delta(y) \delta(x) \\
	& = \pi^2
		\left( -\frac{\cwd}{\cwd \beta^*} \right)^r
		\left( \frac{\cwd}{\cwd \beta} \right)^s
		\delta(\Real \beta) \delta(\Imag \beta).
		\qedhere
\end{eqn}
\end{proof}

It can be proved by expansion in real variables that the formally written rule of integration by parts works for the integral from \defref{c-numbers:integration}:
\begin{eqn}
	\int \upd z\, f \frac{\cwd g}{\cwd z}
	= \int \upd z \frac{\cwd (fg)}{\cwd z} - \int \upd z \frac{\cwd f}{\cwd z} g.
\end{eqn}
Integration by parts will be used extensively in further proofs, and we will need two lemmas that will handle the first term in the right part of the above expression.

\begin{lemma}
\label{lmm:c-numbers:zero-integrals}
	For a square-integrable $f(\beta)$, and a complex $\alpha$
	\begin{eqn*}
		\int \upd^2\beta
			\frac{\cwd}{\cwd \beta} \left(
				\exp(-\beta \alpha^* + \beta^* \alpha)
				f(\beta)
			\right)
		& = 0, \\
		\int \upd^2\beta
			\frac{\cwd}{\cwd \beta^*}
			\left(
				\exp(-\beta \alpha^* + \beta^* \alpha)
				f(\beta)
			\right)
		& = 0.
	\end{eqn*}
\end{lemma}
\begin{proof}
It follows from the square-integrability of $f$ that $\lim_{\Real \beta \rightarrow \infty} = 0$ and $\lim_{\Imag \beta \rightarrow \infty} = 0$, so the statement of the lemma can be proved by transforming to real variables and integrating.
\end{proof}

\begin{lemma}
\label{lmm:c-numbers:zero-delta-integrals}
	For a bounded $f(z)$
	\begin{eqn*}
		\int \upd^2 z
			\frac{\cwd}{\cwd z} \left(
				f(z)
				\left( \frac{\cwd}{\cwd z} \right)^s
				\left( -\frac{\cwd}{\cwd z^*} \right)^r
				\delta(\Real z) \delta(\Imag z)
			\right)
		& = 0, \\
		\int \upd^2 z
			\frac{\cwd}{\cwd z^*}
			\left(
				f(z)
				\left( \frac{\cwd}{\cwd z} \right)^s
				\left( -\frac{\cwd}{\cwd z^*} \right)^r
				\delta(\Real z) \delta(\Imag z)
			\right)
		& = 0.
	\end{eqn*}
\end{lemma}
\begin{proof}
Proved straightforwardly by expanding integrals in real values, separating variables and integrating, using the fact that any derivative of the delta function is zero on the infinity.
\end{proof}

% =============================================================================
\chapter{Functional calculus}
\label{cha:appendix:func-calculus}
% =============================================================================

The definitions of differentiation and integration from~\appref{c-numbers} can be extended to operate on functions and functionals.
This proved to be a useful tool in the derivation of the functional Wigner transformation and expressing accompanied results, as it helps encapsulate bases and mode populations inside wave functions and field operators.
It has been introduced (among other places) in most of the papers treating functional extensions of quasiprobability with varying level of detail.
The most extensive description was made by Dalton~\cite{Dalton2011}.
Although, while these papers cover the foundations quite well, they are missing several important results which are essential for this thesis, and which will be therefore proved in this Appendix.


% =============================================================================
\section{Functional spaces and projections}
% =============================================================================

We will assume we a provided with an arbitrary orthonormal basis $\fullbasis$, consisting of functions $\phi_{\nvec}(\xvec)$, where $\xvec \in \mathbb{R}^D$ is a coordinate vector, and $\nvec \in \fullbasis$ is a mode identifier.
The exact nature of a mode identifier is irrelevant; we only require it to have an equivalence relation defined, and be enumerable.
For example, for a three-dimensional harmonic potential, the mode identifier will be a tuple $(k,l,m)$ of three non-negative integers.

Orthonormality and completeness conditions for basis functions are, respectively,
\begin{eqns}
	\int\limits_A \phi_{\nvec}^*(\xvec) \phi_{\mvec}(\xvec) \upd\xvec & = \delta_{\nvec\mvec}, \\
	\sum_{\nvec} \phi_{\nvec}^*(\xvec) \phi_{\nvec}(\xvec^\prime) & = \delta(\xvec^\prime - \xvec),
\end{eqns}
where the integration area $A$ depends on the basis set (for example, $A$ is the whole space for harmonic oscillator modes, and a box for plane waves).
Hereinafter we assume that the integration $\int \upd\xvec$ is always performed over $A$, unless explicitly stated otherwise. In addition, to avoid clutter, for functions of coordinates the argument list will be omitted (i.e. $f(\xvec) \equiv f$ and $f(\xvec^\prime) \equiv f^\prime$) except where it is necessary for clarity.

Various functions can be combined from all, or the certain subset of basis modes by means of a composition:

\begin{definition}
	For some subset of the full basis $\restbasis \subseteq \fullbasis$, composition transformation creates a function from a vector of mode populations:
	\begin{eqn*}
		\mathcal{C}_{\restbasis}(\balpha)
		= \sum_{\nvec \in \restbasis} \phi_{\nvec} \alpha_{\nvec}.
	\end{eqn*}
	Decomposition transformation, correspondingly, creates a vector of populations out of a function:
	\begin{eqn*}
		(\mathcal{C}_{\restbasis}^{-1}[f])_{\nvec}
		= \int \upd\xvec \phi_{\nvec}^* f,\,{\nvec} \in \restbasis.
	\end{eqn*}
\end{definition}

For any subset of the full basis, there is a certain subspace of functions that can be obtained by composing modes from this subset only.

\begin{definition}
	For some subset of the full basis $\restbasis \subseteq \fullbasis$, $\mathbb{F}_{\restbasis} \equiv (\mathbb{R}^D \rightarrow \mathbb{C})_{\restbasis}$ is a space of all functions of coordinates, which can be obtained from $\mathcal{C}_{\restbasis}$.
	In other words, such functions consist only of modes from $\restbasis$.
	We denote $\mathbb{F}_{\fullbasis} \equiv \mathbb{F}$.
\end{definition}

Using this definition, the type of composition and decomposition transformations can be written as
\begin{eqn}
		\mathcal{C}_{\restbasis} \in \mathbb{C}^{|\restbasis|} \rightarrow \mathbb{F}_{\restbasis},\quad
		\mathcal{C}_{\restbasis}^{-1} \in \mathbb{F} \rightarrow \mathbb{C}^{|\restbasis|}
\end{eqn}
Note also that for any $f \in \mathbb{F}_{\restbasis}$ corresponding composition and decomposition are reversible, i.e. $\mathcal{C}_{\restbasis}(\mathcal{C}_{\restbasis}^{-1}[f]) \equiv f$.
We will refer such functions as restricted functions.

The result of any non-linear transformation of a function $f \in \mathbb{F}_{\restbasis}$ is not guaranteed to belong to $\mathbb{F}_{\restbasis}$ and requires explicit projection to be used with other restricted functions from the same subspace $\mathbb{F}_{\restbasis}$:

\begin{definition}
\label{def:func-calculus:projector}
	An arbitrary function can be projected to $\mathbb{F}_{\restbasis}$ using the projection transformation:
	\begin{eqn*}
		& \proj{\restbasis} \in \mathbb{F} \rightarrow \mathbb{F}_{\restbasis} \\
		& \proj{\restbasis}[f](\xvec)
		\equiv (\mathcal{C}_{\restbasis}(\mathcal{C}_{\restbasis}^{-1}[f])) (\xvec)
		= \sum_{\nvec \in \restbasis} \phi_{\nvec} \int
			d\xvec^\prime\, \phi_{\nvec}^{\prime*} f^\prime.
	\end{eqn*}
	Obviously, $\proj{\fullbasis} \equiv \mathds{1}$.
\end{definition}

Being applied to the delta function, the projection function produces a restricted delta function:

\begin{definition}
\label{def:func-calculus:restricted-delta}
	The restricted delta function $\delta_{\restbasis} \in \mathbb{F}_{\restbasis}$ is defined as
	\begin{eqn*}
		\delta_{\restbasis}(\xvec^\prime, \xvec)
		= \proj{\restbasis}[\delta]
		= \sum_{\nvec \in \restbasis} \phi_{\nvec}^{\prime*} \phi_{\nvec}.
	\end{eqn*}
	Note that in general $\delta_{\restbasis}^*(\xvec^\prime, \xvec) = \delta_{\restbasis}(\xvec, \xvec^\prime)$, so the order of variables is important.
	For a full basis, this definition coincides with the standard delta function: $\delta_{\fullbasis}(\xvec^\prime, \xvec) \equiv \delta(\xvec^\prime - \xvec)$.
\end{definition}

Projection transformation can be, in turn, expressed using the restricted delta function:
\begin{eqn}
	\proj{\restbasis}[f](\xvec) = \int d\xvec^\prime \delta_{\restbasis}(\xvec^\prime, \xvec) f^\prime.
\end{eqn}
The conjugate of $\proj{\restbasis}$ is therefore
\begin{eqn}
	(\proj{\restbasis}[f](\xvec))^*
	= \int d\xvec^\prime \delta_{\restbasis}^*(\xvec^\prime, \xvec) f^{\prime*}
	= \proj{\restbasis}^* [f^*](\xvec).
\end{eqn}


% =============================================================================
\section{Functional differentiation}
% =============================================================================

First we must introduce some operations on functions, which will replace common differentials and integrals used in single and multi-mode cases

Let $\mathcal{F}[f] :: \mathbb{F}_{\restbasis} \rightarrow \mathbb{F}$ be some transformation (note that the result is not guaranteed to belong to the restricted basis).
Because of the bijection between $\mathbb{F}_{\restbasis}$ and $\mathbb{C}^{|\restbasis|}$, $\mathcal{F}$ can be alternatively treated as a function of a vector of complex numbers:
\begin{eqn}
	& \mathcal{F} :: \mathbb{C}^{|\restbasis|} \rightarrow \mathbb{C}^\infty \\
	& \mathcal{F}(\balpha) \equiv \mathcal{C}_{\restbasis}^{-1}[\mathcal{F}[\mathcal{C}_{\restbasis}(\balpha)]].
\end{eqn}
Using this correspondence, we can define the functional differentiation.

\begin{definition}
\label{def:func-calculus:func-diff}
	Functional derivative is defined as
	\begin{eqn*}
		& \frac{\delta}{\delta f^\prime} ::
		\left(
			\mathbb{F}_{\restbasis} \rightarrow \mathbb{F}
		\right)
		\rightarrow
		\left(
			\mathbb{R}^D \rightarrow \mathbb{F}_{\restbasis} \rightarrow \mathbb{F}
		\right) \\
		& \frac{\delta \mathcal{F}[f]}{\delta f^\prime}
		= \sum_{\nvec \in \restbasis} \phi_{\nvec}^{\prime*}
			\frac{\partial \mathcal{F}(\balpha)}{\partial \alpha_{\nvec}}.
	\end{eqn*}
\end{definition}

Note that the transformation being returned differs from the one which was taken:
the result of new transformation is a function depending on two variables from $\mathbb{R}^D$, not one.
The second variable comes from the function we are differentiating by.

Functional derivative definition behaves in many ways similar to common derivative.
\begin{lemma}
	Functional differentiation from~\defref{func-calculus:func-diff} obeys sum, product, quotient, and chain differentiation rules.
\end{lemma}
\begin{proof}
\todo{Sum, product and quotient are more or less obvious; but should we prove chain differentiation?}
\end{proof}

\begin{lemma}
	If $g(t)$ is a function that can be expanded into power series, and functional $\mathcal{F}[f] \equiv g(f)$, $\mathcal{F} \in \mathbb{F}_{\restbasis} \rightarrow \mathbb{F}$, then
	\begin{eqn*}
		\frac{\delta \mathcal{F}[f]}{\delta f(\xvec^\prime)} (\xvec)
		= \delta_{\restbasis}(\xvec^\prime - \xvec)
			\left. \frac{\partial g(t)}{\partial t} \right|_{t = f(\xvec)}
	\end{eqn*}
\end{lemma}
\begin{proof}
We will consider $g(t) = t^k$ case first, which will straightforwardly lead to the statement of the lemma.
For $k = 1$, obviously,
\begin{eqn}
	\frac{\delta f}{\delta f(\xvec^\prime)} (\xvec)
	= \delta_{\restbasis}(\xvec^\prime, \xvec)
\end{eqn}
Then for other values of $k$:
\begin{eqn}
	\frac{\delta \mathcal{F}[f]}{\delta f(\xvec^\prime)} (\xvec)
	& = \frac{\delta f^k}{\delta f(\xvec^\prime)} (\xvec)
	= \sum_{\nvec \in \restbasis} \phi_{\nvec}^{\prime*}
		\frac{\partial f^k}{\partial \alpha_{\nvec}} \\
	& = \sum_{\nvec \in \restbasis} \phi_{\nvec}^{\prime*}
		\frac{\partial f^k}{\partial f}
		\frac{\partial f}{\partial \alpha_{\nvec}}
	= k f^{k-1}
		\sum_{\nvec \in \restbasis} \phi_{\nvec}^{\prime*}
		\frac{\partial f}{\partial \alpha_{\nvec}} \\
	& = k \delta_{\restbasis}(\xvec^\prime, \xvec) f^{k-1}(\xvec)
	= \delta_{\restbasis}(\xvec^\prime, \xvec)
		\left. \frac{\partial t^k}{\partial t} \right|_{t = f(\xvec)}.
	\qedhere
\end{eqn}
\end{proof}

\begin{lemma}
	If $g(z, z^*)$ can be expanded into series of $z^n (z^*)^m$, and functional $\mathcal{F}[f, f^*] \equiv g(f, f^*)$, $\mathcal{F} \in \mathbb{F}_{\restbasis} \rightarrow \mathbb{F}$, then $\delta \mathcal{F} / \delta f^\prime$ and $\delta \mathcal{F} / \delta f^{\prime*}$ can be treated as a partial differentiation of the functional of two independent variables $f$ and $f^*$.
	In other words:
	\begin{eqn*}
		\frac{\delta \mathcal{F}}{\delta f^\prime}
		= \delta_P(\xvec^\prime, \xvec) \left.
			\frac{\partial g(z, z^*)}{\partial z}
		\right|_{z=f(x)},
		\quad
		\frac{\delta \mathcal{F}}{\delta f^{\prime*}}
		= \delta_P^*(\xvec^\prime, \xvec) \left.
			\frac{\partial g(z, z^*)}{\partial z^*}
		\right|_{z=f(x)}
	\end{eqn*}
\end{lemma}
\begin{proof}
Proof is similar to \thmref{c-numbers:independent-vars}.
\end{proof}


% =============================================================================
\section{Functional integration}
% =============================================================================

Functional integration is defined as

\begin{definition}
	\begin{eqn*}
		& \int \delta^2 f :: (\mathbb{F}_{\restbasis} \rightarrow \mathbb{F}) \rightarrow \mathbb{C} \\
		& \int \delta^2 f \mathcal{F}[f]
		= \int d^2\balpha \mathcal{F}(\balpha)
		= \left(
			\prod_{\nvec \in \restbasis} \int d^2\alpha_{\nvec}
		\right) \mathcal{F}(\balpha),
	\end{eqn*}
	where the product of integrals stands for their successive application.
    If the basis contains an infinite number of modes, the integral is treated as a limit $|\restbasis| \rightarrow \infty$.
	\todo{\cite{Dalton2011} has detailed explanation, do we need it here?}
\end{definition}

Functional integration has the Fourier-like property analogous to \lmmref{c-numbers:fourier-of-moments}, but its statement requires the definition of the delta functional:

\begin{definition}
\label{def:func-calculus:delta-functional}
	For a function $\Lambda \in \mathbb{F}_{\restbasis}$ the delta functional is
	\begin{eqn*}
		\Delta_{\restbasis}[\Lambda]
		\equiv \prod_{\nvec \in \restbasis} \delta(\Real \lambda_{\nvec}) \delta(\Imag \lambda_{\nvec}),
	\end{eqn*}
	where $\blambda = \mathcal{C}_{\restbasis}^{-1}[\Lambda]$.
\end{definition}

The delta functional has the same property as the common delta function:
\begin{eqn}
	\int \delta^2 \Lambda \mathcal{F}[\Lambda] \Delta_{\restbasis}[\Lambda]
	& = \left(
			\prod_{\nvec \in \restbasis} \int d^2\lambda_{\nvec}
		\right)
		\mathcal{F}(\blambda)
		\prod_{\nvec \in \restbasis} \delta(\Real \lambda_{\nvec}) \delta(\Imag \lambda_{\nvec}) \\
	& = \left. \mathcal{F}(\blambda) \right|_{\forall \nvec \in \restbasis\, \lambda_{\nvec} = 0} \\
	& = \left. \mathcal{F}[\Lambda] \right|_{\Lambda \equiv 0}
\end{eqn}

\begin{lemma}[Functional extension of \lmmref{c-numbers:fourier-of-moments}]
\label{lmm:func-calculus:fourier-of-moments}
	For $\Psi \in \mathbb{F}_{\restbasis}$ and $\Lambda \in \mathbb{F}_{\restbasis}$, and for any non-negative integers $r$ and $s$:
	\begin{eqn*}
		\int \delta^2\Psi\, \Psi^r (\Psi^*)^s \exp
			\int d\xvec \left( -\Lambda \Psi^* + \Lambda^* \Psi \right)
		= \pi^{2|\restbasis|}
			\left( -\frac{\delta}{\delta \Lambda^*} \right)^r
			\left( \frac{\delta}{\delta \Lambda} \right)^s
			\Delta_{\restbasis}[\Lambda]
	\end{eqn*}
\end{lemma}
\begin{proof}
\begin{eqn}
	& \int \delta^2\Psi\, \Psi^r (\Psi^*)^s \exp
		\int d\xvec \left( -\Lambda \Psi^* + \Lambda^* \Psi \right) \\
	& = \left(
			\prod_{\nvec \in \restbasis} \int d^2\alpha_{\nvec}
		\right)
		\left( \sum_{\nvec \in \restbasis} \phi_{\nvec} \alpha_{\nvec} \right)^r
		\left( \sum_{\nvec \in \restbasis} \phi^*_{\nvec} \alpha_{\nvec}^* \right)^s
		\prod_{\nvec \in \restbasis} \exp(-\lambda_{\nvec} \alpha_{\nvec}^* + \lambda_{\nvec}^* \alpha_{\nvec}).
\end{eqn}
Expanding powers of $\Psi$ and $\Psi^*$ using multinomial theorem:
\begin{eqn2}
	& ={} && \left(
			\prod_{\nvec \in \restbasis} \int d^2\alpha_{\nvec}
		\right)
		\left(
			\sum_{\sum u_{\mvec} = r} \binom{r}{ \left\{ u_{\mvec} \right\} }
			\prod_{\nvec \in \restbasis} \phi_{\nvec}^{u_{\nvec}} \alpha_{\nvec}^{u_{\nvec}}
		\right) \\
	& && \left(
			\sum_{\sum v_{\mvec} = s} \binom{s}{ \left\{ v_{\mvec} \right\} }
			\prod_{\nvec \in \restbasis} (\phi_{\nvec}^*)^{v_{\nvec}} (\alpha_{\nvec}^*)^{v_{\nvec}}
		\right)
		\prod_{\nvec \in \restbasis} \exp(-\lambda_{\nvec} \alpha_{\nvec}^* + \lambda_{\nvec}^* \alpha_{\nvec}),
\end{eqn2}
where $\binom{r}{ \left\{ u_{\mvec} \right\} } \equiv r! / (\prod u_{\mvec}!)$ are multinomial coefficients.
Splitting variables:
\begin{eqn2}
	& ={} && \sum_{ \sum u_{\mvec} = r,\, \sum v_{\mvec} = s }
		\binom{r}{ \left\{ u_{\mvec} \right\} }
		\binom{s}{ \left\{ v_{\mvec} \right\} } \\
	& && \prod_{\nvec \in \restbasis}
			\phi_{\nvec}^{u_{\nvec}} (\phi_{\nvec}^*)^{v_{\nvec}}
			\int d^2\alpha_{\nvec}
				\alpha_{\nvec}^{u_{\nvec}}
				(\alpha_{\nvec}^*)^{v_{\nvec}}
				\exp(-\lambda_{\nvec} \alpha_{\nvec}^* + \lambda_{\nvec}^* \alpha_{\nvec}).
\end{eqn2}
Applying \lmmref{c-numbers:fourier-of-moments}, collapsing sums, and recognizing \defref{func-calculus:func-diff} and \defref{func-calculus:delta-functional}:
\begin{eqn2}
	& ={} && \sum_{\sum u_{\mvec} = r,\, \sum v_{\mvec} = s}
		\binom{r}{ \left\{ u_{\mvec} \right\} }
		\binom{s}{ \left\{ v_{\mvec} \right\} }
		\pi^{2|\restbasis|} \\
	& && \prod_{\nvec \in \restbasis}
			\phi_{\nvec}^{u_{\nvec}} (\phi_{\nvec}^*)^{v_{\nvec}}
			\left( -\frac{\partial}{\partial \lambda_{\nvec}^*} \right)^{u_{\nvec}}
			\left( \frac{\partial}{\partial \lambda_{\nvec}} \right)^{v_{\nvec}}
			\delta(\Real \lambda_{\nvec}) \delta(\Imag \lambda_{\nvec}) \\
	& ={} && \pi^{2|\restbasis|}
		\left( -\sum_{\nvec \in \restbasis} \phi_{\nvec} \frac{\partial}{\partial \lambda_{\nvec}^*} \right)^r
		\left( \sum_{\nvec \in \restbasis} \phi_{\nvec}^* \frac{\partial}{\partial \lambda_{\nvec}} \right)^s
		\prod_{\nvec \in \restbasis} \delta(\Real \lambda_{\nvec}) \delta(\Imag \lambda_{\nvec}) \\
	& ={} && \pi^{2|\restbasis|}
		\left( -\frac{\delta}{\delta \Lambda^*} \right)^r
		\left( \frac{\delta}{\delta \Lambda} \right)^s
		\Delta_{\restbasis}[\Lambda]
	\qedhere
\end{eqn2}
\end{proof}

\begin{definition}
	Displacement functional is
	\begin{eqn}
		& D :: \mathbb{F}_{\restbasis} \rightarrow \mathbb{F}_{\restbasis} \rightarrow \mathbb{C} \\
		& D[\Lambda, \Lambda^*, \Psi, \Psi^*] = \exp \int d\xvec \left(
			-\Lambda \Psi^* + \Lambda^* \Psi
		\right).
	\end{eqn}
\end{definition}

\begin{lemma}[Functional extension of \lmmref{c-numbers:zero-integrals}]
\label{lmm:func-calculus:zero-integrals}
	For a bounded functional $F(\blambda, \blambda^*)$
	\begin{eqn*}
		\int \delta^2\Lambda
			\frac{\delta}{\delta \Lambda^\prime} \left(
				D[\Lambda, \Lambda^*, \Psi, \Psi^*]
				F[\Lambda, \Lambda^*]
			\right)
		& = 0 \\
		\int \delta^2\Lambda
			\frac{\delta}{\delta \Lambda^{\prime*}}
			\left(
				D[\Lambda, \Lambda^*, \Psi, \Psi^*]
				F[\Lambda, \Lambda^*]
			\right)
		& = 0.
	\end{eqn*}
\end{lemma}
\begin{proof}
We will prove the first equation.
Let $\Lambda = \mathcal{C}_{\restbasis}(\blambda)$ and $\Psi = \mathcal{C}_{\restbasis}(\balpha)$.
Displacement functional can be represented as a function of mode vector:
\begin{eqn}
	D[\Lambda, \Lambda^*, \Psi, \Psi^*]
	& = \exp \int dx \sum_{\nvec \in \restbasis,\mvec \in \restbasis} \left(
		- \phi_{\nvec} \phi_{\mvec}^* \lambda_{\nvec} \alpha_{\mvec}^*
		+ \phi_{\nvec}^* \phi_{\mvec} \lambda_{\nvec}^* \alpha_{\mvec}
	\right) \\
	& = \exp \sum_{\nvec \in \restbasis,\mvec \in \restbasis} \left(
		- \delta_{\nvec \mvec} \lambda_{\nvec} \alpha_{\nvec}^*
		+ \delta_{\nvec \mvec} \lambda_{\nvec}^* \alpha_{\nvec}
	\right) \\
	& = \exp \sum_{\nvec \in \restbasis} \left(
		-\lambda_{\nvec} \alpha_{\nvec}^* + \lambda_{\nvec}^* \alpha_{\nvec}
	\right).
\end{eqn}

We introduce the special notation for this lemma to indicate the subset of $\restbasis$ used by operators and functionals.
With this notation, for fixed $\nvec$:
\begin{eqn}
	D[\Lambda, \Lambda^*, \Psi, \Psi^*]
	& = \prod_{\mvec \in \restbasis} \exp \left(
		- \lambda_{\mvec} \alpha_{\mvec}^* + \lambda_{\mvec}^* \alpha_{\mvec}
	\right) \\
	& = \exp \left(
		- \lambda_{\nvec} \alpha_{\nvec}^* + \lambda_{\nvec}^* \alpha_{\nvec}
	\right)
	\prod_{\mvec \in \restbasis, \mvec \ne \nvec} \exp \left(
		- \lambda_{\mvec} \alpha_{\mvec}^* + \lambda_{\mvec}^* \alpha_{\mvec}
	\right) \\
	& = D_{\lnot \nvec} D_{\nvec},
\end{eqn}
and, similarly,
\begin{eqn}
	\Lambda & = \Lambda_{\lnot \nvec} + \Lambda_{\nvec}, \\
	\int d^2 \blambda & = \int d^2 \blambda_{\lnot \nvec} \int d^2 \lambda_{\nvec}.
\end{eqn}

With this notation:
\begin{eqn}
	& \int \delta^2\Lambda
		\frac{\delta}{\delta \Lambda^\prime} \left(
			D[\Lambda, \Lambda^*, \Psi, \Psi^*]
			F[\Lambda, \Lambda^*]
		\right) \\
	& = \int d^2 \blambda
		\sum_{\nvec \in \restbasis} \phi_{\nvec}^{\prime*} \frac{\partial}{\partial \lambda_{\nvec}}
			D_{\lnot \nvec} D_{\nvec}
			F[\Lambda, \Lambda^*] \\
	& = \sum_{\nvec \in \restbasis} \phi_{\nvec}^{\prime*}
		\int d^2 \blambda_{\lnot \nvec} D_{\lnot \nvec}
		\int d^2 \lambda_{\nvec} \frac{\partial}{\partial \lambda_{\nvec}} D_{\nvec} F(\blambda, \blambda^*).
\end{eqn}
For each term the internal is equal to zero because of \lmmref{c-numbers:zero-integrals}, therefore the whole sum is zero.
\end{proof}

\begin{lemma}[Functional extension of \lmmref{c-numbers:zero-delta-integrals}]
\label{lmm:func-calculus:zero-delta-integrals}
	For $\Lambda \in \mathbb{F}_{\restbasis}$ \todo{Again, any limitations on $F$?}
	\begin{eqn*}
		\int \delta^2\Lambda
			\frac{\delta}{\delta \Lambda} \left(
				\left(
					\left( \frac{\delta}{\delta \Lambda} \right)^s
					\left( -\frac{\delta}{\delta \Lambda^*} \right)^r
					\Delta_{\restbasis}[\Lambda]
				\right)
				F[\lambda, \lambda^*]
			\right)
		& = 0 \\
		\int \delta^2\Lambda
			\frac{\delta}{\delta \Lambda^*} \left(
				\left(
					\left( \frac{\delta}{\delta \Lambda} \right)^s
					\left( -\frac{\delta}{\delta \Lambda^*} \right)^r
					\Delta_{\restbasis}[\Lambda]
				\right)
				F[\lambda, \lambda^*]
			\right)
		& = 0 \\
	\end{eqn*}
\end{lemma}
\begin{proof}
Proved by expanding functional integration and differentials into modes and recognizing \lmmref{c-numbers:zero-delta-integrals}.
\end{proof}

In order to perform transformations of master equations in the future, we will need a lemma, which justifies certain operation with Laplacian (which is a part of kinetic term in Hamiltonian).

\begin{lemma}
\label{lmm:func-calculus:move-laplacian}
	If $\forall \nvec \in \restbasis\, \xvec \in \partial A$ $\phi_n(\xvec) = 0$, then for any $\mathcal{F} \in \mathbb{F}_{\restbasis} \rightarrow \mathbb{F}$
	\begin{eqn*}
		\int\limits_A d\xvec \left(
			\nabla^2 \frac{\delta}{\delta \Psi}
		\right) \Psi \mathcal{F}[\Psi, \Psi^*]
		= \int\limits_A d\xvec \frac{\delta}{\delta \Psi}
		( \nabla^2 \Psi ) \mathcal{F}[\Psi, \Psi^*]
	\end{eqn*}
\end{lemma}
\begin{proof}
Integration limits play an important role in this proof, so we will write them explicitly.
\begin{eqn}
	\int\limits_A d\xvec \left(
		\nabla^2 \frac{\delta}{\delta \Psi}
	\right) \Psi
	= \sum_{\nvec \in \restbasis, \mvec \in \restbasis} \left(
			\int\limits_A d\xvec ( \nabla^2 \phi_{\nvec}^* ) \phi_{\mvec}
		\right)
		\frac{\partial}{\partial \alpha_{\nvec}} \alpha_{\mvec} \mathcal{F}(\mathbf{\alpha})
	= (*)
\end{eqn}
Using Green's first identity and the fact that eigenfunctions are equal to zero at the boundary of $A$:
\begin{eqn}
	\int\limits_A d\xvec ( \nabla^2 \phi_{\nvec}^* ) \phi_{\mvec}
	& = \oint\limits_{\partial A} \phi_{\mvec} (\nabla \phi_{\nvec}^* \cdot \mathbf{v}) dS
	- \int\limits_A d\xvec ( \nabla \phi_{\nvec}^* ) ( \nabla \phi_{\mvec} ) \\
	& = 0 - \int\limits_A d\xvec ( \nabla \phi_{\nvec}^* ) ( \nabla \phi_{\mvec} ) \\
	& = \oint\limits_{\partial A} \phi_{\nvec}^* (\nabla \phi_{\mvec} \cdot \mathbf{v}) dS
	- \int\limits_A d\xvec ( \nabla \phi_{\nvec}^* ) ( \nabla \phi_{\mvec} ) \\
	& = \int\limits_A d\xvec \phi_{\nvec}^* ( \nabla^2 \phi_{\mvec} ),
\end{eqn}
where $\mathbf{v}$ is the outward pointing unit normal of surface element $dS$.
Thus
\begin{eqn}
	(*)
	= \sum_{\nvec \in \restbasis, \mvec \in \restbasis} \left(
			\int\limits_A d\xvec \phi_{\nvec}^* ( \nabla^2 \phi_{\mvec} )
		\right)
		\frac{\partial}{\partial \alpha_{\nvec}} \alpha_{\mvec} \mathcal{F}(\mathbf{\alpha})
	= \int\limits_A d\xvec \frac{\delta}{\delta \Psi}
		( \nabla^2 \Psi ) \mathcal{F}[\Psi, \Psi^*].
	\qedhere
\end{eqn}
\end{proof}

Note that this lemma imposes additional requirement for basis functions, but in practical applications it is always satisfied.
For example, in plane wave basis eigenfunctions are equal to zero at the border of the bounding box, and in harmonic oscillator basis they are equal to zero on the infinity (which can be considered the boundary of their integration area).
Hereinafter we will assume that this condition is true for any basis we work with.

% =============================================================================
\chapter{Functional FPE to SDE correspondences}
\label{cha:appendix:fpe-sde}
% =============================================================================

Wigner transformation, to which the majority of this thesis is dedicated to, produces a \abbrev{fpe}, or its functional equivalent, from an initial master equation.
\abbrev{fpe} is an equation in partial derivatives, and, in general, is not easy to solve~--- even numerically.
The major part of the usefullness of the Wigner transformation paired with the Wigner truncation is that it produces \abbrev{fpe} in a special form, which can be further transformed to a set of \abbrev{sde}s, with the Wigner function playing the role of a probability distribution.
Algorithms of solving such equations numerically are much more straightforward.

The actual correspondence between \abbrev{fpe} and \abbrev{sde}s is formulated and proved for real-valued coefficients in literature~\cite{Risken1996}.
In this thesis we will need to transform \abbrev{fpe}s with complex coefficients, or even functional operator ones.
While it is always possible to express them in real-valued form, it is much more convenient to derive correspondence theorems that work directly on such \abbrev{fpe}s.
In this Appendix we will do that by proceeding successively from the initial real-valued theorem to complex-valued and functional correspondences.

In addition, we will do the same for the It\^o formula, which provides the expression for the time derivative of any function of transverse variables.
This formula is useful, among other cases, if one wants to derive the time dependence of some integral observable (for instance, population), without solving \abbrev{sde}s themselves.
Alternatively, it can serve as an additional test of a numerical algorithm used to propagate \abbrev{sde}s in time.

% =============================================================================
\section{Correspondences}
% =============================================================================

We will start by formulating the known real-valued correspondence in a form which is more convenient for further proofs in this section, and is also closer to the results one obtains from the Wigner transformation.

\begin{lemma}[real-valued \abbrev{fpe}--\abbrev{sde}s correspondence in a convenient form.]
\label{lmm:fpe-sde:corr:fpe-sde-real}
	Let $\zvec^T \equiv (z_1 \ldots z_M)$ be a set of real variables.
	Then the \abbrev{fpe}
	\begin{eqn*}
		\frac{\upd W}{\upd t}
		= -\vcwd_{\zvec}^T \cdot \avec W
		+ \frac{1}{2} \Trace{ \vcwd_{\zvec} \vcwd_{\zvec}^T B B^T } W
	\end{eqn*}
	is equivalent to the set of \abbrev{sde}s in the It\^o form
	\begin{eqn*}
		\upd\zvec = \avec \upd t + B \upd\Zvec
	\end{eqn*}
	and to the set of \abbrev{sde}s in the Stratonovich form
	\begin{eqn*}
		\upd\zvec = (\avec - \svec)\upd t + B \upd\Zvec,
	\end{eqn*}
	where the noise-induced (Stratonovich) drift vector $\svec$ has elements
	\begin{eqn*}
		s_j
		= \frac{1}{2} \sum_{k,i} B_{ki} \frac{\cwd}{\cwd z_k} B_{ji}
		= \frac{1}{2} \Trace{B^T \vcwd_{\zvec} \evec_j^T B},
	\end{eqn*}
	with $\evec_j$ being a unit vector with elements $(\evec_j)_i = \delta_{ij}$.
	Here $W \equiv W(\zvec)$ is a probability distribution, $\avec \equiv \avec(\zvec)$ is a vector function, $B \equiv B(\zvec)$ is a matrix function ($B$ having the size $M \times L$, where $L$ is the number of noise sources), $\vcwd_{\zvec}^T \equiv (\upd/\upd z_1 \ldots \upd/\upd z_M)$ is a cogradient vector, and $\Zvec$ is a standard $L$-dimensional Wiener process with $\langle \upd Z_j^2 \rangle = \upd t$.
\end{lemma}
\begin{proof}
For the detailed proof see Risken~\cite{Risken1996}, sections 3.3 and 3.4.
\end{proof}

The above theorem can be extended to work with complex Wirtinger derivatives and complex-valued coefficients.
Of course, in order to produce a real-valued $\upd W/\upd t$ in the left-hand part, an \abbrev{fpe} must have a particular form.

\begin{theorem}
\label{thm:fpe-sde:corr:fpe-sde-complex}
	Let $\balpha^T \equiv (\alpha_1 \ldots \alpha_M)$ be a set of complex variables.
	Then the \abbrev{fpe}
	\begin{eqn*}
		\frac{\upd W}{\upd t}
		= -\vcwd_{\balpha}^T \avec W - \vcwd_{\balpha^*}^T \avec^* W
		+ \Trace{ \vcwd_{\balpha^*} \vcwd_{\balpha}^T B B^H } W
	\end{eqn*}
	is equivalent to the set of \abbrev{sde}s in the It\^o form
	\begin{eqn*}
		\upd\balpha = \avec \upd t + B \upd\Zvec,
	\end{eqn*}
	and to the set of \abbrev{sde}s in the Stratonovich form
	\begin{eqn*}
		\upd\balpha = (\avec - \svec) \upd t + B \upd\Zvec,
	\end{eqn*}
	where the Stratonovich term has elements
	\begin{eqn*}
		s_j = \frac{1}{2} \Trace{ B^H \vcwd_{\balpha^*} \evec_j^T B },
	\end{eqn*}
	and $\Zvec = (\mathbf{X} + i\mathbf{Y}) / \sqrt{2}$ is an $L$-dimensional standard complex-valued Wiener process (with $\langle \upd Z_j \upd Z_k^* \rangle = \delta_{jk} \upd t$), containing two standard $L$-dimensional Wiener processes $\mathbf{X}$ and $\mathbf{Y}$.
\end{theorem}
\begin{proof}
Let us expand the \abbrev{fpe} using real variables: $\balpha = \xvec + i \yvec$, $\avec = \mathbf{u} + i \mathbf{v}$, $B = F + iG$, $\vcwd_{\balpha} = (\vcwd_{\xvec} - i \vcwd_{\yvec}) / 2$.
This results in
\begin{eqn}
	\frac{\upd W}{\upd t}
	={} & - \vcwd_{\xvec}^T \mathbf{u} W
	- \vcwd_{\yvec}^T \mathbf{v} W
	+ \frac{1}{4} \Trace{
		(\vcwd_{\xvec} \vcwd_{\xvec}^T
			+ \vcwd_{\yvec} \vcwd_{\yvec}^T)
		(F F^T + G G^T) \right. \\
	& \left. - (\vcwd_{\xvec} \vcwd_{\yvec}^T
			- \vcwd_{\yvec} \vcwd_{\xvec}^T)
		(F G^T - G F^T)
	} W \\
	& + \frac{i}{4} \Trace{
		(\vcwd_{\xvec} \vcwd_{\xvec}^T
			+ \vcwd_{\yvec} \vcwd_{\yvec}^T)
		(F G^T - G F^T)
	} W \\
	& + \frac{i}{4} \Trace{
		(\vcwd_{\xvec} \vcwd_{\yvec}^T
			- \vcwd_{\yvec} \vcwd_{\xvec}^T)
		(F F^T + G G^T)
	} W.
\end{eqn}
Since $F F^T + G G^T$ and $\vcwd_{\xvec} \vcwd_{\xvec}^T + \vcwd_{\yvec} \vcwd_{\yvec}^T$ are symmetric matrices, and $F G^T - G F^T$ and $\vcwd_{\xvec} \vcwd_{\yvec}^T - \vcwd_{\yvec} \vcwd_{\xvec}^T$ are antisymmetric ones, the corresponding traces are equal to zero, which gives us the \abbrev{fpe} in real variables
\begin{eqn}
	\frac{\upd W}{\upd t}
	={} & - \vcwd_{\xvec}^T \mathbf{u} W
	- \vcwd_{\yvec}^T \mathbf{v} W
	+ \frac{1}{4} \Trace{
		(\vcwd_{\xvec} \vcwd_{\xvec}^T
			+ \vcwd_{\yvec} \vcwd_{\yvec}^T)
		(F F^T + G G^T) \right. \\
	& \left. - (\vcwd_{\xvec} \vcwd_{\yvec}^T
			- \vcwd_{\yvec} \vcwd_{\xvec}^T)
		(F G^T - G F^T)
	} W.
\end{eqn}

In order to use \lmmref{fpe-sde:corr:fpe-sde-real}, we need to merge variables $\xvec$ and $\yvec$ into one variable vector $\zvec \equiv \xvec \oplus \yvec$.
This will give us an equation in the form identical to that from the lemma, with drift vector $\tilde{\avec} \equiv \mathbf{u} \oplus \mathbf{v}$ and diffusion matrix
\begin{eqn}
	\tilde{B} \tilde{B}^T \equiv \frac{1}{2} \begin{pmatrix}
		F F^T + G G^T & F G^T - G F^T \\
		G F^T - F G^T & F F^T + G G^T
	\end{pmatrix},
\end{eqn}
which gives the noise matrix
\begin{eqn}
	\tilde{B} = \frac{1}{\sqrt{2}} \begin{pmatrix}
		F & -G \\
		G & F
	\end{pmatrix}.
\end{eqn}
Therefore, the equivalent \abbrev{sde}s in the It\^o form are
\begin{eqn}
	d\zvec = \tilde{\avec} dt + \tilde{B} d\tilde{\Zvec},
\end{eqn}
where $d\tilde{\Zvec} \equiv d\mathbf{X} \oplus d\mathbf{Y}$.
Returning to the previous variables:
\begin{eqn}
	d\xvec & = \mathbf{u} dt + \frac{1}{\sqrt{2}} F d\mathbf{X} - \frac{1}{\sqrt{2}} G d\mathbf{Y}, \\
	d\yvec & = \mathbf{v} dt + \frac{1}{\sqrt{2}} G d\mathbf{X} + \frac{1}{\sqrt{2}} F d\mathbf{Y}.
\end{eqn}
Multiplying the second equation by $i$ and adding it to the first one:
\begin{eqn}
	d\balpha = \avec dt + \frac{1}{\sqrt{2}} (F + iG) (d\mathbf{X} + id\mathbf{Y}),
\end{eqn}
which leads to the It\^o part of the theorem statement
\begin{eqn}
	d\balpha = \avec dt + B d\Zvec.
\end{eqn}

The noise-induced drift term in the Stratonovich case can be calculated by substituting $\tilde{B}$ into the expression for $s_j$ from \lmmref{fpe-sde:corr:fpe-sde-real}.
We will calculate $s_j$ with $j$ belonging to the $\xvec$ and the $\yvec$ part of the coordinate space separately.
Starting from the $\xvec$ part:
\begin{eqn}
	s_j^{(x)}
	= \frac{1}{4} \Trace{
		\begin{pmatrix}
			F^T & G^T \\ -G^T & F^T
		\end{pmatrix}
		\begin{pmatrix}
			\vcwd_{\xvec} \\
			\vcwd_{\yvec}
		\end{pmatrix}
		\begin{pmatrix}
			\evec_j^T & 0
		\end{pmatrix}
		\begin{pmatrix}
			F & -G \\ G & F
		\end{pmatrix}
	}.
\end{eqn}
Multiplying the matrices, we get:
\begin{eqn}
	={} & \frac{1}{4} \Trace{
		\begin{pmatrix}
			F^T & G^T \\ -G^T & F^T
		\end{pmatrix}
		\begin{pmatrix}
			\vcwd_{\xvec} \\
			\vcwd_{\yvec}
		\end{pmatrix}
		\begin{pmatrix}
			\evec_j^T F & - \evec_j^T G
		\end{pmatrix}
	} \\
	={} & \frac{1}{4} \Trace{
		\begin{pmatrix}
			F^T & G^T \\ -G^T & F^T
		\end{pmatrix}
		\begin{pmatrix}
			\vcwd_{\xvec} \evec_j^T F & - \vcwd_{\xvec} \evec_j^T G \\
			\vcwd_{\yvec} \evec_j^T F & - \vcwd_{\yvec} \evec_j^T G
		\end{pmatrix}
	} \\
	={} & \frac{1}{4} \left(
		\Trace{ F^T \vcwd_{\xvec} \evec_j^T F }
		+ \Trace{ G^T \vcwd_{\yvec} \evec_j^T F } \right. \\
	& \left. + \Trace{ G^T \vcwd_{\xvec} \evec_j^T G }
		- \Trace{ F^T \vcwd_{\yvec} \evec_j^T G }
	\right).
\end{eqn}
Similarly for the $\yvec$ part,
\begin{eqn}
	s_j^{(y)}
	={} & \frac{1}{4} \left(
		\Trace{ F^T \vcwd_{\xvec} \evec_j^T G }
		+ \Trace{ G^T \vcwd_{\yvec} \evec_j^T G } \right. \\
	& \left. - \Trace{ G^T \vcwd_{\xvec} \evec_j^T F }
		+ \Trace{ F^T \vcwd_{\yvec} \evec_j^T F }
	\right).
\end{eqn}
Therefore, the final term in the complex-valued \abbrev{sde}s is
\begin{eqn}
	s_j
	= s_j^{(x)} + i s_j^{(y)}
	= \frac{1}{2} \Trace{ B^H \vcwd_{\balpha^*} \evec_j^T B },
\end{eqn}
which finishes the proof.
\end{proof}

Note the asymmetry in the expression for the Stratonovich term: if $B = B(\alpha)$, then $\mathbf{s} \equiv 0$.
This is initially caused by the asymmetry in the target \abbrev{sde}s.
A truly general form of an \abbrev{fpe} would be
\begin{eqn}
	\frac{\upd W}{\upd t}
	={} & - 2 \Real \left( \vcwd_{\balpha}^T \avec \right) W
	+ \Trace{ \vcwd_{\balpha^*} \vcwd_{\balpha}^T B_1 B_1^H } W
	+ \Trace{ \vcwd_{\balpha^*} \vcwd_{\balpha}^T B_2 B_2^H } W \\
	& + 2 \Real \left(
		\Trace{ \vcwd_{\balpha} \vcwd_{\balpha}^T B_1 B_2^T }
		+ \Trace{ \vcwd_{\balpha} \vcwd_{\balpha}^T B_2 B_1^T }
	\right) W,
\end{eqn}
which corresponds to the system of \abbrev{sde}s
\begin{eqn}
	\upd\balpha = (\avec - \svec) \upd t + B_1 \upd\Zvec + B_2 \upd\Zvec^*,
\end{eqn}
where the Stratonovich term has elements
\begin{eqn}
	s_j ={} & \frac{1}{2} \left(
		\Trace{ B_1^H \vcwd_{\balpha^*} \evec_j^T B_1 }
		+ \Trace{ B_2^H \vcwd_{\balpha^*} \evec_j^T B_2 } \right. \\
		& \left. + \Trace{ B_1^T \vcwd_{\balpha} \evec_j^T B_2 }
		+ \Trace{ B_2^T \vcwd_{\balpha} \evec_j^T B_1 }
	\right).
\end{eqn}
In the theorem above we limited the space of possible \abbrev{sde}s to those with $B_2 \equiv 0$, leading to the observed asymmetry.

In many applications (some of which are discussed in this thesis), it is advantageous to enumerate the state vector of a system using two variables instead of one, namely the mode identifier and the component number.
This helps to describe particles which can occupy the same set of modes, but are otherwise distinguishable.
We will now reformulate the previous theorem, including this component distinction.

\begin{theorem}[multi-component reformulation of \thmref{fpe-sde:corr:fpe-sde-complex}]
\label{thm:fpe-sde:corr:mc-fpe-sde}
	Let $\balpha^{(j)},\, j = 1 \ldots C$ be $C$ sets of complex variables $\balpha^{(j)} \equiv (\alpha_1^{(j)} \ldots \alpha_{M_j}^{(j)})$.
	Then the \abbrev{fpe}
	\begin{eqn*}
		\frac{\upd W}{\upd t}
		={} & - \sum_{j=1}^C \vcwd_{\balpha^{(j)}}^T \avec^{(j)} W
		- \sum_{j=1}^C \vcwd_{(\balpha^{(j)})^*}^T (\avec^{(j)})^* W \\
		& + \sum_{j=1}^C \sum_{k=1}^C
			\Trace{
				\vcwd_{(\balpha^{(j)})^*}
				\vcwd_{\balpha^{(k)}}^T
				B^{(k)} (B^{(j)})^H
			} W
	\end{eqn*}
	is equivalent to the set of \abbrev{sde}s in the It\^o form
	\begin{eqn*}
		\upd\balpha^{(j)} = \avec^{(j)} \upd t + B^{(j)} \upd\Zvec,
	\end{eqn*}
	or to the set of \abbrev{sde}s in the Stratonovich form
	\begin{eqn*}
		\upd\balpha^{(j)} = (\avec^{(j)} - \svec^{(j)}) \upd t + B^{(j)} \upd\Zvec,
	\end{eqn*}
	where the Stratonovich term has elements
	\begin{eqn*}
		s_i^{(j)} = \frac{1}{2} \sum_{k=1}^C
			\Trace{ (B^{(k)})^H \vcwd_{(\balpha^{(k)})^*} \evec_i^T B^{(j)} }.
	\end{eqn*}
	Here $\Zvec$ is an $L$-dimensional standard complex-valued Wiener process, and the noise matrices $B^{(j)}$ have sizes $M_j \times L$.
\end{theorem}
\begin{proof}
Let us join all variable sets $\balpha^{(j)}$ into a single set
\begin{eqn}
	\balpha \equiv \bigoplus_{j=1}^C \balpha^{(j)}.
\end{eqn}
Then we can use \thmref{fpe-sde:corr:fpe-sde-complex} with the drift vector
\begin{eqn}
	\avec = \bigoplus_{j=1}^C \avec^{(j)},
\end{eqn}
the cogradient vector
\begin{eqn}
	\vcwd_{\balpha} = \bigoplus_{j=1}^C \vcwd_{\balpha^{(j)}},
\end{eqn}
and the noise matrix
\begin{eqn}
	B = \begin{pmatrix}
		B^{(1)} \\ \vdots \\ B^{(C)}
	\end{pmatrix}.
\end{eqn}
This gives us \abbrev{sde}s in the It\^o form
\begin{eqn}
	\upd\balpha = \avec \upd t + B \upd\Zvec,
\end{eqn}
where $d\Zvec$ is an $L$-dimensional standard complex-valued Wiener process.
Splitting this equation for different components, we get the It\^o part of the theorem statement.
Substituting $B$ into the expression for the Stratonovich term:
\begin{eqn}
	s_i^{(j)}
	& = \frac{1}{2} \Trace{
		\begin{pmatrix} (B^{(1)})^H & \cdots & (B^{(C)})^H \end{pmatrix}
		\begin{pmatrix}
			\vcwd_{(\balpha^{(1)})^*} \\
			\vdots \\
			\vcwd_{(\balpha^{(C)})^*}
		\end{pmatrix}
		\begin{pmatrix} 0 & \cdots & \evec_i^T & \cdots & 0 \end{pmatrix}
		\begin{pmatrix}
			B^{(1)} \\
			\vdots \\
			B^{(C)}
		\end{pmatrix}
	}.
\end{eqn}
Multiplying the matrices successively, we get
\begin{eqn}
	& = \frac{1}{2} \Trace{
		\begin{pmatrix} (B^{(1)})^H & \cdots & (B^{(C)})^H \end{pmatrix}
		\begin{pmatrix}
			\vcwd_{(\balpha^{(1)})^*} \evec_i^T B^{(j)} \\
			\vdots \\
			\vcwd_{(\balpha^{(C)})^*} \evec_i^T B^{(j)}
		\end{pmatrix}
	} \\
	& = \frac{1}{2} \sum_{k=1}^C \Trace{
		(B^{(k)})^H
		\vcwd_{(\balpha^{(k)})^*}
		\evec_i^T
		B^{(j)}
	},
\end{eqn}
which is the expression from the theorem statement.
\end{proof}

Most of the time we will deal with \abbrev{fpe}s in a functional form, so we will reformulate the correspondence once again, now using functional derivatives.

\begin{theorem}
\label{thm:fpe-sde:corr:fpe-sde-func}
	The \abbrev{fpe} in a functional form
	\begin{eqn*}
		\frac{\upd W}{\upd t}
		={} & \int \upd\xvec \left(
			- \sum_{j=1}^C \frac{\fdelta}{\fdelta f_j} \mathcal{A}_j W
			- \sum_{j=1}^C \frac{\fdelta}{\fdelta f_j^*} \mathcal{A}_j^* W \right. \\
		& \left. + \sum_{j=1}^C \sum_{k=1}^C \frac{\fdelta^2}{\fdelta f_j^* \fdelta f_k}
				\sum_{l=1}^L \mathcal{B}_{kl} \mathcal{B}_{jl}^* W
		\right)
	\end{eqn*}
	is equivalent to the set of \abbrev{sde}s in the It\^o form
	\begin{eqn*}
		\upd f_j = \mathcal{P}_{\restbasis_j} \left[
			\mathcal{A}_j \upd t
			+ \sum_{l=1}^L \mathcal{B}_{jl} \upd Q_l
		\right],
	\end{eqn*}
	or the set of \abbrev{sde}s in the Stratonovich form
	\begin{eqn*}
		\upd f_j = \mathcal{P}_{\restbasis_j} \left[
			(\mathcal{A}_j - \mathcal{S}_j) \upd t
			+ \sum_{l=1}^L \mathcal{B}_{jl} \upd Q_l
		\right],
	\end{eqn*}
	where
	\begin{eqn*}
		\mathcal{S}_j = \frac{1}{2} \sum_{k=1}^C \sum_{l=1}^L
			\mathcal{B}_{kl}^*
			\frac{\fdelta}{\fdelta f_k^*}
			\mathcal{B}_{jl}.
	\end{eqn*}
	Here $f_j \in \mathbb{F}_{\mathbb{M}_j}$, $\mathcal{A}_j \equiv \mathcal{A}_j[\fvec]$ and $\mathcal{B}_{jl} \equiv \mathcal{B}_{jl}[\fvec]$ are functional operators, $W \equiv W[\fvec]$ is a probability functional, and $L$ is the number of noise sources.
	The standard functional Wiener processes $Q_l$ are the compositions of standard complex-valued Wiener processes:
	\begin{eqn*}
		Q_l = \sum_{\nvec \in \fullbasis} \phi_{\nvec} Z_{l,\nvec}.
	\end{eqn*}
\end{theorem}
\begin{proof}
Expanding the functional derivatives according to \defref{func-calculus:func-diff} with $f_j = \sum_{\nvec \in \restbasis_j} \phi_{j,\nvec} \alpha_{j,\nvec}$:
\begin{eqn}
	\frac{\upd W}{\upd t}
	={} & \left(
		- \sum_{j=1}^C \sum_{\nvec \in \restbasis_j}
			\frac{\cwd}{\cwd \alpha_{j,\nvec}}
			\int \upd\xvec\, \phi_{j,\nvec}^* \mathcal{A}_j W
		- \sum_{j=1}^C \sum_{\nvec \in \restbasis_j}
			\frac{\cwd}{\cwd \alpha_{j,\nvec}^*}
			\int \upd\xvec\, \phi_{j,\nvec} \mathcal{A}_j^* W
		\right. \\
	&	\left. + \sum_{j=1}^C \sum_{k=1}^C
			\sum_{\mvec \in \restbasis_j, \nvec \in \restbasis_k}
			\frac{\cwd}{\cwd \alpha_{j,\mvec}^*}
			\frac{\cwd}{\cwd \alpha_{k,\nvec}}
			\int \upd\xvec
			\phi_{j,\mvec} \phi_{k,\nvec}^*
			\sum_{l=1}^L \mathcal{B}_{jl}^* \mathcal{B}_{kl} W
	\right).
\end{eqn}
The diffusion term has to be transformed in order to conform to \thmref{fpe-sde:corr:mc-fpe-sde}:
\begin{eqn}
	\int \upd\xvec \phi_{j,\mvec} \phi_{k,\nvec}^* \sum_{l=1}^L
		\mathcal{B}_{kl} \mathcal{B}_{jl}^*
	& = \int \upd\xvec \int \upd\xvec^\prime
			\phi_{j,\mvec}^\prime \phi_{k,\nvec}^*
			\sum_{l=1}^L \mathcal{B}_{jl}^{\prime *} \mathcal{B}_{kl}
			\delta(\xvec - \xvec^\prime) \\
	& = \int \upd\xvec \int \upd\xvec^\prime
			\phi_{j,\mvec}^\prime \phi_{k,\nvec}^*
			\sum_{l=1}^L \mathcal{B}_{jl}^{\prime *} \mathcal{B}_{kl}
			\sum_{\pvec \in \fullbasis} \phi_{\pvec}^{\prime*} \phi_{\pvec} \\
	& = \sum_{l=1}^L \sum_{\pvec \in \fullbasis}
		\int \upd\xvec\,
			\phi_{j,\mvec} \mathcal{B}_{jl}^* \phi_{\pvec}^*
		\int \upd\xvec\,
			\phi_{k,\nvec}^* \mathcal{B}_{kl} \phi_{\pvec}.
\end{eqn}
Note that we did not specify the index of the full basis used to expand the delta function.
It can be any orthonormal and complete basis, in particular one of $\fullbasis_j$~--- this will not change the result.

Now we have the \abbrev{fpe} in the form required by \thmref{fpe-sde:corr:mc-fpe-sde} with
\begin{eqn}
	a_{\mvec}^{(j)}
	= \int \upd\xvec\, \phi_{j,\mvec}^* \mathcal{A}_j,\,
	\mvec \in \restbasis_j,
\end{eqn}
and
\begin{eqn}
\label{eqn:fpe-sde:corr:func-noise-matrix}
	B_{\mvec,(\pvec,l)}^{(j)}
	= \int \upd\xvec\, \phi_{j,\mvec}^* \mathcal{B}_{jl} \phi_{\pvec},\,
	\mvec \in \restbasis_j, \pvec \in \fullbasis, l \in [1 \ldots L].
\end{eqn}
Note that the columns of $B$ are enumerated using the compound index $\pvec,l$.

Therefore, the initial \abbrev{fpe} is equivalent to the set of \abbrev{sde}s in the It\^o form
\begin{eqn}
	\upd\alpha_{\mvec}^{(j)}
	= \int \upd\xvec\, \phi_{j,\mvec}^* \mathcal{A}_j \upd t
	+ \sum_{\pvec \in \fullbasis, l \in [1 \ldots L]}
		\int \upd\xvec\, \phi_{j,\mvec}^* \mathcal{B}_{jl} \phi_{\pvec} \upd Z_{\pvec,l}.
\end{eqn}
Multiplying by $\phi_{j,\mvec}^\prime$ and grouping by component:
\begin{eqn}
	\sum_{\mvec \in \restbasis_j} \phi_{j,\mvec}^\prime \upd\alpha_{\mvec}^{(j)}
	={} & \sum_{\mvec \in \restbasis_j} \phi_{j,\mvec}^\prime \int \upd\xvec\, \phi_{j,\mvec}^* \mathcal{A}_j \upd t \\
	& + \sum_{\mvec \in \restbasis_j} \phi_{j,\mvec}^\prime \int \upd\xvec\, \phi_{j,\mvec}^*
		\sum_{l=1}^L \sum_{\pvec \in \fullbasis}
			\mathcal{B}_{jl} \phi_{\pvec} \upd Z_{\pvec,l}.
\end{eqn}
Recognising \defref{func-calculus:projector} of the projection operator:
\begin{eqn}
	\upd f_j
	= \proj{\restbasis_j} \left[
		\mathcal{A}_j \upd t
		+ \sum_{l=1}^L \mathcal{B}_{jl}
			\sum_{\pvec \in \fullbasis} \phi_{\pvec} \upd Z_{\pvec,l}
	\right].
\end{eqn}
Defining the standard functional Wiener process as $Q_l = \sum_{\pvec \in \fullbasis} \phi_{\pvec} Z_{\pvec,l}$:
\begin{eqn}
	\upd f_c
	= \proj{\restbasis_j} \left[
		\mathcal{A}_j \upd t
		+ \sum_{l=1}^L \mathcal{B}_{jl} \upd Q_l
	\right].
\end{eqn}

Performing the same multiplication and summation on the Stratonovich term from \thmref{fpe-sde:corr:mc-fpe-sde}:
\begin{eqn}
	\mathcal{S}_j
	= \sum_{\mvec \in \restbasis_j} \phi_{j,\mvec}^\prime s_{\mvec}^{(j)}
	= \frac{1}{2} \sum_{\mvec \in \restbasis_j} \phi_{j,\mvec}^\prime \sum_{k=1}^C \Trace{
		(B^{(k)})^H \vcwd_{(\balpha^{(k)})^*} \evec_{\mvec}^T B^{(j)}
	}.
\end{eqn}
Transforming the trace to a summation:
\begin{eqn}
	\mathcal{S}_j
	= \frac{1}{2} \sum_{\mvec \in \restbasis_c} \phi_{j,\mvec}^\prime \sum_{k=1}^C
		\sum_{\nvec \in \restbasis_k} \sum_{l=1}^L \sum_{\pvec \in \fullbasis}
			(B_{\nvec (\pvec,l)}^{(k)})^*
			\frac{\cwd}{\cwd (\alpha_{\nvec}^{(k)})^*}
			B_{\mvec (\pvec,l)}^{(j)}.
\end{eqn}
Using the multimode form~\eqnref{fpe-sde:corr:func-noise-matrix} of the noise matrix:
\begin{eqn}
	\mathcal{S}_j
	= \frac{1}{2} \sum_{\mvec \in \restbasis_j} \phi_{j,\mvec}^\prime \sum_{k=1}^C
		\sum_{\nvec \in \restbasis_k} \sum_{l=1}^L \sum_{\pvec \in \fullbasis}
			\int \upd\xvec\, \phi_{k,\nvec} \mathcal{B}_{kl}^* \phi_{\pvec}^*
			\int \upd\xvec\, \phi_{j,\mvec}^*
				\frac{\cwd}{\cwd (\alpha_{\nvec}^{(k)})^*}
				\mathcal{B}_{jl} \phi_{\pvec}.
\end{eqn}
Substituting $\sum_{\pvec \in \fullbasis} \phi_{\pvec}^* \phi_{\pvec} = \delta(\xvec - \xvec^\prime)$:
\begin{eqn}
	\mathcal{S}_j
	= \frac{1}{2} \sum_{\mvec \in \restbasis_j} \phi_{j,\mvec}^\prime
		\sum_{k=1}^C \sum_{\nvec \in \restbasis_k} \sum_{l=1}^L
			\int \upd\xvec\,
				\phi_{k,\nvec} \mathcal{B}_{kl}^*
				\phi_{j,\mvec}^* \frac{\cwd}{\cwd (\alpha_{\nvec}^{(k)})^*}
				\mathcal{B}_{jl}.
\end{eqn}
Recognising the projection transformation and the functional differential:
\begin{eqn}
	\mathcal{S}_j
	& = \proj{\restbasis_j} \left[
		\frac{1}{2} \sum_{k=1}^C \sum_{\nvec \in \restbasis_k} \sum_{l=1}^L
			\phi_{k,\nvec} \mathcal{B}_{kl}^*
			\frac{\cwd}{\cwd (\alpha_{\nvec}^{(k)})^*}
			\mathcal{B}_{jl}
	\right] \\
	& = \proj{\restbasis_j} \left[
		\frac{1}{2} \sum_{k=1}^C \sum_{l=1}^L
		\mathcal{B}_{kl}^*
		\frac{\fdelta}{\fdelta f_k^*}
		\mathcal{B}_{jl}
	\right].
	\qedhere
\end{eqn}
\end{proof}

Alternatively, the \abbrev{fpe} from the above theorem can be expressed in a short matrix form.
The \abbrev{fpe} in this case is
\begin{eqn}
	\frac{dW}{dt}
	= \int d\xvec \left(
		- \vfdelta_{\fvec} \cdot \mathbfcal{A} W
		- \vfdelta_{\fvec^*} \cdot \mathbfcal{A}^* W
		+ \Trace{ \vfdelta_{\fvec^*} \vfdelta_{\fvec}^T \mathcal{B} \mathcal{B}^H } W
	\right),
\end{eqn}
where the functional cogradient $\vfdelta_{\fvec} = \left( \fdelta/\fdelta f_1 \ldots \fdelta/\fdelta f_C \right)$, $\mathbf{\mathcal{A}}$ is a vector of $C$ functional operators, and $\mathcal{B}$ is a matrix of $C \times L$ functional operators.
Such \abbrev{fpe} is equivalent to the matrix \abbrev{sde} in the Stratonovich form:
\begin{eqn}
	\upd \fvec = \mathbfcal{P} \left[
		\left( \mathbfcal{A} - \mathbfcal{S} \right) \upd t
		+ \mathcal{B} \upd \Qvec
	\right],
\end{eqn}
with the Stratonovich term
\begin{eqn}
	\mathbfcal{S}_j
	= \frac{1}{2} \Trace{ \mathcal{B}^H \vfdelta_{\fvec^*} \evec_j^T \mathcal{B} },
\end{eqn}
where $\mathbfcal{P}^T \equiv (\proj{\restbasis_1} \ldots \proj{\restbasis_C})$ is a vector of projection operators, and $\Qvec^T \equiv (Q_1 \ldots Q_C)$ is a vector of standard functional Wiener processes.

% =============================================================================
\section{It\^o formula}
% =============================================================================

In this section we will follow a procedure similar to the one in the previous section and derive the It\^o formula for the differential of a functional, based on the standard definition for the multi-variable real-valued case.
Again, we will formulate the real-valued correspondence in a way that is convenient for future proofs.

\begin{lemma}
\label{lmm:fpe-sde:ito-formula:ito-f-real}
	Let $\zvec^T \equiv (z_1 \ldots z_M)$ be a set of real variables, and $\Zvec(t)$ be a standard $L$-dimensional Wiener process.
	For the set of \abbrev{sde}s in the It\^o form
	\begin{eqn*}
		\upd\zvec = \avec(\zvec, t) \upd t + B(\zvec, t) \upd\Zvec(t),
	\end{eqn*}
	the differential of a function $f(\zvec)$ is
	\begin{eqn*}
		\upd f(\zvec) =
			\avec \cdot \vcwd_{\zvec} f(\zvec) \upd t
			+ \frac{1}{2} \Trace{ B B^T \vcwd_{\zvec} \vcwd_{\zvec}^T } f(\zvec) \upd t
			+ \Trace{ B \upd\Zvec \vcwd_{\zvec}^T } f(\zvec).
	\end{eqn*}
\end{lemma}
\begin{proof}
For the detailed proof see Gardiner~\cite{Gardiner1997}, section 4.3.3.
\end{proof}

As a next step, we will extend this lemma to operate on vectors of complex variables and \abbrev{sde}s with complex-valued coefficients from \thmref{fpe-sde:corr:fpe-sde-complex}.

\begin{theorem}
\label{thm:fpe-sde:ito-formula:ito-f-complex}
	Let $\balpha^T \equiv (\alpha_1 \ldots \alpha_M)$ be a set of complex variables, and $\Zvec = (\mathbf{X} + i\mathbf{Y}) / \sqrt{2}$ be an $L$-dimensional standard complex-valued Wiener process, containing two standard $L$-dimensional Wiener processes $\mathbf{X}$ and $\mathbf{Y}$.
	For the set of \abbrev{sde}s in the It\^o form
	\begin{eqn*}
		\upd\balpha = \avec(\balpha, t) \upd t + B(\balpha, t) \upd\Zvec(t),
	\end{eqn*}
	the differential of a function $f(\balpha)$ is
	\begin{eqn*}
		\upd f(\balpha) =
			2 \Real (\avec \cdot \vcwd_{\balpha}) f(\balpha) \upd t
			+ \Trace{ B B^H \vcwd_{\balpha^*} \vcwd_{\balpha}^T } f(\balpha) \upd t
			+ 2 \Real \Trace{ B \upd\Zvec \vcwd_{\balpha}^T } f(\balpha).
	\end{eqn*}
\end{theorem}
\begin{proof}
The proof follows the same scheme as \thmref{fpe-sde:corr:fpe-sde-complex}, just in the opposite direction.
Let $f = g + ih$, $\balpha = \mathbf{x} + i \mathbf{y}$, $\avec = \mathbf{u} + i \mathbf{v}$, $B = F + iG$, $\vcwd_{\balpha} = (\vcwd_{\mathbf{x}} - i \vcwd_{\mathbf{y}}) / 2$.
Then the set of \abbrev{sde}s from the statement is equivalent to
\begin{eqn}
	\upd \begin{pmatrix} \mathbf{x} \\ \mathbf{y} \end{pmatrix}
	= \begin{pmatrix} \mathbf{u} \\ \mathbf{v} \end{pmatrix} \upd t
		+ \frac{1}{\sqrt{2}} \begin{pmatrix} F & -G \\ G & F \end{pmatrix}
			\begin{pmatrix} \upd\mathbf{X} \\ \upd\mathbf{Y} \end{pmatrix}.
\end{eqn}
Applying \lmmref{fpe-sde:ito-formula:ito-f-real} for real-valued functions $g(\mathbf{x}, \mathbf{y})$ and $h(\mathbf{x}, \mathbf{y})$ and combining them into $f = g + ih$, we get
\begin{eqn}
	\upd f ={} &
		\begin{pmatrix} \mathbf{x} \\ \mathbf{y} \end{pmatrix} \cdot
			\begin{pmatrix} \vcwd_{\mathbf{x}} \\ \vcwd_{\mathbf{y}} \end{pmatrix} f \upd t
		+ \frac{1}{4} \Trace{
			\begin{pmatrix} F & -G \\ G & F \end{pmatrix}
			\begin{pmatrix} F^T & G^T \\ -G^T & F^T \end{pmatrix}
			\begin{pmatrix} \vcwd_{\mathbf{x}} \\ \vcwd_{\mathbf{y}} \end{pmatrix}
			\begin{pmatrix} \vcwd_{\mathbf{x}} \\ \vcwd_{\mathbf{y}} \end{pmatrix}^T
		} f \upd t  \\
	& + \frac{1}{\sqrt{2}} \Trace{
			\begin{pmatrix} F & -G \\ G & F \end{pmatrix}
			\begin{pmatrix} \upd\mathbf{X} \\ \upd\mathbf{Y} \end{pmatrix}
			\begin{pmatrix} \vcwd_{\mathbf{x}} \\ \vcwd_{\mathbf{y}} \end{pmatrix}^T
		} f.
\end{eqn}
Now let us match this equation and the lemma statement term by term.

First term:
\begin{eqn}
	2 \Real ( \avec \cdot \vcwd_{\balpha} )
	& = \Real \left(
			\left( \mathbf{u} + i\mathbf{v} \right) \cdot \left( \vcwd_{\mathbf{x}} - i \vcwd_{\mathbf{y}} \right)
		\right) \\
	& = \mathbf{u} \cdot \vcwd_{\mathbf{x}} + \mathbf{v} \cdot \vcwd_{\mathbf{y}} \\
	& = \begin{pmatrix} \mathbf{x} \\ \mathbf{y} \end{pmatrix} \cdot
		\begin{pmatrix} \vcwd_{\mathbf{x}} \\ \vcwd_{\mathbf{y}} \end{pmatrix}.
\end{eqn}

Second term:
\begin{eqn}
	\Trace{ B B^H \vcwd_{\balpha^*} \vcwd_{\balpha}^T }
	={} & \frac{1}{4} \Trace{
		(F F^T + G G^T)
		(\vcwd_{\mathbf{x}} \vcwd_{\mathbf{x}}^T
			+ \vcwd_{\mathbf{y}} \vcwd_{\mathbf{y}}^T)
		} \\
	& - \frac{1}{4} \Trace {
		(F G^T - G F^T)
		(\vcwd_{\mathbf{x}} \vcwd_{\mathbf{y}}^T
			- \vcwd_{\mathbf{y}} \vcwd_{\mathbf{x}}^T)
		} \\
	& + \frac{i}{4} \Trace{
		(F G^T - G F^T)
		(\vcwd_{\mathbf{x}} \vcwd_{\mathbf{x}}^T
			+ \vcwd_{\mathbf{y}} \vcwd_{\mathbf{y}}^T)
	} \\
	& + \frac{i}{4} \Trace{
		(G G^T + F F^T)
		(\vcwd_{\mathbf{x}} \vcwd_{\mathbf{y}}^T
			- \vcwd_{\mathbf{y}} \vcwd_{\mathbf{x}}^T)
	}.
\end{eqn}
Same as in \thmref{fpe-sde:corr:fpe-sde-complex} we notice that $F F^T + G G^T$ and $\vcwd_{\mathbf{x}} \vcwd_{\mathbf{x}}^T + \vcwd_{\mathbf{y}} \vcwd_{\mathbf{y}}^T$ are symmetric matrices, and $F G^T - G F^T$ and $\vcwd_{\mathbf{x}} \vcwd_{\mathbf{y}}^T - \vcwd_{\mathbf{y}} \vcwd_{\mathbf{x}}^T$ are antisymmetric ones.
Therefore, the last two terms contain traces of antisymmetric matrices and are, consequently, equal to zero:
\begin{eqn}
	={} & \frac{1}{4} \Trace{
		(F F^T + G G^T) \vcwd_{\mathbf{x}} \vcwd_{\mathbf{x}}^T
		+ (F G^T - G F^T) \vcwd_{\mathbf{y}} \vcwd_{\mathbf{x}}^T)
		} \\
	& + \frac{1}{4} \Trace {
		(G F^T - F G^T) \vcwd_{\mathbf{x}} \vcwd_{\mathbf{y}}^T
		+ (F F^T + G G^T) \vcwd_{\mathbf{y}} \vcwd_{\mathbf{y}}^T)
		} \\
	={} & \frac{1}{4} \Trace {
		\begin{pmatrix}
			F F^T + G G^T & F G^T - G F^T \\
			G F^T - F G^T & F F^T + G G^T
		\end{pmatrix}
		\begin{pmatrix}
			\vcwd_{\mathbf{x}} \vcwd_{\mathbf{x}}^T & \vcwd_{\mathbf{x}} \vcwd_{\mathbf{y}}^T \\
			\vcwd_{\mathbf{y}} \vcwd_{\mathbf{x}}^T & \vcwd_{\mathbf{y}} \vcwd_{\mathbf{y}}^T
		\end{pmatrix}
	} \\
	={} & \frac{1}{4} \Trace{
		\begin{pmatrix} F & -G \\ G & F \end{pmatrix}
		\begin{pmatrix} F^T & G^T \\ -G^T & F^T \end{pmatrix}
		\begin{pmatrix} \vcwd_{\mathbf{x}} \\ \vcwd_{\mathbf{y}} \end{pmatrix}
		\begin{pmatrix} \vcwd_{\mathbf{x}} \\ \vcwd_{\mathbf{y}} \end{pmatrix}^T
	}.
\end{eqn}

Third term:
\begin{eqn}
	2 \Real \Trace{ B \upd\Zvec \vcwd_{\balpha}^T }
	& = \frac{1}{\sqrt{2}} \Real \Trace{
		(F + iG) (\upd\mathbf{X} + i \upd\mathbf{Y}) (\vcwd_{\mathbf{x}} - i\vcwd_{\mathbf{y}})
	} \\
	& = \frac{1}{\sqrt{2}} \Trace{
		F \upd\mathbf{X} \vcwd_{\mathbf{x}} + F \upd\mathbf{Y} \vcwd_{\mathbf{y}}
		- G \upd\mathbf{Y} \vcwd_{\mathbf{x}} + G \upd\mathbf{X} \vcwd_{\mathbf{y}}
	} \\
	& = \frac{1}{\sqrt{2}} \Trace{
			\begin{pmatrix} F & -G \\ G & F \end{pmatrix}
			\begin{pmatrix} \upd\mathbf{X} \\ \upd\mathbf{Y} \end{pmatrix}
			\begin{pmatrix} \vcwd_{\mathbf{x}} \\ \vcwd_{\mathbf{y}} \end{pmatrix}^T
		}.
\end{eqn}

All terms have matched, thus proving the theorem.
\end{proof}

The above theorem can be reformulated for the multi-component \abbrev{sde}s from \thmref{fpe-sde:corr:mc-fpe-sde}.

\begin{theorem}
\label{thm:fpe-sde:ito-formula:mc-ito-f}
	Let $\balpha^{(j)},\, j = 1 \ldots C$ be $C$ sets of complex variables $\balpha^{(j)} \equiv (\alpha_1^{(j)} \ldots \alpha_{M_j}^{(j)})$.
	For the \abbrev{sde} in the It\^o form
	\begin{eqn*}
		\upd\balpha^{(j)} = \avec^{(j)} \upd t + B^{(j)} \upd\Zvec,
	\end{eqn*}
	the differential of a function $f(\balpha^{(1)}, \ldots, \balpha^{(C)})$ is
	\begin{eqn*}
		\upd f ={} &
			2 \sum_{j=1}^C \Real (\avec^{(j)} \cdot \vcwd_{\balpha^{(j)}}) f \upd t
			+ \sum_{j=1}^C \sum_{k=1}^C \Trace{
				B^{(j)} (B^{(k)})^H \vcwd_{(\balpha^{(k)})^*} \vcwd_{\balpha^{(j)}}^T } f \upd t \\
		& + 2 \sum_{j=1}^C \Real \Trace{ B^{(j)} \upd\Zvec \vcwd_{\balpha^{(j)}}^T } f.
	\end{eqn*}
\end{theorem}
\begin{proof}
Proved analogously to \thmref{fpe-sde:corr:mc-fpe-sde}, by combining $\balpha^{(j)}$ into a single vector and applying \thmref{fpe-sde:ito-formula:ito-f-complex}.
\end{proof}

Finally, we can use the multi-component reformulation to derive the It\^o formula in functional form for the set of \abbrev{sde}s from \thmref{fpe-sde:corr:fpe-sde-func}.

\begin{theorem}
\label{thm:fpe-sde:ito-formula:func-ito-f}
	Given the set of functional \abbrev{sde}s in the It\^o form
	\begin{eqn*}
		\upd f_j = \proj{\restbasis{j}} \left[
			\mathcal{A}_j \upd t + \sum_{l=1}^L \mathcal{B}_{jl} \upd Q_l
		\right],
	\end{eqn*}
	the differential of a functional operator $\mathcal{F}[\fvec]$ is
	\begin{eqn*}
		\upd \mathcal{F}[\fvec]
		={} & \int \upd\xvec^\prime \left(
			2 \sum_{j=1}^C \Real \left(
				\mathcal{A}_j^\prime \frac{\fdelta}{\fdelta f_j^\prime}
			\right) \mathcal{F}[\fvec] \upd t \right. \\
		& + \sum_{j=1}^C \sum_{k=1}^C \sum_{l=1}^L
				\mathcal{B}_{jl}^\prime
				\mathcal{B}_{kl}^{\prime *}
				\frac{\fdelta}{\fdelta f_j^\prime}
				\frac{\fdelta}{\fdelta f_k^{\prime *}} \mathcal{F}[\fvec] \upd t \\
		& \left. + 2 \sum_{j=1}^C \sum_{l=1}^L
			\Real \left(
				\mathcal{B}_{jl}^\prime
				\upd Q_l^\prime
				\frac{\fdelta}{\fdelta f_j^\prime}
			\right)
			\mathcal{F}[\fvec]
		\right).
	\end{eqn*}
	Here $f_j$, $\mathcal{A}_j$, $\mathcal{B}_{jl}$, $L$, and $Q_l$ are defined in the same way as in \thmref{fpe-sde:corr:fpe-sde-func}.
\end{theorem}
\begin{proof}
The set of \abbrev{sde}s can be rewritten in terms of complex vectors as
\begin{eqn}
	\upd\alpha_{\mvec}^{(j)}
	= \int \upd\xvec\, \phi_{j,\mvec}^* \mathcal{A}_j \upd t
	+ \sum_{l=1}^L \sum_{\pvec \in \fullbasis}
		\int \upd\xvec\, \phi_{j,\mvec}^* \mathcal{B}_{jl} \phi_{\pvec} \upd Z_{\pvec,l},\quad
	\mvec \in \restbasis_j.
\end{eqn}
Now, treating the functional operator as a function of $C$ complex vectors
\begin{eqn}
	\mathcal{F} \equiv \mathcal{F}[\mathcal{C}_{\restbasis_1}(\balpha^{(1)}), \ldots, \mathcal{C}_{\restbasis_C}(\balpha^{(C)})],
\end{eqn}
we can use \thmref{fpe-sde:ito-formula:mc-ito-f} with the drift vectors
\begin{eqn}
	a_{\mvec}^{(j)} = \int \upd\xvec\, \phi_{j,\mvec}^* \mathcal{A}_j,
\end{eqn}
and the noise matrices
\begin{eqn}
	B_{\mvec,(\pvec,l)}^{(j)}
	= \int \upd\xvec\, \phi_{j,\mvec}^* \mathcal{B}_{jl} \phi_{\pvec}.
\end{eqn}
Applying \thmref{fpe-sde:ito-formula:mc-ito-f}:
\begin{eqn}
	\upd \mathcal{F}
	={} &
		2 \sum_{j=1}^C \sum_{\mvec \in \restbasis_j} \Real \left(
			\int \upd\xvec^\prime \phi_{j,\mvec}^{\prime*} \mathcal{A}_j^\prime
			\frac{\cwd}{\cwd \alpha_{j,\mvec}}
		\right) \mathcal{F} \upd t \\
	& + \sum_{j=1}^C \sum_{k=1}^C
			\sum_{\mvec \in \restbasis_j} \sum_{\nvec \in \restbasis_k}
			\sum_{l=1}^L \sum_{\pvec \in \fullbasis}
			\int \upd\xvec^\prime \phi_{j,\mvec}^{\prime *} \mathcal{B}_{jl}^\prime \phi_{\pvec}^\prime
			\int \upd\xvec^{\prime\prime} \phi_{k,\nvec}^{\prime\prime} \mathcal{B}_{kl}^{\prime\prime *} \phi_{\pvec}^{\prime\prime *}
			\frac{\cwd^2 \mathcal{F}}{\cwd \alpha_{k,\nvec}^* \cwd \alpha_{j,\mvec}}
			\upd t \\
	& + 2 \sum_{j=1}^C \Real \left(
			\sum_{\mvec \in \restbasis_j}
			\sum_{l=1}^L \sum_{\pvec \in \fullbasis}
			\int \upd\xvec^\prime \phi_{j,\mvec}^{\prime*} \mathcal{B}_{jl}^\prime \phi_{\pvec}^\prime
			\upd Z_{\pvec,l}
			\frac{\cwd}{\cwd \alpha_{j,\mvec}}
			\mathcal{F}
	\right).
\end{eqn}
Recognising definitions of the functional differentials, the functional Wiener process, and the delta function, we get
\begin{eqn}
	={} & 2 \sum_{j=1}^C \Real \left(
			\int \upd\xvec^\prime \mathcal{A}_j^\prime
			\frac{\fdelta}{\fdelta f_j^\prime}
		\right) \mathcal{F} \upd t
	+ \sum_{j=1}^C \sum_{k=1}^C \sum_{l=1}^L
			\int \upd\xvec^\prime \mathcal{B}_{jl}^\prime
			\mathcal{B}_{kl}^{\prime *}
			\frac{\fdelta}{\fdelta f_j^\prime}
			\frac{\fdelta}{\fdelta f_k^{\prime *}} \mathcal{F} \upd t \\
	& + 2 \sum_{j=1}^C \sum_{l=1}^L \Real \left(
			\int \upd\xvec^\prime \mathcal{B}_{jl}^\prime
			\upd Q_l^\prime
			\frac{\fdelta}{\fdelta f_j^\prime}
		\right) \mathcal{F},
\end{eqn}
which leads to the statement of the theorem.
\end{proof}

Alternatively, the functional It\^o formula can be written in the matrix form as
\begin{eqn}
	\upd \mathcal{F}[\fvec]
	={} & \int \upd\xvec^\prime \left(
		2 \Real \left(
			\mathbfcal{A}^\prime \cdot \vfdelta_{\bPsi^\prime}
		\right) \mathcal{F}[\fvec] \upd t
		+ \Trace{
			\mathcal{B}^\prime
			(\mathcal{B}^\prime)^H
			\vfdelta_{\fvec^{\prime *}}
			\vfdelta_{\fvec^\prime}^T
		} \mathcal{F}[\fvec] \upd t \right. \\
	& \left. + 2 \Real \Trace{
			\mathcal{B}^\prime
			\upd\mathbf{Q}^\prime
			\vfdelta_{\fvec^\prime}^T
		} \mathcal{F}[\fvec]
	\right).
\end{eqn}


% =============================================================================
\chapter{Numerical methods}
\label{cha:appendix:numerical}
% =============================================================================

This Appendix outlines the approach to numerical simulations performed for this thesis.
While full texts of the programs used are too long to include verbatim in the text, the main choices made will be described here.


% =============================================================================
\section{Parallel calculations}
% =============================================================================

The phase space methods described in this thesis are inherently parallel.
Such algorithms are suitable to run on modern graphical processing units (\abbrev{gpu}s), as long as the data being processed fits the video memory (which was our case).
We chose nVidia's CUDA as the general purpose \abbrev{gpu} (\abbrev{gpgpu}) platform because of its maturity and the included set of libraries with effective implementations of the \abbrev{fft}, random number generators, and other useful algorithms.
In practice, OpenCL could be used as well: at the moment it is just as fast as CUDA on nVidia video cards, and has an additional benefit of supporting AMD cards.

Furthermore, in order to speed up prototyping we did not use the CUDA language itself (which is, essentially, a superset of C++), but Python bindings to it.
Python is a general-purpose dynamically typed language with a rich set of third-party libraries.
It is quite popular in the academic community owing to the excellent NumPy and SciPy packages~\cite{Oliphant2007}, which provide an extensive toolset for numerical calculations.
PyCUDA~\cite{Klockner2012} augments it with a convenient access to CUDA and its libraries, significantly reducing the amount of boilerplate code and making metaprogramming possible.
Of course, the dynamical facilities of Python make it slower than C++, but this was negligible in our simulations since the majority of calculations was done on \abbrev{gpu}, and the Python language overhead was masked by their asynchronous execution.

The use of \abbrev{gpu} hardware allowed us to reach a hundredfold speedup of calculations on a single nVidia Tesla C2050 as compared to MatLab implementations.
This was improved even further by combining the results calculated on several video cards, such as we did for the calculations in \secref{bell-ineq:ghz}, where we used seven nVidia Tesla M2090.


% =============================================================================
\section{XMDS}
% =============================================================================

For the cases where the raw speed of calculations was less important, and a custom \abbrev{gpgpu} program was not required, we used the XMDS package~\cite{Collecutt2001,Dennis2013}.
XMDS is a code generator for numerical simulation problems, written in Python.
It creates and compiles a C++ program that performs the integration, based on a brief \abbrev{xml} description of the simulation problem.
The strength of XMDS lies in highly optimized integration algorithms it uses (which is improved even more by on-demand compilation), and in its transparent usage of multi-processor interface (\abbrev{mpi}), which allows it to take advantage of multiple cores of a \abbrev{cpu}, or multiple nodes in a cluster.
We used XMDS to integrate the \abbrev{sde}s in \charef{exact}.


% =============================================================================
\section{Stochastic integration}
% =============================================================================

The problem of integrating \abbrev{sde}s is well-studied; for some general information, one can refer to an extensive reviews by Drummond and Mortimer~\cite{Drummond1991}, and by Werner and Drummond~\cite{Werner1997}.
The common approach is to use a method from the Runge-Kutta family, which can be applied without changes to a set of \abbrev{sde}s in the Stratonovich form~\cite{Wilkie2004,Wilkie2005}.

It is possible to employ variable-stepsize methods, which are strongly convergent, provided that stochastic differentials on the subintervals are sampled correctly~\cite{Wilkie2005}.
Another widely-used specialized Runge-Kutta subtype is the \abbrev{rk4} interaction picture method~\cite{CaradocDavies2000}, which can improve convergence properties considerably when applied to stiff equations.
One must note though that the reduction in the number of steps required for a given level of convergence is often neutralized by the increased number of \abbrev{fft}s that have to be performed on each step.

In our case, neither adaptive stepsize, nor a transition to the interaction picture was not necessary, and we settled on a fixed-stepsize fourth order \abbrev{rk} method optimized for low dissipation, low dispersion and low storage~\cite{Berland2006}.
In particular, the latter means that in each substep of the \abbrev{rk} propagation the result depends only on the output of the previous step.
This results in an algorithm requiring 6 steps in order to reach 4th order.
The low-storage property is very important in the \abbrev{gpgpu} world, where one has to keep whole arrays of intermediate results in video memory.


% =============================================================================
\section{Software developed for this thesis}
% =============================================================================

In the process of writing this thesis we have created several libraries used by our simulation programs.
While usefulness of these libraries may decrease with time quickly (in case we choose not to maintain them), and many of the sources are not very well documented, we still feel that it is necessary to reference them here.

The programs used to obtain and process the numerical results in this thesis can be found in the Git repository \href{http://github.com/Manticore/thesis}{github.com/Manticore/thesis}, along with the sources of the thesis itself.
This includes both Python programs and several XMDS scripts used in \charef{exact}.

The processing code for the XMDS scripts was generated by a symbolic calculation library \href{http://github.com/Manticore/wigner}{github.com/Manticore/wigner}, written in Haskell.
This library is able to perform the Wigner transformation of arbitrary operator expressions (including the ones with field operators) and conversions between normally and symmetrically ordered operator products.
In addition to \charef{exact}, it was used to test the theorems from \charef{wigner-spec} and also in several other places in the thesis where ordering conversions of complex operator expressions were needed.

To obtain the results in \charef{bec-noise} and \charef{bec-squeezing} we created \href{http://github.com/Manticore/beclab}{github.com/Manticore/beclab} a framework in Python and CUDA for simulating the dynamics of trapped \abbrev{bec}s.
The low-level part of this library later gave rise to \href{http://github.com/Manticore/reikna}{github.com/Manticore/reikna}, a code generator for \abbrev{gpgpu} algorithms written in Python, which can work both with CUDA and OpenCL.


% unsorted stuff that may be useful; to be removed the in final version
% =============================================================================
\chapter{Unsorted notes (draft only)}
% =============================================================================

% =============================================================================
\section{Minimal lattice}
% =============================================================================

This chapter shows how to calculate minimal lattice which contains all modes below given cutoff $\ecut$.


% =============================================================================
\subsection{Uniform grid}
% =============================================================================

In case of 1D uniform grid, the cutoff condition looks like:
\[
	\frac{\hbar^2 k^2}{2 m} \le \ecut,
\]
\[
	k \le \sqrt{\frac{2 m \ecut}{\hbar^2}} = k_{\mathrm{cut}}.
\]
In FFT algorithm, maximum spatial frequency for given problem size $N$ and lattice step $d$ can be found as:
\[
	k_{\max} = 2 \pi \frac{ \left[ \frac{N}{2} \right] }{N d}.
\]
If $L = d (N - 1)$ is fixed, the condition for even $N$ is easier to find first:
\[
	N_{\min,\mathrm{even}} = 2 \lceil
		\frac{k_{\mathrm{cut}} L}{2 \pi} + \frac{1}{2}
	\rceil.
\]
Then, one should check whether $k_{\max}$ corresponding to $N_{\min,\mathrm{even}} - 1$ is still larger than $k_{\mathrm{cut}}$; if it is, it should be used as minimal lattice size instead.
In 3D case, $N_{\min}$ can be calculated separately for each dimension using corresponding values of $L$.


% =============================================================================
\subsection{Harmonic grid}
% =============================================================================

In 1D case, the cutoff condition is:
\[
	\hbar \omega ( n + \frac{1}{2} ) \le \ecut,
\]
where $\omega$ is the trap frequency.
Therefore minimal lattice size can be found as:
\[
	N_{\min} = \lceil \frac{\ecut}{\hbar \omega} - \frac{1}{2} \rceil.
\]
3D case is a bit more complicated:
\[
	\hbar \left(
		\omega_x (n_x + \frac{1}{2})
		+ \omega_y (n_y + \frac{1}{2})
		+ \omega_z (n_z + \frac{1}{2})
	\right) \le \ecut,
\]
and minimal lattice sizes are:
\begin{equation*}
\begin{split}
	N_{x,\min} & = \lceil
		\frac{\ecut / \hbar - \omega_y / 2 - \omega_z / 2}{\omega_x} - \frac{1}{2}
	\rceil, \\
	N_{y,\min} & = \lceil
		\frac{\ecut / \hbar - \omega_x / 2 - \omega_z / 2}{\omega_y} - \frac{1}{2}
	\rceil, \\
	N_{z,\min} & = \lceil
		\frac{\ecut / \hbar - \omega_x / 2 - \omega_y / 2}{\omega_z} - \frac{1}{2}
	\rceil.
\end{split}
\end{equation*}

% =============================================================================
\section{Sets of single-mode operators}
% =============================================================================

We start from the set of single-mode operators $\hat{a}_j$, which obey bosonic commutation relations:
\begin{eqn}
\label{eqn:wigner:mm-aux:commutators}
	[ \hat{a}_j, \hat{a}_k ] & = [ \hat{a}_j^\dagger, \hat{a}_k^\dagger ] = 0, \\
	[ \hat{a}_j, \hat{a}_k^\dagger ] & = \delta_{jk}.
\end{eqn}

In order to work with the moments of multimode operators we will need the equations for commutators of arbitrary single-mode operator products.

\begin{lemma}
\label{lmm:wigner:mm-aux:high-order-commutators}
	\begin{eqn*}
		[ \hat{a}_n, \hat{a}_{m_1}^\dagger \ldots \hat{a}_{m_k}^\dagger ]
		& = \sum_{i=1}^k \delta_{n m_i}
			\prod_{j=1,j \ne i}^k \hat{a}_{m_j}^\dagger, \\
		[ \hat{a}_n^\dagger, \hat{a}_{m_1} \ldots \hat{a}_{m_k} ]
		& = - \sum_{i=1}^k \delta_{n m_i}
			\prod_{j=1,j \ne i}^k \hat{a}_{m_j}.
	\end{eqn*}
\end{lemma}
\begin{proof}
Let us find the expression for the first commutator by induction.
Providing that we know the expression for $[ \hat{a}_n, \hat{a}_{m_1}^\dagger \ldots \hat{a}_{m_{k-1}}^\dagger ]$,
commutator of order $k$ can be expanded as:
\begin{eqn}
	[ \hat{a}_n, \hat{a}_{m_1}^\dagger \ldots \hat{a}_{m_k}^\dagger ]
	={} & (1 - \delta_{n m_k})
		[ \hat{a}_n, \hat{a}_{m_1}^\dagger \ldots \hat{a}_{m_{k-1}}^\dagger ] \hat{a}_{m_k} \\
	& + \delta_{n m_k} (
		\hat{a}_n \hat{a}_{m_1}^\dagger \ldots \hat{a}_{m_{k-1}}^\dagger \hat{a}_n^\dagger
		- \hat{a}_{m_1}^\dagger \ldots \hat{a}_{m_{k-1}}^\dagger \hat{a}_n^\dagger \hat{a}_n
	).
\end{eqn}
Here we have split the initial commutator into two possible outcomes, depending on whether $n = m_k$.
First term, corresponding to $n \ne m_k$, contains the known commutator of lower order.
In the second term we have substituted $\hat{a}_n$ for $\hat{a}_{m_k}$,
since the delta function outside the parentheses ensures that $n = m_k$.
Swapping $\hat{a}_n^\dagger$ and $\hat{a}_n$ in the last term and, again, recognising the known commutator:
\begin{eqn}
	& = (1 - \delta_{n m_k})
		[ \hat{a}_n, \hat{a}_{m_1}^\dagger \ldots \hat{a}_{m_{k-1}}^\dagger ] \hat{a}_{m_k}
	+ \delta_{n m_k} (
		[ \hat{a}_n, \hat{a}_{m_1}^\dagger \ldots \hat{a}_{m_{k-1}}^\dagger ] \hat{a}_n^\dagger
		+ \hat{a}_{m_1}^\dagger \ldots \hat{a}_{m_{k-1}}^\dagger
	) \\
	& = [ \hat{a}_n, \hat{a}_{m_1}^\dagger \ldots \hat{a}_{m_{k-1}}^\dagger ] \hat{a}_{m_k}
	+ \delta_{n m_k} \hat{a}_{m_1}^\dagger \ldots \hat{a}_{m_{k-1}}^\dagger.
\end{eqn}
Now, starting from the first-order relation $[ \hat{a}_n, \hat{a}_{m_1}^\dagger ] = \delta_{n m_1}$, we can obtain the relation for any order:
\begin{eqn}
	[ \hat{a}_n, \hat{a}_{m_1}^\dagger \hat{a}_{m_2}^\dagger ]
	& = \delta_{n m_1} \hat{a}_{m_2}^\dagger + \delta_{n m_2} \hat{a}_{m_1}^\dagger, \\
	[ \hat{a}_n, \hat{a}_{m_1}^\dagger \hat{a}_{m_2}^\dagger \hat{a}_{m_3}^\dagger ]
	& = \delta_{n m_1} \hat{a}_{m_2}^\dagger \hat{a}_{m_3}^\dagger
	+ \delta_{n m_2} \hat{a}_{m_1}^\dagger \hat{a}_{m_3}^\dagger
	+ \delta_{n m_3} \hat{a}_{m_1}^\dagger \hat{a}_{m_2}^\dagger, \\
	& \ldots
\end{eqn}
Which gives us the statement of the lemma.
\end{proof}

Note that if $n = m_1 = \ldots = m_k$, this boils down to the well-known relation from~\cite{Louisell1990}:
\begin{eqn}
	[ \hat{a}, (\hat{a}^\dagger)^k ] = k (\hat{a}^\dagger)^{k-1}.
\end{eqn}



\bibliographystyle{aip}
\bibliography{thesis}

\end{document}
