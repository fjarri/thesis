\documentclass[a4paper,11pt]{report}
%\usepackage{fullpage}

\usepackage[pdftex]{graphicx}
\usepackage[margin=5pt]{subfig}
\usepackage{amssymb}
\usepackage{amsmath}
\usepackage{amsthm}
\usepackage{amsfonts}
\usepackage[usenames]{color}
\usepackage{bm}
\usepackage{verbatim}
\usepackage{dsfont}
\usepackage{upgreek}
\usepackage{psfrag}
\usepackage[nottoc]{tocbibind}
\usepackage{environ} % gives \NewEnviron macro

% For print
%\usepackage[breaklinks,pdftex,pdftitle={PhD thesis: Two-component BEC dynamics simulations}, pdfauthor={Bogdan Opanchuk}]{hyperref}
% For screen
\definecolor{darkgreen}{rgb}{0.00,0.50,0.25}
\definecolor{darkblue}{rgb}{0.00,0.00,0.67}
\usepackage[breaklinks,pdftex,pdftitle={PhD thesis: Two-component BEC dynamics simulations}, pdfauthor={Bogdan Opanchuk},colorlinks,urlcolor=blue,citecolor=darkgreen,linkcolor=darkblue]{hyperref}

% Page layout
% 1.5 interval
\renewcommand{\baselinestretch}{1.5}
\textwidth=15cm
\evensidemargin=-0.28cm
\oddsidemargin=1.13cm
\pretolerance=2000

\newtheorem{theorem}{Theorem}
\newtheorem{lemma}{Lemma}

\newcommand{\todo}[1]{\textcolor{red}{[#1]}}
\newcommand{\citationneeded}{\textcolor{red}{[citation needed]}}

\newcommand{\jvec}{\boldsymbol{j}}
\newcommand{\xvec}{\boldsymbol{x}}
\newcommand{\kvec}{\boldsymbol{k}}
\newcommand{\lvec}{\boldsymbol{l}}
\newcommand{\mvec}{\boldsymbol{m}}
\newcommand{\nvec}{\boldsymbol{n}}
\newcommand{\pvec}{\boldsymbol{p}}

\newcommand{\ecut}{\epsilon_{\mathrm{cut}}}
\newcommand{\Tr}{\operatorname{Tr}}
\newcommand{\Trace}[1]{\Tr \left\{ #1 \right\}}

\newcommand{\symprod}[1]{\left\{ #1 \right\}_{\mathrm{sym}}}
\newcommand{\pathavg}[1]{\langle #1 \rangle_{\mathrm{paths}}}
\newcommand{\Real}{\operatorname{Re}}
\newcommand{\Imag}{\operatorname{Im}}

\newcommand{\Psivec}{\boldsymbol{\Psi}}
\newcommand{\Psiop}{\hat{\Psi}}
\newcommand{\Psiopvec}{\hat{\boldsymbol{\Psi}}}

% {C, H} - i.e. either complex number or operator from Hilbert space
% making it a macro, because I'm not sure what the letter should be
\newcommand{\BasicType}{\mathbb{B}}

\newcommand{\eqnref}[1]{(\ref{eqn:#1})}
\newcommand{\figref}[1]{Fig.~\ref{fig:#1}}
\newcommand{\charef}[1]{Chapter~\ref{cha:#1}}
\newcommand{\appref}[1]{Appendix~\ref{cha:appendix:#1}}
\newcommand{\thmref}[1]{Theorem~\ref{thm:#1}}
\newcommand{\lmmref}[1]{Lemma~\ref{lmm:#1}}

\NewEnviron{eqn}{%
\begin{equation}\begin{split}
  \BODY
\end{split}\end{equation}
}

\NewEnviron{eqn*}{%
\begin{equation*}\begin{split}
  \BODY
\end{split}\end{equation*}
}

\title{Two-component BEC dynamics simulations}
\author{Bogdan Opanchuk}

\begin{document}
\maketitle

\tableofcontents

\chapter{Operator formalism}
\label{cha:formalism}

% =============================================================================
\section{Relaxed operations with complex numbers}
% =============================================================================

Formally, a function of complex variable has to be holomorphic in order to be complex differentiable.
In many cases it is enough to have less strict ``physicists'\,'' complex differentiation rules.
These rules were developed by Wirtinger in~\cite{Wirtinger1927};
a very good review and a thorough description of their application can be found in~\cite{Kreutz-Delgado2009}.
This section will outline these rules and provide some lemmas based on them, which will be used further.

For a complex variable $z = x + iy$ and a function $f(z) = u(x, y) + iv(x, y)$
\[
	\left( \frac{df(z)}{dz} \right)_{phys}
	= \frac{1}{2} \left(
		\frac{\partial f}{\partial x} - i \frac{\partial f}{\partial y}
	\right).
\]

\begin{lemma}
If $f(z)$ is holomorphic, then the Wirtinger differentiation is equivalent to the formal one.
\end{lemma}

\begin{lemma}
For any ``good'' (even non-holomorphic) $f(z)$, Wirtinger differentiation obeys sum, product, quotient, and chain differentiation rules
(the former one is applied as if $f(z) \equiv f(z, z^*)$).
\end{lemma}

Hereinafter we will use Wirtinger differentiation unless explicitly stated otherwise,
because some important functions we will encounter are not holomorphic.
This differentiation has all intuitively assumed properties, along with some not quite obvious ones.

\begin{lemma}
For any nonnegative integers $a$ and $b$.
\[
	\frac{d}{dz} (z^a (z^*)^b) = a z^{a-1} (z^*)^b,
	\quad
	\frac{d}{dz^*} (z^a (z^*)^b) = b z^a (z^*)^{b-1},
\]
\end{lemma}
\begin{proof}
Let us assume that the statement of the lemma is valid for some $a$ and $b$, then using chain rule
\[
	\frac{d}{dz} (z^{a+1} (z^*)^b)
	= \frac{d}{dz} (z z^a (z^*)^b)
	= z^a (z^*)^b + z \frac{d}{dz} (z^a (z^*)^b)
	= z^a (z^*)^b + a z z^{a-1} (z^*)^b
	= (a + 1) z^a (z^*)^b.
\]
The part for $d/dz^*$ can is proved in the same way.
One can easily prove (by transition to real values) that $d(z z^*)/dz = z^*$ and $d(z z^*)/dz^* = z$.
By induction, the statement is true for any natural $a$ and $b$,
and it is obviously true if $a = 0$ or $b = 0$, which proves the lemma.
\todo{This can be proved for any real $a$ and $b$, if necessary.}
\end{proof}

This is straightforwardly followed by
\begin{lemma}
\label{lmm:formalism:c-numbers:independent-vars}
If $f(z)$ can be expanded into series of $z^n (z^*)^m$, $df(z)/dz$ can be treated as partial differentiation of the function of two independent variables $z$ and $z^*$.
In other words:
\[
	\frac{d}{dz} f(z) \equiv \frac{\partial}{\partial z} f(z, z^*),
	\quad
	\frac{d}{dz^*} f(z) \equiv \frac{\partial}{\partial z^*} f(z, z^*).
\]
\end{lemma}

Now we can prove two lemmas which will help us deal with some integrals.

\begin{lemma}
\label{lmm:formalism:c-numbers:fourier-of-moments}
If $\alpha$ and $\lambda$ are complex variables and $\int d^2\alpha$ stands for the integral over the complex plane, then for any non-negative integers $r$ and $s$:
\[
	\int d^2\alpha\, \alpha^r (\alpha^*)^s \exp(-\lambda \alpha^* + \lambda^* \alpha)
	= \pi^2
		\left( -\frac{\partial}{\partial \lambda^*} \right)^r
		\left( \frac{\partial}{\partial \lambda} \right)^s
		\delta(\Real \lambda) \delta(\Imag \lambda)
\]
\end{lemma}
\begin{proof}
First, changing the variables in the integrals and using known Fourier transform relations, we can prove that for real $x$ and $v$, and non-negative integer $n$
\[
	\int\limits_{-\infty}^{\infty} dv\, v^n \exp(\pm 2 i x v)
	= \pi (\mp i / 2)^n \delta^{(n)}(x),
\]
Note that we have explicitly written integration limits here;
they are swapped when we change the variable in the first integral.

Denoting $\alpha = u + iv$ and $\lambda = x + iy$, we can expand the initial expression as
\begin{equation*}
\begin{split}
	\int d^2\alpha\, \alpha^r (\alpha^*)^s \exp(-\lambda \alpha^* + \lambda^* \alpha)
	& = \int du dv \exp(2ivx - 2iuy)
		\sum_{l=0}^r \binom{r}{l} u^l (iv)^{r-l}
		\sum_{m=0}^s \binom{s}{m} u^m (-iv)^{s-m} \\
	& = \sum_{l=0}^r \sum_{m=0}^s \binom{r}{l} \binom{s}{m}
		i^{r-l} (-i)^{s-m}
		\int du\, u^{l+m} \exp(2ivx)
		\int dv\, v^{r-l+s-m} \exp(-2iuy) \\
	& = \pi^2 \sum_{l=0}^r \sum_{m=0}^s \binom{r}{l} \binom{s}{m}
		i^{r-l} (-i)^{s-m}
		(-i/2)^{l+m} \delta^{(l+m)}(y)
		(i/2)^{r-l+s-m} \delta^{(r-l+s-m)}(x) \\
	& = \pi^2
		\sum_{l=0}^r \binom{r}{l}
			\frac{1}{2^r}
			(-i \partial / \partial y)^l
			(-\partial / \partial x)^{r-l}
		\sum_{m=0}^s \binom{s}{m}
			\frac{1}{2^s}
			(-i \partial / \partial y)^m
			(\partial / \partial x)^{s-m}
		\delta(y) \delta(x) \\
	& = \pi^2
		\left( \frac{1}{2} (-i \partial / \partial y - \partial / \partial x) \right)^r
		\left( \frac{1}{2} (-i \partial / \partial y + \partial / \partial x) \right)^s
		\delta(y) \delta(x) \\
	& = \pi^2
		\left( -\frac{\partial}{\partial \lambda^*} \right)^r
		\left( \frac{\partial}{\partial \lambda} \right)^s
		\delta(\Real \lambda) \delta(\Imag \lambda).
		\qedhere
\end{split}
\end{equation*}
\end{proof}

A notable special case of \lmmref{formalism:c-numbers:fourier-of-moments} is
\[
	\int d^2\alpha \exp(-\lambda \alpha^* + \lambda^* \alpha)
	= \pi^2 \delta(\Real \lambda) \delta(\Imag \lambda).
\]

\begin{lemma}
\label{lmm:formalism:c-numbers:zero-integrals}
For any non-negative integers $r$, $s$ and complex $\alpha$.
\begin{equation*}
\begin{split}
	\int d^2\lambda
		\frac{\partial}{\partial \lambda} \left(
			\exp(-\lambda \alpha^* + \lambda^* \alpha)
			\exp(ixy) x^r y^s
		\right)
	& = 0 \\
	\int d^2\lambda
		\frac{\partial}{\partial \lambda^*}
		\left(
			\exp(-\lambda \alpha^* + \lambda^* \alpha)
			\exp(ixy) x^r y^s
		\right)
	& = 0.
\end{split}
\end{equation*}
\end{lemma}
\begin{proof}
We will prove the first equation.
First, note that complex-valued integral of derivative is evaluated as
\begin{equation*}
\begin{split}
	\int d^2\lambda \frac{\partial}{\partial \lambda} f(\lambda, \lambda^*)
	& = \frac{1}{2} \int\limits_{-\infty}^{\infty} dx \int\limits_{-\infty}^{\infty} dy
		\left( \frac{\partial}{\partial x} - i \frac{\partial}{\partial y} \right)
		g(x, y) \\
	& = \frac{1}{2} \int\limits_{-\infty}^{\infty} dy \int\limits_{-\infty}^{\infty} dx
			\frac{\partial}{\partial x} g(x, y)
		- \frac{i}{2} \int\limits_{-\infty}^{\infty} dx \int\limits_{-\infty}^{\infty} dy
			\frac{\partial}{\partial y} g(x, y) \\
	& =	\frac{1}{2} \int\limits_{-\infty}^{\infty} dy \left(
			\left. g(x, y) \right|_{x=-\infty}^{\infty}
		\right)
		- \frac{i}{2} \int\limits_{-\infty}^{\infty} dx \left(
			\left. g(x, y) \right|_{y=-\infty}^{\infty}
		\right),
\end{split}
\end{equation*}
where we expanded $\lambda = x + iy$.
Thus
\begin{equation*}
\begin{split}
	\int d^2\lambda
		\frac{\partial}{\partial \lambda} \left(
			\exp(-\lambda \alpha^* + \lambda^* \alpha)
			\exp(ixy) x^r y^s
		\right)
	& = \frac{1}{2} \int dy \left. \left(
			\exp(2ixv - 2iyu) \exp(ixy) x^r y^s
		\right) \right|_{x = -\infty}^\infty \\
	& - \frac{i}{2} \int dx \left. \left(
			\exp(2ixv - 2iyu) \exp(ixy) x^r y^s
		\right) \right|_{y = -\infty}^\infty \\
	& = \left(
			\frac{1}{2} \exp(2ixv) x^r \int dy \exp(iy(x-2u)) y^s
		\right)_{x = -\infty}^\infty \\
	& - \left(
			\frac{i}{2} \exp(-2ixy) y^s \int dx \exp(ix(y+2v)) x^r
		\right)_{y = -\infty}^\infty \\
	& = \left(
			\frac{1}{2} \exp(2ixv) x^r 2 \pi i^s \delta^{(s)}(x-2u)
		\right)_{x = -\infty}^\infty \\
	& - \left(
			\frac{i}{2} \exp(-2ixy) y^s 2 \pi i^r \delta^{(r)}(y+2v)
		\right)_{y = -\infty}^\infty \\
	& = 0,
\end{split}
\end{equation*}
because any derivative of delta function is zero on the infinity.
\end{proof}

% =============================================================================
\section{Single-mode Wigner representation}
% =============================================================================

We will need the displacement operator which was first introduced by Weyl~\cite{Weyl1950}:
\begin{equation}
\label{eqn:formalism:sm-wigner:dispacement-op}
	\hat{D}(\lambda, \lambda^*) = \exp(\lambda \hat{a}^\dagger - \lambda^* \hat{a}).
\end{equation}
Using Baker-Hausdorff theorem to split non-commuting operators in the exponent,
one can find that
\begin{equation}
\label{eqn:formalism:sm-wigner:displacement-derivatives}
\begin{split}
	\frac{\partial}{\partial \lambda} \hat{D}(\lambda, \lambda^*)
	& = \hat{D}(\lambda, \lambda^*) (\hat{a}^\dagger + \frac{1}{2} \lambda^*)
	= (\hat{a}^\dagger - \frac{1}{2} \lambda^*) \hat{D}(\lambda, \lambda^*), \\
	-\frac{\partial}{\partial \lambda^*} \hat{D}(\lambda, \lambda^*)
	& = \hat{D}(\lambda, \lambda^*) (\hat{a} + \frac{1}{2} \lambda)
	= (\hat{a} - \frac{1}{2} \lambda) \hat{D}(\lambda, \lambda^*).
\end{split}
\end{equation}

In terms of displacement operator Wigner transformation $\mathcal{W}$ is defined as
\begin{equation}
\label{eqn:formalism:sm-wigner:w-transformation}
	\mathcal{W}[\hat{A}]
	= \frac{1}{\pi^2} \int d^2 \lambda \exp(-\lambda \alpha^* + \lambda^* \alpha)
		\Trace{ \hat{A} \hat{D}(\lambda, \lambda^*) }.
\end{equation}
It transforms an operator $\hat{A}$ on a Hilbert space to a function $\mathcal{W}[\hat{A}](\alpha, \alpha^*)$ on phase space.
The backward transformation (called the Weyl transformation) gives back matrix elements of the operator:
\[
	\langle \alpha \lvert \mathcal{W}^{-1}[f(\alpha, \alpha^*)] \rvert \alpha \rangle
	= \todo{find\,the\,expression},
\]
which is enough, because any operator is determined by its expectation in all coherent states~\cite{Gardiner2004}.

Thus Wigner function can be defined as a result of Wigner transformation of the density matrix: $W(\alpha, \alpha^*) = \mathcal{W}[\hat{\rho}]$.
The Wigner function always exists for any density matrix~\cite{Gardiner2004}.
The correspondence $W \leftrightarrow \hat{\rho}$ is a bijection \todo{prove it?}.

In some cases it will be convenient to use Wigner function in form~\cite{Gardiner2004}
\begin{equation}
\label{eqn:formalism:sm-wigner:w-function}
	W (\alpha, \alpha^*)
	= \frac{1}{\pi^2} \int d^2 \lambda \exp(-\lambda \alpha^* + \lambda^* \alpha)
		\chi_W (\lambda, \lambda^*),
\end{equation}
where $\chi_W (\lambda, \lambda^*)$ is the characteristic function
\[
	\chi_W (\lambda, \lambda^*)
	= \Trace{ \hat{\rho} \hat{D}(\lambda, \lambda^*) }.
\]

\begin{lemma}
\label{lmm:formalism:sm-wigner:zero-integrals}
\begin{equation*}
\begin{split}
	\int d^2\lambda
		\frac{\partial}{\partial \lambda} \left(
			\exp(-\lambda \alpha^* + \lambda^* \alpha)
			\left( \frac{\partial}{\partial \lambda} \right)^m
			\left( -\frac{\partial}{\partial \lambda^*} \right)^n
			\hat{D}(\lambda, \lambda^*)
		\right)
	& = 0 \\
	\int d^2\lambda
		\frac{\partial}{\partial \lambda^*}
		\left(
			\exp(-\lambda \alpha^* + \lambda^* \alpha)
			\left( \frac{\partial}{\partial \lambda} \right)^m
			\left( -\frac{\partial}{\partial \lambda^*} \right)^n
			\hat{D}(\lambda, \lambda^*)
		\right)
	& = 0.
\end{split}
\end{equation*}
\end{lemma}
\begin{proof}
We will prove the first equation.
Expanding $\lambda = x + iy$ and applying Baker-Hausdorff theorem:
\begin{equation*}
\begin{split}
	\hat{D}(\lambda, \lambda^*)
	= \exp(x(\hat{a}^\dagger - \hat{a}) + iy(\hat{a}^\dagger + \hat{a}))
	= \exp(ixy) \exp(x(\hat{a}^\dagger - \hat{a})) \exp(iy(\hat{a}^\dagger + \hat{a}))
\end{split}
\end{equation*}
Thus
\begin{equation*}
\begin{split}
	& \int d^2\lambda
		\frac{\partial}{\partial \lambda} \left(
			\exp(-\lambda \alpha^* + \lambda^* \alpha)
			\left( \frac{\partial}{\partial \lambda} \right)^m
			\left( -\frac{\partial}{\partial \lambda^*} \right)^n
			\hat{D}(\lambda, \lambda^*)
		\right) \\
	& = \int d^2\lambda
		\frac{\partial}{\partial \lambda} \left(
			\exp(-\lambda \alpha^* + \lambda^* \alpha)
			\exp(ixy)
			\frac{1}{4^{m+n}}
			\left( \frac{\partial}{\partial x} - i \frac{\partial}{\partial y} \right)^m
			\left( -\frac{\partial}{\partial x} - i \frac{\partial}{\partial y} \right)^n
			\sum\limits_{r=0}^{\infty} \frac{x^r (\hat{a}^\dagger - \hat{a})^r}{r!}
			\sum\limits_{s=0}^{\infty} \frac{(iy)^s (\hat{a}^\dagger + \hat{a})^s}{s!}
		\right) \\
	& = \frac{1}{4^{m+n}} \sum\limits_{r=0}^{\infty} \sum\limits_{s=0}^{\infty} \left(
		\int d^2\lambda
			\frac{\partial}{\partial \lambda} \left(
				\exp(-\lambda \alpha^* + \lambda^* \alpha)
				\exp(ixy)
				\left( \frac{\partial}{\partial x} - i \frac{\partial}{\partial y} \right)^m
				\left( -\frac{\partial}{\partial x} - i \frac{\partial}{\partial y} \right)^n
				x^r (iy)^s
			\right)
		\right)
		\frac{(\hat{a}^\dagger - \hat{a})^r}{r!}
		\frac{(\hat{a}^\dagger + \hat{a})^s}{s!} \\
	& = 0,
\end{split}
\end{equation*}
where we used \lmmref{formalism:c-numbers:zero-integrals} to evaluate the integral over $\lambda$.
\end{proof}

\begin{lemma}
\label{lmm:formalism:sm-wigner:moments-from-chi}
\[
	\langle \symprod{ \hat{a}^r (\hat{a}^\dagger)^s } \rangle
	= \left.
		\left( \frac{\partial}{\partial \lambda} \right)^s
		\left( -\frac{\partial}{\partial \lambda^*} \right)^r
		\chi_W (\lambda, \lambda^*)
	\right|_{\lambda=0}.
\]
\end{lemma}
\begin{proof}
The exponent in the $\chi_W$ can be expanded as
\[
	\exp (\lambda \hat{a}^\dagger - \lambda^* \hat{a})
	= \sum\limits_{r,s}
		\frac{(-\lambda^*)^r \lambda^s}{r!s!}
		\symprod{ \hat{a}^r (\hat{a}^\dagger)^s }.
\]
Thus
\[
	\chi_W(\lambda, \lambda^*)
	= \sum\limits_{r,s}
		\frac{(-\lambda^*)^r \lambda^s}{r!s!}
		\Trace{
			\hat{\rho} \symprod{ \hat{a}^r (\hat{a}^\dagger)^s }
		}
	= \sum\limits_{r,s}
		\frac{(-\lambda^*)^r \lambda^s}{r!s!}
		\langle \symprod{ \hat{a}^r (\hat{a}^\dagger)^s } \rangle
\]
Apparently, the application of $(\partial / \partial \lambda)^s$ and $(-\partial / \partial \lambda^*)^r$ will eliminate all lower order moments,
and setting $\lambda = 0$ afterwards will eliminate all higher order moments,
leaving only $\symprod{ \hat{a}^r (\hat{a}^\dagger)^s }$:
\[
	\left.
		\left( \frac{\partial}{\partial \lambda} \right)^s
		\left( -\frac{\partial}{\partial \lambda^*} \right)^r
		\chi_W (\lambda, \lambda^*)
	\right|_{\lambda=0}
	= r! s! \frac{1}{r! s!}
		\langle \symprod{ \hat{a}^r (\hat{a}^\dagger)^s } \rangle
	= \langle \symprod{ \hat{a}^r (\hat{a}^\dagger)^s } \rangle.
	\qedhere
\]
\end{proof}

Now we can get the final relation.
\begin{theorem}
\label{thm:formalism:sm-wigner:moments}
\[
	\int d^2\alpha\, \alpha^r (\alpha^*)^s W(\alpha, \alpha^*)
	= \langle \symprod{ \hat{a}^r (\hat{a}^\dagger)^s } \rangle
\]
\end{theorem}
\begin{proof}
By definition of Wigner function:
\begin{equation*}
\begin{split}
	\int d^2\alpha\, \alpha^r (\alpha^*)^s W(\alpha, \alpha^*)
	= \frac{1}{\pi^2} \Trace{ \hat{\rho}
		\int d^2\alpha\, \alpha^r (\alpha^*)^s
		\int d^2\lambda \exp(-\lambda \alpha^* + \lambda^* \alpha)
		\hat{D}(\lambda, \lambda^*)
	}
\end{split}
\end{equation*}
Integrating by parts and eliminating terms which fit \lmmref{formalism:sm-wigner:zero-integrals}:
\[
	= \frac{1}{\pi^2} \Trace{ \hat{\rho}
		\int d^2\alpha \int d^2\lambda
		\exp(-\lambda \alpha^* + \lambda^* \alpha)
		\left( \frac{\partial}{\partial \lambda} \right)^s
		\left( -\frac{\partial}{\partial \lambda^*} \right)^r
		\hat{D} (\lambda, \lambda^*)
	}
\]
Evaluating integral over $\alpha$ using \lmmref{formalism:c-numbers:fourier-of-moments}:
\begin{equation*}
\begin{split}
	& = \int d^2\lambda\,
		\delta (\Real \lambda) \delta (\Imag \lambda)
		\left( \frac{\partial}{\partial \lambda} \right)^s
		\left( -\frac{\partial}{\partial \lambda^*} \right)^r
		\Trace{
			\hat{\rho}
			\hat{D}(\lambda, \lambda^*)
		} \\
	& = \left.
		\left( \frac{\partial}{\partial \lambda} \right)^s
		\left( -\frac{\partial}{\partial \lambda^*} \right)^r
		\chi_W (\lambda, \lambda^*)
	\right|_{\lambda=0}.
\end{split}
\end{equation*}
Now, recognising the final expression as a part of \lmmref{formalism:sm-wigner:moments-from-chi},
we immideately get the statement of the theorem.
\end{proof}

\begin{theorem}[Operator correspondences]
\label{thm:formalism:sm-wigner:correspondences}
\begin{equation*}
\begin{split}
	\mathcal{W} [ \hat{a} \hat{A} ]
		& = \left( \alpha + \frac{1}{2} \frac{\partial}{\partial \alpha^*} \right) \mathcal{W}[\hat{A}],
	\quad
	\mathcal{W} [ \hat{a}^\dagger \hat{A} ]
		= \left( \alpha^* - \frac{1}{2} \frac{\partial}{\partial \alpha} \right) \mathcal{W}[\hat{A}], \\
	\mathcal{W} [ \hat{A} \hat{a} ]
		& = \left( \alpha - \frac{1}{2} \frac{\partial}{\partial \alpha^*} \right) \mathcal{W}[\hat{A}],
	\quad
	\mathcal{W} [ \hat{A} \hat{a}^\dagger ]
		= \left( \alpha^* + \frac{1}{2} \frac{\partial}{\partial \alpha} \right) \mathcal{W}[\hat{A}].
\end{split}
\end{equation*}
\end{theorem}
\begin{proof}
We will prove the first correspondence.
First, let us transform the trace using~\eqnref{formalism:sm-wigner:displacement-derivatives}:
\begin{equation*}
\begin{split}
	\Trace{ \hat{a} \hat{A} \hat{D} }
	= \Trace{ \hat{A} \hat{D} \hat{a}}
	= \Trace{ \hat{A} \left(
		-\frac{\partial}{\partial \lambda^*}
		-\frac{1}{2} \lambda
	\right) \hat{D}}
	= \left(
		-\frac{\partial}{\partial \lambda^*}
		-\frac{1}{2} \lambda
	\right) \Trace{ \hat{A} \hat{D}}
\end{split}
\end{equation*}
Now we need to somehow move this additional multiplier outside the integral in the expression for Wigner function:
\begin{equation*}
\begin{split}
	\mathcal{W} [ \hat{a} \hat{A} ]
	& = \frac{1}{\pi^2} \int d^2 \lambda \exp(-\lambda \alpha^* + \lambda^* \alpha)
		\Trace{ \hat{a} \hat{A} \hat{D}(\lambda, \lambda^*) } \\
	& = \frac{1}{\pi^2} \int d^2 \lambda \exp(-\lambda \alpha^* + \lambda^* \alpha)
		\left(
			-\frac{\partial}{\partial \lambda^*}
			-\frac{1}{2} \lambda
		\right)
		\Trace{ \hat{A} \hat{D}(\lambda, \lambda^*) } \\
	& = \frac{1}{2} \frac{\partial}{\partial \alpha^*} \mathcal{W} [\hat{A}]
	- \frac{1}{\pi^2} \int d^2 \lambda \exp(-\lambda \alpha^* + \lambda^* \alpha)
		\frac{\partial}{\partial \lambda^*}
		\Trace{ \hat{A} \hat{D}(\lambda, \lambda^*) } \\
	& = \frac{1}{2} \frac{\partial}{\partial \alpha^*} \mathcal{W} [\hat{A}]
	+ \frac{1}{\pi^2} \int d^2 \lambda \left(
		\frac{\partial}{\partial \lambda^*} \exp(-\lambda \alpha^* + \lambda^* \alpha)
	\right)
	\Trace{ \hat{A} \hat{D}(\lambda, \lambda^*) } \\
	& = \left( \alpha + \frac{1}{2} \frac{\partial}{\partial \alpha^*} \right) \mathcal{W} [\hat{A}].
\end{split}
\end{equation*}
Notice that we used~\lmmref{formalism:sm-wigner:zero-integrals} to move the partial derivative over $\lambda^*$.
\end{proof}

% =============================================================================
\section{Lemmas for sets of single-mode operators}
% =============================================================================

We start from the set of single-mode operators $\hat{a}_j$, which obey bosonic commutation relations:
\begin{equation}
\label{eqn:formalism:mm-aux:commutators}
\begin{split}
	[ \hat{a}_j, \hat{a}_k ] & = [ \hat{a}_j^\dagger, \hat{a}_k^\dagger ] = 0, \\
	[ \hat{a}_j, \hat{a}_k^\dagger ] & = \delta_{jk}.
\end{split}
\end{equation}

In order to work with the moments of multimode operators we will need the equations for commutators of arbitrary single-mode operator products $[ \hat{a}_n, \hat{a}_{m_1}^\dagger \ldots \hat{a}_{m_k}^\dagger ]$ and $[ \hat{a}_n^\dagger, \hat{a}_{m_1} \ldots \hat{a}_{m_k} ]$.
Let us find the expression for the first commutator by induction.
Providing that we know the expression for $[ \hat{a}_n, \hat{a}_{m_1}^\dagger \ldots \hat{a}_{m_{k-1}}^\dagger ]$,
commutator of order $k$ can be expanded as:
\[
	[ \hat{a}_n, \hat{a}_{m_1}^\dagger \ldots \hat{a}_{m_k}^\dagger ]
	= (1 - \delta_{n m_k})
		[ \hat{a}_n, \hat{a}_{m_1}^\dagger \ldots \hat{a}_{m_{k-1}}^\dagger ] \hat{a}_{m_k}
	+ \delta_{n m_k} (
		\hat{a}_n \hat{a}_{m_1}^\dagger \ldots \hat{a}_{m_{k-1}}^\dagger \hat{a}_n^\dagger
		- \hat{a}_{m_1}^\dagger \ldots \hat{a}_{m_{k-1}}^\dagger \hat{a}_n^\dagger \hat{a}_n
	)
	= (*).
\]
Here we have split the initial commutator into two possible outcomes, depending on whether $n = m_k$.
First term, corresponding to $n \ne m_k$, contains the known commutator of lower order.
In the second term we have substituted $\hat{a}_n$ for $\hat{a}_{m_k}$,
since the delta function outside the parentheses ensures that $n = m_k$.
Swapping $\hat{a}_n^\dagger$ and $\hat{a}_n$ in the last term and, again, recognising the known commutator:
\begin{equation*}
\begin{split}
	(*)
	& = (1 - \delta_{n m_k})
		[ \hat{a}_n, \hat{a}_{m_1}^\dagger \ldots \hat{a}_{m_{k-1}}^\dagger ] \hat{a}_{m_k}
	+ \delta_{n m_k} (
		[ \hat{a}_n, \hat{a}_{m_1}^\dagger \ldots \hat{a}_{m_{k-1}}^\dagger ] \hat{a}_n^\dagger
		+ \hat{a}_{m_1}^\dagger \ldots \hat{a}_{m_{k-1}}^\dagger
	) \\
	& = [ \hat{a}_n, \hat{a}_{m_1}^\dagger \ldots \hat{a}_{m_{k-1}}^\dagger ] \hat{a}_{m_k}
	+ \delta_{n m_k} \hat{a}_{m_1}^\dagger \ldots \hat{a}_{m_{k-1}}^\dagger.
\end{split}
\end{equation*}
Now, starting from the first-order relation $[ \hat{a}_n, \hat{a}_{m_1}^\dagger ] = \delta_{n m_1}$, we can obtain the relation for any order:
\begin{equation*}
\begin{split}
	[ \hat{a}_n, \hat{a}_{m_1}^\dagger \hat{a}_{m_2}^\dagger ]
	& = \delta_{n m_1} \hat{a}_{m_2}^\dagger + \delta_{n m_2} \hat{a}_{m_1}^\dagger, \\
	[ \hat{a}_n, \hat{a}_{m_1}^\dagger \hat{a}_{m_2}^\dagger \hat{a}_{m_3}^\dagger ]
	& = \delta_{n m_1} \hat{a}_{m_2}^\dagger \hat{a}_{m_3}^\dagger
	+ \delta_{n m_2} \hat{a}_{m_1}^\dagger \hat{a}_{m_3}^\dagger
	+ \delta_{n m_3} \hat{a}_{m_1}^\dagger \hat{a}_{m_2}^\dagger, \\
	& \ldots
\end{split}
\end{equation*}
or, in generalised form:
\begin{equation}
\label{eqn:formalism:mm-aux:high-order-commutators}
	[ \hat{a}_n, \hat{a}_{m_1}^\dagger \ldots \hat{a}_{m_k}^\dagger ]
	= \sum\limits_{i=1}^k \delta_{n m_i}
		\prod\limits_{j=1,j \ne i}^k \hat{a}_{m_j}^\dagger.
\end{equation}
Note that if $n = m_1 = \ldots = m_k$, this boils down to the well-known relation from~\cite{Louisell1990}:
\[
	[ \hat{a}, (\hat{a}^\dagger)^k ] = k (\hat{a}^\dagger)^{k-1}.
\]
The general form for the second commutator can be found using the exact same procedure:
\begin{equation}
	[ \hat{a}_n^\dagger, \hat{a}_{m_1} \ldots \hat{a}_{m_k} ]
	= - \sum\limits_{i=1}^k \delta_{n m_i}
		\prod\limits_{j=1,j \ne i}^k \hat{a}_{m_j}.
\end{equation}

Let us now find the expression for high-order commutators of restricted field operators, analogous to equation~\eqnref{formalism:mm-aux:high-order-commutators} for single-mode operators.
It can be done using the similar recursive procedure.
Given that we know the expression for $\left[ \Psiop, ( \Psiop^{\prime\dagger} )^{l-1} \right]$,
the commutator of higher order can be expanded as
\begin{equation*}
\begin{split}
	\left[ \Psiop, ( \Psiop^{\prime\dagger} )^l \right]
	& = \Psiop ( \Psiop^{\prime\dagger} )^l - ( \Psiop^{\prime\dagger} )^l \Psiop \\
	& = (
		\delta_P (\xvec - \xvec^\prime) + \Psiop^{\prime\dagger} \Psiop
	) ( \Psiop^{\prime\dagger} )^{l-1}
	- ( \Psiop^{\prime\dagger} )^l \Psiop \\
	& = \delta_P (\xvec - \xvec^\prime) ( \Psiop^{\prime\dagger} )^{l-1}
	+ \Psiop^{\prime\dagger} (
		\Psiop ( \Psiop^{\prime\dagger} )^{l-1}
		- ( \Psiop^{\prime\dagger} )^{l-1} \Psiop
	) \\
	& = \delta_P (\xvec - \xvec^\prime) ( \Psiop^{\prime\dagger} )^{l-1}
	+ \Psiop^{\prime\dagger} [
		\Psiop, ( \Psiop^{\prime\dagger} )^{l-1}
	].
\end{split}
\end{equation*}
Now we can get the commutator of any order starting from the known relation~\eqnref{formalism:mm-aux:restricted-commutators}:
\[
	\left[ \Psiop, ( \Psiop^{\prime\dagger} )^l \right]
	= l \delta_P (\xvec - \xvec^\prime) ( \Psiop^{\prime\dagger} )^{l-1}.
\]
Accompanying conjugated relation:
\[
	\left[ \Psiop^\dagger, ( \Psiop^\prime )^l \right]
	= - l \delta_P^* (\xvec - \xvec^\prime) ( \Psiop^\prime )^{l-1}.
\]

A further generalisation of these relations is
\begin{equation}
\label{eqn:formalism:mm-aux:functional-commutators}
\begin{split}
	\left[ \Psiop, f( \Psiop^\prime, \Psiop^{\prime\dagger} ) \right]
	& = \delta_P (\xvec - \xvec^\prime) \frac{\partial f}{\partial \Psiop^{\prime\dagger}} \\
	\left[ \Psiop^\dagger, f( \Psiop^\prime, \Psiop^{\prime\dagger} ) \right]
	& = -\delta_P^* (\xvec - \xvec^\prime) \frac{\partial f}{\partial \Psiop^\prime},
\end{split}
\end{equation}
where $f(x, y)$ is a function that can be expanded in the power series of $x$ and $y$.
Let us prove the first relation; the procedure for the second one is the same.
Without loss of generality, we can assume that $f(\Psiop^\prime, \Psiop^{\prime\dagger})$ can be expanded in power series of normally ordered operators (otherwise we can just use commutation relations).
Thus
\begin{equation*}
\begin{split}
	\left[ \Psiop, f( \Psiop^\prime, \Psiop^{\prime\dagger} ) \right]
	& = \sum\limits_{r,s} f_{rs} [ \Psiop, (\Psiop^{\prime\dagger})^r (\Psiop^\prime)^s ] \\
	& = \sum\limits_{r,s} f_{rs} [ \Psiop, (\Psiop^{\prime\dagger})^r ] (\Psiop^\prime)^s \\
	& = \sum\limits_{r,s} f_{rs} r \delta_P(\xvec - \xvec^\prime)
		(\Psiop^{\prime\dagger})^{r-1} (\Psiop^\prime)^s \\
	& = \delta_P (\xvec - \xvec^\prime) \frac{\partial f}{\partial \Psiop^{\prime\dagger}}.
\end{split}
\end{equation*}

% =============================================================================
\section{Multimode Wigner representation}
% =============================================================================


Single-mode definition~\eqnref{formalism:sm-wigner:w-definition} of Wigner representation can be extended to the case of many modes.
Let $\bm{\lambda}$ and $\bm{\alpha}$ be vectors of $\lambda_n$ and $\alpha_n$ values respectively,
with $n = 1 \ldots N$.
Then the definitions of multimode characteristic function and Wigner function are
\[
	\chi_W (\bm{\lambda}, \bm{\lambda}^*)
	= \Trace{ \hat{\rho} \exp \sum\limits_n
		(\lambda_n \hat{a}_n^\dagger - \lambda_n^* \hat{a}_n) },
\]
\begin{equation}
\label{eqn:formalism:mm-wigner:w-definition}
	W (\bm{\alpha}, \bm{\alpha}^*)
	= \frac{1}{\pi^{2N}} \int d^2 \lambda_1 \ldots \int d^2 \lambda_N
		\left(
			\exp \sum\limits_n (-\lambda_n \alpha_n^* + \lambda_n^* \alpha_n)
		\right)
		\chi_W (\bm{\lambda}, \bm{\lambda}^*).
\end{equation}

\begin{lemma}[Multimode extension of~\lmmref{formalism:sm-wigner:moments-from-chi}]
\label{lmm:formalism:mm-wigner:moments-from-chi}
\[
	\langle \symprod{ \prod\limits_n \hat{a}_n^{r_n} (\hat{a}_n^\dagger)^{s_n} } \rangle
	= \left.
		\left(
			\prod\limits_n
			\left( \frac{\partial}{\partial \lambda_n} \right)^{s_n}
			\left( -\frac{\partial}{\partial \lambda_n^*} \right)^{r_n}
		\right)
		\chi_W (\bm{\lambda}, \bm{\lambda}^*)
	\right|_{\bm{\lambda}=0}.
\]
\end{lemma}
\begin{proof}
Mode operators with different indices commute, so
\begin{equation*}
\begin{split}
	\chi_W (\bm{\lambda}, \bm{\lambda}^*)
	& = \Trace{
		\hat{\rho}
		\prod\limits_n
			\exp( \lambda_n \hat{a}_n^\dagger - \lambda_n^* \hat{a}_n)
	} \\
	& = \Trace{
		\hat{\rho}
		\prod\limits_n \sum\limits_{r_n, s_n}
			\frac{(-\lambda_n^*)^{r_n} \lambda_n^{s_n}}{r_n! s_n!}
			\symprod{ \hat{a}_n^{r_n} (\hat{a}_n^\dagger)^{s_n}}
	} \\
	& = \Trace{
		\hat{\rho}
		\sum\limits_{r_1, s_1, \ldots, r_N, s_N} \prod\limits_n
			\frac{(-\lambda_n^*)^{r_n} \lambda_n^{s_n}}{r_n! s_n!}
			\symprod{ \hat{a}_n^{r_n} (\hat{a}_n^\dagger)^{s_n}}
	} \\
	& = \Trace{
		\sum\limits_{r_1, s_1, \ldots, r_N, s_N}
			\left(
				\prod\limits_n \frac{(-\lambda_n^*)^{r_n} \lambda_n^{s_n}}{r_n! s_n!}
			\right)
			\hat{\rho}
			\symprod{ \prod\limits_n \hat{a}_n^{r_n} (\hat{a}_n^\dagger)^{s_n}}
	} \\
	& = \sum\limits_{r_1, s_1, \ldots, r_N, s_N}
		\left(
			\prod\limits_n \frac{(-\lambda_n^*)^{r_n} \lambda_n^{s_n}}{r_n! s_n!}
		\right)
		\langle
			\symprod{ \prod\limits_n \hat{a}_n^{r_n} (\hat{a}_n^\dagger)^{s_n}}
		\rangle,
\end{split}
\end{equation*}
which is straightforwardly followed by the statement of the lemma.
\end{proof}

Moments of multimode Wigner function correspond to the averages of symmetrically ordered products in the same way as for single-mode case.
\begin{theorem}[Multimode extension of~\thmref{formalism:sm-wigner:moments}]
\[
	\int d^2\alpha_1 \ldots \int d^2\alpha_N\,
		\left(
			\prod\limits_n \alpha_n^{r_n} (\alpha_n^*)^{s_n}
		\right) W(\bm{\alpha}, \bm{\alpha}^*)
	= \langle \symprod{ \prod\limits_n \hat{a}_n^{r_n} (\hat{a}_n^\dagger)^{s_n} } \rangle.
\]
\end{theorem}
\begin{proof}
The proof is carried out similarly to~\thmref{formalism:sm-wigner:moments}:
integrals over $\alpha_n$ are eliminated one by one using integration by parts and~\lmmref{formalism:c-numbers:zero-integrals},
until we get the right part of~\lmmref{formalism:mm-wigner:moments-from-chi}.
\end{proof}

% =============================================================================
\section{Functional calculus}
% =============================================================================

Phase-space treatment of multimode problems can be simplified by working with multimode field operators instead of single-mode operators.
It was initially introduced by Graham~\cite{Graham1970,Graham1970a}.
Examples of usage can be found in ~\cite{Steel1998,Norrie2006a}.
Detailed description of functional calculus is given in~\cite{Dalton2011} \todo{anywhere else?}.
Here we only provide some important results which are going to be used later on in this chapter.
\todo{Move to bibliography review and extend.}

First we must introduce some operations on functions,
which will replace common differentials and integrals used in single and multi-mode cases and help encapsulate basis and mode populations inside wave functions and field operators.
In order to do that, we define an orthonormal basis $\phi_{\nvec}$,
where $\nvec$ is a state vector with $D$ elements.
Orthonormality and completeness conditions for basis functions are, respectively,
\[
	\int\limits_A \phi_{\nvec}^*(\xvec) \phi_{\mvec}(\xvec) d\xvec = \delta_{\nvec\mvec},
\]
\[
	\sum_{\nvec} \phi_{\nvec}^*(\xvec) \phi_{\nvec}(\xvec^\prime) = \delta(\xvec^\prime - \xvec),
\]
where the exact nature of integration area $A$ depends on the nature of the basis set
(for example, $A$ is the whole space for harmonic oscillator modes, and a box for plane waves).
Hereinafter we assume that the integration $\int d\xvec$ is always performed over $A$.

Given basis, we can define composition transformation
\[
	\mathcal{C} :: \mathbb{C}^{|L|} \rightarrow (\mathbb{R}^D \rightarrow \mathbb{C})_L
\]
\[
	\mathcal{C}(\bm{\alpha}) = \sum_{\nvec \in L} \phi_{\nvec} \alpha_{\nvec},
\]
where $L$ is some subset of the basis,
and $|L|$ is its cardinality.
Its result is a complex-valued function, which consists only of modes from $L$.
Decomposition transformation is, in turn
\[
	\mathcal{C}^{-1} :: (\mathbb{R}^D \rightarrow \mathbb{C})_L \rightarrow \mathbb{C}^{|L|}
\]
\[
	\mathcal{C}^{-1}[f]_m = \int d\xvec \phi_m^*(\xvec) f(\xvec),\,m \in L
\]
Any function can be projected to subset $L$ using the projection transformation
\[
	\mathcal{P} ::
	(\mathbb{R}^D \rightarrow \mathbb{C}) \rightarrow (\mathbb{R}^D \rightarrow \mathbb{C})_L
\]
\begin{equation}
\label{eqn:formalism:func-calculus:projector}
	\mathcal{P}[f](\xvec)
	= \sum_{\nvec \in L} \phi_{\nvec} (\xvec) \int
		d\xvec^\prime\, \phi_{\nvec}^*(\xvec^\prime) f(\xvec^\prime),
\end{equation}
If $L$ is the whole basis, then, apparently, $\mathcal{P} \equiv \mathds{1}$.

If not explicitly stated otherwise, all functions of $\xvec$ are assumed to belong to basis subset $L$ (restricted basis),
that is $\mathcal{P}[f] \equiv f$, or $f$ has type $(\mathbb{R}^D \rightarrow \mathbb{C})_L$.
Note that the result of any non-linear transformation of a function is not guaranteed to belong to $L$ and requires explicit projection to be used with other restricted functions.
This applies to the delta function which depends on coordinates.
To avoid confusion with delta function of real or complex number,
the restricted delta function is written as $\delta_P$ and defined as
\begin{equation}
\label{eqn:formalism:func-calculus:restricted-delta}
	\delta_P(\xvec^\prime - \xvec)
	= \sum_{\nvec \in L} \phi_{\nvec}^* (\xvec^\prime) \phi_{\nvec} (\xvec).
\end{equation}
Apparently, restricted delta belongs to required type $(\mathbb{R}^D \rightarrow \mathbb{C})_L$.
Note that $\delta_P$ is a Hermitian function: $\delta_P^*(\xvec^\prime - \xvec) = \delta_P(\xvec - \xvec^\prime)$.

Restricted delta function can be used to rewrite equation for $\mathcal{P}$:
\[
	\mathcal{P}[f](\xvec) = \int d\xvec^\prime \delta_P(\xvec^\prime - \xvec) f(\xvec^\prime).
\]
The conjugate of $\mathcal{P}$ is thus defined as
\[
	(\mathcal{P}[f])^*(\xvec)
	= \int d\xvec^\prime \delta_P^*(\xvec^\prime - \xvec) f^*(\xvec^\prime)
	= \mathcal{P}^* [f^*](\xvec).
\]

Let $\mathcal{F}[f] :: (\mathbb{R}^D \rightarrow \mathbb{C})_L \rightarrow (\mathbb{R}^D \rightarrow \mathbb{C})$ be some transformation
(note that the result is not guaranteed to belong to restricted basis).
Because values of types $(\mathbb{R}^D \rightarrow \mathbb{C})_L$ and $\mathbb{C}^{|L|}$ are interchangeable,
$\mathcal{F}$ can be alternatively treated as a function of a vector of complex numbers:
\[
	\mathcal{F} :: \mathbb{C}^{|L|} \rightarrow \mathbb{C}^\infty
\]
\[
	\mathcal{F}(\bm{\alpha}_f) \equiv \mathcal{C}^{-1}[\mathcal{F}[\mathcal{C}(\bm{\alpha}_f)]],
\]
where the subscript $f$ after the vector means that this is the vector which specifies function $f$.

Functional derivative is defined as
\[
	\frac{\delta}{\delta f(\xvec^\prime)} ::
	\left(
		(\mathbb{R}^D \rightarrow \mathbb{C})_L
		\rightarrow
		(\mathbb{R}^D \rightarrow \mathbb{C})
	\right)
	\rightarrow
	\left(
		(\mathbb{R}^D \rightarrow \mathbb{C})_L
		\rightarrow
		(\mathbb{R}^D \rightarrow \mathbb{R}^D \rightarrow \mathbb{C})
	\right)
\]
\begin{equation}
\label{eqn:formalism:func-aux:func-diff}
	\frac{\delta \mathcal{F}[f]}{\delta f(\xvec^\prime)}
	= \sum_{\nvec \in L} \phi_{\nvec}^* (\xvec^\prime)
		\frac{\partial \mathcal{F}(\bm{\alpha}_f)}{\partial \alpha_{f,\nvec}}.
\end{equation}
Note that the transformation being returned differs from the one which was taken:
the result of new transformation is a function depending on two variables from $\mathbb{R}^D$, not one.
The second variable comes from the function we are differentiating by.

Functional derivative definition behaves in many ways similar to common derivative.
\begin{lemma}
Functional differentiation~\eqnref{formalism:func-aux:func-diff} obeys sum, product, quotient, and chain differentiation rules.
\end{lemma}
\begin{proof}
\todo{Sum, product and quotient are more or less obvious; but should we prove chain differentiation?}
\end{proof}

\begin{lemma}
If $g(z)$ is a function that can be expanded into power series,
and functional $\mathcal{F}[f] \equiv g(f)$, then
\[
	\frac{\delta \mathcal{F}[f]}{\delta f(\xvec^\prime)} (\xvec)
	= \delta_P(\xvec^\prime - \xvec)
		\left. \frac{\partial g(z)}{\partial z} \right|_{z = f(\xvec)}
\]
\end{lemma}
\begin{proof}
We will consider $g(z) = z^k$ case first, which will straightforwardly lead to the statement of the lemma.
For $k = 1$, obviously,
\[
	\frac{\delta f}{\delta f(\xvec^\prime)} (\xvec)
	= \delta_P(\xvec^\prime - \xvec)
\]
Then for other values of $k$:
\begin{equation*}
\begin{split}
	\frac{\delta \mathcal{F}[f]}{\delta f(\xvec^\prime)} (\xvec)
	& = \frac{\delta f^k}{\delta f(\xvec^\prime)} (\xvec)
	= \sum_{\nvec \in L} \phi_{\nvec}^{\prime*}
		\frac{\partial f^k}{\partial \alpha_{\nvec}} \\
	& = \sum_{\nvec \in L} \phi_{\nvec}^{\prime*}
		\frac{\partial f^k}{\partial f}
		\frac{\partial f}{\partial \alpha_{\nvec}}
	= k f^{k-1}
		\sum_{\nvec \in L} \phi_{\nvec}^{\prime*}
		\frac{\partial f}{\partial \alpha_{\nvec}} \\
	& = k \delta_P(\xvec^\prime - \xvec) f^{k-1}(\xvec)
	= \delta_P(\xvec^\prime - \xvec)
		\left. \frac{\partial z^k}{\partial z} \right|_{z = f(\xvec)}.
	\qedhere
\end{split}
\end{equation*}
\end{proof}

\begin{lemma}
If $g(z)$ can be expanded into series of $z^n (z^*)^m$,
and functional $\mathcal{F}[f, f^*] \equiv g(f, f^*)$,
then $\delta \mathcal{F} / \delta f^\prime$ and $\delta \mathcal{F} / \delta f^{\prime*}$ can be treated as partial differentiation of the functional of two independent variables $f$ and $f^*$.
In other words:
\[
	\frac{\delta \mathcal{F}}{\delta f^\prime}
	= \delta_P(\xvec^\prime - \xvec) \left.
		\frac{\partial g(z, z^*)}{\partial z}
	\right|_{z=f(x)},
	\quad
	\frac{\delta \mathcal{F}}{\delta f^{\prime*}}
	= \delta_P^*(\xvec^\prime - \xvec) \left.
		\frac{\partial g(z, z^*)}{\partial z^*}
	\right|_{z=f^*(x)}
\]
\end{lemma}
\begin{proof}
Proof is similar to \lmmref{formalism:c-numbers:independent-vars}.
\end{proof}

Functional integration is defined as
\[
	\int \delta f ::
	(\mathbb{R}^D \rightarrow \mathbb{C})_L	\rightarrow \mathbb{C}
\]
\[
	\int \delta^2 f \mathcal{F}[f]
	= \int d^2\bm{\alpha}_f \mathcal{F}(\bm{\alpha}_f)
	= \int \ldots \int d^2\alpha_{f,1} \ldots d^2\alpha_{f,N} \mathcal{F}(\bm{\alpha}_f).
\]
If the basis contains infinite number of modes, the integral is treated as a limit $N \rightarrow \infty$ \todo{\cite{Dalton2011} has detailed explanation, do we need it here?}.

We will need delta functional:
\[
	\Delta[\Lambda]
	\equiv \prod_{\nvec \in L} \delta(\Real \lambda_{\nvec}) \delta(\Imag \lambda_{\nvec}).
\]
Note that it is really a functional and not a transformation:
it depends only on $\bm{\lambda}$, but not on coordinate.
It has the same property as common delta function:
\[
	\int \delta^2 \Lambda \mathcal{F}[\Lambda] \Delta[\Lambda]
	= \int \ldots \int d^2\lambda_1 \ldots d^2\lambda_N \mathcal{F}(\bm{\lambda})
		\prod_{\nvec \in L} \delta(\Real \lambda_{\nvec}) \delta(\Imag \lambda_{\nvec})
	= \left. \mathcal{F}(\bm{\lambda}) \right|_{\forall \nvec\, \lambda_{\nvec} = 0}
	= \left. \mathcal{F}[\Lambda] \right|_{\Lambda \equiv 0}
\]

\begin{lemma}[Functional extension of \lmmref{formalism:c-numbers:fourier-of-moments}]
\label{lmm:formalism:func-calculus:fourier-of-moments}
If $\Psi$ and $\Lambda$ are complex-valued functions of coordinate $\xvec$,
then for any non-negative integers $r$ and $s$:
\[
	\int \delta^2\Psi\, \Psi^r (\Psi^*)^s \exp
		\int d\xvec \left( -\Lambda \Psi^* + \Lambda^* \Psi \right)
	= \pi^{2N}
		\left( -\frac{\delta}{\delta \Lambda^*} \right)^r
		\left( \frac{\delta}{\delta \Lambda} \right)^s
		\Delta[\Lambda]
\]
\end{lemma}
\begin{proof}
\begin{equation*}
\begin{split}
	& \int \delta^2\Psi\, \Psi^r (\Psi^*)^s \exp
		\int d\xvec \left( -\Lambda \Psi^* + \Lambda^* \Psi \right) \\
	& = \int \ldots \int d^2\alpha_1 \ldots d^2\alpha_N
		\left( \sum_{\nvec \in L} \phi_{\nvec} \alpha_{\nvec} \right)^r
		\left( \sum_{\nvec \in L} \phi^*_{\nvec} \alpha_{\nvec}^* \right)^s
		\prod_{\nvec \in L} \exp(-\lambda_{\nvec} \alpha_{\nvec}^* + \lambda_{\nvec}^* \alpha_{\nvec}) \\
	& = \int \ldots \int d^2\alpha_1 \ldots d^2\alpha_N
		\sum_{u_1 + \ldots + u_N = r} \binom{r}{u_1, \ldots, u_N}
			\prod_{\nvec \in L} \phi_{\nvec}^{u_{\nvec}} \alpha_{\nvec}^{u_{\nvec}} \\
	&	\sum_{v_1 + \ldots + v_N = s} \binom{s}{v_1, \ldots, v_N}
			\prod_{\nvec \in L} (\phi_{\nvec}^*)^{v_{\nvec}} (\alpha_{\nvec}^*)^{v_{\nvec}}
		\prod_{\nvec \in L} \exp(-\lambda_{\nvec} \alpha_{\nvec}^* + \lambda_{\nvec}^* \alpha_{\nvec}) \\
	& = \sum_{u_1 + \ldots + u_N = r}
		\sum_{v_1 + \ldots + v_N = s}
		\binom{r}{u_1, \ldots, u_N}
		\binom{s}{v_1, \ldots, v_N}
		\prod_{\nvec \in L}
			\phi_{\nvec}^{u_{\nvec}} (\phi_{\nvec}^*){v_{\nvec}}
			\int d^2\alpha_{\nvec}
				\alpha_{\nvec}^{u_{\nvec}}
				(\alpha_{\nvec}^*)^{v_{\nvec}}
				\exp(-\lambda_{\nvec} \alpha_{\nvec}^* + \lambda_{\nvec}^* \alpha_{\nvec}) \\
	& = \sum_{u_1 + \ldots + u_N = r}
		\sum_{v_1 + \ldots + v_N = s}
		\binom{r}{u_1, \ldots, u_N}
		\binom{s}{v_1, \ldots, v_N}
		\pi^{2N}
		\prod_{\nvec \in L}
			\phi_{\nvec}^{u_{\nvec}} (\phi_{\nvec}^*)^{v_{\nvec}}
			\left( -\frac{\partial}{\partial \lambda_{\nvec}^*} \right)^{u_{\nvec}}
			\left( \frac{\partial}{\partial \lambda_{\nvec}} \right)^{v_{\nvec}}
			\delta(\Real \lambda_{\nvec}) \delta(\Imag \lambda_{\nvec}) \\
	& = \pi^{2N}
		(-\sum_{\nvec \in L} \phi_{\nvec} \frac{\partial}{\partial \lambda_{\nvec}^*})^r
		(\sum_{\nvec \in L} \phi_{\nvec}^* \frac{\partial}{\partial \lambda_{\nvec}})^s
		\prod_{\nvec \in L} \delta(\Real \lambda_{\nvec}) \delta(\Imag \lambda_{\nvec}) \\
	& = \pi^{2N}
		\left( -\frac{\delta}{\delta \Lambda^*} \right)^r
		\left( \frac{\delta}{\delta \Lambda} \right)^s
		\Delta[\Lambda]
	\qedhere
\end{split}
\end{equation*}
\end{proof}

In order to perform transformations of master equations in the future,
we will need a lemma, which justifies certain operation with Laplacian
(which is a part of kinetic term in Hamiltonian).

\begin{lemma}
\label{lmm:formalism:func-calculus:move-laplacian}
If $\forall n \in L, \xvec \in \partial A$ $\phi_n(\xvec) = 0$, then
\[
	\int\limits_A d\xvec \left(
		\nabla^2 \frac{\delta}{\delta \Psi}
	\right) \Psi \mathcal{F}[\Psi, \Psi^*]
	= \int\limits_A d\xvec \frac{\delta}{\delta \Psi}
	( \nabla^2 \Psi ) \mathcal{F}[\Psi, \Psi^*]
\]
\end{lemma}
\begin{proof}
Integration limits play an important role in this proof,
so we will write them explicitly.
\begin{equation*}
\begin{split}
	\int\limits_A d\xvec \left(
		\nabla^2 \frac{\delta}{\delta \Psi}
	\right) \Psi
	= \sum_{\nvec, \mvec} \left(
			\int\limits_A d\xvec ( \nabla^2 \phi_{\nvec}^* ) \phi_{\mvec}
		\right)
		\frac{\partial}{\partial \alpha_{\nvec}} \alpha_{\mvec} \mathcal{F}(\bm{\alpha})
	= (*)
\end{split}
\end{equation*}
Using Green's first identity and the fact that eigenfunctions are equal to zero at the boundary of $A$:
\begin{equation*}
\begin{split}
	\int\limits_A d\xvec ( \nabla^2 \phi_{\nvec}^* ) \phi_{\mvec}
	& = \oint\limits_{\partial A} \phi_{\mvec} (\nabla \phi_{\nvec}^* \cdot \bm{v}) dS
	- \int\limits_A d\xvec ( \nabla \phi_{\nvec}^* ) ( \nabla \phi_{\mvec} ) \\
	& = 0 - \int\limits_A d\xvec ( \nabla \phi_{\nvec}^* ) ( \nabla \phi_{\mvec} ) \\
	& = \oint\limits_{\partial A} \phi_{\nvec}^* (\nabla \phi_{\mvec} \cdot \bm{v}) dS
	- \int\limits_A d\xvec ( \nabla \phi_{\nvec}^* ) ( \nabla \phi_{\mvec} ) \\
	& = \int\limits_A d\xvec \phi_{\nvec}^* ( \nabla^2 \phi_{\mvec} ),
\end{split}
\end{equation*}
where $\bm{v}$ is the outward pointing unit normal of surface element $dS$.
Thus
\[
	(*)
	= \sum_{\nvec, \mvec} \left(
			\int\limits_A d\xvec \phi_{\nvec}^* ( \nabla^2 \phi_{\mvec} )
		\right)
		\frac{\partial}{\partial \alpha_{\nvec}} \alpha_{\mvec} \mathcal{F}(\bm{\alpha})
	= \int\limits_A d\xvec \frac{\delta}{\delta \Psi}
		( \nabla^2 \Psi ) \mathcal{F}[\Psi, \Psi^*].
	\qedhere
\]
\end{proof}

Note that this lemma imposes additional requirement for basis functions,
but in practical applications it is always satisfied.
For example, in plane wave basis eigenfunctions are equal to zero at the border of the bounding box,
and in harmonic oscillator basis they are equal to zero on the infinity
(which can be considered the boundary of their integration area).
Hereinafter we will assume that this condition is true for any basis we work with.

% =============================================================================
\section{Field operators and restricted basis}
% =============================================================================

Multimode fields are described by operators $\Psiop_j^{\dagger}(\xvec)$ and $\Psiop_j(\xvec)$,
where $\Psiop_j^{\dagger}(\xvec)$ creates a bosonic atom of spin $j$ at location $\xvec$,
and $\Psiop_j(\xvec)$ destroys one;
the commutators are
\begin{equation}
\label{eqn:formalism:func-aux:commutators}
	[ \Psiop_j(\xvec), \Psiop_k^{\dagger}(\xvec^\prime) ]
	= \delta_{jk} \delta(\xvec^\prime-\xvec).
\end{equation}
Field operators can be decomposed using a single-particle basis \todo{explanation needed?}:
\[
	\Psiop_j(\xvec) = \sum_{\nvec} \phi_{\nvec}(\xvec) \hat{a}_{j,\nvec}.
\]
Single mode operators $\hat{a}_{j,\nvec}$ obey commutation relations~\eqnref{formalism:mm-aux:commutators},
the pair $j,\nvec$ serving as a mode identifier.

Now suppose we want to consider only modes from some subset $L$.
Projection transformation~\eqnref{formalism:func-calculus:projector} can be extended to work on operators.
The expression remains the same, and the type becomes
\[
	\hat{\mathcal{P}} ::
	(\mathbb{R}^D \rightarrow \mathbb{H}) \rightarrow (\mathbb{R}^D \rightarrow \mathbb{H}_L),
\]
where $\mathbb{H}_L$ is the restricted subset of modes.
Being applied to the annihilation operator $\Psiop_j$, this projection returns the restricted annihilation operator
\[
	\hat{\mathcal{P}} [\Psiop_j]
	= \sum_{\nvec \in L} \phi_{\nvec} (\xvec) \hat{a}_{j,\nvec}
	= \Psiop_{jP} (\xvec),
\]
containing only modes from subset $L$.
Same as with functions, we will consider all field operators to be restricted and omit the index $P$.

Because of the restricted nature of the operator, commutation relations~\eqnref{formalism:func-aux:commutators} no longer apply.
The following ones should be used instead:
\begin{equation}
\label{eqn:formalism:func-aux:restricted-commutators}
\begin{split}
	\left[ \Psiop_j(\xvec), \Psiop_k(\xvec^\prime) \right]
	& = \left[ \Psiop_j^\dagger(\xvec), \Psiop_k^\dagger(\xvec^\prime) \right] = 0, \\
	\left[ \Psiop_j(\xvec), \Psiop_k^\dagger(\xvec^\prime) \right]
	& = \delta_{jk} \delta_P(\xvec^\prime - \xvec).
\end{split}
\end{equation}

Let us now find the expression for high-order commutators of restricted field operators, analogous to \lmmref{formalism:mm-aux:high-order-commutators} for single-mode operators.
It can be done using the similar recursive procedure.

\begin{lemma}
\begin{equation*}
\begin{split}
	\left[ \Psiop, ( \Psiop^{\prime\dagger} )^l \right]
	& = l \delta_P (\xvec^\prime - \xvec) ( \Psiop^{\prime\dagger} )^{l-1}, \\
	\left[ \Psiop^\dagger, ( \Psiop^\prime )^l \right]
	& = - l \delta_P^* (\xvec^\prime - \xvec) ( \Psiop^\prime )^{l-1}.
\end{split}
\end{equation*}
\end{lemma}
\begin{proof}
Given that we know the expression for $\left[ \Psiop, ( \Psiop^{\prime\dagger} )^{l-1} \right]$,
the commutator of higher order can be expanded as
\begin{equation*}
\begin{split}
	\left[ \Psiop, ( \Psiop^{\prime\dagger} )^l \right]
	& = \Psiop ( \Psiop^{\prime\dagger} )^l - ( \Psiop^{\prime\dagger} )^l \Psiop \\
	& = (
		\delta_P (\xvec^\prime - \xvec) + \Psiop^{\prime\dagger} \Psiop
	) ( \Psiop^{\prime\dagger} )^{l-1}
	- ( \Psiop^{\prime\dagger} )^l \Psiop \\
	& = \delta_P (\xvec^\prime - \xvec) ( \Psiop^{\prime\dagger} )^{l-1}
	+ \Psiop^{\prime\dagger} (
		\Psiop ( \Psiop^{\prime\dagger} )^{l-1}
		- ( \Psiop^{\prime\dagger} )^{l-1} \Psiop
	) \\
	& = \delta_P (\xvec^\prime - \xvec) ( \Psiop^{\prime\dagger} )^{l-1}
	+ \Psiop^{\prime\dagger} [
		\Psiop, ( \Psiop^{\prime\dagger} )^{l-1}
	].
\end{split}
\end{equation*}
Now we can get the commutator of any order starting from the known relation~\eqnref{formalism:func-aux:restricted-commutators}.
\end{proof}

A further generalisation of these relations is
\begin{lemma}
\label{lmm:formalism:func-aux:functional-commutators}
\begin{equation*}
\begin{split}
	\left[ \Psiop, f( \Psiop^\prime, \Psiop^{\prime\dagger} ) \right]
	& = \delta_P (\xvec^\prime - \xvec) \frac{\partial f}{\partial \Psiop^{\prime\dagger}} \\
	\left[ \Psiop^\dagger, f( \Psiop^\prime, \Psiop^{\prime\dagger} ) \right]
	& = -\delta_P^* (\xvec^\prime - \xvec) \frac{\partial f}{\partial \Psiop^\prime},
\end{split}
\end{equation*}
where $f(x, y)$ is a function that can be expanded in the power series of $x$ and $y$.
\end{lemma}
\begin{proof}
Let us prove the first relation; the procedure for the second one is the same.
Without loss of generality, we can assume that $f(\Psiop^\prime, \Psiop^{\prime\dagger})$ can be expanded in power series of normally ordered operators (otherwise we can just use commutation relations).
Thus
\begin{equation*}
\begin{split}
	\left[ \Psiop, f( \Psiop^\prime, \Psiop^{\prime\dagger} ) \right]
	& = \sum_{r,s} f_{rs} [ \Psiop, (\Psiop^{\prime\dagger})^r (\Psiop^\prime)^s ] \\
	& = \sum_{r,s} f_{rs} [ \Psiop, (\Psiop^{\prime\dagger})^r ] (\Psiop^\prime)^s \\
	& = \sum_{r,s} f_{rs} r \delta_P(\xvec^\prime - \xvec)
		(\Psiop^{\prime\dagger})^{r-1} (\Psiop^\prime)^s \\
	& = \delta_P (\xvec^\prime - \xvec) \frac{\partial f}{\partial \Psiop^{\prime\dagger}}.
	\qedhere
\end{split}
\end{equation*}
\end{proof}

% =============================================================================
\section{Functional Wigner representation}
% =============================================================================

First, we will define functional analogue of the displacement operator~\eqnref{formalism:sm-wigner:dispacement-op}:
\[
	\hat{D}[\Lambda, \Lambda^*] = \exp \int d\xvec \left(
		\Lambda(\xvec) \Psiop^\dagger(\xvec) - \Lambda^*(\xvec) \Psiop(\xvec)
	\right),
\]
where $\Lambda(\xvec) = \sum\limits_{\nvec \in L} \phi_n(\xvec) \lambda_n$ is some function from restricted mode space $L$.
It can be shown that the displacement operator has properties similar to~\eqnref{formalism:sm-wigner:displacement-derivatives}.

Functional Wigner transformation $\mathcal{W}$ is defined as
\begin{equation}
\label{eqn:formalism:func-wigner:w-transformation}
	\mathcal{W}[\hat{A}]
	= \frac{1}{\pi^2} \int \delta^2 \Lambda \left(
		\exp \int d\xvec (-\Lambda \Psi^* + \Lambda^* \Psi)
	\right)
	\Trace{ \hat{A} \hat{D}[\Lambda, \Lambda^*] }.
\end{equation}
It transforms an operator $\hat{A}$ on a Hilbert space to a functional $(\mathcal{W}[\hat{A}])[\Psi, \Psi^*]$.
The backward transformation (called the Weyl transformation) gives back matrix elements of the operator:
\[
	\langle \Psi \lvert \mathcal{W}^{-1}[f[\Psi, \Psi^*]] \rvert \Psi \rangle
	= \todo{find\,the\,expression}.
\]

Equivalent definition:
\begin{equation}
\label{eqn:formalism:func-wigner:w-function}
	W [\Psi, \Psi^*]
	\equiv \mathcal{W}[\hat{\rho}]
	= \frac{1}{\pi^2} \int \delta^2 \Lambda \left(
		\exp \int d\xvec (-\Lambda \Psi^* + \Lambda^* \Psi)
	\right)
	\chi_W [\Lambda, \Lambda^*],
\end{equation}
where $\chi_W (\Lambda, \Lambda^*)$ is the characteristic functional
\[
	\chi_W [\Lambda, \Lambda^*]
	= \Trace{ \hat{\rho} \hat{D}[\Lambda, \Lambda^*] }.
\]

\begin{lemma}
\begin{equation*}
\begin{split}
	\frac{\delta}{\delta \Lambda^\prime} \hat{D}[\Lambda, \Lambda^*]
	& = \hat{D}[\Lambda, \Lambda^*] (\Psiop^{\prime\dagger} + \frac{1}{2} \Lambda^{\prime*})
	= (\Psiop^{\prime\dagger} - \frac{1}{2} \Lambda^{\prime*}) \hat{D}[\Lambda, \Lambda^*], \\
	-\frac{\delta}{\delta \Lambda^{\prime*}} \hat{D}[\Lambda, \Lambda^*]
	& = \hat{D}(\Lambda, \Lambda^*) (\Psiop^\prime + \frac{1}{2} \Lambda^\prime)
	= (\Psiop^\prime - \frac{1}{2} \Lambda^\prime) \hat{D}[\Lambda, \Lambda^*].
\end{split}
\end{equation*}
\end{lemma}
\begin{proof}
We will prove the second part of the first equation.
Using Baker-Hausdorff theorem:
\begin{equation*}
\begin{split}
	\hat{D}[\Lambda, \Lambda^*]
	& = \exp \int d\xvec \Lambda(\xvec) \Psiop(\xvec)^\dagger
		\exp \left( -\int d\xvec \Lambda^*(\xvec) \Psiop(\xvec) \right)
		\exp \frac{1}{2} \left[
			\int d\xvec^\prime \Lambda(\xvec^\prime) \Psiop(\xvec^\prime)^\dagger,
			\int d\xvec \Lambda^*(\xvec) \Psiop(\xvec)
		\right] \\
	& = \exp \int d\xvec \Lambda \Psiop^\dagger
		\exp \left( -\int d\xvec \Lambda^* \Psiop \right)
		\exp \left(
			-\frac{1}{2} \int \int d\xvec d\xvec^\prime
			\Lambda^\prime \Lambda^* \delta_P(\xvec^\prime - \xvec)
		\right) \\
	& = \exp \int d\xvec \Lambda \Psiop^\dagger
		\exp \left( -\int d\xvec \Lambda^* \Psiop \right)
		\exp \left(
			-\frac{1}{2} \int d\xvec \Lambda \Lambda^*
		\right).
\end{split}
\end{equation*}
Note that, since $\Lambda$ belongs to restricted mode space, it projects to itself.
Thus
\begin{equation*}
\begin{split}
	\frac{\delta}{\delta \Lambda^\prime} \hat{D}[\Lambda, \Lambda^*]
	= \left(
		\int dx \Psiop^\dagger(\xvec) \delta_P(\xvec^\prime - \xvec)
		- \frac{1}{2} \int dx \Lambda^*(\xvec) \delta_P(\xvec^\prime - \xvec)
	\right) \hat{D}[\Lambda, \Lambda^*]
	= (\Psiop^\dagger(\xvec^\prime) - \frac{1}{2} \Lambda^*(\xvec^\prime)) \hat{D}[\Lambda, \Lambda^*]
	\qedhere
\end{split}
\end{equation*}
\end{proof}

\begin{lemma}[Functional extension of \lmmref{formalism:sm-wigner:moments-from-chi}]
\label{lmm:formalism:func-wigner:moments-from-chi}
\[
	\langle \symprod{ (\Psiop^\prime)^r (\Psiop^{\prime\dagger})^s } \rangle
	= \left.
		\left( \frac{\delta}{\delta \Lambda^\prime} \right)^s
		\left( -\frac{\delta}{\delta \Lambda^{\prime*}} \right)^r
		\chi_W (\Lambda, \Lambda^*)
	\right|_{\Lambda \equiv 0}.
\]
\end{lemma}
\begin{proof}
The proof follows the same general scheme as in single-mode case.
The exponent in the $\chi_W$ can be expanded as
\[
	\exp (\Lambda \Psiop^\dagger - \Lambda^* \Psiop)
	= \sum\limits_{r,s}
		\frac{
			\symprod{
				\left( \int d\xvec \Lambda \Psiop^\dagger \right)^r
				\left( -\int d\xvec \Lambda^* \Psiop \right)^s
			}
		}
		{r!s!}.
\]
We can swap functional derivative with both integration and multiplication by independent function, so:
\[
	\frac{\delta}{\delta \Lambda^\prime} \left( \int d\xvec \Lambda \Psiop^\dagger \right)^r
	= r \int d\xvec \frac{\delta \Lambda}{\delta \Lambda^\prime} \Psiop^\dagger
		\left( \int d\xvec \Lambda \Psiop^\dagger \right)^{r-1}
	= r \int d\xvec \delta_P(\xvec^\prime - \xvec) \Psiop^\dagger
		\left( \int d\xvec \Lambda \Psiop^\dagger \right)^{r-1}
	= r \Psiop^{\prime\dagger} \left( \int d\xvec \Lambda \Psiop^\dagger \right)^{r-1},
\]
and multiple application of the differential gives us
\[
	\left( \frac{\delta}{\delta \Lambda^\prime} \right)^r
	\left( \int d\xvec \Lambda \Psiop^\dagger \right)^r
	= r! ( \Psiop^{\prime\dagger} )^r.
\]
Similarly for the other differential:
\[
	\left( -\frac{\delta}{\delta \Lambda^{\prime*}} \right)^s
	\left( -\int d\xvec \Lambda \Psiop^\dagger \right)^s
	= s! ( \Psiop^{\prime\dagger} )^s.
\]

Thus, same as in single-mode case,
differentiation will eliminate all lower order terms in the expansion,
and all higher order terms will be eliminated by setting $\Lambda \equiv 0$,
leaving only one operator product with required order:
\[
	\left.
		\left( \frac{\delta}{\delta \Lambda^\prime} \right)^s
		\left( -\frac{\delta}{\delta \Lambda^{\prime*}} \right)^r
		\chi_W (\Lambda, \Lambda^*)
	\right|_{\Lambda \equiv 0}
	= r! s! \frac{1}{r! s!}
		\langle \symprod{ (\Psiop^\prime)^r (\Psiop^{\prime\dagger})^s } \rangle
	= \langle \symprod{ (\Psiop^\prime)^r (\Psiop^{\prime\dagger})^s } \rangle.
	\qedhere
\]
\end{proof}


\chapter{Mean-field approximation}
\label{cha:mean-field}


% =============================================================================
\section{Thomas-Fermi approximation}
% =============================================================================

We are looking for the ground state of the system with pseudopotential model hamiltonian~\cite{Pitaevskii2003}:
\[
	\hat{H} =
		- \frac{\hbar^2}{2m} \frac{\partial^2}{\partial \xvec^2}
		+ V(\xvec)
		+ g_{11} \lvert \Psi(\xvec) \rvert^2,
\]
\[
	g_{11} = \frac{4 \pi \hbar^2 a_{11}}{m},
\]
\begin{equation}
\label{eqn:mean-field:trap-potential}
	V(\xvec) = \frac{m}{2} \left(
		\omega_x^2 x^2 + \omega_y^2 y^2 + \omega_z^2 z^2
	\right).
\end{equation}
where $V(\xvec)$ is the potential energy of parabolic trap, $g_{11}$ is the interaction coefficient,
and $a_{11}$ is the scattering length for atoms in non-excited state.

The ground state satisfies the Gross-Pitaevskii equation~\cite{Pitaevskii2003}:
\begin{equation}
\label{eqn:mean-field:gs-shroedinger}
	\hat{H} \Psi = \mu \Psi,
\end{equation}
where $\mu$ is the chemical potential of the state.
To get the first approximation of the state function,
we consider the kinetic term to be small as compared to other terms and omit it.
The conditions for this operation to be valid will be determined later in this section.
Thus we get the simple equation:
\[
	\left( V(\xvec) + g_{11} \lvert \Psi(\xvec) \rvert^2 \right) \Psi(\xvec) = \mu \Psi(\xvec),
\]
which leads us to the state function:
\begin{equation}
\label{eqn:mean-field:tf-gs}
	\lvert \Psi(\xvec) \rvert^2 = \frac{1}{g_{11}} \max \left( \mu - V(\xvec), 0 \right).
\end{equation}
The condition for $V(\xvec)$ defines the shape of the condensate---it is the ellipsoid with the following radii:
\begin{equation}
\label{eqn:mean-field:tf-radii}
	r_x = \sqrt{\frac{2\mu}{m \omega_x^2}},\,
	r_y = \sqrt{\frac{2\mu}{m \omega_y^2}},\,
	r_z = \sqrt{\frac{2\mu}{m \omega_z^2}}.
\end{equation}

Normalisation condition for the ground state function gives us the connection
between the number of atoms in the condensate and the chemical potential:
\[
	\mu =
		\left( \frac{15 N}{8 \pi} \right)^\frac{2}{5}
		\left( \frac{m \bar{\omega}^2}{2} \right)^\frac{3}{5}
		{g_{11}}^\frac{2}{5},
\]
where $\bar{\omega} = \sqrt[3]{\omega_x \omega_y \omega_z}$.

Now we can roughly estimate the conditions necessary to drop the kinetic term from equation.
Substituting approximate solution~\eqnref{mean-field:tf-gs} to~\eqnref{mean-field:gs-shroedinger}
and comparing kinetic and potential term, we can get the following inequation:
\begin{equation}
\label{eqn:mean-field:tf-inequation}
	\frac{\hbar^2}{2m} \left(
		\frac{m \left( \omega_x^2 + \omega_y^2 + \omega_z^2 \right)}{2}
		+ \frac{m^2 \left( \omega_x^4 x^2 + \omega_y^4 y^2 + \omega_z^4 z^2 \right)}
			{4 \left( \mu - V(\xvec) \right)}
	\right) \ll
	\mu \left(\mu - V(\xvec)\right).
\end{equation}
Near the centre of the condensate this inequation simplifies to
\begin{equation}
\label{eqn:mean-field:tf-condition}
	\mu \gg \frac{\hbar}{2} \sqrt{\omega_x^2 + \omega_y^2 + \omega_z^2}.
\end{equation}

On the other hand, near the edges of the cloud the left-hand side of the inequation~\eqnref{mean-field:tf-inequation} diverges,
while the right-hand side equals zero there.
This means that near the edges Thomas-Fermi approximation fails regardless of the conditions.
Fortunately, the density of the particles there is low, so we can estimate the width $h$ of the "belt"
where our first approximation of the state function is significantly incorrect.
If it happens to be small as compared to the size of the condensate, the approximation can be considered valid.

The first term at the left-hand side of the inequation~\eqnref{mean-field:tf-inequation}
is constant and can be dropped in the limit of $V(\xvec) \rightarrow \mu$.
Then, for the sake of simplicity, we consider two of three coordinates to be zero and the third one to equal to $r - h$,
where $r$ is the corresponding radius of the condensate.
After replacing ``$\ll$'' by ``$\approx$'' and assuming $h$ to be small as compared to $r$,
we obtain the conditions for each coordinate:
\[
	h_x \approx \sqrt{\frac{\hbar^2}{2 \mu m}},\,\ldots
\]
They have to be much smaller than corresponding radii, which gives us:
\[
	\mu \gg \frac{1}{2} \hbar \omega_x,\ldots
\]
These conditions are less strict than the condition for the center of the condensate.
Therefore, we have only one condition justifying the application of Thomas-Fermi approximation is~\eqnref{mean-field:tf-condition}.

\begin{figure}
\begin{center}
\subfloat[100,000 atoms]{\includegraphics[width=0.5\textwidth]{%
	figures_generated/mean_field/ground_states_100k.eps}}
\subfloat[1,000 atoms]{\includegraphics[width=0.5\textwidth]{%
	figures_generated/mean_field/ground_states_1k.eps}}
\end{center}
\caption{Numerically calculated and Thomas-Fermi approximated ground states}
\label{fig:mean-field:tf-vs-accurate}
\end{figure}

Let us use some real-life experimental parameters and check how well Thomas-Fermi approximation works.
For three-dimensional trap with frequencies $f_x = f_y = 97.6 \textrm{ Hz}$ and $f_x = 11.96 \textrm{ Hz}$
and $10^5$ rubidium atoms (which have scattering length $a_{11} = 100.4 a_0$, where $a_0$ is the Bohr radius),
we have $\mu \approx 7.67 \hbar \omega_x$.
This means that Thomas-Fermi approximation produces solution which is close to the real one.
But for lower amount of atoms, say $10^3$, we get $\mu \approx 1.28 \hbar \omega_x$,
which is a sign that the we are reaching the limit of the approximation's applicability.
\figref{mean-field:tf-vs-accurate} shows the density along the z axis for both cases:
for $10^5$ atoms Thomas-Fermi approximation is very close to accurately calculated ground state (see the following section for details),
and for $10^3$ atoms it differs significantly, as expected.


% =============================================================================
\section{Ground state calculation}
% =============================================================================

The ground state obtained using Thomas-Fermi approximation is good for estimation purposes,
but not for real-life calculations---for example, it does not have continuous first derivative everywhere
(namely, near the edges of the condensate).
That is why we have to employ numerical calculations in order to find precise (to a certain extent) solution
of the Gross-Pitaevskii equation~\eqnref{mean-field:gs-shroedinger}.
One of the possible ways, the propagation in imaginary time, will be described in this section.

The idea of the method is that propagating the system state using the time-dependent GPE,
but with the substitution $t \rightarrow \tau = it$, diminishes energy of the system;
therefore after the sufficient amount of time this propagation will lead us to ground state.
The rigorous proof of this method can be found in \cite{Bao2004}, but there is a simple "hand-waving" explanation.
It assumes the superposition principle works for GPE, though it does not because of the nonlinearity.

Let us say we have the system with Hamiltonian $\hat{H}$, whose eigenvalues are $\mu_1 < \mu_2 < ...$.
They do not have to correspond to real states of BEC, we just know that this Hamiltonian must have discrete spectre
(because of the form of the potential) and the lowest eigenvalue corresponds to ground state we want to find.
The steady solution of time-dependent GPE
\[
	i \hbar \frac{\partial \psi}{\partial t} = \hat{H} \Psi
\]
then looks like
\[
	\Psi(\xvec, t) = \sum_k e^{-\frac{i}{\hbar}\mu_k t} f_k(\xvec),
\]
where $f_k$ are eigenfunctions of $\hat{H}$, corresponding to eigenvalues $\mu_k$.
Now consider the substitution $t \rightarrow \tau = it$; after it the steady solution will become fading,
with higher-energy components fading faster:
\[
	\Psi(\xvec, \tau) = \sum_k e^{-\frac{1}{\hbar}\mu_k \tau} f_k(\xvec).
\]

Therefore, if we take some random initial solution and propagate it for a sufficient amount of time,
higher-energy components will eventually die out (in comparison with the lowest-energy state)
and leave us with desired ground state.
The state obtained from Thomas-Fermi approximated GPE can be taken as an initial one,
since it is rather close to the desired one (and, therefore, higher-energy components are already quite small).

Since the energy will decrease exponentially after each step and the precision of numerical calculations is limited,
renormalisation after each step will be required.
Known total number of atoms in ground state serves best in this case
(because we will have to renormalise the final ground state anyway):
\[
	\int\limits_V \lvert \Psi(\tau, \xvec) \rvert^2 dV = N.
\]

Propagation is terminated when the total energy of the state stops changing
(that is, only one component with the lowest energy is left out).
So, we need to calculate the total energy after each step:
\[
	E(\Psi) = \int\limits_V \Psi^* \hat{H} \Psi dV
	= \int\limits_V \left(
		-\frac{\hbar^2}{2 m} \Psi^* \nabla^2 \Psi + V(\xvec) n + \frac{g_{11}}{2} n^2
	\right) dV,
\]
where $n = \lvert \Psi \rvert^2$,
and compare it to the previous value, waiting for the desired precision to be reached.

Now how do we propagate the state of the system?
There are a lot of possibilities, one of which is split-step Fourier method (see \appref{split-step}).
The propagation in imaginary time can be described as:
\[
	\frac{\partial \Psi}{\partial \tau} = - \frac{1}{\hbar} \hat{H} \Psi.
\]
Therefore, differential and nonlinear operator, necessary for split-step method, are:
\[
	\hat{D} = \frac{\hbar}{2 m}\nabla^2,\,
	\hat{N} = -\frac{1}{\hbar}\left( V(\xvec) + g_{11} \lvert \Psi(\xvec) \rvert^2 \right).
\]


% =============================================================================
\section{Two-component condensate}
% =============================================================================

Hereafter we are discussing $^{87}$Rb condensate and two of its $5^2S_{1/2}$ states: $\vert1,-1\rangle$ and $\vert2,1\rangle$,
called $\vert1\rangle$ and $\vert2\rangle$, correspondingly.
The energy of the mixture of two states is~\cite{Pitaevskii2003}:
\begin{equation}
\label{eqn:mean-field:two-comp-energy}
\begin{split}
	E(\Psi) = & \int\limits_V \left(
		- \frac{\hbar^2 \Psi_1^* \nabla^2 \Psi_1}{2m}
		- \frac{\hbar^2 \Psi_2^* \nabla^2 \Psi_2}{2m}
	\right. \\
	& \left.
		+ V_1 n_1 + ( V_2 + V_{hf} ) n_2
		+ \frac{g_{11}}{2} n_1^2 + \frac{g_{22}}{2} n_2^2 + g_{12} n_1 n_2
	\right) d\xvec.
\end{split}
\end{equation}
External trap potential $V_1$ and $V_2$ can be different for each component, depending on experimental setup.
Interaction coefficients $g_{11}$, $g_{12}$ and $g_{22}$ depend on corresponding scattering lengths:
\[
	g_{ij} = \frac{4 \pi \hbar^2 a_{ij}}{m}.
\]
The quantity $V_{hf}$ is the hyperfine splitting for $5^2S_{1/2}$ and equals $V_{hf} / h \approx 6.8 \textrm{GHz}$~\cite{Steck2009}.

One can obtain coupled GPEs from~\eqnref{mean-field:two-comp-energy} using the variational principle $i \hbar \partial \Psi_i / \partial t = \delta E / \delta \Psi_i^*$:
\begin{align}
\label{eqn:mean-field:two-comp-cgpes}
\begin{split}
	i \hbar \frac{\partial \Psi_1}{\partial t} & = \left(
		-\frac{\hbar^2 \nabla^2}{2 m} + V_1
		+ g_{11} \lvert \Psi_1 \rvert^2 + g_{12} \lvert \Psi_2 \rvert^2
	\right) \Psi_1 \\
	i \hbar \frac{\partial \Psi_2}{\partial t} & = \left(
		-\frac{\hbar^2 \nabla^2}{2 m} + V_2 + V_{hf}
		+ g_{22} \lvert \Psi_2 \rvert^2 + g_{12} \lvert \Psi_1 \rvert^2
	\right) \Psi_2
\end{split}
\end{align}

\begin{figure}
\begin{center}
\includegraphics[width=0.5\textwidth]{figures_generated/mean_field/two_comp_gs.eps}
\caption{Two-component ground state for immiscible regime.}
\label{fig:mean-field:two-comp-gs}
\end{center}
\end{figure}

Ground state for two-component condensate can be found by propagating these equations in imaginary time simultaneously,
waiting for total energy~\eqnref{mean-field:two-comp-energy} to stop changing.
\figref{mean-field:two-comp-gs} shows the axial projection of two-component ground state for a mixture of 40,000 $\vert 1 \rangle$ and 40.000 $\vert 2 \rangle$ atoms;
scattering lengths were taken to be equal to $a_{11} = 100.40\ a_0$, $a_{22} = 95.68\ a_0$ and $a_{12} = 98.13\ a_0$, where $a_0$ is the Bohr radius.


% =============================================================================
\section{Two-component evolution}
% =============================================================================

Full evolution equations can be constructed from the basic form~\eqnref{mean-field:two-comp-energy},
with the inclusion of electromagnetic coupling terms~\cite{Pitaevskii2003}
and loss terms~\cite{Mertes2007}:
\begin{align}
\label{eqn:mean-field:two-comp-evolution-cgpes}
\begin{split}
	i \hbar \frac{\partial \Psi_1}{\partial t} & = \left(
		-\frac{\hbar^2 \nabla^2}{2 m} + V_1
		+ g_{11} \lvert \Psi_1 \rvert^2
		+ g_{12} \lvert \Psi_2 \rvert^2
		- i \hbar \Gamma_1
	\right) \Psi_1 \\
	& + \left(
		\frac{\hbar \tilde{\Omega}}{2} e^{i \omega t}
		+ \textrm{c.c.}
	\right) \Psi_2, \\
	i \hbar \frac{\partial \Psi_2}{\partial t} & = \left(
		-\frac{\hbar^2 \nabla^2}{2 m} + V_2 + V_{hf}
		+ g_{22} \lvert \Psi_2 \rvert^2
		+ g_{12} \lvert \Psi_1 \rvert^2
		- i \hbar \Gamma_2
	\right) \Psi_2 \\
	& + \left(
		\frac{\hbar \tilde{\Omega}}{2} e^{i \omega t}
		+ \textrm{c.c.}
	\right) \Psi_1,
\end{split}
\end{align}
where
$\Gamma_1 = \left( \gamma_{111} n_1^2 + \gamma_{12} n_2 \right) / 2$ and
$\Gamma_2 = \left( \gamma_{12} n_1 + \gamma_{22} n_2 \right) / 2$.
Loss rates $\gamma$ can be found in~\cite{Mertes2007} and~\cite{Burt1997}:
\[
	\gamma_{111} = 5.4(11) \times 10^{-30}\ \textrm{cm}^6/\textrm{s},\,
	\gamma_{12} = 0.780(19) \times 10^{-13}\ \textrm{cm}^3/\textrm{s},\,
	\gamma_{22} = 1.194(19) \times 10^{-13}\ \textrm{cm}^3/\textrm{s}.
\]
Coupling frequency $\omega$ is slightly detuned from the hyperfine frequency in the experiment:
$\omega = \omega_{hf} + \delta,\, \delta \ll \omega_{hf}$.
Coupling coefficient $\tilde{\Omega}$ is a complex number $\Omega e^{i \alpha}$,
where $\Omega$ is the Rabi frequency (its exact value depends on the nature of coupling process),
and $\alpha$ is the phase of the coupling field.

The fact that $V_{hf} \gg V_2$ can cause problems when performing calculations with low precision.
Therefore it is convenient to use equations~\eqnref{mean-field:two-comp-evolution-cgpes}
in a slightly different coordinate system:
$\Psi_2 \rightarrow \Psi_2 e^{-i \omega_{hf} t}$.
This transformation eliminates $V_{hf}$ from the equations and does not change single-time observable values;
but one must remember that it does change relative phase of the components,
which may be significant in some cases.
Transformed equations look like:
\begin{align*}
\begin{split}
	i \hbar \frac{\partial \psi_1}{\partial t} & = \left(
		-\frac{\hbar^2 \nabla^2}{2 m} + V_1
		+ g_{11} \lvert \Psi_1 \rvert^2
		+ g_{12} \lvert \Psi_2 \rvert^2
		- i \hbar \Gamma_1
	\right) \Psi_1 \\
	& + \left(
		\frac{\hbar \tilde{\Omega}}{2} e^{i (\omega + \omega_{hf}) t}
		+ \frac{\hbar \tilde{\Omega}^*}{2} e^{-i \delta t}
	\right) \Psi_2 \\
	i \hbar \frac{\partial \Psi_2}{\partial t} & = \left(
		-\frac{\hbar^2 \nabla^2}{2 m} + V_2
		+ g_{22} \lvert \Psi_2 \rvert^2
		+ g_{12} \lvert \Psi_1 \rvert^2
		- i \hbar \Gamma_2
	\right) \Psi_2 \\
	& + \left(
		\frac{\hbar \tilde{\Omega}}{2} e^{i \delta t}
		+ \frac{\hbar \tilde{\Omega}^*}{2} e^{-i (\omega + \omega_{hf})}
	\right) \Psi_1
\end{split}
\end{align*}

In the experiment coupling field is applied for short periods of time $t_{pulse}$,
where $1 / \omega \ll t_{pulse} \ll 1 / \delta$.
This allows us to neglect fast oscillating terms:
\begin{align}
\label{eqn:mean-field:cgpes_simplified}
\begin{split}
	i \hbar \frac{\partial \Psi_1}{\partial t} & = \left(
		-\frac{\hbar^2 \nabla^2}{2 m} + V
		+ g_{11} \lvert \Psi_1 \rvert^2
		+ g_{12} \lvert \Psi_2 \rvert^2
		- i \hbar \Gamma_1
	\right) \Psi_1
	+ \frac{\hbar \Omega}{2} e^{-i \delta t - \alpha} \Psi_2, \\
	i \hbar \frac{\partial \Psi_2}{\partial t} & = \left(
		-\frac{\hbar^2 \nabla^2}{2 m} + V
		+ g_{22} \lvert \Psi_2 \rvert^2
		+ g_{12} \lvert \Psi_1 \rvert^2
		- i \hbar \Gamma_2
	\right) \Psi_2 +
	\frac{\hbar \Omega}{2} e^{i \delta t + \alpha} \Psi_1,
\end{split}
\end{align}
where $\alpha$ is the starting phase of the coupling field.
When pulse is applied twice using the same coupling field (which is the case for Ramsey interferometry),
it is the same as just setting $\Omega$ to zero after the first pulse and then restoring its value for the time of the second pulse;
therefore $\alpha$ stays the same too.
If one wants to apply pulse with the different detuning, phase information is lost,
and the value of $\alpha$ has to become random before this pulse.

Application of the coupling field can be simplified, if certain additional conditions are valid, namely:
\begin{enumerate}
	\item $\mu / \hbar \ll \Omega$, where $\mu$ is the chemical potential of the first component;
	\item $\delta \ll \Omega$;
	\item mean field interaction can be neglected \textcolor{red}{[mathematical condition needed]}.
\end{enumerate}
This allows us to use ``instantaneous'' pulse, multiplying state vector by rotation matrix:
\begin{equation}
\label{eqn:mean-field:rotation-matrix}
	\begin{pmatrix}
		\Psi^\prime_1 \\ \Psi^\prime_2
	\end{pmatrix} =
	\begin{pmatrix}
		\cos \frac{\theta}{2} & -i e^{-i \phi} \sin \frac{\theta}{2} \\
		-i e^{i \phi} \sin \frac{\theta}{2} & \cos \frac{\theta}{2}
	\end{pmatrix}
	\begin{pmatrix}
		\Psi_1 \\ \Psi_2
	\end{pmatrix},
\end{equation}
where $\theta = \Omega t_{pulse}$, and $\phi$ is the phase of the coupling field at the beginning of the pulse.
In particular, for two-pulse Ramsey scheme, $\phi_2 = \phi_1 + \delta (t_{R} + t_{pulse}) \approx \phi_1 + \delta t_{R}$.

Coupled equations~\eqnref{mean-field:cgpes_simplified} require slightly improved split-step method,
because nonlinear matrix $\hat{N}$ is no longer diagonal.
See \appref{runge-kutta} for details.

But if one uses ``instantaneous'' pulses, evolution without coupling terms can be simulated with the simple split-step method.
Differential and nonlinear operators will look as following then:
\[
	\hat{D} = \frac{i \hbar}{2m} \nabla^2,
\]
\[
	\hat{N}_1 = -\frac{i}{\hbar} \left( V + g_{11} n_1 + g_{12} n_2 \right) - \Gamma_1,
\]
\[
	\hat{N}_2 = -\frac{i}{\hbar} \left( V + g_{12} n_1 + g_{22} n_2 \right) - \Gamma_2.
\]

Gross-Pitaevskii equations give a good approximation of BEC behaviour.

\chapter{Wigner representation of BEC}
\label{cha:wigner-bec}

% =============================================================================
\section{Hamiltonian}
% =============================================================================

The second-quantized Hamiltonian of a $C$-component \abbrev{bec} in $D$ effective dimensions is expressed using quantum field operators $\Psiopf_j^{\dagger}(\xvec)$ and $\Psiopf_j(\xvec)$ defined by~\eqnref{wigner:op-calculus:field}, with the commutators~\eqnref{wigner:op-calculus:commutators}.
\begin{eqn}
\label{eqn:wigner-bec:hamiltonian:H}
	\hat{H} = \int \upd \xvec \sum_{j=1}^C \sum_{k=1}^C \left\{
		\Psiopf_j^{\dagger} K_{jk} \Psiopf_k
		+ \frac{1}{2} \int \upd \xvec^\prime
			\Psiopf_j^\dagger \Psiopf_k^{\prime\dagger}
			U_{jk}(\xvec - \xvec^\prime)
			\Psiopf_j^\prime \Psiopf_k
	\right\}.
\end{eqn}
Here we use the Einstein summation convention of summing over repeated indices.
$U_{jk}$ is the two-body scattering potential, and $K_{jk}$ is the single-particle Hamiltonian:
\begin{eqn}
	K_{jk} = \left(
			-\frac{\hbar^2}{2m} \nabla^2 + \hbar \omega_j + V_j(\xvec)
		\right) \delta_{jk}
		+ \hbar \tilde{\Omega}_{jk}(t),
\end{eqn}
where $V_j$ is the external trapping potential for spin $j$,
$\omega_j$ is the internal energy of spin $j$,
and $\tilde{\Omega}_{jk}$ represents a time-dependent coupling that is used to rotate one spin projection into another.
For details about the source and the concrete form of this term see \secref{bec-noise:mean-field}.

If we impose an energy cutoff $\ecut$ and only take into account low-energy modes of the field,
the general scattering potential $U_{jk}(\xvec - \xvec^\prime)$ can be replaced by the contact potential $U_{jk} \delta(\xvec - \xvec^\prime)$~\cite{Morgan2000}, giving the effective Hamiltonian
\begin{eqn}
\label{eqn:wigner-bec:hamiltonian:effective-H}
	\hat{H} = \int \upd \xvec \sum_{j=1}^C \sum_{k=1}^C \left\{
		\Psiop_j^{\dagger} K_{jk} \Psiop_k
		+ \frac{U_{jk}}{2} \Psiop_j^\dagger \Psiop_k^\dagger \Psiop_j \Psiop_k
	\right\},
\end{eqn}
where $\Psiopf_j^{\dagger}(\xvec)$ and $\Psiopf_j(\xvec)$ are restricted quantum field operators defined by~\eqnref{wigner:op-calculus:restricted-field}, with the commutators~\eqnref{wigner:op-calculus:restricted-commutators}.

For $s$-wave scattering in three dimensions the coefficient is $U_{jk} = 4 \pi \hbar a_{jk} / m$,
where $a_{jk}$ is the scattering length.
Note that in general case the coefficient must be renormalised depending on the grid~\cite{Kokkelmans2002,Sinatra2002}, but the change is small if $dx_{i}\gg a_{jk}$, where $dx_{i}$ is the grid step in dimension $i$.

% =============================================================================
\section{Energy cutoff}
% =============================================================================

As was noted earlier, in order to use contact interactions, an energy cutoff has to be imposed.
We use two different bases in numerical simulations, plane waves and harmonic oscillator modes (see \appref{bases} for details).
Both have analytical expressions for modes and corresponding energies, which makes the selection of the modes straightforward.

Besides being a requirement for using the contact interaction in the Hamiltonian, the energy cutoff has other important functions.
First, it allows one to check for the convergence of integration with respect to decreasing step of the spatial grid.
It effectively separates the propagation in momentum space (which remains constant) from the propagating of the nonlinear parts of the equation in coordinate space (which, hopefully, becomes more precise when the grid step is decreased).
Alternatively, the energy cutoff can work in a different direction, lowering the amount of modes under consideration while keeping the spatial grid constant.
This helps to satisfy the Wigner truncation condition (see \secref{wigner-bec:truncation} for details).

% =============================================================================
\section{Master equation}
% =============================================================================

Hereinafter field operators and wave functions will be assumed to be defined in restricted basis, unless explicitly stated otherwise.
The Markovian master equation for the system with the inclusion of losses can be written as~\cite{Jack2002}
\begin{eqn}
\label{eqn:wigner-bec:master-eqn:master-eqn}
	\frac{d\hat{\rho}}{dt} =
		- \frac{i}{\hbar} \left[ \hat{H}, \hat{\rho} \right]
		+ \sum_{\lvec} \kappa_{\lvec} \int d\xvec
			\mathcal{L}_{\lvec} \left[ \hat{\rho} \right],
\end{eqn}
where $\lvec = (l_1, l_2, \ldots, l_C)$ is a vector containing the number of particles of the corresponding components participating in the interaction, $C$ being the number of components, and we have introduced local Liouville loss terms,
\begin{eqn}
	\mathcal{L}_{\lvec} \left[ \hat{\rho} \right] =
		2\hat{O}_{\lvec} \hat{\rho} \hat{O}_{\lvec}^\dagger
		- \hat{O}_{\lvec}^\dagger \hat{O}_{\lvec} \hat{\rho}
		- \hat{\rho} \hat{O}_{\lvec}^\dagger \hat{O}_{\lvec}.
\end{eqn}
The reservoir coupling operators $\hat{O}_{\lvec}$ are the distinct $n$-fold products of local field annihilation operators, $\hat{O}_{\lvec} = \hat{O}_{\lvec} (\Psiopvec) = \prod_{c=1}^C \Psiop_c^{l_c}$, describing local $(\sum l_c)$-body collision losses.

The master equation allows us to derive an important property.

\begin{theorem}
	\begin{eqn*}
		\frac{d}{dt} \langle \Psiop_j \rangle
		= \mathcal{P}_{\restbasis_j} \left[
			\langle
				-\frac{i}{\hbar} \left(
					K_{jm} \Psiop_m
					+ U_{jm} \Psiop_m^\dagger \Psiop_m \Psiop_j
				\right)
				- \sum_{\lvec} \kappa_{\lvec}
					\frac{\partial \hat{O}_{\lvec}^\dagger}{\partial \Psiop_j^\dagger} \hat{O}_{\lvec}
			\rangle
		\right]
	\end{eqn*}
\end{theorem}
\begin{proof}
\begin{eqn}
	\frac{d}{dt} \langle \Psiop_j \rangle
	={} & \frac{d}{dt} \Trace{ \hat{\rho} \Psiop_j }
	= \Trace{ \frac{d\hat{\rho}}{dt} \Psiop_j } \\
	={} & \Trace{ -\frac{i}{\hbar} \left[ \hat{H}, \hat{\rho} \right] \Psiop_j }
	+ \sum_{\lvec} \kappa_{\lvec} \int d\xvec^\prime
		\Trace{
			\mathcal{L}_{\lvec}^\prime \left[ \hat{\rho} \right]
			\Psiop_j
		} \\
	={} & \int d\xvec^\prime \left(
		- \frac{i}{\hbar} \Trace{
			\left[
				\Psiop_l^{\prime\dagger} K_{lm}^\prime \Psiop_m^\prime,
				\hat{\rho}
			\right] \Psiop_j
		}
		- \frac{i}{2\hbar} U_{lm} \Trace{
			\left[
				\Psiop_l^{\prime\dagger} \Psiop_m^{\prime\dagger}
				\Psiop_l^\prime \Psiop_m^\prime,
				\hat{\rho}
			\right] \Psiop_j
		} \right. \\
	& \left. + \sum_{\lvec} \kappa_{\lvec}
			\Trace{
				\mathcal{L}_{\lvec}^\prime \left[ \hat{\rho} \right]
				\Psiop_j
			}
	\right),
\end{eqn}
where $K_{lm}^\prime \equiv K_{lm}(\xvec^\prime)$, $\mathcal{L}_{\lvec}^\prime \equiv \mathcal{L} [ O_{\lvec}^\prime ] \equiv \mathcal{L} [ O_{\lvec} ( \Psiopvec^\prime ) ]$.

Let us transform each term separately.
We will make extensive use of the fact that trace is invariant under cyclic permutations to re-order the operators in terms.
In addition, the transformations are based on commutation relations proved in \lmmref{op-calculus:functional-commutators}.

Noticing that $[ K_{lm}^\prime, \Psiop_j ] \equiv 0$, the first term can be transformed as:
\begin{eqn}
	\Trace{
		\left[
			\Psiop_l^{\prime\dagger} K_{lm}^\prime \Psiop_m^\prime,
			\hat{\rho}
		\right] \Psiop_j
	}
	& = \Trace{
		\Psiop_l^{\prime\dagger} K_{lm}^\prime \Psiop_m^\prime \hat{\rho} \Psiop_j
		- \hat{\rho} \Psiop_l^{\prime\dagger} K_{lm}^\prime \Psiop_m^\prime \Psiop_j
	} \\
	& = \Trace{
		\hat{\rho} \left(
			\Psiop_j \Psiop_l^{\prime\dagger} K_{lm}^\prime \Psiop_m^\prime
			- \Psiop_l^{\prime\dagger} K_{lm}^\prime \Psiop_m^\prime \Psiop_j
		\right)
	} \\
	& = \Trace{
		\hat{\rho} \left[
			\Psiop_j \Psiop_l^{\prime\dagger}
		\right] K_{lm}^\prime \Psiop_m^\prime
	} \\
	& = \Trace{
		\hat{\rho} \delta_{jl} \delta_{\restbasis_j}(\xvec^\prime - \xvec) K_{lm}^\prime \Psiop_m^\prime
	}
	= \delta_{\restbasis_j}(\xvec^\prime - \xvec) \langle K_{jm}^\prime \Psiop_m^\prime \rangle
\end{eqn}

Second (nonlinear) term:
\begin{eqn}
	U_{lm} \Trace{
		\left[
			\Psiop_l^{\prime\dagger} \Psiop_m^{\prime\dagger}
			\Psiop_l^\prime \Psiop_m^\prime,
			\hat{\rho}
		\right] \Psiop_j
	}
	& = U_{lm} \Trace{
		\Psiop_l^{\prime\dagger} \Psiop_m^{\prime\dagger}
		\Psiop_l^\prime \Psiop_m^\prime \hat{\rho} \Psiop_j
		- \hat{\rho} \Psiop_l^{\prime\dagger} \Psiop_m^{\prime\dagger}
		\Psiop_l^\prime \Psiop_m^\prime \Psiop_j
	} \\
	& = U_{lm} \Trace{
		\hat{\rho} \left(
			\Psiop_j \Psiop_l^{\prime\dagger} \Psiop_m^{\prime\dagger}
			\Psiop_l^\prime \Psiop_m^\prime
			- \Psiop_l^{\prime\dagger} \Psiop_m^{\prime\dagger}
			\Psiop_l^\prime \Psiop_m^\prime \Psiop_j
		\right)
	} \\
	& = U_{lm} \Trace{
		\hat{\rho} \left[
			\Psiop_j, \Psiop_l^{\prime\dagger} \Psiop_m^{\prime\dagger}
		\right] \Psiop_l^\prime \Psiop_m^\prime
	} \\
	& = U_{lm} \Trace{
		\hat{\rho} \delta_{\restbasis_j}(\xvec^\prime - \xvec) \left(
			\delta_{jl} \Psiop_m^{\prime\dagger}
			+ \delta_{jm} \Psiop_l^{\prime\dagger}
		\right) \Psiop_l^\prime \Psiop_m^\prime
	} \\
	& = 2 U_{jm} \delta_{\restbasis_j}(\xvec^\prime - \xvec) \Trace{
		\hat{\rho} \Psiop_m^{\prime\dagger} \Psiop_m^\prime \Psiop_j^\prime
	} \\
	& = 2 U_{jm} \delta_{\restbasis_j}(\xvec^\prime - \xvec) \langle
		\Psiop_m^{\prime\dagger} \Psiop_m^\prime \Psiop_j^\prime
	\rangle
\end{eqn}

The third term, coming from the losses, can be calculated as
\begin{eqn}
	\Trace{
		\mathcal{L}_{\lvec}^\prime \left[ \hat{\rho} \right]
		\Psiop_j
	}
	& = \Trace{
		2 \hat{O}_{\lvec}^\prime \hat{\rho} \hat{O}_{\lvec}^{\prime\dagger} \Psiop_j
		- \hat{O}_{\lvec}^{\prime\dagger} \hat{O}_{\lvec}^\prime \hat{\rho} \Psiop_j
		- \hat{\rho} \hat{O}_{\lvec}^{\prime\dagger} \hat{O}_{\lvec}^\prime \Psiop_j
	} \\
	& = \Trace{
		2 \hat{\rho} \hat{O}_{\lvec}^{\prime\dagger} \Psiop_j \hat{O}_{\lvec}^\prime
		- \hat{O}_{\lvec}^{\prime\dagger} \hat{O}_{\lvec}^\prime \hat{\rho} \Psiop_j
		- \hat{\rho} \hat{O}_{\lvec}^{\prime\dagger} \hat{O}_{\lvec}^\prime \Psiop_j
	} \\
	& = \Trace{
		\hat{\rho} \hat{O}_{\lvec}^{\prime\dagger} \hat{O}_{\lvec}^\prime \Psiop_j
		+ \hat{\rho} \hat{O}_{\lvec}^{\prime\dagger} \Psiop_j \hat{O}_{\lvec}^\prime
		- \hat{\rho} \Psiop_j \hat{O}_{\lvec}^{\prime\dagger} \hat{O}_{\lvec}^\prime
		- \hat{\rho} \hat{O}_{\lvec}^{\prime\dagger} \hat{O}_{\lvec}^\prime \Psiop_j
	} \\
	& = \Trace{
		\hat{\rho} \left[
			\hat{O}_{\lvec}^{\prime\dagger}, \Psiop_j
		\right] \hat{O}_{\lvec}^\prime
	} \\
	& = -\delta_{\restbasis_j}(\xvec^\prime - \xvec) \Trace{
		\hat{\rho} \frac{\partial \hat{O}_{\lvec}^{\prime\dagger}}{\partial \Psiop_j^{\prime\dagger}}
		\hat{O}_{\lvec}^\prime
	} \\
	& = -\delta_{\restbasis_j}(\xvec^\prime - \xvec) \langle
		\frac{\partial \hat{O}_{\lvec}^{\prime\dagger}}{\partial \Psiop_j^{\prime\dagger}}
		\hat{O}_{\lvec}^\prime
	\rangle.
\end{eqn}

Thus the full relation is
\begin{eqn}
	\frac{d}{dt} \langle \Psiop_j \rangle
	& = \int d\xvec^\prime \delta_{\restbasis_j}(\xvec^\prime - \xvec) \left(
		- \frac{i}{\hbar} \langle K_{jm}^\prime \Psiop_m^\prime \rangle
		- \frac{i U_{jm}}{\hbar} \langle
			\Psiop_m^{\prime\dagger} \Psiop_m^\prime \Psiop_j^\prime
		\rangle
		- \sum_{\lvec} \kappa_{\lvec} \langle
			\frac{\partial \hat{O}_{\lvec}^{\prime\dagger}}{\partial \Psiop_j^{\prime\dagger}}
			\hat{O}_{\lvec}^\prime
		\rangle
	\right),
\end{eqn}
which is equivalent to the statement of the theorem.
\end{proof}

% =============================================================================
\section{Wigner truncation}
\label{sec:wigner-bec:truncation}
% =============================================================================

In order to solve operator equation~\eqnref{wigner-bec:master-eqn:master-eqn} with the Hamiltonian~\eqnref{wigner-bec:hamiltonian:effective-H} numerically, we will transform it to an ordinary differential equation using the Wigner transformation from \defref{wigner:mc:w-transformation}.

Namely, the term with $K_{jk}$ is transformed using \thmref{wigner-spec:w-commutator1} and \thmref{wigner-spec:w-laplacian-commutator1} (since $K_{jk}$ is basically a sum of the Laplacian operator and functions of $\xvec$):
\begin{eqn}
	\mathcal{W} \left[ [ \int \upd\xvec \Psiop_j^\dagger K_{jk} \Psiop_k, \hat{\rho} ] \right]
	= \int \upd\xvec \left(
			- \frac{\delta}{\delta \Psi_j} K_{jk} \Psi_k
			+ \frac{\delta}{\delta \Psi_k^*} K_{jk} \Psi_j^*
		\right)
		W,
\end{eqn}
where Wigner function $W = \mathcal{W}[\hat{\rho}]$.
Nonlinear term is transformed with \thmref{wigner-spec:w-commutator2} (assuming the locality of the interaction, and $U_{kj} = U_{jk}$):
\begin{eqn}
\label{eqn:wigner-bec:truncation:full-nonlinear}
	\mathcal{W} \left[
		[
			\int \upd\xvec \frac{U_{jk}}{2}
				\Psiop_j^\dagger \Psiop_k^\dagger \Psiop_j \Psiop_k,
			\hat{\rho}
		]
	\right]
	& = \int \upd\xvec U_{jk} \left(
		\frac{\delta}{\delta \Psi_j} \left(
			- \Psi_j \Psi_k \Psi_k^*
			+ \frac{\delta_{\restbasis_k}(\xvec, \xvec)}{2} ( \delta_{jk} \Psi_k + \Psi_j )
		\right) \right. \\
	&	\left. + \frac{\delta}{\delta \Psi_j^*} \left(
			\Psi_j^* \Psi_k \Psi_k^*
			- \frac{\delta_{\restbasis_k}(\xvec, \xvec)}{2} ( \delta_{jk} \Psi_k^* + \Psi_j^* )
		\right) \right. \\
	&	\left.
			+ \frac{\delta}{\delta \Psi_j}
			\frac{\delta}{\delta \Psi_j^*}
			\frac{\delta}{\delta \Psi_k}
			\frac{1}{4} \Psi_k
			- \frac{\delta}{\delta \Psi_j}
			\frac{\delta}{\delta \Psi_j^*}
			\frac{\delta}{\delta \Psi_k^*}
			\frac{1}{4} \Psi_k^*
		\right) W.
\end{eqn}
Loss operator is transformed with \thmref{wigner-spec:w-losses} and result in a similar equation, with a finite number of differential terms up to order $2n$ for $n-$body collisional losses.

Assuming that $K_{jk}$, $U_{jk}$, and $\kappa_{\lvec}$ are real-valued, all the transformations described above result in a partial differential equation for $W$ of the form
\begin{eqn}
\label{eqn:wigner-bec:truncation:untruncated-fpe}
	\frac{\upd W}{\upd t}
	= \int \upd\xvec \left\{
		- \sum_{j=1}^{C} \frac{\fdelta}{\fdelta \Psi_j} \mathcal{A}_j
		- \sum_{j=1}^{C} \frac{\fdelta}{\fdelta \Psi_j^*} \mathcal{A}_j^*
		+ \sum_{j=1}^{C} \sum_{k=1}^{C}
			\frac{\fdelta^2}{\fdelta \Psi_j^* \fdelta \Psi_k} \mathcal{D}_{jk}
		+ \mathcal{O} \left[ \frac{\fdelta^3}{\fdelta \Psi_j^3} \right]
	\right\} W.
\end{eqn}
Terms of order higher than $2$ are produced both by the nonlinear term in the Hamiltonian, and loss terms.
Such an equation could be solved perturbatively if there were only orders up to $3$ (which means an absence of nonlinear losses)~\cite{Polkovnikov2003}, but in most cases all terms except for first- and second-order ones are truncated.
In order to justify this truncation in a consistent way, we develop an order-by-order expansion in $1/N_j$, where $N_j$ is a characteristic particle number in a physical interaction volume, and truncate terms of formal order $1/N_j^2$.
This is achieved by use of the formal definition of a scaled Wigner function $W^{\psi}$~\cite{Drummond1993}, satisfying a scaled equation in terms of dimensionless scaled fields $\psi$, with:
\begin{eqn}
	\tau & = t / t_c, \\
	\psi_{j} & = \Psi_{j}\sqrt{\ell_c / N_j}, \\
	\mathcal{A}_j^{\psi} & = t_c \sqrt{\ell_c / N_j} \mathcal{A}_j
		+ \mathcal{O} \left( 1 / N_j^2 \right), \\
	\mathcal{D}_{jk}^{\psi} & = t_c \left( \ell_c / N_j \right) \mathcal{D}_{jk}
		+ \mathcal{O} \left( 1 / N_j^2 \right).
\end{eqn}
Here $t_c$ is a characteristic interaction time and $\ell_c$ is a characteristic interaction length.
These would normally be chosen as the healing time and healing length respectively in a \abbrev{bec} calculation.
Typically the cell size is chosen as proportional to the healing length, for optimum accuracy in resolving spatial detail.
Using this expansion, a consistent order-by-order expansion in $1/N_j$ can be obtained,
of form:
\begin{eqn}
	\frac{\upd W^{\psi}}{\upd \tau}
	= \int \upd\xvec \left\{
		- \sum_{j=1}^C \frac{\fdelta}{\fdelta \psi_j} \mathcal{A}_j^{\psi}
		- \sum_{j=1}^C \frac{\fdelta}{\fdelta \psi_j^*} \mathcal{A}_j^{\psi*}
		+ \sum_{j=1}^C \sum_{k=1}^C \frac{\fdelta^2}{\fdelta \psi_j^* \fdelta \psi_k}
			\mathcal{D}_{jk}^{\psi}
		+ \mathcal{O} \left[ \frac{1}{N_j^2} \right]
	\right\} W^{\psi}.
\end{eqn}

With the assumption of the state being coherent, the simple condition for truncation~--- that is, omitting terms of the order $\mathcal{O}(1/N_j^2)$~--- can be shown to be~\cite{Sinatra2002}
\begin{eqn}
	N_j \gg |\restbasis_j|,
\end{eqn}
where $N_j$ is the total number of atoms of the component $j$.
The inclusion of the mode factor is caused by the fact that the number of additional terms increases as the number of modes increases, which may be needed to treat convergence of the method for large momentum cutoff.
We see immediately that there are subtleties involved if one wishes to include larger numbers of high-momentum modes, since this increases the mode number while leaving the numbers unchanged.
In other words, the truncation technique is inherently restricted in its ability to resolve fine spatial details in the high-momentum cutoff limit.

The $1/N_j$ is equivalent to an expansion in the inverse density, which requires the inequality~\cite{Norrie2006}
\begin{eqn}
\label{eqn:wigner-bec:truncation:delta-condition}
	\delta_{\restbasis_j}(\xvec, \xvec)
	\ll |\Psi_j|^2.
\end{eqn}
The coherency assumption does not, of course, encompass all possible states that can be produced during evolution, which means that the condition above is more of a guide than a restriction.
For certain systems the truncation was shown to work even when~\eqnref{wigner-bec:truncation:delta-condition} is violated~\cite{Ruostekoski2005}.
The validity may also depend on the simulation time~\cite{Javanainen2013}, and other physically
relevant factors.

A common example of such relevant factors is that there can be a large difference in the size of the original parameters.
To illustrate this issue, one may have a situation where $\kappa_1 \approx \kappa_2 N_j$ even though $N_j \gg 1$.
Under these conditions, it is essential to include a scaling of the parameters in calculating the formal order, so that the scaled parameters have comparable sizes.
This allows one to identify correctly which terms are negligible in a given physical problem, and which terms must be included.

In general, one can estimate the validity of truncation for the particular problem and the particular observable by calculating the quantum correction~\cite{Polkovnikov2010}.
Other techniques for estimating validity include comparison with the exact positive-P simulation method~\cite{Drummond1993}, and examining results for unphysical behaviour such as negative occupation numbers~\cite{Deuar2007}.
It is generally the case for unitary evolution that errors caused by truncation grow in time, leading to a finite time horizon for applicability, as explained in the introduction.

The use of this Wigner truncation allows us to simplify the results of \thmref{wigner-spec:w-commutator2} and \thmref{wigner-spec:w-losses}.
Wigner truncation is an expansion up to the order $1/N_j$, so during the simplification, along with the higher order derivatives, we drop all components with $\delta_{\restbasis_j}$ of order higher than $1$ in the drift terms, and of order higher than $0$ in the diffusion terms.

\begin{theorem}
Assuming the conditions for Wigner truncation are satisfied, the result of the Wigner transformation of the nonlinear term is
\begin{eqn*}
	\mathcal{W} \left[
		[
			\int \upd\xvec \frac{U_{jk}}{2}
				\Psiop_j^\dagger \Psiop_k^\dagger \Psiop_j \Psiop_k,
			\hat{\rho}
		]
	\right]
	\approx{} & \int \upd\xvec U_{jk} \left(
		\frac{\fdelta}{\fdelta \Psi_j} \left(
			- \Psi_j \Psi_k \Psi_k^*
			+ \frac{\delta_{\restbasis_k}(\xvec, \xvec)}{2} ( \delta_{jk} \Psi_k + \Psi_j )
		\right) \right. \\
	&	\left. + \frac{\fdelta}{\fdelta \Psi_j^*} \left(
			\Psi_j^* \Psi_k \Psi_k^*
			- \frac{\delta_{\restbasis_k}(\xvec, \xvec)}{2} ( \delta_{jk} \Psi_k^* + \Psi_j^* )
		\right) \right) W.
\end{eqn*}
\end{theorem}
\begin{proof}
A straigtforward result of neglecting high-order terms in~\eqnref{wigner-bec:truncation:full-nonlinear}.
\end{proof}

\begin{theorem}
Assuming the conditions for Wigner truncation are satisfied, the result of the Wigner transformation of the loss term is
\begin{eqn*}
	\mathcal{W}[\mathcal{L}_{\lvec}[\hat{\rho}]]
	\approx{} & \left(
		\sum_{m=1}^C \frac{\fdelta}{\fdelta\Psi_m^*}
		\left(
			\frac{\upp O_{\lvec}}{\upp \Psi_m} O_{\lvec}^*
			- \frac{1}{2} \sum_{n=1}^{C} \delta_{\restbasis_n}(\xvec, \xvec)
				\frac{\upp^2 O_{\lvec}}{\upp\Psi_m \upp\Psi_n}
				\frac{\upp O_{\lvec}^*}{\upp\Psi_n^*}
		\right)
	\right. \\
	& + \sum_{m=1}^C \frac{\fdelta}{\fdelta\Psi_m}
	\left(
		\frac{\upp O_{\lvec}^*}{\upp \Psi_m^*} O_{\lvec}
		- \frac{1}{2}\sum_{n=1}^C \delta_{\restbasis_n}(\xvec, \xvec)
			\frac{\upp^2 O_{\lvec}^*}{\upp\Psi_m^* \upp\Psi_n^*}
			\frac{\upp O_{\lvec}}{\upp\Psi_n}
	\right) \\
	& \left. + \sum_{m=1}^C \sum_{n=1}^C
		\frac{\fdelta^2}{\fdelta\Psi_m^* \fdelta\Psi_n}
		\frac{\upp O_{\lvec}}{\upp\Psi_m} \frac{\upp O_{\lvec}^*}{\upp\Psi_n^*}
	\right) W,
	\end{eqn*}
where coupling functionals $O_{\lvec} \equiv O_{\lvec}[\Psivec] = \prod_{c=1}^C \Psi_c^{l_c}$.
\end{theorem}
\begin{proof}
The proof is basically a simplification of the result of \thmref{wigner-spec:w-losses} under two conditions following from neglecting the terms smaller than $1 / N$.
First, we are dropping all terms with high order differentials, which can be expressed as limiting $\sum j_c + \sum k_c \le 2$.
Second, we are only considering the terms with the restricted delta function of up to first order in the drift part (containing the functional differentials of order $1$), and terms with no restricted delta functions in the diffusion part (containing the functional differentials of order $2$).

The only combinations of $j_c$ and $k_c$ for which $Z_{\lvec,\jvec,\kvec}$ is not zero are thus $\{ j_c = \delta_{cm}, k_c = 0, m \in [1, C] \}$, $\{ j_c = 0, k_c = \delta_{cm}, m \in [1, C] \}$ and $\{ j_c = \delta_{cm}, k_c = \delta_{cn}, m \in [1, C], n \in [1, C] \}$.
These combinations produce terms with $\frac{\delta}{\delta \Psi_n^*}$, $\frac{\delta}{\delta \Psi_n}$ and $\frac{\delta^2}{\delta \Psi_p \delta \Psi_n^*}$ respectively:
\begin{eqn}
\label{eqn:wigner-bec:truncation:truncated-losses}
	\mathcal{W}[\mathcal{L}_{\lvec}[\hat{\rho}]]
	\approx{} & \left(
		\sum_{m=1}^C \frac{\fdelta}{\fdelta \Psi_m^*} H[l_m - 1] Z_{\lvec, \evec_m, \mathbf{0}}
		+ \sum_{m=1}^C \frac{\fdelta}{\fdelta \Psi_m} H[l_m - 1] Z_{\lvec, \mathbf{0}, \evec_m}
	\right. \\
	& \left. + \sum_{m=1}^C \sum_{n=1}^C \frac{\fdelta^2}{\fdelta \Psi_m^* \fdelta \Psi_n^*}
			H[l_m - 1] H[l_n - 1] Z_{\lvec, \evec_m, \evec_n}
	\right) W,
\end{eqn}
where $\evec_n$ is a vector consisting of zeros and a single $1$ at the $n$-th position, and $H[n]$ is the discrete Heavyside function.

Evaluating the partially parametrized $Z$ function in the first term using the expression in \thmref{wigner-spec:w-losses}:
\begin{eqn}
	Z_{\lvec, \evec_m, \mathbf{0}}
	= l_m \prod_{c=1}^C
		\exp \left(
			-\frac{\delta_{\restbasis_c}(\xvec, \xvec)}{2}
			\frac{\upp^2}{\upp \Psi_c \upp \Psi_c^*}
		\right)
		\Psi_c^{l_c - \delta_{cm}} (\Psi_c^*)^{l_c}.
\end{eqn}
Expanding the exponent in series and discarding the terms with more than one restricted delta function:
\begin{eqn}
	Z_{\lvec, \evec_m, \mathbf{0}}
	& \approx l_m \left(
		\prod_{c=1}^C \Psi_c^{l_c - \delta_{cm}} (\Psi_c^*)^{l_c}
		- \frac{1}{2} \sum_{n=1}^C
			\delta_{\restbasis_n}(\xvec, \xvec)
			\frac{\upp^2}{\upp \Psi_n \upp \Psi_p^*}
			\prod_{c=1}^C
				\Psi_c^{l_c - \delta_{cm}} (\Psi_c^*)^{l_c}
	\right).
\end{eqn}
Multiplied by the Heavyside function, this can be expressed using derivatives of the coupling functional $O_{\lvec}$:
\begin{eqn}
	H[l_m - 1] Z_{\lvec, \evec_m, \mathbf{0}}
	\approx \frac{\upp O_{\lvec}}{\upp \Psi_m} O_{\lvec}^*
		- \frac{1}{2} \sum_{n=1}^C
			\delta_{\restbasis_n}(\xvec, \xvec)
			\frac{\upp^2 O_{\lvec}}{\upp \Psi_m \upp \Psi_n}
			\frac{\upp O_{\lvec}^*}{\upp \Psi_n^*}.
\end{eqn}

Analogously, for the second term in~\eqnref{wigner-bec:truncation:truncated-losses} we get
\begin{eqn}
	H[l_m - 1] Z_{\lvec, \mathbf{0}, \evec_m}
	\approx \frac{\upp O_{\lvec}^*}{\upp \Psi_m^*} O_{\lvec}
		- \frac{1}{2} \sum_{n=1}^C
			\delta_{\restbasis_n}(\xvec, \xvec)
			\frac{\upp^2 O_{\lvec}^*}{\upp \Psi_m^* \upp \Psi_n^*}
			\frac{\upp O_{\lvec}}{\upp \Psi_n},
\end{eqn}
and for the third term (discarding all terms with the restricted delta function in the expansion of the exponent):
\begin{eqn}
	H[l_m - 1] H[l_n - 1] Z_{\lvec, \evec_m, \evec_n}
	\approx \frac{\upp O_{\lvec}}{\upp\Psi_m} \frac{\upp O_{\lvec}^*}{\upp\Psi_n^*}.
\end{eqn}
Substituting these expressions back into~\eqnref{wigner-bec:truncation:truncated-losses}, we get the statement of the theorem.
\end{proof}

% =============================================================================
\section{Fokker-Planck equation}
% =============================================================================

The general approach to numerical solution of the Fokker-Planck equation~\eqnref{wigner-bec:truncation:fpe} is to transform it to the equivalent set of stochastic differential equations (SDEs) for $\Psi_j$.
Since the transformation is defined for real-valued variables only \todo{citation needed}, we have to modify the equation.

First, noticing that $K_{jk}$, $U_{jk}$ and $\kappa_{\lvec}$ are real-valued (which is important for the further transformations), we can rewrite equation~\eqnref{wigner-bec:truncation:fpe} as
\begin{eqn}
	\frac{dW}{dt}
	= \int d\xvec \left(
		- \sum_{j=1}^C \frac{\delta}{\delta \Psi_j} A_j
		- \sum_{j=1}^C \frac{\delta}{\delta \Psi_j^*} A_j^*
		+ \sum_{j=1}^C \sum_{k=1}^C \frac{\delta^2}{\delta \Psi_j^* \delta \Psi_k} D_{jk}
	\right) W,
\end{eqn}
where
\begin{eqn}
	A_j = -\frac{i}{\hbar} \left(
			\sum_{k=1}^C K_{jk} \Psi_k
			+ \sum_{k=1}^C U_{jk} \Psi_j \Psi_k \Psi_k^*
		\right)
		- \sum_{\lvec} \kappa_{\lvec} \frac{\partial O_{\lvec}^*}{\partial \Psi_j^*} O_{\lvec},
\end{eqn}
and
\begin{eqn}
	D_{jk} = \sum_{\lvec} \kappa_{\lvec}
		\frac{\partial O_{\lvec}}{\partial \Psi_j}
		\frac{\partial O_{\lvec}^*}{\partial \Psi_k^*}.
\end{eqn}
Considering $\Psi_j = \sum_{\nvec \in \restbasis_j} \phi_{j,\nvec} \alpha_{j,\nvec}$ and replacing functional derivatives with ordinary ones:
\begin{eqn}
	\frac{dW}{dt}
	={} & \left(
		- \sum_{j=1}^C \sum_{\nvec \in L}
			\frac{\partial}{\partial \alpha_{j,\nvec}}
			\int d\xvec \phi_{j,\nvec}^* A_j
		- \sum_{j=1}^C \sum_{\nvec \in L}
			\frac{\partial}{\partial \alpha_{j,\nvec}^*}
			\int d\xvec \phi_{j,\nvec} A_j^* \right. \\
	&	\left. + \sum_{j=1}^C \sum_{k=1}^C
			\sum_{\mvec \in \restbasis_j,\nvec \in \restbasis_k}
			\frac{\partial}{\partial \alpha_{j,\mvec}^*}
			\frac{\partial}{\partial \alpha_{k,\nvec}}
			\int d\xvec
			\phi_{j,\mvec} \phi_{k,\nvec}^* D_{jk}
	\right) W.
\end{eqn}

\begin{lemma}[FPE--SDEs correspondence in convenient form.]
\label{lmm:wigner-bec:fpe:fpe-sde-real}
	If $\zvec^T \equiv (z_1 \ldots z_M)$ is a set of real variables, Fokker-Planck equation
	\begin{eqn*}
		\frac{dW}{dt}
		= -\bpartial_{\zvec}^T \avec W
		+ \frac{1}{2} \Trace{ \bpartial_{\zvec} \bpartial_{\zvec}^T B B^T } W
	\end{eqn*}
	is equivalent to a set of stochastic differential equations in It\^{o} form
	\begin{eqn*}
		d\zvec = \avec dt + B d\Zvec
	\end{eqn*}
	and to a set of stochastic differential equations in Stratonovich form
	\begin{eqn*}
		d\zvec = (\avec - \svec)dt + B d\Zvec,
	\end{eqn*}
	where the noise-induced (or spurious) drift vector $\svec$ has elements
	\begin{eqn*}
		s_i
		= \sum_{k,j} B_{kj} \frac{\partial}{\partial z_k} B_{ij}
		= \Trace{B^T \bpartial_z \evec_i^T B},
	\end{eqn*}
	$\evec_i$ being the unit vector with elements $(\evec_i)_j = \delta_{ij}$.
	\todo{Is there a better way to express $\svec$ in terms of matrices?}
	Here $W \equiv W(\zvec)$ is a probability distribution, $\avec \equiv \avec(\zvec)$ is a vector function, $B \equiv B(\zvec)$ is a matrix function ($B$ having size $M \times L$, where $L$ corresponds to the number of noise sources), $\bpartial_{\zvec}^T \equiv (\partial_{z_1} \ldots \partial_{z_M})$ is a vector differential, and $\Zvec$ is a standard $L$-dimensional Wiener process.
\end{lemma}
\begin{proof}
For details see~\cite{Risken1996}, sections 3.3 and 3.4.
\todo{Consider the case of colored noise (\cite{Risken1996}, 3.1 and appendix A).}
\end{proof}

\begin{theorem}
\label{thm:wigner-bec:fpe:fpe-sde-complex}
	If $\balpha^T \equiv (\alpha_1 \ldots \alpha_M)$ is a set of complex variables,
	Fokker-Planck equation
	\begin{eqn*}
		\frac{dW}{dt}
		= -\bpartial_{\balpha}^T \avec W - \bpartial_{\balpha^*}^T \avec^* W
		+ \Trace{ \bpartial_{\balpha^*} \bpartial_{\balpha}^T B B^H } W
	\end{eqn*}
	is equivalent to a set of stochastic differential equations in It\^{o} form
	\begin{eqn*}
		d\balpha = \avec dt + B d\Zvec,
	\end{eqn*}
	or to Stratonovich form
	\begin{eqn*}
		d\balpha = (\avec - \svec) dt + B d\Zvec,
	\end{eqn*}
	where noise-induced drift term is
	\begin{eqn*}
		s_j = \Trace{ B^H \bpartial_{\balpha^*} \evec_j^T B },
	\end{eqn*}
	and $\Zvec = (\bm{X} + i\bm{Y}) / \sqrt{2}$ is an $L$-dimensional complex-valued Wiener process,
	containing two standard $L$-dimensional Wiener processes $\bm{X}$ and $\bm{Y}$.
\end{theorem}
\begin{proof}
Let us expand the FPE using real values $\balpha = \bm{x} + i \bm{y}$, $\avec = \bm{u} + i \bm{v}$, $B = F + iG$, $\bpartial_{\balpha} = (\bpartial_{\bm{x}} - i \bpartial_{\bm{y}}) / 2$.
Thus
\begin{eqn}
	\frac{dW}{dt}
	={} & - \bpartial_{\bm{x}}^T \bm{u} W
	- \bpartial_{\bm{y}}^T \bm{v} W
	+ \frac{1}{4} \Trace{
		(\bpartial_{\bm{x}} \bpartial_{\bm{x}}^T
			+ \bpartial_{\bm{y}} \bpartial_{\bm{y}}^T)
		(F F^T + G G^T) \right. \\
	& \left. - (\bpartial_{\bm{x}} \bpartial_{\bm{y}}^T
			- \bpartial_{\bm{y}} \bpartial_{\bm{x}}^T)
		(F G^T - G F^T)
	} W \\
	& + \frac{i}{4} \Trace{
		(\bpartial_{\bm{x}} \bpartial_{\bm{x}}^T
			+ \bpartial_{\bm{y}} \bpartial_{\bm{y}}^T)
		(F G^T - G F^T)
	} W \\
	& + \frac{i}{4} \Trace{
		(\bpartial_{\bm{x}} \bpartial_{\bm{y}}^T
			- \bpartial_{\bm{y}} \bpartial_{\bm{x}}^T)
		(F F^T + G G^T)
	} W.
\end{eqn}
Since $F F^T + G G^T$ and $\bpartial_{\bm{x}} \bpartial_{\bm{x}}^T + \bpartial_{\bm{y}} \bpartial_{\bm{y}}^T$ are symmetric matrices, and $F G^T - G F^T$ and $\bpartial_{\bm{x}} \bpartial_{\bm{y}}^T - \bpartial_{\bm{y}} \bpartial_{\bm{x}}^T$ are antisymmetric, corresponding traces are equal to zero, which gives us FPE in real variables
\begin{eqn}
	\frac{dW}{dt}
	={} & - \bpartial_{\bm{x}}^T \bm{u} W
	- \bpartial_{\bm{y}}^T \bm{v} W
	+ \frac{1}{4} \Trace{
		(\bpartial_{\bm{x}} \bpartial_{\bm{x}}^T
			+ \bpartial_{\bm{y}} \bpartial_{\bm{y}}^T)
		(F F^T + G G^T) \right. \\
	& \left. - (\bpartial_{\bm{x}} \bpartial_{\bm{y}}^T
			- \bpartial_{\bm{y}} \bpartial_{\bm{x}}^T)
		(F G^T - G F^T)
	} W.
\end{eqn}

In order to use \lmmref{wigner-bec:fpe:fpe-sde-real},
we need to join variables $\bm{x}$ and $\bm{y}$ into the one variable vector $\zvec^T \equiv \bm{x}^T \oplus \bm{y}^T$.
This will give us an equation identical to one from the lemma, with the drift vector $\tilde{\avec}^T \equiv \bm{u}^T \oplus \bm{v}^T$ and the diffusion matrix
\begin{eqn}
	\tilde{B} \tilde{B}^T \equiv \frac{1}{2} \begin{pmatrix}
		F F^T + G G^T & F G^T - G F^T \\
		G F^T - F G^T & F F^T + G G^T
	\end{pmatrix},
\end{eqn}
which gives a noise matrix
\begin{eqn}
	\tilde{B} = \frac{1}{\sqrt{2}} \begin{pmatrix}
		F & -G \\
		G & F
	\end{pmatrix}.
\end{eqn}
Therefore the equivalent SDEs in It\^{o} form are
\begin{eqn}
	d\zvec = \tilde{\avec} dt + \tilde{B} d\tilde{\Zvec},
\end{eqn}
where $d\tilde{\Zvec}^T \equiv d\bm{X}^T \oplus d\bm{Y}^T$.
Returning to our previous variables:
\begin{eqn}
	d\bm{x} & = \bm{u} dt + \frac{1}{\sqrt{2}} F d\bm{X} - \frac{1}{\sqrt{2}} G d\bm{Y} \\
	d\bm{y} & = \bm{v} dt + \frac{1}{\sqrt{2}} G d\bm{X} + \frac{1}{\sqrt{2}} F d\bm{Y}.
\end{eqn}
Multiplying the second equation by $i$ and adding it to the first one:
\begin{eqn}
	d\balpha = \avec dt + \frac{1}{\sqrt{2}} (F + iG) (d\bm{X} + id\bm{Y}),
\end{eqn}
which leads to the It\^{o} part of the lemma statement.
\begin{eqn}
	d\balpha = \avec dt + B d\Zvec.
\end{eqn}

Noise-induced drift term in Stratonovich case can be calculated as
\begin{eqn}
	s_j^{(x)}
	& = \frac{1}{2} \Trace{
		\begin{pmatrix}
			F^T & G^T \\ -G^T & F^T
		\end{pmatrix}
		\begin{pmatrix}
			\bpartial_{\bm{x}} \\
			\bpartial_{\bm{y}}
		\end{pmatrix}
		\begin{pmatrix}
			\evec_j^T & 0
		\end{pmatrix}
		\begin{pmatrix}
			F & -G \\ G & F
		\end{pmatrix}
	} \\
	& = \frac{1}{2} \Trace{
		\begin{pmatrix}
			F^T & G^T \\ -G^T & F^T
		\end{pmatrix}
		\begin{pmatrix}
			\bpartial_{\bm{x}} \\
			\bpartial_{\bm{y}}
		\end{pmatrix}
		\begin{pmatrix}
			\evec_j^T F & - \evec_j^T G
		\end{pmatrix}
	} \\
	& = \frac{1}{2} \Trace{
		\begin{pmatrix}
			F^T & G^T \\ -G^T & F^T
		\end{pmatrix}
		\begin{pmatrix}
			\bpartial_{\bm{x}} \evec_j^T F & - \bpartial_{\bm{x}} \evec_j^T G \\
			\bpartial_{\bm{y}} \evec_j^T F & - \bpartial_{\bm{y}} \evec_j^T G
		\end{pmatrix}
	} \\
	& = \frac{1}{2} \left(
		\Trace{ F^T \bpartial_{\bm{x}} \evec_j^T F }
		+ \Trace{ G^T \bpartial_{\bm{y}} \evec_j^T F }
		+ \Trace{ G^T \bpartial_{\bm{x}} \evec_j^T G }
		- \Trace{ F^T \bpartial_{\bm{y}} \evec_j^T G }
	\right).
\end{eqn}
Similarly,
\begin{eqn}
	s_j^{(y)}
	= \frac{1}{2} \left(
		\Trace{ F^T \bpartial_{\bm{x}} \evec_j^T G }
		+ \Trace{ G^T \bpartial_{\bm{y}} \evec_j^T G }
		- \Trace{ G^T \bpartial_{\bm{x}} \evec_j^T F }
		+ \Trace{ F^T \bpartial_{\bm{y}} \evec_j^T F }
	\right).
\end{eqn}
Therefore the final term in complex-valued SDEs is
\begin{eqn}
	s_j
	= s_j^{(x)} + i s_j^{(y)}
	= \Trace{ B^H \bpartial_{\balpha^*} \evec_j^T B }.
	\qedhere
\end{eqn}
\end{proof}

\begin{theorem}[Multi-component reformulation of \thmref{wigner-bec:fpe:fpe-sde-complex}]
\label{thm:wigner-bec:fpe:mc-fpe-sde}
	If $\balpha^{(c)},\, c = 1..C$ are $C$ sets of complex variables $\balpha^{(c)} \equiv (\alpha_1^{(c)} \ldots \alpha_{M_c}^{(c)})$,
	then Fokker-Planck equation
	\begin{eqn*}
		\frac{dW}{dt}
		= - \sum_{c=1}^C \bpartial_{\balpha^{(c)}}^T \avec^{(c)} W
		- \sum_{c=1}^C \bpartial_{(\balpha^{(c)})^*}^T (\avec^{(c)})^* W
		+ \sum_{m=1}^C \sum_{n=1}^C
			\Trace{
				\bpartial_{\balpha^{(m)}}
				\bpartial_{(\balpha^{(n)})^*}^T
				B^{(n)} (B^{(m)})^H
			} W
	\end{eqn*}
	is equivalent to a set of stochastic differential equations in It\^{o} form
	\begin{eqn*}
		d\balpha^{(c)} = \avec^{(c)} dt + B^{(c)} d\Zvec,\, c = 1..C
	\end{eqn*}
	or to Stratonovich form
	\begin{eqn*}
		d\balpha^{(c)} = (\avec^{(c)} - \svec^{(c)}) dt + B^{(c)} d\Zvec,
	\end{eqn*}
	where noise-induced drift term is
	\begin{eqn*}
		s_j^{(c)} = \sum_{n=1}^C
			\Trace{ (B^{(n)})^H \bpartial_{(\balpha^{(n)})^*} \evec_j^T B^{(c)} },
	\end{eqn*}
	and $d\Zvec$ is an $L$-dimensional complex-valued Wiener process.
\end{theorem}
\begin{proof}
Let us join all variable sets $\balpha^{(c)}$ into one set
\begin{eqn}
	\balpha \equiv \bigoplus_{c=1}^C \balpha^{(c)}.
\end{eqn}
Then we can use \thmref{wigner-bec:fpe:fpe-sde-complex} with drift vector
\begin{eqn}
	\avec = \bigoplus_{c=1}^C \avec^{(c)},
\end{eqn}
differentials vector
\begin{eqn}
	\bpartial_{\balpha} = \bigoplus_{c=1}^C \bpartial_{\balpha^{(c)}},
\end{eqn}
and noise matrix
\begin{eqn}
	B = \begin{pmatrix}
		B^{(1)} \\ \vdots \\ B^{(C)}
	\end{pmatrix}.
\end{eqn}
This gives us SDEs in It\^{o} form
\begin{eqn}
	d\balpha = \avec dt + B d\Zvec,
\end{eqn}
where $d\Zvec$ is an $L$-dimensional complex-valued Wiener process.
Splitting this equation for different components, we get the statement of the lemma.
Stratonovich variant is obtained in the same way.
Noise-induced drift term requires some work:
\begin{eqn}
	s_j^{(c)}
	& = \Trace{
		\begin{pmatrix} (B^{(1)})^H & \cdots & (B^{(C)})^H \end{pmatrix}
		\begin{pmatrix}
			\bpartial_{(\balpha^{(1)})^*} \\
			\vdots \\
			\bpartial_{(\balpha^{(C)})^*}
		\end{pmatrix}
		\begin{pmatrix} 0 & \cdots & \evec_j^T & \cdots & 0 \end{pmatrix}
		\begin{pmatrix}
			B^{(1)} \\
			\vdots \\
			B^{(C)}
		\end{pmatrix}
	} \\
	& = \Trace{
		\begin{pmatrix} (B^{(1)})^H & \cdots & (B^{(C)})^H \end{pmatrix}
		\begin{pmatrix}
			\bpartial_{(\balpha^{(1)})^*} \evec_j^T B^{(c)} \\
			\vdots \\
			\bpartial_{(\balpha^{(C)})^*} \evec_j^T B^{(c)}
		\end{pmatrix}
	} \\
	& = \sum_{n=1}^C \Trace{
		(B^{(n)})^H
		\bpartial_{(\balpha^{(n)})^*}
		\evec_j^T
		B^{(c)}
	}.
\end{eqn}
\end{proof}

\begin{theorem}
	Functional FPE
	\begin{eqn*}
		\frac{dW}{dt}
		= \int d\xvec \left(
			- \sum_{j=1}^C \frac{\delta}{\delta \Psi_j} \mathcal{A}^{(j)}
			- \sum_{j=1}^C \frac{\delta}{\delta \Psi_j^*} (\mathcal{A}^{(j)})^*
			+ \sum_{j=1}^C \sum_{k=1}^C \frac{\delta^2}{\delta \Psi_j \delta \Psi_k^*}
				\sum_{\lvec} \mathcal{B}_{\lvec}^{(k)} (\mathcal{B}_{\lvec}^{(j)})^*
		\right) W
	\end{eqn*}
	is equivalent to the set of SDEs in It\^{o} form
	\begin{eqn*}
		d\Psi_j = \mathcal{P}_{\restbasis_j} \left[
			\mathcal{A}^{(j)} dt + \sum_{\lvec} \mathcal{B}_{\lvec}^{(j)} dQ_{\lvec}
		\right],
	\end{eqn*}
	or in Stratonovich form
	\begin{eqn*}
		d\Psi_j = \mathcal{P}_{\restbasis_j} \left[
			(\mathcal{A}^{(j)} - \mathcal{S}^{(j)}) dt + \sum_{\lvec} \mathcal{B}_{\lvec}^{(j)} dQ_{\lvec}
		\right],
	\end{eqn*}
	where
	\begin{eqn*}
		\mathcal{S}^{(j)} = \sum_{n=1}^C \sum_{\lvec}
			(\mathcal{B}_{\lvec}^{(n)})^*
			\frac{\delta}{\delta \Psi_n^*}
			\mathcal{B}_{\lvec}^{(j)},
	\end{eqn*}
	and $Q_{\lvec}$ is a functional Wiener process:
	\begin{eqn*}
		Q_{\lvec} = \sum_{\nvec \in \fullbasis} \phi_{\nvec} Z_{\lvec,\nvec}.
	\end{eqn*}
\end{theorem}
\begin{proof}
Considering $\Psi_j = \sum_{\nvec \in \restbasis_j} \phi_{j,\nvec} \alpha_{j,\nvec}$ and replacing functional derivatives with ordinary ones:
\begin{eqn}
	\frac{dW}{dt}
	={} & \left(
		- \sum_{j=1}^C \sum_{\nvec \in \restbasis_j}
			\frac{\partial}{\partial \alpha_{j,\nvec}}
			\int d\xvec \phi_{j,\nvec}^* \mathcal{A}^{(j)}
		- \sum_{j=1}^C \sum_{\nvec \in \restbasis_j}
			\frac{\partial}{\partial \alpha_{j,\nvec}^*}
			\int d\xvec \phi_{j,\nvec} (\mathcal{A}^{(j)})^*
		\right. \\
	&	\left. + \sum_{j=1}^C \sum_{k=1}^C
			\sum_{\mvec \in \restbasis_j, \nvec \in \restbasis_k}
			\frac{\partial}{\partial \alpha_{j,\mvec}}
			\frac{\partial}{\partial \alpha_{k,\nvec}^*}
			\int d\xvec
			\phi_{k,\nvec} \phi_{j,\mvec}^*
			\sum_{\lvec} \mathcal{B}_{\lvec}^{(k)} (\mathcal{B}_{\lvec}^{(j)})^*
	\right) W.
\end{eqn}
The diffusion term has to be transformed in order to conform to \thmref{wigner-bec:fpe:mc-fpe-sde}:
\begin{eqn}
	\int d\xvec \phi_{k,\nvec} \phi_{j,\mvec}^* \sum_{\lvec} \mathcal{B}_{\lvec}^{(k)} (\mathcal{B}_{\lvec}^{(j)})^*
	& = \int d\xvec \int d\xvec^\prime
			\phi_{k,\nvec}^\prime \phi_{j,\mvec}^*
			\sum_{\lvec} \mathcal{B}_{\lvec}^{(k)} (\mathcal{B}_{\lvec}^{(j)})^{\prime*}
			\delta(\xvec - \xvec^\prime) \\
	& = \int d\xvec \int d\xvec^\prime
			\phi_{k,\nvec}^\prime \phi_{j,\mvec}^*
			\sum_{\lvec} \mathcal{B}_{\lvec}^{(k)} (\mathcal{B}_{\lvec}^{(j)})^{\prime*}
			\sum_{\pvec \in \fullbasis} \phi_{\pvec}^{\prime*} \phi_{\pvec} \\
	& = \sum_{\pvec \in \fullbasis, \lvec}
		\int d\xvec
			\phi_{k,\nvec}^* \mathcal{B}_{\lvec}^{(k)} \phi_{\pvec}
		\int d\xvec
			\phi_{j,\mvec} (\mathcal{B}_{\lvec}^{(j)})^* \phi_{\pvec}^*
\end{eqn}
Note that we did not specify the index of the full basis used to expand the delta function.
It can be any orthonormal and complete basis, in particular one of $\fullbasis_j$, this will not change the result.

Now we have the FPE from \thmref{wigner-bec:fpe:mc-fpe-sde} with
\begin{eqn}
	\avec_{\mvec}^{(c)} = \int d\xvec \phi_{c,\mvec}^* \mathcal{A}^{(c)},\,\mvec \in \restbasis_c
\end{eqn}
and
\begin{eqn}
\label{eqn:wigner-bec:fpe:func-noise-matrix}
	B_{\mvec,(\nvec,\lvec)}^{(c)} = \int d\xvec \phi_{c,\mvec}^* \mathcal{B}_{\lvec}^{(c)} \phi_{\nvec},\,
	\mvec \in \restbasis_c, \nvec \in \fullbasis.
\end{eqn}
Note that columns of $B$ are enumerated using compound index $\nvec,\lvec$.

Therefore the initial FPE is equivalent to the set of SDEs in It\^{o} form
\begin{eqn}
	d\alpha_{\mvec}^{(c)}
	= \int d\xvec \phi_{c,\mvec}^* \mathcal{A}^{(c)} dt
	+ \sum_{\nvec \in \fullbasis, \lvec}
		\int d\xvec \phi_{c,\mvec}^* \mathcal{B}_{\lvec}^{(c)} \phi_{\nvec} dZ_{\nvec,\lvec}.
\end{eqn}
Multiplying by $\phi_{c,\mvec}^\prime$ and grouping by component:
\begin{eqn}
	\sum_{\mvec \in \restbasis_c} \phi_{c,\mvec}^\prime d\alpha_{\mvec}^{(c)}
	= \sum_{\mvec \in \restbasis_c} \phi_{c,\mvec}^\prime \int d\xvec \phi_{c,\mvec}^* \mathcal{A}^{(c)} dt
	+ \sum_{\mvec \in \restbasis_c} \phi_{c,\mvec}^\prime \int d\xvec \phi_{c,\mvec}^*
		\sum_{\nvec \in \fullbasis, \lvec} \mathcal{B}_{\lvec}^{(c)} \phi_{\nvec} dZ_{\nvec,\lvec}.
\end{eqn}
Recognizing \defref{func-calculus:projector} of projection transformation:
\begin{eqn}
	d\Psi_c
	= \mathcal{P}_{\restbasis_c} \left[
		\mathcal{A}^{(c)} dt
		+ \sum_{\lvec} \mathcal{B}_{\lvec}^{(c)}
			\sum_{\nvec \in \fullbasis} \phi_{\nvec} dZ_{\nvec,\lvec}
	\right].
\end{eqn}
Defining functional Wiener process $Q_{\lvec} = \sum_{\nvec \in \fullbasis} \phi_{\nvec} dZ_{\nvec,\lvec}$:
\begin{eqn}
	d\Psi_c
	= \mathcal{P}_{\restbasis_c} \left[
		\mathcal{A}^{(c)} dt
		+ \sum_{\lvec} \mathcal{B}_{\lvec}^{(c)} dQ_{\lvec}
	\right].
\end{eqn}

Performing the same multiplication and summation on Stratonovich term from \thmref{wigner-bec:fpe:mc-fpe-sde}:
\begin{eqn}
	\mathcal{S}^{(c)}
	= \sum_{\mvec \in \restbasis_c} \phi_{c,\mvec}^\prime s_{\mvec}^{(c)}
	= \sum_{\mvec \in \restbasis_c} \phi_{c,\mvec}^\prime \sum_{n=1}^C \Trace{
		(B^{(n)})^H \bpartial_{(\balpha^{(n)})^*} \evec_{\mvec}^T B^{(c)}
	}.
\end{eqn}
Transforming trace to summation:
\begin{eqn}
	= \sum_{\mvec \in \restbasis_c} \phi_{c,\mvec}^\prime \sum_{n=1}^C
		\sum_{\jvec \in \restbasis_n} \sum_{\pvec \in \fullbasis, \lvec}
			(B_{\jvec (\pvec,\lvec)}^{(n)})^*
			\frac{\partial}{\partial (\alpha_{\jvec}^{(n)})^*}
			B_{\mvec (\pvec,\lvec)}^{(c)}.
\end{eqn}
Using the multimode form~\eqnref{wigner-bec:fpe:func-noise-matrix} of the noise matrix:
\begin{eqn}
	= \sum_{\mvec \in \restbasis_c} \phi_{c,\mvec}^\prime \sum_{n=1}^C
		\sum_{\jvec \in \restbasis_n} \sum_{\pvec \in \fullbasis, \lvec}
			\int d\xvec \phi_{n,\jvec} (\mathcal{B}_{\lvec}^{(n)})^* \phi_{\pvec}^*
			\int d\xvec \phi_{c,\mvec}^*
				\frac{\partial}{\partial (\alpha_{\jvec}^{(n)})^*}
				\mathcal{B}_{\lvec}^{(c)} \phi_{\pvec}
\end{eqn}
Substituting $\sum_{\pvec \in \fullbasis} \phi_{\pvec}^* \phi_{\pvec} = \delta(\xvec - \xvec^\prime)$:
\begin{eqn}
	= \sum_{\mvec \in \restbasis_c} \phi_{c,\mvec}^\prime
		\sum_{n=1}^C \sum_{\jvec \in \restbasis_n} \sum_{\lvec}
			\int d\xvec
				\phi_{n,\jvec} (\mathcal{B}_{\lvec}^{(n)})^*
				\phi_{c,\mvec}^* \frac{\partial}{\partial (\alpha_{\jvec}^{(n)})^*}
				\mathcal{B}_{\lvec}^{(c)}
\end{eqn}
Recognizing the projection transformation and the functional differential:
\begin{eqn}
	& = \mathcal{P}_{\restbasis_c} \left[
		\sum_{n=1}^C \sum_{\jvec \in \restbasis_n} \sum_{\lvec}
			\phi_{n,\jvec} (\mathcal{B}_{\lvec}^{(n)})^*
			\frac{\partial}{\partial (\alpha_{\jvec}^{(n)})^*}
			\mathcal{B}_{\lvec}^{(c)}
	\right] \\
	& = \mathcal{P}_{\restbasis_c} \left[
		\sum_{n=1}^C \sum_{\lvec}
		(\mathcal{B}_{\lvec}^{(n)})^*
		\frac{\delta}{\delta \Psi_n^*}
		\mathcal{B}_{\lvec}^{(c)}
	\right].
	\qedhere
\end{eqn}
\end{proof}

\todo{This means that Stratonovich term is equal to zero for our $B \equiv B(\Psivec)$.}


% =============================================================================
\section{It\^{o} formula}
% =============================================================================

In this section we will derive the It\^{o} formula for the differential of a functional, based on the standard definition for multi-variable real-valued case.

\begin{theorem}
\label{thm:wigner-bec:ito-formula:ito-f-real}
	Let $\zvec^T \equiv (z_1 \ldots z_M)$ be a set of real variables, and $\Zvec(t)$ be a standard $L$-dimensional Wiener process.
	For the SDE in It\^{o} form
	\begin{eqn*}
		d\zvec = \avec(\zvec, t) dt + B(\zvec, t) d\Zvec(t),
	\end{eqn*}
	the differential of a function $f(\zvec)$ is
	\begin{eqn*}
		df(\zvec) = \left(
			\avec \cdot \bpartial_{\zvec} dt
			+ \frac{1}{2} \Trace{ B B^T \bpartial_{\zvec} \bpartial_{\zvec}^T } dt
			+ \Trace{ B d\Zvec \bpartial_{\zvec}^T }
		\right) f(\zvec).
	\end{eqn*}
\end{theorem}
\begin{proof}
This is just a statement from~\cite{Gardiner1997} expressed in matrix form.
\end{proof}

This theorem can be extended to complex variables.

\begin{theorem}
\label{thm:wigner-bec:ito-formula:ito-f-complex}
	Let $\balpha^T \equiv (\alpha_1 \ldots \alpha_M)$ be a set of complex variables, and $\Zvec = (\bm{X} + i\bm{Y}) / \sqrt{2}$ be an $L$-dimensional complex-valued Wiener process, containing two standard $L$-dimensional Wiener processes $\bm{X}$ and $\bm{Y}$.
	For the SDE in It\^{o} form
	\begin{eqn*}
		d\balpha = \avec(\balpha, t) dt + B(\balpha, t) d\Zvec(t),
	\end{eqn*}
	the differential of a function $f(\balpha)$ is
	\begin{eqn*}
		df(\balpha) = \left(
			2 \Real (\avec \cdot \bpartial_{\balpha}) dt
			+ \Trace{ B B^H \bpartial_{\balpha^*} \bpartial_{\balpha}^T } dt
			+ 2 \Real \Trace{ B d\Zvec \bpartial_{\balpha}^T }
		\right) f(\balpha).
	\end{eqn*}
\end{theorem}
\begin{proof}
The proof follows the same scheme as \thmref{wigner-bec:fpe:fpe-sde-complex}, just in the opposite direction.
Let $f = g + ih$, $\balpha = \bm{x} + i \bm{y}$, $\avec = \bm{u} + i \bm{v}$, $B = F + iG$, $\bpartial_{\balpha} = (\bpartial_{\bm{x}} - i \bpartial_{\bm{y}}) / 2$.
Then the set of SDEs from the statement is equivalent to
\begin{eqn}
	d \begin{pmatrix} \bm{x} \\ \bm{y} \end{pmatrix}
	= \begin{pmatrix} \bm{u} \\ \bm{v} \end{pmatrix} dt
		+ \frac{1}{\sqrt{2}} \begin{pmatrix} F & -G \\ G & F \end{pmatrix}
			\begin{pmatrix} d\bm{X} \\ d\bm{Y} \end{pmatrix}.
\end{eqn}
Applying \thmref{wigner-bec:ito-formula:ito-f-real} for real-valued functions $g(\bm{x}, \bm{y})$ and $h(\bm{x}, \bm{y})$ and combining them into $f = g + ih$:
\begin{eqn}
	df ={} &
		\begin{pmatrix} \bm{x} \\ \bm{y} \end{pmatrix} \cdot
			\begin{pmatrix} \bpartial_{\bm{x}} \\ \bpartial_{\bm{y}} \end{pmatrix} f dt
		+ \frac{1}{4} \Trace{
			\begin{pmatrix} F & -G \\ G & F \end{pmatrix}
			\begin{pmatrix} F^T & G^T \\ -G^T & F^T \end{pmatrix}
			\begin{pmatrix} \bpartial_{\bm{x}} \\ \bpartial_{\bm{y}} \end{pmatrix}
			\begin{pmatrix} \bpartial_{\bm{x}} \\ \bpartial_{\bm{y}} \end{pmatrix}^T
		} f dt  \\
	& + \frac{1}{\sqrt{2}} \Trace{
			\begin{pmatrix} F & -G \\ G & F \end{pmatrix}
			\begin{pmatrix} d\bm{X} \\ d\bm{Y} \end{pmatrix}
			\begin{pmatrix} \bpartial_{\bm{x}} \\ \bpartial_{\bm{y}} \end{pmatrix}^T
		} f
\end{eqn}
Now let us match this equation and the lemma statement term by term.

First term:
\begin{eqn}
	2 \Real ( \avec \cdot \bpartial_{\balpha} )
	& = \Real \left(
			\left( \bm{u} + i\bm{v} \right) \cdot \left( \bpartial_{\bm{x}} - i \bpartial_{\bm{y}} \right)
		\right) \\
	& = \bm{u} \cdot \bpartial_{\bm{x}} + \bm{v} \cdot \bpartial_{\bm{y}} \\
	& = \begin{pmatrix} \bm{x} \\ \bm{y} \end{pmatrix} \cdot
		\begin{pmatrix} \bpartial_{\bm{x}} \\ \bpartial_{\bm{y}} \end{pmatrix}
\end{eqn}

Second term:
\begin{eqn}
	\Trace{ B B^H \bpartial_{\balpha^*} \bpartial_{\balpha}^T }
	={} & \frac{1}{4} \Trace{
		(F F^T + G G^T)
		(\bpartial_{\bm{x}} \bpartial_{\bm{x}}^T
			+ \bpartial_{\bm{y}} \bpartial_{\bm{y}}^T)
		} \\
	& - \frac{1}{4} \Trace {
		(F G^T - G F^T)
		(\bpartial_{\bm{x}} \bpartial_{\bm{y}}^T
			- \bpartial_{\bm{y}} \bpartial_{\bm{x}}^T)
		} \\
	& + \frac{i}{4} \Trace{
		(F G^T - G F^T)
		(\bpartial_{\bm{x}} \bpartial_{\bm{x}}^T
			+ \bpartial_{\bm{y}} \bpartial_{\bm{y}}^T)
	} \\
	& + \frac{i}{4} \Trace{
		(G G^T + F F^T)
		(\bpartial_{\bm{x}} \bpartial_{\bm{y}}^T
			- \bpartial_{\bm{y}} \bpartial_{\bm{x}}^T)
	}
\end{eqn}
Same as in \thmref{wigner-bec:fpe:fpe-sde-complex} we notice that $F F^T + G G^T$ and $\bpartial_{\bm{x}} \bpartial_{\bm{x}}^T + \bpartial_{\bm{y}} \bpartial_{\bm{y}}^T$ are symmetric matrices, and $F G^T - G F^T$ and $\bpartial_{\bm{x}} \bpartial_{\bm{y}}^T - \bpartial_{\bm{y}} \bpartial_{\bm{x}}^T$ are antisymmetric.
Therefore the last two terms contain traces of antisymmetric matrices and are equal to zero.
\begin{eqn}
	={} & \frac{1}{4} \Trace{
		(F F^T + G G^T) \bpartial_{\bm{x}} \bpartial_{\bm{x}}^T
		+ (F G^T - G F^T) \bpartial_{\bm{y}} \bpartial_{\bm{x}}^T)
		} \\
	& + \frac{1}{4} \Trace {
		(G F^T - F G^T) \bpartial_{\bm{x}} \bpartial_{\bm{y}}^T
		+ (F F^T + G G^T) \bpartial_{\bm{y}} \bpartial_{\bm{y}}^T)
		} \\
	={} & \frac{1}{4} \Trace {
		\begin{pmatrix}
			F F^T + G G^T & F G^T - G F^T \\
			G F^T - F G^T & F F^T + G G^T
		\end{pmatrix}
		\begin{pmatrix}
			\bpartial_{\bm{x}} \bpartial_{\bm{x}}^T & \bpartial_{\bm{x}} \bpartial_{\bm{y}}^T \\
			\bpartial_{\bm{y}} \bpartial_{\bm{x}}^T & \bpartial_{\bm{y}} \bpartial_{\bm{y}}^T
		\end{pmatrix}
	} \\
	={} & \frac{1}{4} \Trace{
		\begin{pmatrix} F & -G \\ G & F \end{pmatrix}
		\begin{pmatrix} F^T & G^T \\ -G^T & F^T \end{pmatrix}
		\begin{pmatrix} \bpartial_{\bm{x}} \\ \bpartial_{\bm{y}} \end{pmatrix}
		\begin{pmatrix} \bpartial_{\bm{x}} \\ \bpartial_{\bm{y}} \end{pmatrix}^T
	}.
\end{eqn}

Third term:
\begin{eqn}
	2 \Real \Trace{ B d\Zvec \bpartial_{\balpha}^T }
	& = \frac{1}{\sqrt{2}} \Real \Trace{
		(F + iG) (d\bm{X} + id\bm{Y}) (\bpartial_{\bm{x}} - i\bpartial_{\bm{y}})
	} \\
	& = \frac{1}{\sqrt{2}} \Trace{
		F d\bm{X} \bpartial_{\bm{x}} + F d\bm{Y} \bpartial_{\bm{y}}
		- G d\bm{Y} \bpartial_{\bm{x}} + G d\bm{X} \bpartial_{\bm{y}}
	} \\
	& = \frac{1}{\sqrt{2}} \Trace{
			\begin{pmatrix} F & -G \\ G & F \end{pmatrix}
			\begin{pmatrix} d\bm{X} \\ d\bm{Y} \end{pmatrix}
			\begin{pmatrix} \bpartial_{\bm{x}} \\ \bpartial_{\bm{y}} \end{pmatrix}^T
		}.
\end{eqn}

All terms have matched, thus proving the theorem.
\end{proof}

\begin{theorem}
\label{thm:wigner-bec:ito-formula:mc-ito-f}
	Let $\balpha^{(c)},\, c = 1..C$ be $C$ sets of complex variables $\balpha^{(c)} \equiv (\alpha_1^{(c)} \ldots \alpha_{M_c}^{(c)})$.
	For the SDE in It\^{o} form
	\begin{eqn*}
		d\balpha^{(c)} = \avec^{(c)} dt + B^{(c)} d\Zvec,
	\end{eqn*}
	the differential of a function $f(\balpha^{(1)}, \ldots, \balpha^{(C)})$ is
	\begin{eqn*}
		df ={} & \left(
			2 \sum_{c=1}^C \Real (\avec^{(c)} \cdot \bpartial_{\balpha^{(c)}}) dt
			+ \sum_{m=1}^C \sum_{n=1}^C \Trace{
				B^{(m)} (B^{(n)})^H \bpartial_{(\balpha^{(n)})^*} \bpartial_{\balpha^{(m)}}^T } dt \right. \\
		& \left. + 2 \sum_{c=1}^C \Real \Trace{ B^{(c)} d\Zvec \bpartial_{\balpha^{(c)}}^T }
		\right) f.
	\end{eqn*}
\end{theorem}
\begin{proof}
Proved analogously to \thmref{wigner-bec:fpe:mc-fpe-sde}, by combining $\balpha^{(c)}$ into a single vector	and applying \thmref{wigner-bec:ito-formula:ito-f-complex}.
\end{proof}

\begin{theorem}
\label{thm:wigner-bec:ito-formula:func-ito-f}
	Given functional SDEs in It\^{o} form
	\begin{eqn*}
		d\Psi^{(c)} = \mathcal{A}^{(c)} dt + \sum_{\lvec} \mathcal{B}_{\lvec}^{(c)} dQ_{\lvec},
	\end{eqn*}
	the differential of a functional $F[\Psivec]$ is
	\begin{eqn*}
		dF[\Psivec]
		={} & \int d\xvec^\prime \left(
			2 \sum_{c=1}^C \Real \left(
				\mathcal{A}^{(c)\prime} \frac{\delta}{\delta \Psi_c^\prime}
			\right) dt
			+ \sum_{i=1}^C \sum_{j=1}^C \sum_{\lvec}
				\mathcal{B}_{\lvec}^{(i)\prime}
				\mathcal{B}_{\lvec}^{(j)\prime *}
				\frac{\delta}{\delta \Psi_i^\prime}
				\frac{\delta}{\delta \Psi_j^{\prime *}} dt
			\right. \\
		& \left. + 2 \sum_{c=1}^C \sum_{\lvec} \Real \left(
				\mathcal{B}_{\lvec}^{(i)\prime}
				dQ_{\lvec}^\prime
				\frac{\delta}{\delta \Psi_c^\prime}
			\right)
		\right) F[\Psivec]
	\end{eqn*}
	\todo{Consider rewriting it as
	\begin{eqn*}
		dF[\Psivec]
		= \int d\xvec^\prime \left(
			2 \Real \left(
				\vec{\mathcal{A}}^\prime \frac{\delta}{\delta \Psi_c^\prime}
			\right) dt
			+ \Trace{
				\mathcal{B}^\prime
				(\mathcal{B}^\prime)^H
				\bdelta_{\Psivec^{\prime *}}
				\bdelta_{\Psivec^\prime}^T
			} dt
			+ 2 \Real \Trace{
				\mathcal{B}^\prime
				d\vec{Q}^\prime
				\bdelta_{\Psivec^\prime}^T
			}
		\right) F[\Psivec].
	\end{eqn*}
	}
\end{theorem}
\begin{proof}
In terms of complex vectors SDEs can be rewritten as
\begin{eqn}
	d\alpha_{\mvec}^{(c)}
	= \int d\xvec \phi_{c,\mvec}^* \mathcal{A}^{(c)} dt
	+ \sum_{\pvec \in \fullbasis, \lvec}
		\int d\xvec \phi_{c,\mvec}^* \mathcal{B}_{\lvec}^{(c)} \phi_{\pvec} dZ_{\pvec,\lvec}.
\end{eqn}
Now, treating the functional as a function of complex vector $F \equiv F(\balpha^{(1)}, \ldots, \balpha^{(C)})$, we can use \thmref{wigner-bec:ito-formula:mc-ito-f} with
\begin{eqn}
	(\bm{a}^{(c)})_{\mvec} = \int d\xvec \phi_{c,\mvec}^* \mathcal{A}^{(c)},
\end{eqn}
and
\begin{eqn}
	(B^{(c)})_{\mvec,(\pvec,\lvec)}
	= \int d\xvec \phi_{c,\mvec}^* \mathcal{B}_{\lvec}^{(c)} \phi_{\pvec}.
\end{eqn}
This gives us
\begin{eqn}
	dF
	={} & \left(
		2 \sum_{c=1}^C \sum_{\mvec \in \restbasis_c} \Real \left(
			\int d\xvec^\prime \phi_{c,\mvec}^{\prime*} \mathcal{A}^{(c)\prime}
			\frac{\partial}{\partial \alpha_{c,\mvec}}
		\right) \right. \\
	& \left. + \sum_{i=1}^C \sum_{j=1}^C
			\sum_{\mvec \in \restbasis_i} \sum_{\nvec \in \restbasis_j}
			\sum_{\pvec \in \fullbasis, \lvec}
			\int d\xvec^\prime \phi_{i,\mvec}^{\prime *} \mathcal{B}_{\lvec}^{(i)\prime} \phi_{\pvec}^\prime
			\int d\xvec^{\prime\prime} \phi_{j,\nvec}^{\prime\prime} \mathcal{B}_{\lvec}^{(j)\prime\prime *} \phi_{\pvec}^{\prime\prime *}
			\frac{\partial}{\partial_{\alpha_{j,\nvec}^*}}
			\frac{\partial}{\partial_{\alpha_{i,\mvec}}} \right. \\
	& \left. + 2 \sum_{c=1}^C \Real \left(
			\sum_{\mvec \in \restbasis_c}
			\sum_{\pvec \in \fullbasis, \lvec}
			\int d\xvec^\prime \phi_{i,\mvec}^{\prime*} \mathcal{B}_{\lvec}^{(i)\prime} \phi_{\pvec}^\prime
			dZ_{\pvec,\lvec}
			\partial_{\alpha_{c,\mvec}}
		\right)
	\right) F
\end{eqn}
Recognizing definitions of functional differentials, functional Wiener process, and the delta function, we get
\begin{eqn}
	={} & \left(
		2 \sum_{c=1}^C \Real \left(
			\int d\xvec^\prime \mathcal{A}^{(c)\prime}
			\frac{\delta}{\delta \Psi_c^\prime}
		\right) \right. \\
	& \left. + \sum_{i=1}^C \sum_{j=1}^C \sum_{\lvec}
			\int d\xvec^\prime \mathcal{B}_{\lvec}^{(i)\prime}
			\mathcal{B}_{\lvec}^{(j)\prime *}
			\frac{\delta}{\delta \Psi_i^\prime}
			\frac{\delta}{\delta \Psi_j^{\prime *}}
		\right. \\
	& \left. + 2 \sum_{c=1}^C \sum_{\lvec} \Real \left(
			\int d\xvec^\prime \mathcal{B}_{\lvec}^{(i)\prime}
			dQ_{\lvec}^\prime
			\frac{\delta}{\delta \Psi_c^\prime}
		\right)
	\right) F
\end{eqn}
Which leads to the statement of the theorem.
\end{proof}

% =============================================================================
\section{Fokker-Planck equation for the BEC}
% =============================================================================

With the Wigner truncation applied, we can get rid of the third- and higher-order functional derivatives in~\eqnref{wigner-bec:truncation:untruncated-fpe}.
Under the reasonable assumption of $K_{jk}$, $U_{jk}$ and $\kappa_{\lvec}$ being real-valued,
this results in the functional equation
\begin{eqn}
\label{eqn:wigner-bec:truncation:fpe}
	\frac{\upd W}{\upd t}
	= \int \upd\xvec \left(
		- \sum_{j=1}^C \frac{\fdelta}{\fdelta \Psi_j} \mathcal{A}_j
		- \sum_{j=1}^C \frac{\fdelta}{\fdelta \Psi_j^*} \mathcal{A}_j^*
		+ \sum_{j=1}^C \sum_{k=1}^C \frac{\fdelta^2}{\fdelta \Psi_j^* \fdelta \Psi_k}
			\mathcal{D}_{jk}
	\right) W,
\end{eqn}
with the drift terms
\begin{eqn}
\label{eqn:wigner-bec:truncation:drift-term}
	\mathcal{A}_j
	={} & -\frac{i}{\hbar} \left(
		\sum_{k=1}^C K_{jk} \Psi_k
		+ \Psi_j \sum_{k=1}^C U_{jk} \left(
			|\Psi_k|^2 - \frac{\delta_{jk} + 1}{2} \delta_{\restbasis_k}(\xvec, \xvec)
		\right)
	\right) \\
	& - \sum_{\lvec \in L} \kappa_{\lvec} \left(
		\frac{\upp O_{\lvec}^*}{\upp \Psi_j^*} O_{\lvec}
		- \frac{1}{2} \sum_{k=1}^C \delta_{\restbasis_k}(\xvec, \xvec)
			\frac{\upp^2 O_{\lvec}^*}{\upp \Psi_j^* \upp \Psi_k^*}
			\frac{\upp O_{\lvec}}{\upp \Psi_k}
	\right),
\end{eqn}
and the diffusion matrix
\begin{eqn}
\label{eqn:wigner-bec:truncation:diffusion-term}
	\mathcal{D}_{jk} = \sum_{\lvec \in L} \kappa_{\lvec}
		\frac{\upp O_{\lvec}}{\upp \Psi_j}
		\frac{\upp O_{\lvec}^*}{\upp \Psi_k^*}.
\end{eqn}

It can be shown (see \appref{fpe-sde} for details) that this truncated equation has a positive-definite diffusion matrix $\mathcal{D}$ and is therefore a Fokker-Planck equation (\abbrev{fpe}), with its solution $W(t)$ being a probability distribution (provided that $W(0)$ was a probability distribution).
This differs from the original Wigner function(al), which can be negative, and is the result of the truncation.
Thus the solution of this equation may not be equivalent to the solution of the orignal master equation~\eqnref{wigner-bec:master-eqn:master-eqn} (if the corresponding density matrices have non-positive Wigner functions), but, given that the truncation conditions are satisfied, the effect is negligible.

Direct solution of the above \abbrev{fpe} is generally impractical, and a Monte Carlo or sampled calculation is called for. The equation can be further transformed to the equivalent set of stochastic differential equations using \thmref{fpe-sde:corr:fpe-sde-func} with
\begin{eqn}
	\mathcal{B}_{j\lvec}
	= \sqrt{\kappa_{\lvec}} \frac{\partial O_{\lvec}^*}{\partial \Psi_j^*}.
\end{eqn}
This results in the set of functional \abbrev{sde}s in It\^{o} form
\begin{eqn}
\label{eqn:fpe-sde:corr-bec:sde}
	\upd\Psi_j = \mathcal{P}_{\restbasis_j} \left[
		\mathcal{A}_j \upd t
		+ \sum_{\lvec \in L} \mathcal{B}_{j\lvec} \upd Q_{\lvec}
	\right],
\end{eqn}
or, alternatively, in Stratonovich form
\begin{eqn}
\label{eqn:fpe-sde:corr-bec:sde-stratonovich}
	\upd\Psi_j = \mathcal{P}_{\restbasis_j} \left[
		(\mathcal{A}_j - \mathcal{S}_j) \upd t
		+ \sum_{\lvec \in L} \mathcal{B}_{j\lvec} \upd Q_{\lvec}
	\right],
\end{eqn}
where the Stratonovich term is
\begin{eqn}
	\mathcal{S}_j
	& = \frac{1}{2} \sum_{k=1}^C \sum_{\lvec \in L}
		\mathcal{B}_{k\lvec}^*
		\frac{\fdelta}{\fdelta \Psi_k^*}
		\mathcal{B}_{j\lvec} \\
	& = \frac{1}{2} \sum_{k=1}^C \sum_{\lvec \in L}
		\delta_{\restbasis_k}(\xvec, \xvec)
		\frac{\upp^2 O_{\lvec}^*}{\upp \Psi_k^* \upp \Psi_j^*}
		\frac{\upp O_{\lvec}}{\upp \Psi_k}.
\end{eqn}
Note that the Stratonovich term is exactly equal to the correction in the loss-induced part of the drift term~\eqnref{wigner-bec:truncation:drift-term}.
This means that in Stratonovich form the \abbrev{sde}s are actually simpler.

The equations~\eqnref{fpe-sde:corr-bec:sde} can be solved by conventional integration methods for \abbrev{sde}s. \todo{reference the numericals appendix?}
The required expectations of symmetrically ordered operator products can be obtained by averaging results from multiple independent solutions according to \thmref{wigner:mc:moments}, since the truncated $W$ function is a probability distribution:
\begin{eqn}
\label{eqn:wigner-bec:fpe-bec:moments}
	\langle \symprod{ \prod_{j=1}^C \Psiop_j^{r_j} (\Psiop_j^\dagger)^{s_j} } \rangle
	& = \int \fdelta^2 \bPsi\,
		\left( \prod_{j=1}^C \Psi_j^{r_j} (\Psi_j^*)^{s_j} \right) W[\bPsi] \\
	& \approx \pathavg{
		\prod_{j=1}^C \Psi_j^{r_j} (\Psi_j^*)^{s_j}
	},
\end{eqn}
where $\{r_j\}$ and $\{s_j\}$ are some sets of non-negative integers.
Naturally, the second approximate equality becomes exact in the limit of the infinite number of integration paths.


% =============================================================================
\subsection{Integral averages}
% =============================================================================

It is interesting to get an expression for time dependence of some simple observables using \abbrev{sde}s~\eqnref{fpe-sde:corr-bec:sde} and It\^{o}'s formula (\thmref{fpe-sde:ito-formula:func-ito-f}).
Namely, we are interested in population $N_i = \int \upd\xvec \langle \Psiop_i^\dagger \Psiop_i \rangle$.

\begin{theorem}
	In a \abbrev{bec} with the evolution governed by the set of \abbrev{sde}s~\eqnref{fpe-sde:corr-bec:sde}, the population changes in time as
	\begin{eqn*}
		\frac{\upd N_i}{\upd t}
		={} & - \sum_{\lvec \in L} \kappa_{\lvec} \int \upd\xvec \pathavgleft
			2 \frac{\upp O_{\lvec}^*}{\upp \Psi_i^*} O_{\lvec} \Psi_i^*
				- \sum_{k=1}^C \delta_{\restbasis_k}(\xvec, \xvec)
					\frac{\upp^2 O_{\lvec}^*}{\upp \Psi_i^* \upp \Psi_k^*}
					\frac{\upp O_{\lvec}}{\upp \Psi_k}
					\Psi_i^*
			\right. \\
		& \quad \left.
			- \frac{\partial O_{\lvec}}{\partial \Psi_i}
				\frac{\partial O_{\lvec}^*}{\partial \Psi_i^*}
				\delta_{\restbasis_i}(\xvec, \xvec)
		\pathavgright.
	\end{eqn*}
\end{theorem}
\begin{proof}
Let us apply \thmref{fpe-sde:ito-formula:func-ito-f} with $f_j \equiv \Psi_j$ to $\mathcal{F} \equiv \Psi_i^* \Psi_i$.
Since in the end we are interested in the average of $\mathcal{F}$, and $\pathavg{ dQ_{\lvec} } \equiv 0$, we can discard the third term in It\^{o} formula.
The resulting expression for the differential is
\begin{eqn}
	\upd (\Psi_i^* \Psi_i)
	={} & \int \upd \xvec^\prime \left(
		\sum_{j=1}^C \mathcal{A}_j^\prime
			\frac{\fdelta (\Psi_i^* \Psi_i)}{\fdelta \Psi_j^\prime}
		+ \sum_{j=1}^C \mathcal{A}_j^{\prime *}
			\frac{\fdelta (\Psi_i^* \Psi_i)}{\fdelta \Psi_j^{\prime *}} \right. \\
	& \quad \left. + \sum_{j=1}^C \sum_{k=1}^C \sum_{\lvec \in L}
			\mathcal{B}_{j\lvec}^\prime \mathcal{B}_{k\lvec}^{\prime *}
			\frac{\fdelta^2 (\Psi_i^* \Psi_i)}{\fdelta \Psi_j^\prime \fdelta \Psi_k^{\prime *}}
		\right) \upd t.
\end{eqn}
The derivatives are evaluated as
\begin{eqn}
	\frac{\fdelta (\Psi_i^* \Psi_i)}{\fdelta \Psi_j^\prime}
	& = \delta_{ij} \Psi_i^* \delta_{\restbasis_i}(\xvec^\prime, \xvec), \\
	\frac{\fdelta (\Psi_i^* \Psi_i)}{\fdelta \Psi_j^{\prime *}}
	& = \delta_{ij} \Psi_i \delta_{\restbasis_i}^*(\xvec^\prime, \xvec), \\
	\frac{\fdelta^2 (\Psi_i^* \Psi_i)}{\fdelta \Psi_j^\prime \fdelta \Psi_k^{\prime *}}
	& = \delta_{ij} \delta_{ik} \delta_{\restbasis_i}(\xvec^\prime, \xvec) \delta_{\restbasis_i}^*(\xvec^\prime, \xvec).
\end{eqn}

From~\eqnref{wigner-bec:fpe-bec:moments} it follows that
\begin{eqn}
	\frac{\upd N_i}{\upd t}
	& = \int \upd \xvec \frac{\upd \langle \Psiop_i^* \Psiop_i \rangle}{\upd t} \\
	& \approx \int \upd \xvec \frac{\upd (
		\pathavg{ \Psi_i^* \Psi_i } - \frac{1}{2} \delta_{\restbasis_i}(\xvec, \xvec)
		)}{\upd t}
	= \int \upd \xvec
		\pathavg{ \frac{\upd (\Psi_i^* \Psi_i)}{\upd t} }.
\end{eqn}
Substituting the expression for the differential of $\Psi_i^* \Psi_i$, we get
\begin{eqn}
	\frac{\upd N_i}{\upd t}
	={} & \iint \upd \xvec\, \upd \xvec^\prime \pathavgleft
		\mathcal{A}_i^\prime
			\Psi_i^* \delta_{\restbasis_i}(\xvec^\prime, \xvec)
		+ \mathcal{A}_i^{\prime *}
			\Psi_i \delta_{\restbasis_i}^*(\xvec^\prime, \xvec) \right. \\
	& \quad + \sum_{\lvec \in L} \left.
			\mathcal{B}_{i\lvec}^\prime \mathcal{B}_{i\lvec}^{\prime *}
			\delta_{\restbasis_i}(\xvec^\prime, \xvec) \delta_{\restbasis_i}^*(\xvec^\prime, \xvec)
		\pathavgright,
\end{eqn}
where we were able to move the integral over $\xvec$ since the drift and diffusion terms depend on $\xvec^\prime$.
Integrating by $\xvec$, and expanding $\Psi_i$ and restricted delta functions:
\begin{eqn}
	\iint \upd\xvec\, \upd\xvec^\prime
		\mathcal{A}_i^\prime \Psi_i^* \delta_{\restbasis_i}(\xvec^\prime, \xvec)
	& = \iint d\xvec d\xvec^\prime \mathcal{A}_i^\prime
		\sum_{\nvec \in \restbasis_i} \phi_{i,\nvec}^* \alpha_{i,\nvec}^*
		\sum_{\mvec \in \restbasis_i} \phi_{i,\mvec} \phi_{i,\mvec}^{\prime *} \\
	& = \int \upd\xvec^\prime \mathcal{A}_i^\prime
		\sum_{\mvec \in \restbasis_i} \alpha_{i,\nvec}^* \phi_{i,\nvec}^{\prime *} \\
	& = \int \upd\xvec \mathcal{A}_i \Psi_i^*.
\end{eqn}
Similarly,
\begin{eqn}
	\iint \upd\xvec\, \upd\xvec^\prime
		\mathcal{A}_i^{\prime *} \Psi_i \delta_{\restbasis_i}^*(\xvec^\prime, \xvec)
	= \int \upd\xvec \mathcal{A}_i^* \Psi_i,
\end{eqn}
and
\begin{eqn}
	\iint \upd\xvec\, \upd\xvec^\prime
		\mathcal{B}_{i\lvec}^\prime \mathcal{B}_{i\lvec}^{\prime *}
		\delta_{\restbasis_i}(\xvec^\prime, \xvec) \delta_{\restbasis_i}^*(\xvec^\prime, \xvec)
	= \int \upd\xvec \mathcal{B}_{i\lvec} \mathcal{B}_{i\lvec}^*
		\delta_{\restbasis_i}(\xvec, \xvec).
\end{eqn}
The integration thus gives us
\begin{eqn}
	\frac{\upd N_i}{\upd t}
	= \int \upd\xvec \pathavgleft
		\mathcal{A}_i \Psi_i^*
		+ \mathcal{A}_i^* \Psi_i
		+ \sum_{\lvec \in L} \mathcal{B}_{i\lvec} \mathcal{B}_{i\lvec}^*
			\delta_{\restbasis_i}(\xvec, \xvec)
	\pathavgright.
\end{eqn}

Now we can substitute known expressions for drift terms~\eqnref{wigner-bec:truncation:drift-term} and diffusion terms~\eqnref{wigner-bec:truncation:diffusion-term}.
From the form of the expression above it is clear that the unitary evolution part of the drift term does not contribute to the population change rate (which agrees with the intuition).
Thus the drift terms can be safely simplified to
\begin{eqn}
	\mathcal{A}_i^{\mathrm{(loss)}}
	= - \sum_{\lvec \in L} \kappa_{\lvec} \left(
		\frac{\upp O_{\lvec}^*}{\upp \Psi_i^*} O_{\lvec}
		- \frac{1}{2} \sum_{k=1}^C \delta_{\restbasis_k}(\xvec, \xvec)
			\frac{\upp^2 O_{\lvec}^*}{\upp \Psi_i^* \upp \Psi_k^*}
			\frac{\upp O_{\lvec}}{\upp \Psi_k}
		\right),
\end{eqn}
which gives the population change rate
\begin{eqn}
	\frac{\upd N_i}{\upd t}
	={} & - \sum_{\lvec \in L} \kappa_{\lvec} \int \upd\xvec \pathavgleft
		2 \frac{\upp O_{\lvec}^*}{\upp \Psi_i^*} O_{\lvec} \Psi_i^*
			- \sum_{k=1}^C \delta_{\restbasis_k}(\xvec, \xvec)
				\frac{\upp^2 O_{\lvec}^*}{\upp \Psi_i^* \upp \Psi_k^*}
				\frac{\upp O_{\lvec}}{\upp \Psi_k}
				\Psi_i^*
		\right. \\
	& \quad \left.
		- \frac{\partial O_{\lvec}}{\partial \Psi_i}
			\frac{\partial O_{\lvec}^*}{\partial \Psi_i^*}
			\delta_{\restbasis_i}(\xvec, \xvec)
	\pathavgright,
\end{eqn}
where we have used the fact that $O_{\lvec}$ are products of integer powers of $\Psi_j$, which makes $\mathcal{A}_i^{\mathrm{(loss)}} \Psi_j^* \equiv (\mathcal{A}_i^{\mathrm{(loss)}})^* \Psi_j$.
\end{proof}

\todo{It should be possible to proof that this formula agrees with the classical one up to the $1/N$ order, same as we do for particular cases later or in the theoretical paper.}

In practice, it is more convenient to use the above equation using more intuitive quantities, namely real component densities $n_i = \langle \Psiop_i^\dagger \Psiop_i \rangle$.
To do that, we have to rewrite the resulting path averages of the moments of $\Psi_j$ as averages of symmetric products of $\Psiop_j$, transform them to the normal order using the analogue of the ordering transformation formula~\cite{Cahill1969} for field operators
\begin{eqn}
	\symprod{
		(\Psiop_j^\dagger)^r \Psiop_j^s
	}
	= \sum_{k=0}^{\min(r,s)} \frac{k!}{2^{k}} \binom{r}{k} \binom{s}{k}
		(\Psiop^\dagger)^{r-k} \Psiop^{s-k} \delta_{\restbasis_j}^k,
\end{eqn}
and simplify the resulting averages of normally ordered operators using correlation factors $g^{(k)} = \langle (\Psiop^\dagger)^k \Psiop^k \rangle / \langle \Psiop^\dagger \Psiop \rangle$.
In other words, coefficients $g^{(k)}$ define the degree of high-order correlations.
In the ideal condensate they are equal to $1$, and increase with the temperature (the so called photon bunching, or atom bunching).
For example, the theoretical maximum value for $g_{111}$ is $3!=6$~\cite{Kagan1985}, which has been confirmed experimentally~\cite{Burt1997}.

% =============================================================================
\section{Initial states}
% =============================================================================

Before integrating the evolution equations~\eqnref{wigner-bec:fpe-bec:sde} or~\eqnref{wigner-bec:fpe-bec:sde-stratonovich}, the initial value of $\Psi_j$ at $t=0$ has to be sampled.
The general procedure is to take the density matrix of the desired initial state and find its Wigner transformation using \defref{wigner:mc:w-transformation}.
The resulting Wigner function is then sampled.


% =============================================================================
\subsection{Coherent state}
% =============================================================================

The simplest case of an initial state is a coherent state.

\begin{theorem}
\label{thm:wigner-bec:initial-state:coherent-state}
	The Wigner distribution for a multimode coherent state with the expectation value $\Psi^{(0)} \equiv \sum_{\nvec \in restbasis} \alpha_{\nvec}^{(0)} \phi_{\nvec}$ is
	\begin{eqn*}
		W_{\mathrm{coh}} [\Psi]
		= \left( \frac{2}{\pi} \right)^{|\restbasis|} \prod_{\nvec \in \restbasis}
			\exp(-2 |\alpha_{\nvec} - \alpha_{\nvec}^{(0)}|^2).
	\end{eqn*}
\end{theorem}
\begin{proof}
The density matrix of the state is
\begin{eqn}
	\hat{\rho}
	= \vert \alpha_{\nvec}^{(0)},\, \nvec \in \restbasis \rangle
		\langle \alpha_{\nvec}^{(0)},\, \nvec \in \restbasis \vert
	= \left( \prod_{\nvec \in \restbasis} \vert \alpha_{\nvec}^{(0)} \rangle \right)
		\left( \prod_{\nvec \in \restbasis} \langle \alpha_{\nvec}^{(0)} \vert \right).
\end{eqn}
Then the characteristic functional for this state can be expressed as
\begin{eqn}
	\chi_W [\Lambda]
	& = \Trace{
		\left( \prod_{\nvec \in \restbasis} \vert \alpha_{\nvec}^{(0)} \rangle \right)
		\left( \prod_{\nvec \in \restbasis} \langle \alpha_{\nvec}^{(0)} \vert \right)
		\left( \prod_{\nvec \in \restbasis} \hat{D}_{\nvec} (\lambda_{\nvec}, \lambda_{\nvec}^*) \right)
	} \\
	& = \prod_{\nvec \in \restbasis}
		\langle \alpha_{\nvec}^{(0)} \vert
		\hat{D}_{\nvec} (\lambda_{\nvec}, \lambda_{\nvec}^*)
		\vert \alpha_{\nvec}^{(0)} \rangle,
\end{eqn}
where $\Lambda = \sum_{\nvec \in \restbasis} \lambda_{\nvec} \phi_{\nvec}$.
Displacement operators obey the multiplication law~\cite{Cahill1969}
\begin{eqn}
	\hat{D}(\lambda, \lambda^*) \hat{D}(\alpha, \alpha^*)
	= \hat{D}(\lambda + \alpha, \lambda^* + \alpha^*)
		\exp(\frac{1}{2} (\lambda \alpha^* - \lambda^* \alpha)),
\end{eqn}
and the scalar product of two coherent state is calculated as~\cite{Cahill1969}
\begin{eqn}
	\langle \beta \vert \alpha \rangle
	= \exp(-\frac{1}{2} |\alpha|^2 - \frac{1}{2} |\beta|^2 + \beta^* \alpha).
\end{eqn}
Therefore,
\begin{eqn}
	\hat{D}(\lambda, \lambda^*) \vert \alpha \rangle
	& = \hat{D}(\lambda, \lambda^*) \hat{D}(\alpha, \alpha^*) \vert 0 \rangle \\
	& = \exp(\frac{1}{2} (\lambda \alpha^* - \lambda^* \alpha))
		\vert \lambda + \alpha \rangle,
\end{eqn}
and the characteristic functional can be simplified as:
\begin{eqn}
	\chi_W [\Lambda]
	& = \prod_{\nvec \in \restbasis}
		\exp(\frac{1}{2} (\lambda_{\nvec} (\alpha_{\nvec}^{(0)})^*
			- \lambda_{\nvec}^* \alpha_{\nvec}^{(0)}))
		\langle \alpha_{\nvec}^{(0)} \vert \lambda_{\nvec} + \alpha_{\nvec}^{(0)} \rangle \\
	& = \prod_{\nvec \in \restbasis}
		\exp(
			- \lambda_{\nvec}^* \alpha_{\nvec}^{(0)}
			+ \lambda_{\nvec} (\alpha_{\nvec}^{(0)})^*
			- \frac{1}{2} |\lambda|^2
		).
\end{eqn}

Finally, the Wigner functional is
\begin{eqn}
	W_c [\Psi]
	& = \frac{1}{\pi^{2|\restbasis|}} \prod_{\nvec \in \restbasis} \left(
		\int \upd^2\lambda_{\nvec}
			\exp(
				- \lambda_{\nvec} (\alpha_{\nvec}^* - (\alpha_{\nvec}^{(0)})^*)
				+ \lambda_{\nvec}^* (\alpha_{\nvec} - \alpha_{\nvec}^{(0)})
				- \frac{1}{2} |\lambda|^2
			)
	\right) \\
	& = \left( \frac{2}{\pi} \right)^{|\restbasis|} \prod_{\nvec \in \restbasis}
		\exp(-2 |\alpha_{\nvec} - \alpha_{\nvec}^{(0)}|^2).
	\qedhere
\end{eqn}
\end{proof}

The resulting Wigner distribution is a product of independent complex-valued Gaussian distributions for each mode, with an expectation value equal to the expectation value of the mode, and the variance equal to $\frac{1}{2}$.
Therefore, the initial state can be sampled as
\begin{eqn}
	\alpha_{\nvec} = \alpha_{\nvec}^{(0)} + \frac{1}{\sqrt{2}} \eta_{\nvec},
\end{eqn}
where $\eta_{\nvec}$ are normally distributed complex random numbers with zero mean, $\langle \eta_{\mvec} \eta_{\nvec} \rangle = 0$ and $\langle \eta_{\mvec} \eta_{\nvec}^* \rangle = \delta_{\mvec,\nvec}$, or, in other words, with components distributed independently with variance $\frac{1}{2}$.
This looks like adding half a ``vacuum particle'' to each mode.
In the functional form, this can be written as
\begin{eqn}
	\Psi_j(\xvec, 0)
	= \Psi_j^{(0)}(\xvec, 0)
		+ \sum_{\nvec \in \restbasis} \frac{\eta_{j,\nvec}}{\sqrt{2}} \phi_{\nvec}(\xvec),
\end{eqn}
where $\Psi_j^{(0)}(\xvec, 0)$ is the ``classical'' ground state of the system.


% =============================================================================
\subsection{Other cases}
% =============================================================================

In particular, a numerically efficient way to sample a Wigner distribution for Bogoliubov states was developed by Sinatra~\textit{et~al}~\cite{Sinatra2002}.
More involved examples, including thermalized states and Bogoliubov states, are reviewed by Blakie~\textit{et~al}~\cite{Blakie2008}, Olsen and Bradley~\cite{Olsen2009}, and Ruostekoski and Martin~\cite{Ruostekoski2010}.



\appendix
\part*{Appendices}

% =============================================================================
\chapter{Wirtinger differentiation}
% =============================================================================

Formally, a function of complex variable has to be holomorphic in order to be complex differentiable.
In many cases it is enough to have less strict ``physicists'\,'' complex differentiation rules.
These rules were developed by Wirtinger in~\cite{Wirtinger1927};
a very good review and a thorough description of their application can be found in~\cite{Kreutz-Delgado2009}.
Further extension of these rules to vectors and matrices is done in~\cite{Hjorungnes2007}.
This section will outline these rules and provide some lemmas based on them, which will be used further.

\begin{definition}
\label{def:c-numbers:wirtinger}
	For a complex variable $z = x + iy$ and a function $f(z) = u(x, y) + iv(x, y)$
	\begin{eqn*}
		\left( \frac{\partial f(z)}{\partial z} \right)
		= \frac{1}{2} \left(
			\frac{\partial f}{\partial x} - i \frac{\partial f}{\partial y}
		\right).
	\end{eqn*}
\end{definition}

\begin{lemma}
	If $f(z)$ is holomorphic, then the Wirtinger differentiation is equivalent to the standard definition.
\end{lemma}

\begin{lemma}
	For any ``good'' (even non-holomorphic) $f(z)$, Wirtinger differentiation obeys sum, product, quotient, and chain differentiation rules
	(the former one is applied as if $f(z) \equiv f(z, z^*)$).
\end{lemma}

Hereinafter we will use Wirtinger differentiation unless explicitly stated otherwise,
because some important functions we will encounter are not holomorphic.
This differentiation has all intuitively assumed properties, along with some not quite obvious ones.

\begin{lemma}
	For any nonnegative integers $a$ and $b$.
	\begin{eqn*}
		\frac{\partial}{\partial z} (z^a (z^*)^b) = a z^{a-1} (z^*)^b,
		\quad
		\frac{\partial}{\partial z^*} (z^a (z^*)^b) = b z^a (z^*)^{b-1},
	\end{eqn*}
\end{lemma}
\begin{proof}
Let us assume that the statement of the lemma is valid for some $a$ and $b$, then using product rule
\begin{eqn}
	\frac{\partial}{\partial z} (z^{a+1} (z^*)^b)
	& = \frac{\partial}{\partial z} (z z^a (z^*)^b)
	= z^a (z^*)^b + z \frac{\partial}{\partial z} (z^a (z^*)^b) \\
	& = z^a (z^*)^b + a z z^{a-1} (z^*)^b
	= (a + 1) z^a (z^*)^b.
\end{eqn}
The part for $\partial/\partial z^*$ can is proved in the same way.
One can easily prove (by transition to real values) that $\partial(z z^*)/\partial z = z^*$ and $\partial (z z^*)/\partial z^* = z$.
By induction, the statement is true for any natural $a$ and $b$,
and it is obviously true if $a = 0$ or $b = 0$, which proves the lemma.
\todo{This can be proved for any real $a$ and $b$, if necessary.}
\end{proof}

This is straightforwardly followed by
\begin{lemma}
\label{lmm:c-numbers:independent-vars}
	If $f(z)$ can be expanded into series of $z^n (z^*)^m$, $\partial f(z)/\partial z$ and $\partial f(z)/\partial z^*$ can be treated as partial differentiation of the function of two independent variables $z$ and $z^*$.
	In other words:
	\begin{eqn*}
		\frac{\partial}{\partial z} f(z, z^*) \equiv \frac{\partial}{\partial u} f(u, v),
		\quad
		\frac{\partial}{\partial z^*} f(z, z^*) \equiv \frac{\partial}{\partial v} f(u, v).
	\end{eqn*}
\end{lemma}

Therefore, although technically $f(z, z^*)$ depends on one complex variable, when differentiated it behaves as if it was a function of two complex variables.
Now we can prove two lemmas which will help us deal with some integrals.

\begin{definition}
	For a complex variable $z = x + iy$ the integral
	\begin{eqn*}
		\int d^2 z \equiv \int_{-\infty}^{\infty} \int_{-\infty}^{\infty} dx\, dy,
	\end{eqn*}
	or, in other words, stands for the two-dimensional integral over the complex plane.
\end{definition}

\begin{lemma}
\label{lmm:c-numbers:fourier-of-moments}
	For complex $\lambda$ and any non-negative integers $r$ and $s$:
	\begin{eqn*}
		\int d^2\alpha\, \alpha^r (\alpha^*)^s \exp(-\lambda \alpha^* + \lambda^* \alpha)
		= \pi^2
			\left( -\frac{\partial}{\partial \lambda^*} \right)^r
			\left( \frac{\partial}{\partial \lambda} \right)^s
			\delta(\Real \lambda) \delta(\Imag \lambda)
	\end{eqn*}
\end{lemma}
\begin{proof}
First, changing the variables in the integrals and using known Fourier transform relations, we can prove that for real $x$ and $v$, and non-negative integer $n$
\begin{eqn}
\label{eqn:c-numbers:fourier-real}
	\int\limits_{-\infty}^{\infty} dv\, v^n \exp(\pm 2 i x v)
	= \pi (\mp i / 2)^n \delta^{(n)}(x),
\end{eqn}
Note that we have explicitly written integration limits here;
they are swapped when we change the variable in the first integral.

Denoting $\alpha = u + iv$ and $\lambda = x + iy$, we can expand the initial expression as
\begin{eqn}
	& \int d^2\alpha\, \alpha^r (\alpha^*)^s \exp(-\lambda \alpha^* + \lambda^* \alpha) \\
	& = \int du dv \exp(2ivx - 2iuy)
		\sum_{l=0}^r \binom{r}{l} u^l (iv)^{r-l}
		\sum_{m=0}^s \binom{s}{m} u^m (-iv)^{s-m} \\
	& = \sum_{l=0}^r \sum_{m=0}^s \binom{r}{l} \binom{s}{m}
		i^{r-l} (-i)^{s-m}
		\int du\, u^{l+m} \exp(2ivx)
		\int dv\, v^{r-l+s-m} \exp(-2iuy).
\end{eqn}
Applying~\eqnref{c-numbers:fourier-real} and grouping differentials:
\begin{eqn}
	& = \pi^2 \sum_{l=0}^r \sum_{m=0}^s \binom{r}{l} \binom{s}{m}
		i^{r-l} (-i)^{s-m}
		(-i/2)^{l+m} \delta^{(l+m)}(y)
		(i/2)^{r-l+s-m} \delta^{(r-l+s-m)}(x) \\
	& = \pi^2
		\sum_{l=0}^r \binom{r}{l}
			\frac{1}{2^r}
			(-i \partial / \partial y)^l
			(-\partial / \partial x)^{r-l}
		\sum_{m=0}^s \binom{s}{m}
			\frac{1}{2^s}
			(-i \partial / \partial y)^m
			(\partial / \partial x)^{s-m}
		\delta(y) \delta(x).
\end{eqn}
Collapsing sums and recognizing \defref{c-numbers:wirtinger}:
\begin{eqn}
	& = \pi^2
		\left( \frac{1}{2} (-i \partial / \partial y - \partial / \partial x) \right)^r
		\left( \frac{1}{2} (-i \partial / \partial y + \partial / \partial x) \right)^s
		\delta(y) \delta(x) \\
	& = \pi^2
		\left( -\frac{\partial}{\partial \lambda^*} \right)^r
		\left( \frac{\partial}{\partial \lambda} \right)^s
		\delta(\Real \lambda) \delta(\Imag \lambda).
		\qedhere
\end{eqn}
\end{proof}

\begin{lemma}
\label{lmm:c-numbers:zero-integrals}
	If $f(\lambda, \lambda^*)$ is bounded, then for any complex $\alpha$:
	\begin{eqn*}
		\int d^2\lambda
			\frac{\partial}{\partial \lambda} \left(
				\exp(-\lambda \alpha^* + \lambda^* \alpha)
				f(\lambda, \lambda^*)
			\right)
		& = 0 \\
		\int d^2\lambda
			\frac{\partial}{\partial \lambda^*}
			\left(
				\exp(-\lambda \alpha^* + \lambda^* \alpha)
				f(\lambda, \lambda^*)
			\right)
		& = 0.
	\end{eqn*}
\end{lemma}
\begin{proof}
\todo{Needs proof!}
\end{proof}

\begin{lemma}
\label{lmm:c-numbers:zero-delta-integrals}
	\todo{Any limitations on $f$? It is bounded (on account of being $\chi_W$), for one.}
	\begin{eqn*}
		\int d^2\lambda
			\frac{\partial}{\partial \lambda} \left(
				\left(
					\left( \frac{\partial}{\partial \lambda} \right)^s
					\left( -\frac{\partial}{\partial \lambda^*} \right)^r
					\delta(\Real \lambda) \delta(\Imag \lambda)
				\right)
				f(\lambda, \lambda^*)
			\right)
		& = 0 \\
		\int d^2\lambda
			\frac{\partial}{\partial \lambda^*}
			\left(
				\left(
					\left( \frac{\partial}{\partial \lambda} \right)^s
					\left( -\frac{\partial}{\partial \lambda^*} \right)^r
					\delta(\Real \lambda) \delta(\Imag \lambda)
				\right)
				f(\lambda, \lambda^*)
			\right)
		& = 0.
	\end{eqn*}
\end{lemma}
\begin{proof}
Proved straightforwardly by expanding integrals in real values, separating variables and integrating, using the fact that any derivative of the delta function is zero on the infinity.
\end{proof}

\chapter{Natural units}
\label{cha:appendix:natural-units}

Nondimensionalization can be performed in many ways;
the parameters of the harmonic trap are the natural choice for the new units.
Taking into account the usual symmetry $\omega_x = \omega_y = \omega_\rho$ which exists in experiments,
we can set the following units for length, time and energy:
\[
	l_\rho = \sqrt{\frac{\hbar}{m\omega_\rho}},\,
	t_\rho = \frac{1}{\omega_\rho},\,
	E_\rho = \hbar \omega_\rho.
\]

Wave function can be made dimensionless differently depending on the normalisation.
We will use the transformation
\[
	\psi = \psi^\prime l_\rho^\frac{3}{2},
\]
that keeps the integral of wave function over space equal to number of particles in the system:
\[
	\int\limits_V \psi dV = \int\limits_V \psi^\prime dV^\prime = N,
\]
where the variables with primes are dimensionless.

\chapter{Notation}
\label{cha:appendix:notation}

This chapters contains general principles of notation used in this work.

\paragraph{Function types.}
In some places types of functions, functionals and operators are written explicitly for the sake of clarification.
They are expressed using Haskell style for function types.
Basic types are complex scalar $\mathbb{C}$ and Hilbert space element $\mathbb{H}$.
Thus, for example, common complex-valued function has type $f(z) :: \mathbb{C} \rightarrow \mathbb{C}$,
wavefunction in $D$-dimensional space has type $\Psi(\xvec) :: \mathbb{C}^D \rightarrow \mathbb{C}$,
and field annihilation operator has type $\Psiop(\xvec) :: \mathbb{C}^D \rightarrow \mathbb{H}$.
As an example of slightly more complex dependency,
an integration functional has type $\int d\xvec f(\xvec) :: (\mathbb{C}^D \rightarrow \mathbb{C}) \rightarrow \mathbb{C}$,
because it maps function to complex number.

\paragraph{Operators.}
There is an ambiguity associated with the word ``operator'':
in the context of this work it could mean either element of Hilbert space, or a mapping from function to function.
Strictly speaking, these are the same, but in many cases it is more convenient to extract quantum-mechanical operators in a separate group, without going into details about what they actually do.
Therefore we are using the term ``operator'' for quantum-mechanical operators,
and term ``transformation'' for anything else we define.
Note that function that takes an operator as one of the parameters is a transformation too.

\paragraph{Notation for operators and transformations.}
In order to distinguish operators and transformations, they are written in different form.
Transformations use letters from calligraphic set, for instance $\mathcal{W}$ or $\mathcal{F}$.
Anything which has $\mathbb{H}$ as a final parameter is marked by hat symbol.
This applies for transformations which are known to produce operator result in known context;
for example, projection $\mathcal{P}$ can take either function or operator function,
and in the latter case it is marked with the hat: $\mathcal{P}[\Psi]$, but $\hat{\mathcal{P}}[\Psiop]$.

\paragraph{Operands.}
In general, functions can have multiple operands with basic types, vector types or function types.
Operands with types $\mathbb{C}$ and $\mathbb{C}^D$ are written in parentheses,
while all other operands are written in square brackets.
For example, in case of projection $\mathcal{P}[\Psi] \equiv \mathcal{P}[\Psi](\xvec)$.
Operands can be partially applied.
Single-mode Wigner transformation has type $\mathcal{W}[\hat{A}](\alpha) :: \mathbb{H} \rightarrow \mathbb{C} \rightarrow \mathbb{C}$.
Therefore Wigner function, which is a Wigner transformation applied to density matrix, can be considered a partially applied $\mathcal{W}$,
which, as it is obvious from its type, is a complex-valued function: $W(\alpha) \equiv \mathcal{W}[\hat{\rho}](\alpha)$.

\chapter{Split-step propagation}
\label{cha:appendix:split-step}


% =============================================================================
\section{General form}
% =============================================================================

Split-step propagation method~\citationneeded is used to obtain the numerical solution of the following differential equation:
\begin{equation}
\label{eqn:split-step:general-eqn}
	\frac{d\Psi(\xvec, t)}{dt} = \left(
		\hat{D}(\nabla) + \hat{N}(\xvec, t)
	\right) \Psi(\xvec, t)
\end{equation}
given the initial state $\psi(\xvec, 0)$,
where $\hat{D}$ is the differential operator, and $\hat{N}$ is the nonlinear operator,
which is just a common function.

The equation~\eqnref{split-step:general-eqn} can be rewritten as:
\[
	\Psi(\xvec, t + dt) \simeq \exp ( ( \hat{D} + \hat{N} ) dt ) \Psi(\xvec, dt).
\]
The idea of split-step Fourier method is to separate the calculation of both terms:
\begin{equation}
\label{eqn:split-step:split-eqn}
	\Psi(\xvec, t + dt) \simeq \exp(dt \hat{D}) \exp(dt \hat{N}) \Psi(\xvec, t)
\end{equation}
and calculate the differential factor in Fourier domain, where spatial derivative can be replaced by simple multiplication:
\[
	\Psi(\xvec, t + dt) \simeq \left\{
		\hat{F}^{-1} \exp \left[
			d\tau \hat{D}(i k)
		\right] \hat{F}
	\right\}
	\exp(dt \hat{N}) \Psi(\xvec, t).
\]
Here $\hat{D}(i k)$ is obtained by replacing differential operator by $i k$,
where $k$ is a spatial frequency in Fourier domain.
The equation is approximate, because splitting operators in this way ignores the fact that they do not commute.

In order to improve the accuracy of the method, equation~\eqnref{split-step:split-eqn} can be rewritten as~\cite{Sinkin2003}
\[
	\Psi(\xvec, t + dt) \simeq
	\exp \left( \frac{dt}{2} \hat{D} \right)
	\exp \left( \int\limits^{t + dt}_t \hat{N} (t^\prime) dt^\prime \right)
	\exp \left( \frac{dt}{2} \hat{D} \right) \Psi(\xvec, t).
\]
This method is called the symmetrized split-step Fourier method, and it has the global error of $\mbox{O}(dt^2)$.
Integral can be evaluated either by approximating it with $dt \hat{N}(t)$ or by using slightly more accurate method,
like trapezoidal rule
\[
	\int\limits^{t + dt}_t \hat{N} (t^\prime) dt^\prime \simeq
	\frac{dt}{2} \left( \hat{N}(t) + \hat{N}(t + dt) \right).
\]
Since the value of $\hat{N}(t + dt)$ is unknown at the time of the calculation
(it is performed in the middle of the step), an iterative procedure is necessary.


% =============================================================================
\section{Two-component form}
% =============================================================================

The case of two-component cGPEs with coupling terms (like~\eqnref{mean-field:cgpes_simplified})
has to be described in detail.
We have the following system of equations:
\begin{equation}
\label{eqn:split-step:two-comp-eqn}
	\frac{d \Psivec}{dt} = \hat{D} \Psivec + \hat{N} \Psivec,
\end{equation}
where $\Psivec$ is a vector of two functions $(\Psi_1, \Psi_2)$,
$\hat{D}$ is a differential operator:
\[
	\hat{D} = \begin{pmatrix}
		\frac{i \hbar}{2 m} \nabla^2 & 0 \\
		0 & \frac{i \hbar}{2 m} \nabla^2
	\end{pmatrix},
\]
and $\hat{N}$ is a nonlinear operator:
\[
	\hat{N} = \begin{pmatrix}
		-\frac{i}{\hbar} \left( V + g_{11} n_1 + g_{12} n_2 \right) - \Gamma_1 &
		-i \frac{\Omega}{2} e^{-i (\delta t + \alpha)} \\
		-i \frac{\Omega}{2} e^{i (\delta t + \alpha)} &
		-\frac{i}{\hbar} \left( V + g_{12} n_1 + g_{22} n_2 \right) - \Gamma_2
	\end{pmatrix}
	= \begin{pmatrix}
		N_1 & -i \frac{\Omega}{2} e^{-i \phi} \\ - i \frac{\Omega}{2} e^{i \phi} & N_2
	\end{pmatrix}
\]

In order to integrate equation~\eqnref{split-step:two-comp-eqn},
we have to calculate the matrix exponent $\exp \hat{N}$.
Eigenvalues:
\begin{equation*}
\begin{split}
	\lambda_1 & = \frac{1}{2} \left(
		N_1 + N_2 - \sqrt{(N_1 - N_2)^2 - \Omega^2}
	\right), \\
	\lambda_2 & = \frac{1}{2} \left(
		N_1 + N_2 + \sqrt{(N_1 - N_2)^2 - \Omega^2}
	\right). \\
\end{split}
\end{equation*}
Eigenvectors:
\begin{equation*}
\begin{split}
	v_1 = \begin{pmatrix}
		1 \\
		-\frac{i e^{i \phi}}{\Omega} \left(
			\sqrt{(N_1 - N_2)^2 - \Omega^2} + N_1 - N_2
		\right)
	\end{pmatrix}
	= \begin{pmatrix}
		1 \\ a_1
	\end{pmatrix}, \\
	v_2 = \begin{pmatrix}
		1 \\
		\frac{i e^{i \phi}}{\Omega} \left(
			\sqrt{(N_1 - N_2)^2 - \Omega^2} - N_1 + N_2
		\right)
	\end{pmatrix}
	= \begin{pmatrix}
		1 \\ a_2
	\end{pmatrix}.
\end{split}
\end{equation*}
Thus
\begin{equation}
\begin{split}
	\exp \hat{N} & = \begin{pmatrix}
		1 & 1 \\ a_1 & a_2
	\end{pmatrix}
	\begin{pmatrix}
		e^{\lambda_1} & 0 \\ 0 & e^{\lambda_2}
	\end{pmatrix}
	\begin{pmatrix}
		1 & 1 \\ a_1 & a_2
	\end{pmatrix}^{-1} \\
	& = \frac{1}{a_1 - a_2}
	\begin{pmatrix}
		1 & 1 \\ a_1 & a_2
	\end{pmatrix}
	\begin{pmatrix}
		e^{\lambda_1} & 0 \\ 0 & e^{\lambda_2}
	\end{pmatrix}
	\begin{pmatrix}
		-a_2 & 1 \\ a_1 & -1
	\end{pmatrix} \\
	& = \frac{1}{a_1 - a_2}
	\begin{pmatrix}
		e^{\lambda_2} a_1 - e^{\lambda_1} a_2 &
		e^{\lambda_1} - e^{\lambda_2} \\
		a_1 a_2 (e^{\lambda_2} - e^{\lambda_1}) &
		e^{\lambda_1} a_1 - e^{\lambda_2} a_2.
	\end{pmatrix}
\end{split}
\end{equation}
This formula now can be used in split-step integration~\eqnref{split-step:split-eqn}.

\chapter{Minimal lattice}
\label{cha:appendix:minimal-lattice}

This chapter shows how to calculate minimal lattice which contains all modes below given cutoff $\ecut$.


% =============================================================================
\section{Uniform grid}
% =============================================================================

In case of 1D uniform grid, the cutoff condition looks like:
\[
	\frac{\hbar^2 k^2}{2 m} \le \ecut,
\]
\[
	k \le \sqrt{\frac{2 m \ecut}{\hbar^2}} = k_{\mathrm{cut}}.
\]
In FFT algorithm, maximum spatial frequency for given problem size $N$ and lattice step $d$ can be found as:
\[
	k_{\max} = 2 \pi \frac{ \left[ \frac{N}{2} \right] }{N d}.
\]
If $L = d (N - 1)$ is fixed, the condition for even $N$ is easier to find first:
\[
	N_{\min,\mathrm{even}} = 2 \lceil
		\frac{k_{\mathrm{cut}} L}{2 \pi} + \frac{1}{2}
	\rceil.
\]
Then, one should check whether $k_{\max}$ corresponding to $N_{\min,\mathrm{even}} - 1$ is still larger than $k_{\mathrm{cut}}$; if it is, it should be used as minimal lattice size instead.
In 3D case, $N_{\min}$ can be calculated separately for each dimension using corresponding values of $L$.


% =============================================================================
\section{Harmonic grid}
% =============================================================================

In 1D case, the cutoff condition is:
\[
	\hbar \omega ( n + \frac{1}{2} ) \le \ecut,
\]
where $\omega$ is the trap frequency.
Therefore minimal lattice size can be found as:
\[
	N_{\min} = \lceil \frac{\ecut}{\hbar \omega} - \frac{1}{2} \rceil.
\]
3D case is a bit more complicated:
\[
	\hbar \left(
		\omega_x (n_x + \frac{1}{2})
		+ \omega_y (n_y + \frac{1}{2})
		+ \omega_z (n_z + \frac{1}{2})
	\right) \le \ecut,
\]
and minimal lattice sizes are:
\begin{equation*}
\begin{split}
	N_{x,\min} & = \lceil
		\frac{\ecut / \hbar - \omega_y / 2 - \omega_z / 2}{\omega_x} - \frac{1}{2}
	\rceil, \\
	N_{y,\min} & = \lceil
		\frac{\ecut / \hbar - \omega_x / 2 - \omega_z / 2}{\omega_y} - \frac{1}{2}
	\rceil, \\
	N_{z,\min} & = \lceil
		\frac{\ecut / \hbar - \omega_x / 2 - \omega_y / 2}{\omega_z} - \frac{1}{2}
	\rceil.
\end{split}
\end{equation*}

\chapter{Moments of field operator in Wigner representation}
\label{cha:appendix:moments-calculation}


This section shows how to calculate different kinds of observables from wavefunctions in Wigner representation.
From the definition of Wigner function~\cite{Gardiner2004}:
\[
	\langle \symprod{ \hat{a}^r ( \hat{a}^\dagger)^s } \rangle
	= \int \alpha^r (\alpha^*)^s W (\alpha, \alpha^*) d^2\alpha ,
\]
where $\{\}_{\mathrm{sym}}$ stands for symmetrically ordered operator product.
It can be shown that similar relation applies for the multimode field operator:
\[
	\langle \symprod{ \Psiop^r ( \Psiop^\dagger)^s } \rangle
	= \int \Psi^r (\Psi^*)^s W (\Psi, \Psi^*) \delta^2\Psi.
\]
This equation can be further generalised for multi-component field.
In simulations, Wigner function $W$ can be treated as the probability distribution, allowing to replace the integral by average over simulation paths:
\[
	\int \Psi^r (\Psi^*)^s W (\Psi, \Psi^*) \delta^2\Psi
	= \pathavg{ \Psi^r (\Psi^*)^s }
	= \frac{1}{N_{\mathrm{paths}}} \sum\limits_{j=1}^{N_{\mathrm{paths}}}
		\Psi^{(j)r} (\Psi^{(j)*})^s,
\]
where superscript $(j)$ denotes the value taken from $j$-th simulation path.


% =============================================================================
\section{Number of atoms}
% =============================================================================

First example is the calculation of atom density:
\begin{equation*}
\begin{split}
		\langle \hat{n} (\xvec) \rangle
		& = \langle \Psiop^\dagger (\xvec) \Psiop (\xvec) \rangle \\
		& = \langle
				\symprod{ \Psiop^\dagger \Psiop }
			\rangle - \frac{1}{2} \delta_P (\xvec, \xvec) \\
		& = \pathavg{ \Psi^* (\xvec) \Psi (\xvec) }
			- \frac{1}{2} \delta_P (\xvec, \xvec) \\
		& = \pathavg{ n (\xvec) }
			- \frac{1}{2} \delta_P (\xvec, \xvec).
\end{split}
\end{equation*}
Defining population operator $\hat{N}$ as
\[
	\hat{N} = \int \hat{n} (\xvec) d\xvec,
\]
we can get the average of total population:
\[
		\langle \hat{N} \rangle
		= \int \langle \hat{n}(\xvec) \rangle d\xvec
		= \int \pathavg{ n(\xvec) } d\xvec - \frac{M}{2}
		= \pathavg{ \int n(\xvec) d\xvec } d\xvec - \frac{M}{2}
		= \pathavg{N} - \frac{M}{2},
\]
where $\delta_P (\xvec, \xvec)$ is a restricted delta function~\eqnref{multimode-formalism:restricted-delta}.
Its integral over space equals to the number of modes $M$ in the restricted basis.

Variance of total number $N$ is expressed in a slightly more complicated way.
\[
	(\Delta N)^2
		= \langle \hat{N}^2 \rangle - \langle \hat{N} \rangle^2
\]
Average of $\hat{N}^2$ requires some work.
Denoting $\Psiop(\xvec) \equiv \Psiop$ and $\Psiop(\xvec^\prime) \equiv \Psiop^\prime$ for simplicity:
\[
	\hat{N}^2
		= \int \Psiop^\dagger \Psiop d\xvec
			\int \Psiop^\dagger \Psiop d\xvec
		= \int
			\Psiop^\dagger \Psiop
			\Psiop^{\prime\dagger} \Psiop^\prime
			d\xvec d\xvec^\prime
\]
\begin{equation*}
\begin{split}
	\langle
		\Psiop^\dagger \Psiop \Psiop^{\prime\dagger} \Psiop^\prime
	\rangle
	& = \langle
		\symprod{ \Psiop^{\prime\dagger} \Psiop^\prime \Psiop^\dagger \Psiop}
		- \frac{\delta_P(\xvec^\prime,\xvec^\prime)}{2} \symprod{\Psiop^\dagger \Psiop}
		- \frac{\delta_P(\xvec,\xvec)}{2} \symprod{\Psiop^{\prime\dagger} \Psiop^\prime} \\
	& - \frac{\delta_P(\xvec,\xvec^\prime)}{2} \symprod{\Psiop^{\prime\dagger} \Psiop}
		+ \frac{\delta_P(\xvec^\prime,\xvec)}{2} \symprod{\Psiop^\dagger \Psiop^\prime}
		+ \frac{\delta_P(\xvec,\xvec) \delta_P(\xvec^\prime,\xvec^\prime)}{2}
	\rangle.
\end{split}
\end{equation*}
Therefore the average of $\hat{N}^2$ is:
\begin{equation*}
\begin{split}
	\langle \hat{N}^2 \rangle & = \int
		\langle
			\Psiop^\dagger \Psiop \Psiop^{\prime\dagger} \Psiop^\prime
		\rangle
	d\xvec d\xvec^\prime \\
	& = \int \pathavg{
		\Psi^* \Psi \Psi^{\prime *} \Psi^\prime
		- \frac{\delta_P(\xvec^\prime,\xvec^\prime)}{2} \Psi^* \Psi
		- \frac{\delta_P(\xvec,\xvec) }{2} \Psi^{\prime *} \Psi^\prime \\
	&	- \frac{\delta_P(\xvec,\xvec^\prime)}{2} \Psi^{\prime *} \Psi
		+ \frac{\delta_P(\xvec^\prime,\xvec)}{2} \Psi^* \Psi^\prime
		+ \frac{\delta_P(\xvec,\xvec) \delta_P(\xvec^\prime,\xvec^\prime)}{2}
	} d\xvec d\xvec^\prime \\
	& = \pathavg{
		\int \Psi^* \Psi d\xvec \int \Psi^* \Psi d\xvec
		- \frac{M}{2} \int \Psi^* \Psi d\xvec
		- \frac{M}{2} \int \Psi^* \Psi d\xvec \\
	&	- \int \frac{\delta_P(\xvec,\xvec^\prime)}{2} \Psi^{\prime *} \Psi d\xvec d\xvec^\prime
		+ \int \frac{\delta_P(\xvec^\prime,\xvec)}{2} \Psi^* \Psi^\prime d\xvec d\xvec^\prime
		+ \frac{M^2}{2}
	} \\
	& = \pathavg{N^2 - M N + \frac{M^2}{2}}.
\end{split}
\end{equation*}
Here we used the correspondence between the average of the symmetric product of field operators and the average of wavefunctions over simulation paths.
Then we can split variables in double integrals, allowing us to group terms and simplify the whole equation.
Substituting this into equation for $(\Delta N)^2$:
\begin{equation}
\label{eqn:moments-calculation:delta-N}
	(\Delta N)^2
		= \langle \hat{N}^2 \rangle - \langle \hat{N} \rangle^2
		= \pathavg{N^2 - M N + \frac{M^2}{2}} - (\pathavg{N} - \frac{M}{2})^2
		= \pathavg{N^2} - \pathavg{N}^2 + \frac{M^2}{4}
\end{equation}


% =============================================================================
\section{Spin vector}
% =============================================================================

Another example is the spin vector, whose averages and variances are required for squeezing calculation~\cite{Li2009}.
Spin operators are defined as following:
\begin{equation}
\label{eqn:moments-calculation:spin-operators}
\begin{split}
	\hat{S}_x & = \frac{1}{2} \int \left(
		\Psiop^\dagger_2 \Psiop_1 + \Psiop^\dagger_1 \Psiop_2
	\right) d\xvec, \\
	\hat{S}_y & = \frac{i}{2} \int \left(
		\Psiop^\dagger_2 \Psiop_1 - \Psiop^\dagger_1 \Psiop_2
	\right) d\xvec, \\
	\hat{S}_z & = \frac{1}{2} \int \left(
		\Psiop^\dagger_1 \Psiop_1 - \Psiop^\dagger_2 \Psiop_2
	\right) d\xvec.
\end{split}
\end{equation}
Averages of spin operators can be calculated straightforwardly (using the fact that interspecies commutators $[\Psiop_1, \Psiop_2] = [\Psiop^\dagger_1, \Psiop_2] = 0$):
\begin{equation*}
\begin{split}
	\langle \hat{S}_x \rangle
		& = \pathavg{\Real \int \Psi^*_1 \Psi_2 d\xvec }
		= \pathavg{\Real I}
		= \pathavg{S_x}, \\
	\langle \hat{S}_y \rangle
		& = \pathavg{\Imag \int \Psi^*_1 \Psi_2 d\xvec }
		= \pathavg{\Imag I}
		= \pathavg{S_y}, \\
	\langle \hat{S}_z \rangle
		& = \frac{1}{2} \pathavg{\int \Psi^*_1 \Psi_1 d\xvec - \int \Psi^*_2 \Psi_2 d\xvec}
		= \frac{1}{2} (\pathavg{N_1 - N_2})
		= \pathavg{S_z},
\end{split}
\end{equation*}
where we introduced auxiliary per-path interaction values $I^{j}$ and per-path spin component values, whose definitions are an intuitive consequence of equations~\eqnref{moments-calculation:spin-operators}:
\begin{equation*}
\begin{split}
	S^{(j)}_x & = \frac{1}{2} \int \left(
		\Psi^{(j)*} \Psi^{(j)}_1 + \Psi^{(j)*}_1 \Psi^{(j)}_2
	\right) d\xvec, \\
	S^{(j)}_y & = \frac{i}{2} \int \left(
		\Psi^{(j)*}_2 \Psi^{(j)}_1 - \Psi^{(j)*}_1 \Psi^{(j)}_2
	\right) d\xvec, \\
	S^{(j)}_z & = \frac{1}{2} \int \left(
		\Psi^{(j)*}_1 \Psi^{(j)}_1 - \Psi^{(j)*}_2 \Psi^{(j)}_2
	\right) d\xvec,
\end{split}
\end{equation*}
where $j$ stands for the number of the simulation path.

Second-order moments of spin operators can be obtained similarly to second-order moment of population operator, by transforming normally ordered field operator products to symmetrically ordered ones, substituting them for path averages of wavefunction moments and grouping terms.
\begin{equation*}
\begin{split}
	\langle \hat{S}^2_x \rangle
	& = \frac{1}{4} \langle \int \left(
		\Psiop^\dagger_2 \Psiop_1 + \Psiop^\dagger_1 \Psiop_2
	\right)
	\left(
		\Psiop^{\prime\dagger}_2 \Psiop^\prime_1 + \Psiop^{\prime\dagger}_1 \Psiop^\prime_2
	\right) d\xvec d\xvec^\prime \rangle \\
	& = \frac{1}{4} \langle \int \left(
		\symprod{ \Psiop^\dagger_2 \Psiop_1 \Psiop^{\prime\dagger}_2 \Psiop^\prime_1 }
		+ \symprod{ \Psiop^\dagger_1 \Psiop_2 \Psiop^{\prime\dagger}_2 \Psiop^\prime_1 }
		+ \symprod{ \Psiop^\dagger_1 \Psiop_2 \Psiop^{\prime\dagger}_1 \Psiop^\prime_2 }
		+ \symprod{ \Psiop^\dagger_2 \Psiop_1 \Psiop^{\prime\dagger}_1 \Psiop^\prime_2 }
	\right. \\
	& \left.
		+ \frac{\delta_P(\xvec,\xvec^\prime)}{2} \left(
			- \symprod{ \Psiop_2 \Psiop^{\prime\dagger}_2 }
			- \symprod{ \Psiop_1 \Psiop^{\prime\dagger}_1 }
			+ \symprod{ \Psiop^\dagger_2 \Psiop^\prime_1 }
			+ \symprod{ \Psiop^\dagger_1 \Psiop^\prime_2 }
		\right)
	\right) d\xvec d\xvec^\prime \rangle \\
	& = \frac{1}{4} \pathavg{
		\int \Psi^*_2 \Psi_1 d\xvec \int \Psi^*_2 \Psi_1 d\xvec
		+ \int \Psi^*_1 \Psi_2 d\xvec \int \Psi^*_2 \Psi_1 d\xvec \\
	&	+ \int \Psi^*_1 \Psi_2 d\xvec \int \Psi^*_1 \Psi_2 d\xvec
		+ \int \Psi^*_2 \Psi_1 d\xvec \int \Psi^*_1 \Psi_2 d\xvec } \\
	& = \frac{1}{4} \pathavg{ (I^*)^2 + I I^* + I^2 + I^* I } \\
	& = \pathavg{ (\Real I)^2 } = \pathavg{ S^2_x }
\end{split}
\end{equation*}

\begin{equation*}
\begin{split}
	\langle \hat{S}^2_y \rangle
	& = - \frac{1}{4} \langle \int \left(
		\Psiop^\dagger_2 \Psiop_1 - \Psiop^\dagger_1 \Psiop_2
	\right)
	\left(
		\Psiop^{\prime\dagger}_2 \Psiop^\prime_1 - \Psiop^{\prime\dagger}_1 \Psiop^\prime_2
	\right) d\xvec d\xvec^\prime \rangle \\
	& = - \frac{1}{4} \langle \int \left(
		\symprod{ \Psiop^\dagger_2 \Psiop_1 \Psiop^{\prime\dagger}_2 \Psiop^\prime_1 }
		- \symprod{ \Psiop^\dagger_1 \Psiop_2 \Psiop^{\prime\dagger}_2 \Psiop^\prime_1 }
		+ \symprod{ \Psiop^\dagger_1 \Psiop_2 \Psiop^{\prime\dagger}_1 \Psiop^\prime_2 }
		- \symprod{ \Psiop^\dagger_2 \Psiop_1 \Psiop^{\prime\dagger}_1 \Psiop^\prime_2 }
	\right. \\
	& \left.
		+ \frac{\delta_P(\xvec,\xvec^\prime)}{2} \left(
			\symprod{ \Psiop_2 \Psiop^{\prime\dagger}_2 }
			+ \symprod{ \Psiop_1 \Psiop^{\prime\dagger}_1 }
			- \symprod{ \Psiop^\dagger_2 \Psiop^\prime_2 }
			- \symprod{ \Psiop^\dagger_1 \Psiop^\prime_1 }
		\right)
	\right) d\xvec d\xvec^\prime \rangle \\
	& = - \frac{1}{4} \pathavg{
		\int \Psi^*_2 \Psi_1 d\xvec \int \Psi^*_2 \Psi_1 d\xvec
		- \int \Psi^*_1 \Psi_2 d\xvec \int \Psi^*_2 \Psi_1 d\xvec \\
	&	+ \int \Psi^*_1 \Psi_2 d\xvec \int \Psi^*_1 \Psi_2 d\xvec
		- \int \Psi^*_2 \Psi_1 d\xvec \int \Psi^*_1 \Psi_2 d\xvec } \\
	& = - \frac{1}{4} \pathavg{ (I^*)^2 - I I^* + I^2 - I^* I } \\
	& = \pathavg{ ( \Imag I )^2 } = \pathavg{ S^2_y }
\end{split}
\end{equation*}

\begin{equation*}
\begin{split}
	\langle \hat{S}^2_z \rangle
	& = \frac{1}{4} \langle \int \left(
		\Psiop^\dagger_1 \Psiop_1 - \Psiop^\dagger_2 \Psiop_2
	\right)
	\left(
		\Psiop^{\prime\dagger}_1 \Psiop^\prime_1 - \Psiop^{\prime\dagger}_2 \Psiop^\prime_2
	\right) d\xvec d\xvec^\prime \rangle \\
	& = \frac{1}{4} \langle \int \left(
		\symprod{ \Psiop^\dagger_1 \Psiop_1 \Psiop^{\prime\dagger}_1 \Psiop^\prime_1 }
		- \symprod{ \Psiop^\dagger_1 \Psiop_1 \Psiop^{\prime\dagger}_2 \Psiop^\prime_2 }
		- \symprod{ \Psiop^\dagger_2 \Psiop_2 \Psiop^{\prime\dagger}_1 \Psiop^\prime_1 }
		+ \symprod{ \Psiop^\dagger_2 \Psiop_2 \Psiop^{\prime\dagger}_2 \Psiop^\prime_2 }
	\right. \\
	& \left.
		+ \frac{\delta_P(\xvec,\xvec^\prime)}{2} \left(
			\symprod{ \Psiop^\dagger_1 \Psiop^\prime_1 }
			- \symprod{ \Psiop_2 \Psiop^{\prime\dagger}_2 }
			+ \symprod{ \Psiop^\dagger_2 \Psiop^\prime_2 }
			- \symprod{ \Psiop_1 \Psiop^{\prime\dagger}_1 }
		\right)
	\right) d\xvec d\xvec^\prime \rangle \\
	& = \frac{1}{4} \pathavg{
		\int \Psi^*_1 \Psi_1 d\xvec \int \Psi^*_1 \Psi_1 d\xvec
		- \int \Psi^*_1 \Psi_1 d\xvec \int \Psi^*_2 \Psi_2 d\xvec \\
	&	- \int \Psi^*_2 \Psi_2 d\xvec \int \Psi^*_1 \Psi_1 d\xvec
		+ \int \Psi^*_2 \Psi_2 d\xvec \int \Psi^*_2 \Psi_2 d\xvec } \\
	& = \frac{1}{4} \pathavg{ N^2_1 - N_1 N_2 - N_2 N_1 + N^2_2 } \\
	& = \frac{1}{4} \pathavg{ (N_1 - N_2)^2 } = \pathavg{ S^2_z }
\end{split}
\end{equation*}

\begin{equation*}
\begin{split}
	\langle \hat{S}_x \hat{S}_y + \hat{S}_y \hat{S}_x \rangle
	& = \frac{i}{4} \langle \int \left(
		\left(
			\Psiop^\dagger_2 \Psiop_1 + \Psiop^\dagger_1 \Psiop_2
		\right)
		\left(
			\Psiop^{\prime\dagger}_2 \Psiop^\prime_1 - \Psiop^{\prime\dagger}_1 \Psiop^\prime_2
		\right)
		+ \left(
			\Psiop^\dagger_2 \Psiop_1 - \Psiop^\dagger_1 \Psiop_2
		\right)
		\left(
			\Psiop^{\prime\dagger}_2 \Psiop^\prime_1 + \Psiop^{\prime\dagger}_1 \Psiop^\prime_2
		\right)
	\right) d\xvec d\xvec^\prime \rangle \\
	& = \frac{i}{2} \langle \int \left(
		\symprod{ \Psiop^\dagger_2 \Psiop_1 \Psiop^{\prime\dagger}_2 \Psiop^\prime_1 }
		- \symprod{ \Psiop^\dagger_1 \Psiop_2 \Psiop^{\prime\dagger}_1 \Psiop^\prime_2 }
	\right) d\xvec d\xvec^\prime \rangle \\
	& = \frac{i}{2} \pathavg{
		\int \Psi^*_2 \Psi_1 d\xvec \int \Psi^*_2 \Psi_1 d\xvec
		- \int \Psi^*_1 \Psi_2 d\xvec \int \Psi^*_1 \Psi_2 d\xvec } \\
	& = \frac{i}{2} \pathavg{ (I^*)^2 - I^2 } \\
	& = 2 \pathavg{ \Real I \, \Imag I } = 2 \pathavg{ S_x S_y }
\end{split}
\end{equation*}

\begin{equation*}
\begin{split}
	\langle \hat{S}_x \hat{S}_z + \hat{S}_z \hat{S}_x \rangle
	& = \frac{1}{4} \langle \int \left(
		\left(
			\Psiop^\dagger_2 \Psiop_1 + \Psiop^\dagger_1 \Psiop_2
		\right)
		\left(
			\Psiop^{\prime\dagger}_1 \Psiop^\prime_1 - \Psiop^{\prime\dagger}_2 \Psiop^\prime_2
		\right)
		+ \left(
			\Psiop^\dagger_1 \Psiop_1 - \Psiop^\dagger_2 \Psiop_2
		\right)
		\left(
			\Psiop^{\prime\dagger}_2 \Psiop^\prime_1 + \Psiop^{\prime\dagger}_1 \Psiop^\prime_2
		\right)
	\right) d\xvec d\xvec^\prime \rangle \\
	& = \frac{1}{4} \langle \int \left(
		\symprod{ \Psiop^\dagger_1 \Psiop_1 \Psiop^{\prime\dagger}_2 \Psiop^\prime_1 }
		- \symprod{ \Psiop^\dagger_2 \Psiop_2 \Psiop^{\prime\dagger}_1 \Psiop^\prime_2 }
		+ \symprod{ \Psiop^\dagger_2 \Psiop_1 \Psiop^{\prime\dagger}_1 \Psiop^\prime_1 }
		+ \symprod{ \Psiop^\dagger_1 \Psiop_2 \Psiop^{\prime\dagger}_1 \Psiop^\prime_1 }
	\right. \\
	& \left.
		- \symprod{ \Psiop^\dagger_2 \Psiop_1 \Psiop^{\prime\dagger}_2 \Psiop^\prime_2 }
		+ \symprod{ \Psiop^\dagger_1 \Psiop_1 \Psiop^{\prime\dagger}_1 \Psiop^\prime_2 }
		- \symprod{ \Psiop^\dagger_2 \Psiop_2 \Psiop^{\prime\dagger}_2 \Psiop^\prime_1 }
		- \symprod{ \Psiop^\dagger_1 \Psiop_2 \Psiop^{\prime\dagger}_2 \Psiop^\prime_2 }
	\right) d\xvec d\xvec^\prime \rangle \\
	& = \frac{1}{4} \pathavg{
		\int \Psi^*_1 \Psi_1 d\xvec \int \Psi^*_2 \Psi_1 d\xvec
		- \int \Psi^*_2 \Psi_2 d\xvec \int \Psi^*_1 \Psi_2 d\xvec
		+ \int \Psi^*_2 \Psi_1 d\xvec \int \Psi^*_1 \Psi_1 d\xvec
		+ \int \Psi^*_1 \Psi_2 d\xvec \int \Psi^*_1 \Psi_1 d\xvec \\
	&	- \int \Psi^*_2 \Psi_1 d\xvec \int \Psi^*_2 \Psi_2 d\xvec
		+ \int \Psi^*_1 \Psi_1 d\xvec \int \Psi^*_1 \Psi_2 d\xvec
		- \int \Psi^*_2 \Psi_2 d\xvec \int \Psi^*_2 \Psi_1 d\xvec
		- \int \Psi^*_1 \Psi_2 d\xvec \int \Psi^*_2 \Psi_2 d\xvec
	} \\
	& = \frac{1}{4} \pathavg{
		N_1 I^*
		- N_2 I
		+ I^* N_1
		+ I N_1
		- I^* N_2
		+ N_1 I
		- N_2 I^*
		- I N_2
	} \\
	& = \pathavg{ (N_1 - N_2) \Real I } = 2 \pathavg{ S_x S_z }
\end{split}
\end{equation*}

\begin{equation*}
\begin{split}
	\langle \hat{S}_y \hat{S}_z + \hat{S}_z \hat{S}_y \rangle
	& = \frac{i}{4} \langle \int \left(
		\left(
			\Psiop^\dagger_2 \Psiop_1 - \Psiop^\dagger_1 \Psiop_2
		\right)
		\left(
			\Psiop^{\prime\dagger}_1 \Psiop^\prime_1 - \Psiop^{\prime\dagger}_2 \Psiop^\prime_2
		\right)
		+ \left(
			\Psiop^\dagger_1 \Psiop_1 - \Psiop^\dagger_2 \Psiop_2
		\right)
		\left(
			\Psiop^{\prime\dagger}_2 \Psiop^\prime_1 - \Psiop^{\prime\dagger}_1 \Psiop^\prime_2
		\right)
	\right) d\xvec d\xvec^\prime \rangle \\
	& = \frac{i}{4} \langle \int \left(
		\symprod{ \Psiop^\dagger_1 \Psiop_1 \Psiop^{\prime\dagger}_2 \Psiop^\prime_1 }
		+ \symprod{ \Psiop^\dagger_2 \Psiop_2 \Psiop^{\prime\dagger}_1 \Psiop^\prime_2 }
		+ \symprod{ \Psiop^\dagger_2 \Psiop_1 \Psiop^{\prime\dagger}_1 \Psiop^\prime_1 }
		- \symprod{ \Psiop^\dagger_1 \Psiop_2 \Psiop^{\prime\dagger}_1 \Psiop^\prime_1 }
	\right. \\
	& \left.
		- \symprod{ \Psiop^\dagger_2 \Psiop_1 \Psiop^{\prime\dagger}_2 \Psiop^\prime_2 }
		- \symprod{ \Psiop^\dagger_1 \Psiop_1 \Psiop^{\prime\dagger}_1 \Psiop^\prime_2 }
		- \symprod{ \Psiop^\dagger_2 \Psiop_2 \Psiop^{\prime\dagger}_2 \Psiop^\prime_1 }
		+ \symprod{ \Psiop^\dagger_1 \Psiop_2 \Psiop^{\prime\dagger}_2 \Psiop^\prime_2 }
	\right) d\xvec d\xvec^\prime \rangle \\
	& = \frac{i}{4} \pathavg{
		\int \Psi^*_1 \Psi_1 d\xvec \int \Psi^*_2 \Psi_1 d\xvec
		+ \int \Psi^*_2 \Psi_2 d\xvec \int \Psi^*_1 \Psi_2 d\xvec
		+ \int \Psi^*_2 \Psi_1 d\xvec \int \Psi^*_1 \Psi_1 d\xvec
		- \int \Psi^*_1 \Psi_2 d\xvec \int \Psi^*_1 \Psi_1 d\xvec \\
	&	- \int \Psi^*_2 \Psi_1 d\xvec \int \Psi^*_2 \Psi_2 d\xvec
		- \int \Psi^*_1 \Psi_1 d\xvec \int \Psi^*_1 \Psi_2 d\xvec
		- \int \Psi^*_2 \Psi_2 d\xvec \int \Psi^*_2 \Psi_1 d\xvec
		+ \int \Psi^*_1 \Psi_2 d\xvec \int \Psi^*_2 \Psi_2 d\xvec
	} \\
	& = \frac{i}{4} \pathavg{
		N_1 I^*
		+ N_2 I
		+ I^* N_1
		- I N_1
		- I^* N_2
		- N_1 I
		- N_2 I^*
		+ I N_2
	} \\
	& = \pathavg{ (N_1 - N_2) \Imag I } = 2 \pathavg{ S_y S_z }
\end{split}
\end{equation*}

As it turns out, unlike the equation~\eqnref{moments-calculation:delta-N}, formulas for second-order moments for spin operators do not contain any additional terms depending on $M$.
Now we can calculate all spin correlations from~\cite{Li2009}:
\begin{equation*}
\begin{split}
	\Delta S^2_i
		& = \langle \hat{S}^2_i \rangle - \langle \hat{S}_i \rangle^2
		= \pathavg{ S^2_i } - \pathavg{ S_i }^2, \\
	\Delta_{ij}
		& = \langle \hat{S}_i \hat{S}_j + \hat{S}_j \hat{S}_i \rangle
		- 2 \langle \hat{S}_i \rangle \langle \hat{S}_j \rangle
		= 2 ( \pathavg{ S_i S_j } - \pathavg{ S_i } \pathavg { S_j } )
\end{split}
\end{equation*}
In other words, we proved that in the simulator application we can first calculate spin components $S^{(j)}_i$ for each simulation path, and then use common average and variance functions on resulting arrays to obtain required correlations.

\chapter{Wrapped distributions}
\label{cha:appendix:distributions}


This chapter contains methods for finding properties of the wrapped distributions on a Bloch sphere.
\todo{Need to find papers about this, should be some standard way.
Or, at least, give clear definition of mean and variance for wrapped distributions.}


% =============================================================================
\section{One-dimensional distribution}
% =============================================================================

Suppose we have the set of angles $\phi_j$, and we want to find mean and standard deviation for this distribution,
keeping in mind that it is wrapped,
i.e. for any $j$ we can replace $\phi_j$ by $\phi_j + 2\pi n$, leaving the distribution parameters intact.

There is an approximate method, which works well if the distribution has a distinguished maximum.
First, map the angles to the unit circle in the complex plane:
\[
	p_j = e^{i\phi_j}.
\]
Now if we take the mean value
\[
	p_{mean} = \langle p_j \rangle,
\]
it will be located close to the part of the circle where points are more clustered.
Then, normalisation
\[
	\tilde{p}_{mean} = p_{mean} / | p_{mean} |
\]
gives us a unit vector pointing to approximate maximum of the distribution.
Returning back to angles,
\[
	\phi_{mean} = arg(\tilde{p}_{mean}).
\]

\chapter{Transformation to harmonic oscillator basis}
\label{cha:appendix:harmonic-transform}


This chapter briefly describes transformations to and from harmonic basis,
which were explained in detail in~\cite{Dion2003}.


% =============================================================================
\section{One-dimensional oscillator}
% =============================================================================

Consider one-dimensional harmonic oscillator with the Hamiltonian
\[
	H = -\frac{\hbar^2 \nabla^2}{2 m} + \frac{m \omega^2 x^2}{2},
\]
where $m$ is the mass of a particle and $\omega$ is the oscillator frequency.
Characteristic length for this oscillator is
\[
	l_x = \sqrt{\frac{\hbar}{m \omega}}.
\]
Then the eigenfunctions of the Hamiltonian are
\begin{equation}
\label{eqn:harmonic-transform:harmonic-modes}
	\phi_n = \frac{1}{\sqrt{2^n n! l_x} \sqrt[4]{\pi}} H_n(x / l_x)
		\exp \left( -\frac{x^2}{2 l_x^2} \right),
\end{equation}
where $H_n$ is ``physicists'\,'' Hermite polynomial of order $n$.
Corresponding eigenvalues are
\[
	E_n = \hbar \omega (n + \frac{1}{2}).
\]
One can easily check that this set is orthonormal:
\[
	\int\limits_{-\infty}^{\infty} \phi_m(x) \phi_n(x) dx = \delta_{mn}.
\]

Given some function $f(x)$, one can expand it into harmonic oscillator basis as
\[
	C_n = \int f(x) \phi_n(x) dx,
\]
and the backward transformation is, obviously,
\[
	f(x) = \sum_{n} C_n \phi_n(x).
\]
In general, when we do not know anything about $f(x)$,
the value of the integral can be calculated only approximately.
But we can obtain exact results for functions of a certain type,
and by choosing points where we want to sample $f(x)$.

The method is based on a Gauss-Hermite quadrature \todo{citation needed?},
which states that the value of the integral can be approximated as
\[
	\int\limits_{-\infty}^{\infty} g(x) e^{-x^2} dx
	= \sum_{i=1}^N w_i g(r_i) + E,
\]
where $N$ is the number of sample points.
Weights $w_i$ are calculated as
\[
	w_i = \frac{2^{N-1} N! \sqrt{\pi}}{N^2 (H_{N-1}(r_i))^2},
\]
and sample points $r_i$ are roots of Hermite polynomial $H_N$.
The error term is
\[
	E = \frac{N! \sqrt{\pi}}{2^N (2N)!} g^{(2N)}(\xi).
\]
Therefore if $g(x)$ is a polynomial of order $M$,
one can eliminate the error term by choosing $N$ so that $2N \ge M + 1$,
thus making the integration exact.

Now let us say we want to find population of the first $M$ modes for $f(x) = \Psi(x)^s$,
where $\Psi(x) = \sum_{m=0}^{M-1} C_m \phi_m(x)$ and $s$ is a natural number.
This means that wee need to integrate
\[
	F_m = \int \Psi(x)^s \phi_m(x) dx.
\]
By definition of mode functions~\eqnref{harmonic-transform:harmonic-modes},
\[
	\Psi(x)^s \phi_m(x) = P(x / l_x) \exp \left( -\frac{(s+1) x^2}{2 l_x^2} \right),
\]
where $P(x)$ is the polynomial of order less than $(M-1)s + m$.
Since we want to have the same set of sample points for any $m \in [0, M-1]$,
we will consider the worst case $m = M-1$,
which makes the order of $P(x)$ limited by $(M-1)(s+1)$.
The integral can now be transformed to the form necessary to apply Gauss-Hermite quadrature:
\begin{equation*}
\begin{split}
	F_m
	& = \int P(x / l_x) \exp \left( -\frac{(s+1) x^2}{2 l_x^2} \right) dx \\
	& = \int l_x \sqrt{\frac{2}{s+1}} P \left( y \sqrt{\frac{2}{s+1}} \right) e^{-y^2} dy \\
	& = \sum_{i=1}^N w_i P \left( r_i \sqrt{\frac{2}{s+1}} \right) l_x \sqrt{\frac{2}{s+1}} \\
	& = \sum_{i=1}^N w_i
		\Psi \left( l_x r_i \sqrt{\frac{2}{s+1}} \right)^s
		\phi_m \left( l_x r_i \sqrt{\frac{2}{s+1}} \right)
		\exp(r_i^2) l_x \sqrt{\frac{2}{s+1}} \\
	& = \sum_{i=1}^N \tilde{w}_i \phi_m(\tilde{x}_i) f(\tilde{x}_i).
\end{split}
\end{equation*}
Here the sample points are
\[
	\tilde{x}_i = l_x r_i \sqrt{\frac{2}{s+1}},
\]
and modified weights are
\[
	\tilde{w}_i = w_i l_x \sqrt{\frac{2}{s+1}} \exp(r_i^2).
\]
The number of sampling points is determined by the order of $P(x)$:
\[
	N \ge \frac{(M - 1)(s + 1)}{2}.
\]

Since we usually need population for all modes at once,
it is more convenient to use the decomposition in matrix form:
\[
	\bm{F} = \Phi\,\mathrm{diag}(\tilde{\bm{w}}) \bm{f},
\]
where $\Phi_{mi} = \phi_m(\tilde{x}_i)$,
$\tilde{\bm{w}}$ is a vector of elements $\tilde{w}_i$,
and $\bm{f}$ is a vector of elements $f(\tilde{x}_i)$.
Corresponding backward transform is then expressed as
\[
	\bm{f} = \Phi^T \bm{F}.
\]

Note that the same method is applicable for $f(x) = \Psi(x)^a (\Psi^*(x))^b$,
in which case $s = a + b$.



\bibliographystyle{unsrt}
\bibliography{thesis}

\end{document}