\newtheorem{theorem}{Theorem}
\newtheorem{lemma}{Lemma}
\newtheorem{definition}{Definition}

\newcommand{\todo}[1]{\textcolor{red}{[#1]}}
\newcommand{\citationneeded}{\textcolor{red}{[citation needed]}}

\newcommand{\balpha}{\mathbf{\alpha}}
\newcommand{\bdelta}{\mathbf{\delta}}
\newcommand{\blambda}{\mathbf{\lambda}}
\newcommand{\bfeta}{\mathbf{\eta}}
\newcommand{\bxi}{\mathbf{\xi}}
\newcommand{\bPsi}{\mathbf{\Psi}}
\newcommand{\bLambda}{\mathbf{\Lambda}}
\newcommand{\bPhi}{\mathbf{\Phi}}
\newcommand{\bXi}{\mathbf{\Xi}}

\newcommand{\avec}{\mathbf{a}}
\newcommand{\evec}{\mathbf{e}}
\newcommand{\jvec}{\mathbf{j}}
\newcommand{\kvec}{\mathbf{k}}
\newcommand{\lvec}{\mathbf{l}}
\newcommand{\mvec}{\mathbf{m}}
\newcommand{\nvec}{\mathbf{n}}
\newcommand{\pvec}{\mathbf{p}}
\newcommand{\svec}{\mathbf{s}}
\newcommand{\xvec}{\mathbf{x}}
\newcommand{\zvec}{\mathbf{z}}
\newcommand{\Zvec}{\mathbf{Z}}

\newcommand{\bpartial}{\mathbf{\partial}}

\newcommand{\ecut}{\epsilon_{\mathrm{cut}}}
\newcommand{\Tr}{\operatorname{Tr}}
\newcommand{\Trace}[1]{\Tr \left\{ #1 \right\}}

\newcommand{\symprod}[1]{\left\{ #1 \right\}_{\mathrm{sym}}}
\newcommand{\pathavg}[1]{\langle #1 \rangle_{\mathrm{paths}}}
\newcommand{\Real}{\operatorname{Re}}
\newcommand{\Imag}{\operatorname{Im}}

\newcommand{\Psivec}{\mathbf{\Psi}}
\newcommand{\Psiop}{\hat{\Psi}}
\newcommand{\Psiopvec}{\hat{\mathbf{\Psi}}}

% {C, H} - i.e. either complex number or operator from Hilbert space
% making it a macro, because I'm not sure what the letter should be
\newcommand{\BasicType}{\todo{remove that}}
\newcommand{\fullbasis}{\mathbb{B}}
\newcommand{\restbasis}{\mathbb{M}}

\newcommand{\eqnref}[1]{(\ref{eqn:#1})}
\newcommand{\figref}[1]{Fig.~\ref{fig:#1}}
\newcommand{\charef}[1]{Chapter~\ref{cha:#1}}
\newcommand{\appref}[1]{Appendix~\ref{cha:appendix:#1}}
\newcommand{\thmref}[1]{Theorem~\ref{thm:#1}}
\newcommand{\lmmref}[1]{Lemma~\ref{lmm:#1}}
\newcommand{\defref}[1]{Definition~\ref{def:#1}}

\NewEnviron{eqn}{\begin{align}\begin{split}\BODY\end{split}\end{align}}
\NewEnviron{eqns}{\begin{align} \BODY \end{align}}
\NewEnviron{eqn*}{\begin{equation*}\begin{split} \BODY \end{split}\end{equation*}}

% Split equation with three columns
% - use && as the second separator
% - write equal signs as ={} to get correct spacing
\NewEnviron{eqn2}{\begin{equation}\begin{alignedat}{2} \BODY \end{alignedat}\end{equation}}
\NewEnviron{eqn2*}{\begin{equation*}\begin{alignedat}{2} \BODY \end{alignedat}\end{equation*}}
